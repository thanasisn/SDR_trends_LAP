%  LaTeX support: latex@mdpi.com
%  For support, please attach all files needed for compiling as well as the log file, and specify your operating system, LaTeX version, and LaTeX editor.

%=================================================================
% pandoc conditionals added to preserve backwards compatibility with previous versions of rticles

\documentclass[applsci,article,submit,moreauthors,pdftex]{Definitions/mdpi}


%% Some pieces required from the pandoc template
\setlist[itemize]{leftmargin=*,labelsep=5.8mm}
\setlist[enumerate]{leftmargin=*,labelsep=4.9mm}


%--------------------
% Class Options:
%--------------------

%---------
% article
%---------
% The default type of manuscript is "article", but can be replaced by:
% abstract, addendum, article, book, bookreview, briefreport, casereport, comment, commentary, communication, conferenceproceedings, correction, conferencereport, entry, expressionofconcern, extendedabstract, datadescriptor, editorial, essay, erratum, hypothesis, interestingimage, obituary, opinion, projectreport, reply, retraction, review, perspective, protocol, shortnote, studyprotocol, systematicreview, supfile, technicalnote, viewpoint, guidelines, registeredreport, tutorial
% supfile = supplementary materials

%----------
% submit
%----------
% The class option "submit" will be changed to "accept" by the Editorial Office when the paper is accepted. This will only make changes to the frontpage (e.g., the logo of the journal will get visible), the headings, and the copyright information. Also, line numbering will be removed. Journal info and pagination for accepted papers will also be assigned by the Editorial Office.

%------------------
% moreauthors
%------------------
% If there is only one author the class option oneauthor should be used. Otherwise use the class option moreauthors.

%---------
% pdftex
%---------
% The option pdftex is for use with pdfLaTeX. Remove "pdftex" for (1) compiling with LaTeX & dvi2pdf (if eps figures are used) or for (2) compiling with XeLaTeX.

%=================================================================
% MDPI internal commands - do not modify
\firstpage{1}
\makeatletter
\setcounter{page}{\@firstpage}
\makeatother
\pubvolume{1}
\issuenum{1}
\articlenumber{0}
\pubyear{2023}
\copyrightyear{2023}
%\externaleditor{Academic Editor: Firstname Lastname}
\datereceived{ }
\daterevised{ } % Comment out if no revised date
\dateaccepted{ }
\datepublished{ }
%\datecorrected{} % For corrected papers: "Corrected: XXX" date in the original paper.
%\dateretracted{} % For corrected papers: "Retracted: XXX" date in the original paper.
\hreflink{https://doi.org/} % If needed use \linebreak
%\doinum{}
%\pdfoutput=1 % Uncommented for upload to arXiv.org

%=================================================================
% Add packages and commands here. The following packages are loaded in our class file: fontenc, inputenc, calc, indentfirst, fancyhdr, graphicx, epstopdf, lastpage, ifthen, float, amsmath, amssymb, lineno, setspace, enumitem, mathpazo, booktabs, titlesec, etoolbox, tabto, xcolor, colortbl, soul, multirow, microtype, tikz, totcount, changepage, attrib, upgreek, array, tabularx, pbox, ragged2e, tocloft, marginnote, marginfix, enotez, amsthm, natbib, hyperref, cleveref, scrextend, url, geometry, newfloat, caption, draftwatermark, seqsplit
% cleveref: load \crefname definitions after \begin{document}

%=================================================================
% Please use the following mathematics environments: Theorem, Lemma, Corollary, Proposition, Characterization, Property, Problem, Example, ExamplesandDefinitions, Hypothesis, Remark, Definition, Notation, Assumption
%% For proofs, please use the proof environment (the amsthm package is loaded by the MDPI class).

%=================================================================
% Full title of the paper (Capitalized)
\Title{Trends from 30-year observations of downward solar irradiance in
Thessaloniki, Greece}

% MDPI internal command: Title for citation in the left column
\TitleCitation{Trends from 30-year observations of downward solar
irradiance in Thessaloniki, Greece}

% Author Orchid ID: enter ID or remove command
%\newcommand{\orcidauthorA}{0000-0000-0000-000X} % Add \orcidA{} behind the author's name
%\newcommand{\orcidauthorB}{0000-0000-0000-000X} % Add \orcidB{} behind the author's name


% Authors, for the paper (add full first names)
\Author{Athanasios
Natsis$^{1}$\href{https://orcid.org/0000-0002-5199-4119}
{\orcidicon}, Alkiviadis Bais$^{1,*}$, Charikleia Meleti$^{1}$}


%\longauthorlist{yes}


% MDPI internal command: Authors, for metadata in PDF
\AuthorNames{Athanasios Natsis, Alkiviadis Bais, Charikleia Meleti}

% MDPI internal command: Authors, for citation in the left column
%\AuthorCitation{Lastname, F.; Lastname, F.; Lastname, F.}
% If this is a Chicago style journal: Lastname, Firstname, Firstname Lastname, and Firstname Lastname.
\AuthorCitation{Natsis, A.; Bais, A.; Meleti, C.}

% Affiliations / Addresses (Add [1] after \address if there is only one affiliation.)
\address{%
$^{1}$ \quad Aristotle University of Thessaloniki - Laboratory of
Atmospheric Physics, Campus Box 149, 54124 Thessaloniki,
Greece; \href{mailto:natsisphysicist@gmail.com}{\nolinkurl{natsisphysicist@gmail.com}}
(A.N.); \href{mailto:abais@auth.gr}{\nolinkurl{abais@auth.gr}} (A.B.);
\href{mailto:meleti@auth.gr}{\nolinkurl{meleti@auth.gr}} (C.M.)\\
}

% Contact information of the corresponding author
\corres{Correspondence: \href{mailto:abais@auth.gr}{\nolinkurl{abais@auth.gr}}}

% Current address and/or shared authorship








% The commands \thirdnote{} till \eighthnote{} are available for further notes

% Simple summary

%\conference{} % An extended version of a conference paper

% Abstract (Do not insert blank lines, i.e. \\)
\abstract{The shortwave downward solar irradiance (SDR) is an important
factor that drives climate processes, production and can affect all
living organisms. Observations of SDR at different locations around the
world with different environmental characteristics have been used to
investigate its long-term variability and trends at different time
scales. Periods of positive trends are referred as brightening periods
and of negative trends as dimming periods. Here we studied 30 years of
pyranometer data in Thessaloniki, Greece, under three types of sky
conditions (clear sky, cloudy sky and all sky). The clear-sky data were
identified by applying a cloud screening algorithm. We found a positive
trend of \(0.38\,\%/\text{year}\) for all-sky, \(0.35\,\%/\text{year}\)
for clear-sky conditions, and \(-0.28\,\%/\text{year}\) for cloudy
conditions. We have also investigated the consistency of these trends,
their seasonal variability, and the effect of the solar zenith angle. We
have found that for all-sky and clear-sky conditions the SDR trend is
positive in winter (\(0.7\) and \(0.8\,\%/\text{year}\), respectively)
and autumn (\(~0.4\,\%/\text{year}\)), while under cloudy skies the
trend is negative (\(-0.9\,\%/\text{year}\) in winter and
\(-0.4\,\%/\text{year}\) in autumn). In spring and summer the trend is
very close to zero, irrespective of sky conditions. The SDR trend is
increasing with increasing solar zenith angle, except under cloudy skies
where the trend is highly variable and close to zero. Finally, we
identified some anomalies in the long term SDR trends for all sky
conditions by examining the cumulative sums of monthly anomalies from
the climatological mean.}


% Keywords
\keyword{GHI; SDR; solar radiation; Solar Brigthening/Dimming; aerosols;
clouds.}

% The fields PACS, MSC, and JEL may be left empty or commented out if not applicable
%\PACS{J0101}
%\MSC{}
%\JEL{}

%%%%%%%%%%%%%%%%%%%%%%%%%%%%%%%%%%%%%%%%%%
% Only for the journal Diversity
%\LSID{\url{http://}}

%%%%%%%%%%%%%%%%%%%%%%%%%%%%%%%%%%%%%%%%%%
% Only for the journal Applied Sciences

%%%%%%%%%%%%%%%%%%%%%%%%%%%%%%%%%%%%%%%%%%

%%%%%%%%%%%%%%%%%%%%%%%%%%%%%%%%%%%%%%%%%%
% Only for the journal Data



%%%%%%%%%%%%%%%%%%%%%%%%%%%%%%%%%%%%%%%%%%
% Only for the journal Toxins


%%%%%%%%%%%%%%%%%%%%%%%%%%%%%%%%%%%%%%%%%%
% Only for the journal Encyclopedia


%%%%%%%%%%%%%%%%%%%%%%%%%%%%%%%%%%%%%%%%%%
% Only for the journal Advances in Respiratory Medicine
%\addhighlights{yes}
%\renewcommand{\addhighlights}{%

%\noindent This is an obligatory section in “Advances in Respiratory Medicine”, whose goal is to increase the discoverability and readability of the article via search engines and other scholars. Highlights should not be a copy of the abstract, but a simple text allowing the reader to quickly and simplified find out what the article is about and what can be cited from it. Each of these parts should be devoted up to 2~bullet points.\vspace{3pt}\\
%\textbf{What are the main findings?}
% \begin{itemize}[labelsep=2.5mm,topsep=-3pt]
% \item First bullet.
% \item Second bullet.
% \end{itemize}\vspace{3pt}
%\textbf{What is the implication of the main finding?}
% \begin{itemize}[labelsep=2.5mm,topsep=-3pt]
% \item First bullet.
% \item Second bullet.
% \end{itemize}
%}


%%%%%%%%%%%%%%%%%%%%%%%%%%%%%%%%%%%%%%%%%%


% tightlist command for lists without linebreak
\providecommand{\tightlist}{%
  \setlength{\itemsep}{0pt}\setlength{\parskip}{0pt}}



\usepackage{subcaption}
\captionsetup[sub]{position=bottom, labelfont={bf, small, stretch=1.17}, labelsep=space, textfont={small, stretch=1.17}, aboveskip=6pt,  belowskip=-6pt, singlelinecheck=off, justification=justified}
\usepackage{placeins}
\usepackage{longtable}
\usepackage{booktabs}
\usepackage{array}
\usepackage{multirow}
\usepackage{wrapfig}
\usepackage{float}
\usepackage{colortbl}
\usepackage{pdflscape}
\usepackage{tabu}
\usepackage{threeparttable}
\usepackage{threeparttablex}
\usepackage[normalem]{ulem}
\usepackage{makecell}
\usepackage{xcolor}

\begin{document}



%%%%%%%%%%%%%%%%%%%%%%%%%%%%%%%%%%%%%%%%%%

\hypertarget{introduction.}{%
\section{Introduction.}\label{introduction.}}

The shortwave downward solar irradiance (SDR) at Earth's surface plays a
significant role, on its climate. Changes of the SDR can be related to
changes on Earth's energy budget, the mechanisms of climate change, and
water and carbon cycles \citep{Wild2009}. It can also affect solar and
agricultural production, and all living organisms. Studies of SDR
variability, have identified some distinct SDR trends on different
regions of the world on different time periods. The term `brightening'
is generally used to describe periods of positive SDR trend, and
`dimming' for negative trend \citep{Wild2009}. There are many cases in
the long term records of irradiance, showing a systematic change in the
magnitude of the trend, occurring roughly in the last decades of the
20th century. On multiple stations in China, a dimming period was
reported until about 2000, followed by a brightening period
\citep{Yang2021}. A similar pattern was identified, with the breaking
point around 1980, for stations in Central Europe \citep{Wild2021} and
Brazil \citep{Yamasoe2021}. On global scale, an artificial Intelligence
aided spatial analysis on continental level with data from multiple
stations reach similar conclusions for these regions and for the global
trend \citep{Yuan2021}.

There is a consensus among researchers that the major factor affecting
the variability of SDR attenuation is the interactions of solar
radiation with atmospheric aerosols and clouds. Those interactions,
among other factors, have been analysed with models
\citep{Li2016, Samset2018}, showing the existence of feedback mechanisms
between the two. Similar findings have been shown in observational data
\citep[ and references
therein]{Schwarz2020, Ohvril2009, Zerefos2009, Xia2007}.

Due to the significant spatial and temporal variability of the trends,
and the contributing factors, there is a constant need to monitor and
investigate SDR in different sites in order to estimate the degree of
variability, and its relation to the local conditions. In this study, we
examine the trends of SDR, using ground-based measurements at
Thessaloniki, Greece, for the period 1993 to 2023, as derived from a
CM-21 pyranometer. We reevaluated and extended the dataset used by
\citet{Bais2013}, we applied a different algorithm for the
identification of clear-/cloud-sky instances
\citep{Reno2016, Reno2012a}, and we derived the SDR trends for the
period of study, under different sky conditions (all-sky, clear-sky and
cloud-sky). Finally, we investigated the dependence of the trends on
solar zenith angle and season.

\hypertarget{observational-data-and-methodology.}{%
\section{Observational data and
methodology.}\label{observational-data-and-methodology.}}

The SDR data were measured with a Kipp \& Zonen CM-21 pyranometer
operating continuously at the Laboratory of Atmospheric Physics of the
Aristotle University of Thessaloniki (\(40^\circ\,38'\,\)N,
\(22^\circ\,57'\,\)E, \(80\,\)m~a.s.l.) in the period from 1993-04-13 to
2023-04-12. The monitoring site is located near the city centre, and we
expect to be affected by the urban environment. During the study period,
the pyranometer has been independently calibrated three times at the
Meteorologisches Observatorium Lindenberg, DWD, when it was verified the
stability of the instrument to within better than \(0.7\%\) relative to
the initial calibration by the manufacturer. Along with SDR, the direct
beam radiation (DNI) was also measured by a collocated Kipp \& Zonen
CHP-1 pyrheliometer, for the period 2016-04-01 to 2023-04-12. Although,
we have performed a similar analysis to the DNI data, the results are
not presented here, as they lack the appropriate statistical
significance, due to the sorter duration of the data. However, the DNI
data were used as auxiliary data, in the clear sky identification
algorithm (CSid), which is discussed later, for the selection of the
appropriate thresholds. It is noted that despite the capability of the
CSid algorithm to use the DNI as a characterization parameter, we
haven't utilized it here, to avoid any selection bias, due to unequal
length of the two datasets. There are four distinct steps in the
creation of the dataset analysed here: a)~the acquisition of radiation
measurements from the sensors, b)~the data quality check, c)~the
identification of ``clear sky'' conditions from the radiometric data,
and d)~the aggregation of data and trend analysis.

For the acquisition of radiometric data, the signal of the pyranometer
is sampled with a rate of \(1\,\text{Hz}\). The mean and the standard
deviation of these samples are recorded every minute. The measurements
are corrected for the zero offset (``dark signal'' in volts). The ``dark
signal'' is calculated by averaging all measurements recorded for a
period of \(3\,\text{h}\), before (morning) or after (evening) the Sun
reaches an elevation angle of \(-10^\circ\). The signal is converted to
irradiance using a ramped value of the instrument's sensitivity between
calibrations.

A manual screening was performed, to remove inconsistent and erroneous
recordings that can occur stochastically or systematically, during the
continuous operation of the instruments. The manual screening is aided
by a radiation data quality assurance procedure, adjusted for the site,
which is based on the methods of Long and
Shi~\citetext{\citeyear{Long2008a}; \citeyear{Long2006}}. Thus,
problematic recordings have been excluded from further processing.
Although it is impossible to detect all false data, the large number of
available data, and the aggregation scheme we used, ensures the good
quality of the radiation measurements used in this study.

In order to be able to estimate the effect of the sky conditions on the
long term variability of SDR, we created three datasets, by
characterizing each one-minute measurement with a corresponding sky
condition (i.e., all-sky, clear-sky and cloudy-sky). To identify the
clear-sky conditions we used a method proposed by \citet{Long2000} and
by \citet{Reno2016}, which were adapted and configured for the site, as
the authors suggest.

We have to note, that the definition of clear or cloudy sky, has some
subjectivity, in any method of characterization. As a result, the
details of the definition are site specific, it relies on a combination
of thresholds and comparisons with ideal radiation models and
statistical analysis of different signal metrics. The CSid algorithm was
calibrated with the main focus, to identify the presence of clouds on
the sky. Despite the fine-tuning of the procedure, a few marginal cases
exist, that have been identified manually as false positive or false
negative but cannot affect the final results of the study.

For completeness, we will provide below a brief overview of the clear
sky identification algorithm (CSid), along with the site specific
thresholds. To calculate the reference clear sky
\(\text{SDR}_\text{CSref}\) we used the \(\text{SDR}_\text{Haurwitz}\)
derived by the radiation model of \citet{Haurwitz1945}, adjusted for our
site with a factor \(a\) (Eq.~\ref{eq:ahau}), resulted by an iterative
optimization process, as described by \citet{Long2000} and
\citet{Reno2016}. The target of the optimization was the minimization of
a function \(f(a)\) (Eq.~\ref{eq:minf}) and was accomplished with the
algorithmic function ``optimise'', which is an implementation based on
the work of \citet{Brent1973}, from the library ``stats'' of the R
programming language \citep{RCT2023}. \begin{equation}
f(a) = \frac{1}{n}\sum_{i=1}^{n} ( \text{SDR}_{\text{CSid},i} - a \times \text{SDR}_{\text{testCSref},i} )^2 \label{eq:minf}
\end{equation} where: \(n\) is the total number of daylight records,
\(\text{SDR}_{\text{CSid},i}\) are the records identified as clear sky
by CSid, \(a\) is a hypothetical adjustment factor, and
\(\text{SDR}_{\text{testCSref},i}\) is any of the tested clear sky
radiation models.

The optimization and the selection of the clear sky reference model, was
performed on SDR observations for the period 2016 - 2021. During the
optimization, eight simple clear sky radiation models were tested
(namely, Daneshyar-Paltridge-Proctor, Kasten-Czeplak, Haurwitz,
Berger-Duffie, Adnot-Bourges-Campana-Gicquel, Robledo-Soler, Kasten and
Ineichen-Perez), with a wide range of factors. These models are
described in more details by \citet{Reno2012} and evaluated by
\citet{Reno2016}. We found, that Haurwitz's model, adjusted with the
factor \(a = 0.965\) yields one of the lowest root mean squared errors
(RMSE), while the procedure, manages to characterize the majority of the
data. Thus, our clear sky reference is derived by the Eq.~\ref{eq:ahau}.
\begin{equation}
\text{SDR}_\text{CSref} = a \times \text{SDR}_\text{Haurwitz} = 0.965 \times 1098 \times \cos(\theta) \times \exp \left( \frac{ - 0.057}{\cos(\theta)} \right) \label{eq:ahau}
\end{equation} where: \(\text{SDR}_\text{CSref}\) is the reference clear
sky SDR, in \(\text{W}\,\text{m}^{-2}\) and \(\theta\) is the solar
zenith angle (SZA).

The criteria that were used to identify whether a measurement was taken
under clear-sky conditions are presented below. A data point is flagged
as ``clear-sky'' if all criteria are satisfied, otherwise it is
considered to be ``cloud-sky''. Each criterion was applied for a running
window of \(11\) consecutive one-minute measurements, and the
characterization is assigned to the central value of the window. Each
parameter, was calculated both from the observations and the reference
clear sky model, for each comparison. The allowable range of variation
is defined by the model-derived value of the parameter multiplied by a
factor plus an offset. The factors and the offsets were determined
empirically, by manually inspecting each filter's performance on
selected days and adjusting them accordingly during an iterative
process. The criteria are:

\begin{enumerate}
\def\labelenumi{\alph{enumi})}
\tightlist
\item
  Mean of the measured \(\overline{\text{SDR}}_i\) (Eq.
  \ref{eq:MeanVIP}). \begin{equation}
  0.91 \times \overline{\text{SDR}}_{\text{CSref},i} - 20\,Wm^{-2}
  < \overline{\text{SDR}}_i <
  1.095 \times \overline{\text{SDR}}_{\text{CSref},i} + 30\,Wm^{-2}
  \label{eq:MeanVIP}
  \end{equation}
\item
  Maximum measured value \(M_{i}\) (Eq.~\ref{eq:MaxVIP}).
  \begin{equation}
  1 \times M_{\text{CSref},i} - 75\,Wm^{-2}
  < M_{i} <
  1 \times M_{\text{CSref},i} + 75\,Wm^{-2}
  \label{eq:MaxVIP}
  \end{equation}
\item
  Length \(L_i\) of the sequential line segments, connecting the points
  of the \(11\) SDR values (Eq. \ref{eq:VILeq}). \begin{equation}
  L_i = \sum_{i=1}^{n-1}\sqrt{\left ( \text{SDR}_{i+1} - \text{SDR}_{i}\right )^2 + \left ( t_{i+1} - t_i \right )^2}
  \label{eq:VILeq}
  \end{equation} \begin{equation}
  1 \times L_{\text{CSref},i} - 5 < L_i < 1.3 \times L_{\text{CSref},i} + 13
  \label{eq:VILcr}
  \end{equation} where: \(t_i\) is the time stamp of each SDR
  measurement.
\item
  Standard deviation \(\sigma_i\) of the slope (\(s_i\)) between the
  \(11\) sequential points, normalized by the mean
  \(\overline{\text{SDR}}_i\) (Eq.~\ref{eq:VCT1}). \begin{gather}
    \sigma_i = \frac {\sqrt{\frac{1}{n-1} \sum_{i=1}^{n-1} \left( s_i - \bar{s} \right)^2}} {\overline{\text{SDR}}_i} \label{eq:VCT1} \\
    s_i = \frac{\text{SDR}_{i+1} - \text{SDR}_{i}}{t_{i+1} - t_i},\;\;   \bar{s} = \frac{1}{n-1} \sum_{i=1}^{n-1} s_i,\;\;\forall i \in \left \{ 1, 2, \ldots, n-1 \right \}\;\;
  \end{gather} For this criterion, \(\sigma_i\) should be below a
  certain threshold (Eq.~\ref{eq:VCTcr}): \begin{equation}
    \sigma_i < \ensuremath{1.1\times 10^{-4}} \label{eq:VCTcr}
  \end{equation}
\item
  Maximum difference \(X_i\) between the change in measured irradiance
  and the change in clear sky irradiance over each measurement interval.
  \begin{gather}
    X_i = \max{\left \{ \left | x_i - x_{\text{CSref},i} \right | \right \}} \label{eq:VSM3} \\
    x_i = \text{SDR}_{i+1} - \text{SDR}_{i} \forall i \in \left \{ 1, 2, \ldots, n-1 \right \} \label{eq:VSM1} \\
    x_{\text{CSref},i} = \text{SDR}_{\text{CSref},i+1} - \text{SDR}_{\text{CSref},i} \forall i \in \left \{ 1, 2, \ldots, n-1 \right \} \label{eq:VSM2}
  \end{gather} For this criterion, \(X_i\) should be below a certain
  threshold (Eq.~\ref{eq:VSMcr}): \begin{equation}
    X_i < 7.5\,Wm^{-2} \label{eq:VSMcr}
  \end{equation}
\end{enumerate}

Due to the significant measurement uncertainty when the Sun is near the
horizon, we have excluded all measurements with SZA greater than
\(85^\circ\). Moreover, due to some obstructions around the site (hills
and buildings), we excluded data with azimuth angle between \(58^\circ\)
and \(120^\circ\) with SZA greater than \(78^\circ\). On the latter
instances, Sun is systematically not visible from the instrument's
location. To make the measurements comparable throughout the dataset, we
adjusted all one-minute radiometric values to the mean Sun - Earth
distance. Subsequently, we made all measurements relative to the Total
Solar Irradiance (TSI) at \(1\,\text{au}\), in order to compensate for
the Sun's intensity variability, using a time series of satellite TSI
observations. The TSI data we used are part of the ``NOAA Climate Data
Record of Total Solar Irradiance'' dataset \citep{Coddington2005}. The
initial daily values of this dataset were interpolated to match the time
step of our measurements. In the final dataset \(38\%\) of the 1-minute
data were identified as under clear-sky conditions and \(57.8\%\) as
under cloud-sky conditions.

In order to investigate the SDR trends, we implemented an appropriate
aggregation scheme to the 1-minute data to derive a series in coarser
time-scales. To preserve the representativeness of the data we used the
following criteria: a) we accept only days with more than 50\% of the
daytime measurements present and valid, b) on the dataset of clear- and
cloudy-skies, we included only days with more than 20\% of the daytime
measurements identified as clear or cloudy respectively, c) monthly
values were computed from daily means only when at least 20 days were
available. To create the daily and monthly climatological means, we
averaged the data based on the day of year and calendar month,
respectively. For the seasonal means we averaged the mean daily values
in each season (Winter: December - February, Spring: March - May, etc.).
Finally, each data set was deseasonalized by subtracting the
corresponding climatological annual cycle (daily or monthly) from the
actual data. To estimate the SZA effect on the SDR trends, the
one-minute data were aggregated in \(1^\circ\) SZA bins, separately for
the morning and afternoon hours, and then were deseasonalized as
mentioned above.

\hypertarget{results}{%
\section{Results}\label{results}}

\hypertarget{long-term-sdr-trends}{%
\subsection{Long-term SDR trends}\label{long-term-sdr-trends}}

We calculated the linear SDR trends, from the departures of the mean
daily values from the daily climatology and for the three sky conditions
(Table~\ref{tab:trendtable}). In Figure~\ref{fig:trendALL} we present
only the time series under all-sky conditions; the plots for clear-sky
and cloud-sky conditions, are very similar and are shown in the Appendix
(Figures~\ref{fig:trendCLEAR} and~ \ref{fig:trendCLOUD}). We observe a
positive trend for all-sky conditions (\(0.38\,\%/y\)), a very close but
smaller trend for clear-skies (\(0.024\,\%/y\)) and a negative weaker
trend for cloudy-skies (\(0.34\,\%/y\)). In the studied period, there is
no significant break or change in the variability of the time series.
Other studies for the European region reported a change of the SDR
slope, around 1980 \citep{Wild2021, Yuan2021, Ohmura2009}, a few years
before the start of our records. It is interesting to note, that for the
observations period, the trend of the TSI is \(-0.00024\,\%/y\), and
thus we can not attribute any major effect on SDR trend to Solar
variability.

\begin{table}[H]

\caption{\label{tab:trendtable}Trends in SDR daily means for different sky conditions for the period 1993 - 2023.}
\begin{tabu} to \linewidth {>{\centering\arraybackslash}p{8em}>{\raggedleft}X>{\raggedleft}X>{\raggedleft}X>{\raggedleft}X}
\toprule
Sky conditions & Trend [\%/year] & Trend p-value & Pearson cor. & Days\\
\midrule
All skies & 0.3756 & 0.000 & 0.0911 & 10250\\
Clear skies & 0.0239 & 0.189 & 0.0172 & 5803\\
Cloudy skies & 0.3416 & 0.000 & 0.0725 & 8113\\
\bottomrule
\end{tabu}
\end{table}

\begin{figure}[h!]

{\centering \includegraphics[width=.70\linewidth]{./images/LongtermTrends-2} 

}

\caption{Anomalies (\%) of the daily all-sky SDR, relative to climatological values for 1993 - 2023. The black line shows the long term linear trend.}\label{fig:trendALL}
\end{figure}

Although the year-to-year variability of the anomalies (Figure
\ref{fig:trendALL} and Figures \ref{fig:trendCLEAR},
\ref{fig:trendCLOUD} in Appendix), shows a rather homogeneous behaviour,
plots of the cumulative sums (CUSUM) \citep{Regier2019} of the anomalies
can reveal different structures in the records of all three sky
conditions. In the cases of all-sky and clear-sky conditions (Figures
\ref{fig:cusummonth-1} and \ref{fig:cusummonth-2}), we observe three
macroscopic periods. A downward part from the start until about 2005, a
relatively steady part until about 2016 and, finally, a steep upward
part until the present. For cloud-sky (Figure~\ref{fig:cusummonth-3}),
we have a different pattern; it begins with a relatively steady part
until 1997, followed by an upward part until 2005, and a long decline
until 2020, with a small positive slope until the present. For a uniform
trend, we would expect the CUSUMs of the anomalies to have a symmetric
`V' shape. This would indicate that the anomalies are evenly distributed
around the climatological mean, and for a positive uniform trend, the
first half to be below and the other half above the climatological mean.
In our case, there is a more complex evolution of the anomalies. Another
distinct feature of the CUSUMs, is the different pattern of the
cloudy-sky dataset which peaks around the middle of the period.
Although, there seems to exist a complementary relation to the CUSUMs of
the clear- and all-sky cases, we can not assert that clouds are the main
driver for this relation due to the great difference in the number of
observational data between the two datasets
(Table~\ref{tab:trendtable}).

\begin{figure}[h!]
    \begin{adjustwidth}{-\extralength}{0cm}
        {\centering 
        \subfloat[All skies.\label{fig:cusummonth-1}]
            {\includegraphics[width=.32\linewidth]{./images/CumulativeMonthlyCuSum-1}}\hfill
        \subfloat[Clear skies.\label{fig:cusummonth-2}]
            {\includegraphics[width=.32\linewidth]{./images/CumulativeMonthlyCuSum-5}}\hfill
        \subfloat[Cloudy skies.\label{fig:cusummonth-3}]
            {\includegraphics[width=.32\linewidth]{./images/CumulativeMonthlyCuSum-9}}\hfill
        }
\caption{Cumulative sum plots of the monthly SDR anomalies in (\%) for different sky conditions.}\label{fig:cusummonth}
\end{adjustwidth}
\end{figure}

In order to investigate further the features of the CUSUMs, we created
another set of CUSUM plots by subtracting the corresponding long term
trend from the SDR anomaly data, prior to the CUSUM calculation
(Figure~\ref{fig:cusumnotrendmonthly}). With this approach periods when
the CUSUMs diverge from zero can be interpreted as a systematic
variation of SDR from the climatological mean. When the CUSUM is
increasing, the added values are above the climatological values of the
SDR trend and vice versa. Overall, for all- and clear-sky conditions
(Figures~\ref{fig:cusumnotrendmonthly-1}
and~\ref{fig:cusumnotrendmonthly-2}) we observe periods when the
anomalies diverge from the climatological value, each lasting for
several years. The pattern in both datasets is very similar, suggesting
prevalence in clear skies over Thessaloniki. It is interesting that in
the period 1993 - 2016 the anomalies have a high variability around
zero, while after 2016, the range of the variability is decreased to
about one third of the prior period. For cloudy-sky conditions
(Figure~\ref{fig:cusumnotrendmonthly-3}) the period 1997 - 2008 is
dominated by positive CUSUMs, suggesting a reduced effect of clouds on
SDR. From 1997 to mid-2000s CUSUMs are increasing, likely due to a
continuous decrease in the optical thickness of clouds, followed by a
period of rapid increase (within 3 years) in cloud optical thickness
lasting up to 2008. The following stable period spans for about 15 years
up to 2021 when CUSUMs start increasing again.

\begin{figure}[h!]
    \begin{adjustwidth}{-\extralength}{0cm}
        {\centering 
            \subfloat[All skies.\label{fig:cusumnotrendmonthly-1}]
                {\includegraphics[width=.32\linewidth]{./images/CumulativeMonthlyCuSumNOtrend-1} }\hfill
            \subfloat[Clear skies.\label{fig:cusumnotrendmonthly-2}]
                {\includegraphics[width=.32\linewidth]{./images/CumulativeMonthlyCuSumNOtrend-5} }\hfill
            \subfloat[Cloudy skies.\label{fig:cusumnotrendmonthly-3}]
                {\includegraphics[width=.32\linewidth]{./images/CumulativeMonthlyCuSumNOtrend-9} }
        }
        \caption{Cumulative sum plots of monthly SDR anomalies in (\%) for different sky conditions after removing the long-term linear trend.}\label{fig:cusumnotrendmonthly}
\end{adjustwidth}
\end{figure}

\hypertarget{effects-of-the-solar-zenith-angle-on-sdr.}{%
\subsection{Effects of the solar zenith angle on
SDR.}\label{effects-of-the-solar-zenith-angle-on-sdr.}}

The solar zenith angle is a major factor of SDR reaching the ground, due
to the enhancement of the radiation path in the atmosphere, especially
in urban environments with human activities emitting aerosols
\citep{Wang2021}. In order to estimate the effect of the SZA on the SDR
trends, we grouped the anomaly data in bins of \(1^\circ\) SZA, and
calculated the overall trend for each bin before noon and after noon
(Figure~\ref{fig:szatrends}). Although there are seasonal dependencies
of the minimum SZA (see Appendix, Figure~\ref{fig:SZAtrendSeason}),
these dependencies would not be further examined here. For all-sky and
clear-sky conditions the brightening effect of SDR (positive trend) is
stronger for large SZAs (Figures~\ref{fig:szatrends-1} and
\ref{fig:szatrends-2}). The trends in the morning and afternoon hours
are more or less consistent with small differences, which can be
attributed to systematic diurnal variations of aerosols, particularly
during the warm period of the year \citep{Wang2021}. For cloudy-sky
conditions (Figure~\ref{fig:szatrends-3}), we can not discern any
significant dependence of the SDR trend with SZA. For SZAs \(16^\circ\)
- \(50^\circ\), the trends range within about \(\pm 0.2\,\%/y\), with a
weak statistical significance. Between \(50^\circ\) and \(75^\circ\) SZA
the trends for the period before noon are stronger and negative,
possibly associated with stronger attenuation by clouds under oblique
incidence angles.

\begin{figure}[h!]
    \begin{adjustwidth}{-\extralength}{0cm}
        {\centering 
            \subfloat[All skies.\label{fig:szatrends-1}]
                {\includegraphics[width=.32\linewidth]{./images/SzaTrends-1}}\hfill
            \subfloat[Clear skies.\label{fig:szatrends-2}]
                {\includegraphics[width=.32\linewidth]{./images/SzaTrends-4}}\hfill
            \subfloat[Cloudy skies.\label{fig:szatrends-3}]
                {\includegraphics[width=.32\linewidth]{./images/SzaTrends-7}}
        }
        \caption{Long term trends of SDR as a function of SZA separately from morning and afternoon periods. Solid shapes  represent statistically significant trends ($p < 0.005$).}\label{fig:szatrends}
    \end{adjustwidth}
\end{figure}

\hypertarget{long-term-trends-by-season}{%
\subsection{Long term trends by
season}\label{long-term-trends-by-season}}

Similarly to the long term trends discussed above, we have calculated
the trend of the anomalies for the three different sky conditions, and
for each season of the year, using the corresponding mean monthly values
(Figure~\ref{fig:seasonalALL} and Table~\ref{tab:trendseasontable}). For
all-sky conditions the trend in SDR in winter is the largest
(\(0.69\,\%/y\)), followed by the trend in autumn (\(0.43\,\%/y\), a
value close to the long term trend) both statistically significant above
the \(99\,\%\) confidence level. In spring and summer, the trends are
much smaller and of lesser statistical significance. These seasonal
differences indicate a possible relation of the trends in SDR to trends
of clouds during winter and autumn. For clear-skies, the trend in winter
is \(0.5\,\%/y\), larger than for all-skies (\(0.69\,\%/y\)), which is
another indication of a decreasing trend in cloud optical thickness.
Moreover, the trends under clear- and cloudy-sky conditions are almost
complementary to each other, particularly for winter and autumn, where
the signal is stronger. During spring and summer the statistical
significance is very low and the actual trend too small for a meaningful
comparison.

\begin{figure}[h!]
    \begin{adjustwidth}{-\extralength}{0cm}
        {\centering 
            \includegraphics[width=1\linewidth]{./images/SeasonalTrendsTogether3-2} 
        }
        \caption{Linear trends (black lines) of monthly mean anomalies of SDR by season (rows of plots) for the three sky conditions (columns of plots).}\label{fig:seasonalALL}
    \end{adjustwidth}
\end{figure}

\begin{table}[!h]

\caption{\label{tab:trendseasontable}SDR linear trends of monthly anomalies for each season of the year.}
\begin{tabu} to \linewidth {>{\centering\arraybackslash}p{8em}>{\centering}X>{\raggedleft}X>{\raggedleft}X>{\centering}X>{\centering}X>{\raggedleft}X}
\toprule
Trend [\%/year] & slope.p & Rsqrd & cor.estimate & Sky condition & Season & slope.stat\_sig\\
\midrule
\cellcolor{gray!6}{0.6870} & \cellcolor{gray!6}{1.33e-03} & \cellcolor{gray!6}{0.3120} & \cellcolor{gray!6}{0.5587342} & \cellcolor{gray!6}{All skies} & \cellcolor{gray!6}{Winter} & \cellcolor{gray!6}{99.9}\\
\cmidrule{1-7}
0.1360 & 2.19e-01 & 0.0515 & 0.2270307 & All skies & Spring & 78.1\\
\cmidrule{1-7}
\cellcolor{gray!6}{0.1230} & \cellcolor{gray!6}{9.61e-02} & \cellcolor{gray!6}{0.0958} & \cellcolor{gray!6}{0.3094390} & \cellcolor{gray!6}{All skies} & \cellcolor{gray!6}{Summer} & \cellcolor{gray!6}{90.4}\\
\cmidrule{1-7}
0.4300 & 6.67e-03 & 0.2350 & 0.4844650 & All skies & Autumn & 99.3\\
\cmidrule{1-7}
\cellcolor{gray!6}{0.5050} & \cellcolor{gray!6}{4.16e-04} & \cellcolor{gray!6}{0.3640} & \cellcolor{gray!6}{0.6033797} & \cellcolor{gray!6}{Clear skies} & \cellcolor{gray!6}{Winter} & \cellcolor{gray!6}{100.0}\\
\cmidrule{1-7}
-0.1450 & 1.49e-01 & 0.0706 & -0.2656359 & Clear skies & Spring & 85.1\\
\cmidrule{1-7}
\cellcolor{gray!6}{-0.2310} & \cellcolor{gray!6}{3.01e-05} & \cellcolor{gray!6}{0.4690} & \cellcolor{gray!6}{-0.6846238} & \cellcolor{gray!6}{Clear skies} & \cellcolor{gray!6}{Summer} & \cellcolor{gray!6}{100.0}\\
\cmidrule{1-7}
0.0432 & 5.79e-01 & 0.0111 & 0.1054619 & Clear skies & Autumn & 42.1\\
\cmidrule{1-7}
\cellcolor{gray!6}{0.4720} & \cellcolor{gray!6}{2.01e-02} & \cellcolor{gray!6}{0.1780} & \cellcolor{gray!6}{0.4221299} & \cellcolor{gray!6}{Cloudy skies} & \cellcolor{gray!6}{Winter} & \cellcolor{gray!6}{98.0}\\
\cmidrule{1-7}
0.1780 & 1.25e-01 & 0.0792 & 0.2814813 & Cloudy skies & Spring & 87.5\\
\cmidrule{1-7}
\cellcolor{gray!6}{0.2130} & \cellcolor{gray!6}{2.20e-02} & \cellcolor{gray!6}{0.1740} & \cellcolor{gray!6}{0.4167034} & \cellcolor{gray!6}{Cloudy skies} & \cellcolor{gray!6}{Summer} & \cellcolor{gray!6}{97.8}\\
\cmidrule{1-7}
0.4140 & 2.46e-02 & 0.1680 & 0.4094891 & Cloudy skies & Autumn & 97.5\\
\bottomrule
\end{tabu}
\end{table}

\hypertarget{conclusions}{%
\section{Conclusions}\label{conclusions}}

We have demonstrated that in the period 1993 - 2023, there is a positive
trend in SDR of (\(0.38\,\%/y\)) (brightening) (positive trend) in
Thessaloniki, Greece, under all-sky conditions. A previous study
\citep{Bais2013} for the period 1993 - 2011 found also a positive trend
of \(0.33\,\%/y\). The increase of this trend indicates that the
brightening of SDR continues and is probably caused by continuing
decreases in aerosol optical depth and the optical thickness of clouds
over the area. Moreover, we found a similar trend under clear-sky
conditions (\(0.024\,\%/y\)) that further supports the assumption that
the brightening is caused mainly by decreasing aerosols. Unfortunately,
for the entire period there is no available data for the aerosols, in
order to quantify their effect on SDR. However, \citet{Siomos2020} have
shown that aerosol optical depth over Thessaloniki is decreasing
constantly at least up to 2018. The attenuation of SDR by aerosols over
Europe have been proposed as major factor by \citet{Wild2021}. The
dimming effect on SDR under cloudy-sky conditions (\(0.34\,\%/y\)),
suggests that cloud optical thickness is decreasing during this period.
Because we have no adequate data to investigate the long term changes of
cloud thickness in the region, we cannot verify if the negative SDR
trend we observe under under cloudy-skies can be attributed solely to
changes in clouds.

The observed brightening on SDR over Thessaloniki is dependent on SZA
(larger SZAs lead to stronger brightening). The trend is also dependent
on season, with winter showing the strongest statistically significant
trend of \(0.69\) and \(0.5\,\%/y\) for all- and clear-skies,
respectively, in contrast to spring and summer. The trends for autumn
are also significant but smaller ( \(0.43\) and \(0.043\,\%/y\) for all-
and clear-skies, respectively). Our findings are in agreement with other
studies for the region.

Using the CUSUMs of the monthly departures for all- and clear-skies, we
observed periods where the CUSUMs remain relatively stable, with a steep
decline before and a steep increase after. This is an indication that
the whole brightening effect does not follow a smooth development over
time.

Continued observations with a collocated pyrheliometer, which started in
2016, will allow us to further investigate the variability of solar
radiation at ground level in Thessaloniki. Also, additional data of
cloudiness, aerosols, atmospheric water vapour, etc., will allow better
attribution and quantification of the effects of these factors on SRD.

%%%%%%%%%%%%%%%%%%%%%%%%%%%%%%%%%%%%%%%%%%

\vspace{6pt}

%%%%%%%%%%%%%%%%%%%%%%%%%%%%%%%%%%%%%%%%%%
%% optional

% Only for the journal Methods and Protocols:
% If you wish to submit a video article, please do so with any other supplementary material.
% \supplementary{The following supporting information can be downloaded at: \linksupplementary{s1}, Figure S1: title; Table S1: title; Video S1: title. A supporting video article is available at doi: link.}

%%%%%%%%%%%%%%%%%%%%%%%%%%%%%%%%%%%%%%%%%%

\funding{This research received no external funding.}



\dataavailability{Data as daily sums are available through the WRDC
database \url{http://wrdc.mgo.rssi.ru}. One minute data are available on
request from the corresponding author. The data are not publicly
available for protection against unmonitored commercial use.}



%%%%%%%%%%%%%%%%%%%%%%%%%%%%%%%%%%%%%%%%%%
%% Optional

%% Only for journal Encyclopedia

\abbreviations{Abbreviations}{
The following abbreviations are used in this manuscript:\\

\noindent
\begin{tabular}{@{}ll}
DNI & Direct beam/normal irradiance \\
CSid & Clear sky identification algorithm \\
CUSUM & Cumulative sum \\
SDR & Solar downward radiation \\
SZA & Solar zenith angle \\
\end{tabular}}

%%%%%%%%%%%%%%%%%%%%%%%%%%%%%%%%%%%%%%%%%%
%% Optional
\input{"appendix.tex"}
%%%%%%%%%%%%%%%%%%%%%%%%%%%%%%%%%%%%%%%%%%
\begin{adjustwidth}{-\extralength}{0cm}

%\printendnotes[custom] % Un-comment to print a list of endnotes


\reftitle{References}
\bibliography{manualreferences.bib}

% If authors have biography, please use the format below
%\section*{Short Biography of Authors}
%\bio
%{\raisebox{-0.35cm}{\includegraphics[width=3.5cm,height=5.3cm,clip,keepaspectratio]{Definitions/author1.pdf}}}
%{\textbf{Firstname Lastname} Biography of first author}
%
%\bio
%{\raisebox{-0.35cm}{\includegraphics[width=3.5cm,height=5.3cm,clip,keepaspectratio]{Definitions/author2.jpg}}}
%{\textbf{Firstname Lastname} Biography of second author}

%%%%%%%%%%%%%%%%%%%%%%%%%%%%%%%%%%%%%%%%%%
%% for journal Sci
%\reviewreports{\\
%Reviewer 1 comments and authors’ response\\
%Reviewer 2 comments and authors’ response\\
%Reviewer 3 comments and authors’ response
%}
%%%%%%%%%%%%%%%%%%%%%%%%%%%%%%%%%%%%%%%%%%
\PublishersNote{}
\end{adjustwidth}


\end{document}
