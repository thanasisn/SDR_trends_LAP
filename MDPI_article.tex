% Options for packages loaded elsewhere
\PassOptionsToPackage{unicode}{hyperref}
\PassOptionsToPackage{hyphens}{url}
%
\documentclass[
]{article}
\usepackage{amsmath,amssymb}
\usepackage{iftex}
\ifPDFTeX
  \usepackage[T1]{fontenc}
  \usepackage[utf8]{inputenc}
  \usepackage{textcomp} % provide euro and other symbols
\else % if luatex or xetex
  \usepackage{unicode-math} % this also loads fontspec
  \defaultfontfeatures{Scale=MatchLowercase}
  \defaultfontfeatures[\rmfamily]{Ligatures=TeX,Scale=1}
\fi
\usepackage{lmodern}
\ifPDFTeX\else
  % xetex/luatex font selection
\fi
% Use upquote if available, for straight quotes in verbatim environments
\IfFileExists{upquote.sty}{\usepackage{upquote}}{}
\IfFileExists{microtype.sty}{% use microtype if available
  \usepackage[]{microtype}
  \UseMicrotypeSet[protrusion]{basicmath} % disable protrusion for tt fonts
}{}
\makeatletter
\@ifundefined{KOMAClassName}{% if non-KOMA class
  \IfFileExists{parskip.sty}{%
    \usepackage{parskip}
  }{% else
    \setlength{\parindent}{0pt}
    \setlength{\parskip}{6pt plus 2pt minus 1pt}}
}{% if KOMA class
  \KOMAoptions{parskip=half}}
\makeatother
\usepackage{xcolor}
\usepackage[margin=1in]{geometry}
\usepackage{longtable,booktabs,array}
\usepackage{calc} % for calculating minipage widths
% Correct order of tables after \paragraph or \subparagraph
\usepackage{etoolbox}
\makeatletter
\patchcmd\longtable{\par}{\if@noskipsec\mbox{}\fi\par}{}{}
\makeatother
% Allow footnotes in longtable head/foot
\IfFileExists{footnotehyper.sty}{\usepackage{footnotehyper}}{\usepackage{footnote}}
\makesavenoteenv{longtable}
\usepackage{graphicx}
\makeatletter
\def\maxwidth{\ifdim\Gin@nat@width>\linewidth\linewidth\else\Gin@nat@width\fi}
\def\maxheight{\ifdim\Gin@nat@height>\textheight\textheight\else\Gin@nat@height\fi}
\makeatother
% Scale images if necessary, so that they will not overflow the page
% margins by default, and it is still possible to overwrite the defaults
% using explicit options in \includegraphics[width, height, ...]{}
\setkeys{Gin}{width=\maxwidth,height=\maxheight,keepaspectratio}
% Set default figure placement to htbp
\makeatletter
\def\fps@figure{htbp}
\makeatother
\setlength{\emergencystretch}{3em} % prevent overfull lines
\providecommand{\tightlist}{%
  \setlength{\itemsep}{0pt}\setlength{\parskip}{0pt}}
\setcounter{secnumdepth}{5}
\newlength{\cslhangindent}
\setlength{\cslhangindent}{1.5em}
\newlength{\csllabelwidth}
\setlength{\csllabelwidth}{3em}
\newlength{\cslentryspacingunit} % times entry-spacing
\setlength{\cslentryspacingunit}{\parskip}
\newenvironment{CSLReferences}[2] % #1 hanging-ident, #2 entry spacing
 {% don't indent paragraphs
  \setlength{\parindent}{0pt}
  % turn on hanging indent if param 1 is 1
  \ifodd #1
  \let\oldpar\par
  \def\par{\hangindent=\cslhangindent\oldpar}
  \fi
  % set entry spacing
  \setlength{\parskip}{#2\cslentryspacingunit}
 }%
 {}
\usepackage{calc}
\newcommand{\CSLBlock}[1]{#1\hfill\break}
\newcommand{\CSLLeftMargin}[1]{\parbox[t]{\csllabelwidth}{#1}}
\newcommand{\CSLRightInline}[1]{\parbox[t]{\linewidth - \csllabelwidth}{#1}\break}
\newcommand{\CSLIndent}[1]{\hspace{\cslhangindent}#1}
\usepackage{booktabs}
\usepackage{longtable}
\usepackage{array}
\usepackage{multirow}
\usepackage{wrapfig}
\usepackage{float}
\usepackage{colortbl}
\usepackage{pdflscape}
\usepackage{tabu}
\usepackage{threeparttable}
\usepackage{threeparttablex}
\usepackage[normalem]{ulem}
\usepackage{makecell}
\usepackage{xcolor}
\ifLuaTeX
  \usepackage{selnolig}  % disable illegal ligatures
\fi
\IfFileExists{bookmark.sty}{\usepackage{bookmark}}{\usepackage{hyperref}}
\IfFileExists{xurl.sty}{\usepackage{xurl}}{} % add URL line breaks if available
\urlstyle{same}
\hypersetup{
  pdftitle={Full title of the paper (Capitalized)},
  pdfkeywords={GHI; SDR; Solar Brigthening/Dimming (list three to ten pertinent keywords specific
to the article, yet reasonably common within the subject discipline.).},
  hidelinks,
  pdfcreator={LaTeX via pandoc}}

\title{Full title of the paper (Capitalized)}
\author{true \and true}
\date{}

\begin{document}
\maketitle
\begin{abstract}
The shortwave downward solar irradiance (SDR) is an important factor that drives climate processes, production and can affect all living organisms.
While monitoring the long term variability, there are observations of upward and downward SDR trends on different locations around the world for different time periods.
Periods of positive tredns are refered as brightening periods and with negative tredns as dimming periods.
We studied 29 years of CHP1 data from Thessaloniki, Greece, under three sky conditions (clear sky, cloudy sky and all sky conditions), applying a cloud sky identification algorithm.
We found a positive trend for all-sky and clear-sky conditions, and also, investigated the consistency of those trends, the effect of the solar zenith angle, and the variation of the trends for the seasons of the year.
We indentified that there are some anomalies in the long term SDR trends, for all sky conditions.
\end{abstract}

{
\setcounter{tocdepth}{2}
\tableofcontents
}
\hypertarget{introduction.}{%
\section{Introduction.}\label{introduction.}}

The shortwave downward solar irradiance (SDR) at Earth's surface play a significant role, on its climate.
Changes of the SDR can be related to changes on Earth's energy budget, the mechanisms of climate change, and water and carbon cycle (Wild 2009).
Can also affect, solar and agricultural production, and all living organisms.
Studies of SDR variability, have identified some distinct SDR trends on different regions of the world on different time periods.
The term `brightening' is generally used to describe periods of positive SDR trend, and `dimming' for negative trend.
There are many cases on the long term records of irradiance, showing a systematic change of SDR's trend slope, occurring roughly at the last decades of the 20th century.
On multiple station in China, a dimming period was reported until about 2000, followed by a brightening period (Yang et al. 2021).
A similar pattern was identified, with the breaking point around 1980, for stations in Central Europe (Wild et al. 2021) and Brazil (Yamasoe et al. 2021).
Also, on global scale, an AI aided continental level spatial analysis, with data from multiple station, reach similar conclusions for the above regions and for the global trend (Yuan, Leirvik, and Wild 2021).

There is a consensus, among researchers, that the major factors of SDR attenuation is the interaction of Sun radiation with atmospheric aerosols and clouds.
Those interactions, among other factors, have been analysed with models (Li et al. 2016; Samset et al. 2018), showing the existence of feedback mechanisms between the two.
Similar finds, have been showed in observational data (Schwarz et al. 2020; Ohvril et al. 2009; Zerefos et al. 2009; Xia et al. 2007 and references therein).

Due to the variability of the phenomenon, and its contributing factors, there is a constant need to investigate and monitor SDR, in different sites, to estimate its magnitude, and its relation to the local conditions.
In this study, we examine the trends of SDR, with ground-based measurements at Thessaloniki, Greece for the period 1993 to 2023, as derived from a CM-21 pyranometer.
We reevaluated and extended the dataset used by Bais et al. (2013), applying a different algorithm for the identification of clear-/cloud-sky instances (Reno and Hansen 2016; Reno, Hansen, and Stein 2012a), and we derive the radiation trends for the period 1993 to 2023, under different sky conditions (all-sky, clear-sky and cloud-sky).

\hypertarget{observational-data-and-methodology.}{%
\section{Observational data and methodology.}\label{observational-data-and-methodology.}}

The SDR data were measured with a Kipp \& Zonen CM-21 pyranometer operating continuously at the Laboratory of Atmospheric Physics of the Aristotle University of Thessaloniki
(\(40^\circ\,38'\,\)N, \(22^\circ\,57'\,\)E, \(80\,\)m~a.s.l.)
in the period from
1993-04-13
to
2023-04-13.
The monitoring site is located near the city centre, and we expect to be affected by the urban environment.
During the study period, the pyranometer has been independently calibrated three times at the Meteorologisches Observatorium Lindenberg, DWD, when it was verified the stability of the instrument to within better than \(0.7\%\) relative to the initial calibration by the manufacturer.
Along with SDR, the direct beam radiation (DNI) was also measured by a collocated Kipp \& Zonen CHP-1 pyrheliometer, for the period
2016-04-01
to
2023-04-13.
Although, we have performed a similar analysis to the DNI data the results are not presented here, as they lack the appropriate statistical significance, due to the sorter duration of the data.
However, the DNI data were used as auxiliary data, in the clear sky identification algorithm (CSid), which is discussed later, for the selection of the appropriate thresholds.
It is noted that despite the capability of the CSid algorithm to use the DNI as a characterization parameter, we haven't utilized it here, to avoid any selection bias, due to unequal length of the two datasets.
There are four distinct steps in the creation of the dataset analysed here:
a)~the acquisition of radiation measurements from the sensors,
b)~the data quality check,
c)~the identification of ``clear sky'' conditions from the radiometric data, and
d)~the aggregation of data and trend analysis.

For the acquisition of radiometric data, the signal of the pyranometer is sampled with a rate of \(1\,\text{Hz}\).
The mean and the standard deviation of these samples are recorded every minute.
The measurements are corrected for the zero offset (``dark signal'' in volts).
The ``dark signal'' is calculated by averaging all measurements recorded for a period of
\(3\,\text{h}\),
before (morning) or after (evening) the Sun reaches an elevation angle of
\(-10^\circ\).
The signal is converted to irradiance using a ramped value of the instrument's sensitivity between calibrations.

A manual screening was performed, to remove inconsistent and erroneous recordings that can occur stochastically or systematically, during the continuous operation of the instruments.
The manual screening is aided by a radiation data quality assurance procedure, adjusted for the site, which is based on the methods of
Long and Shi~(2008, 2006).
Thus, problematic recordings have been excluded from further processing.
Although it is impossible to detect all false data, the large number of available data, and the aggregation scheme we used, ensures the good quality of the radiometric measurements used in this study.

In order to be able to estimate the effect of the sky condition on the long term variability of SDR, we created three datasets, by characterizing each one-minute measurement with a corresponding sky condition (i.e., all-sky, clear-sky and cloudy-sky).
To identify the clear-sky conditions we used a method proposed by
Long and Ackerman (2000)
and by
Reno and Hansen (2016),
which was adapted and configured for the site, as the authors suggest.

We have to note, that the definition of what constitutes as clear or cloudy sky, has some subjectivity, in any method of characterization.
As a result, the details of the definition are site specific, it relies on a combination of thresholds and comparisons with ideal actinometric models and statistical analysis on different signal metrics.
The CSid algorithm was calibrated with the main focus, to identify the presence of clouds on the sky dome.
Although the fine-tuning of the procedure, few marginal cases exist, that have been identified manually as false positive or false negative but cannot affect the final results of the study.

For completeness, we will provide below a brief overview of the
clear sky identification algorithm (CSid),
along with the site specific thresholds.
To calculate the reference clear sky
\(\text{SDR}_\text{CSref}\) we used the \(\text{SDR}_\text{Haurwitz}\) derived by
the radiation model of Haurwitz (1945), adjusted for our site with a
factor \(a\) (Eq.~\ref{eq:ahau}), resulted by an iterative optimization process, as described
by Long and Ackerman (2000) and Reno and Hansen (2016).
The target of
the optimization was the minimization of a function \(f(a)\) (Eq.~\ref{eq:minf}) and
was accomplished with the algorithmic function ``optimise'', which is an implementation based on the work of Brent (1973), from the library ``stats'' of the R programming language (R Core Team 2023).
The optimization and the selection of the clear sky reference model, was performed on SDR observations for the period 2016 - 2021.
During the optimization, eight simple clear sky radiation models were tested (Daneshyar-Paltridge-Proctor, Kasten-Czeplak, Haurwitz, Berger-Duffie, Adnot-Bourges-Campana-Gicquel, Robledo-Soler, Kasten and Ineichen-Perez), with a wide range of factors.
These models are described in more details by Reno, Hansen, and Stein (2012b) and evaluated by Reno and Hansen (2016).
We found, that Haurwitz's model, adjusted with the factor \(a = 0.965\) yields one of the lowest root mean squared errors (RMSE),
while the procedure, manages to characterize the majority of the data.
Thus, our clear sky reference is derived by the Eq.~\ref{eq:ahau}.

\begin{equation}
f(a) = \frac{1}{n}\sum_{i=1}^{n} ( \text{SDR}_{\text{CSid},i} - a \times \text{SDR}_{\text{testCSref},i} )^2 \label{eq:minf}
\end{equation}
where: \(n\) is the total number of daylight records, \(\text{SDR}_{\text{CSid},i}\) are the records identified as clear sky by CSid, \(a\) is a hypothetical adjustment factor, and \(\text{SDR}_{\text{testCSref},i}\) is any of the tested clear sky radiation models.

\begin{equation}
\text{SDR}_\text{CSref} = a \times \text{SDR}_\text{Haurwitz} = 0.965 \times 1098 \times \cos(\theta) \times \exp \left( \frac{ - 0.057}{\cos(\theta)} \right) \label{eq:ahau}
\end{equation}
where: \(\text{SDR}_\text{CSref}\) is the reference clear sky SDR, in \(\text{w}\,\text{m}^{-2}\) and \(\theta\) is the solar zenith angle (SZA).

The criteria that were used to identify whether a measurement was taken
under clear-sky conditions are presented below.
A data point is flagged
as ``clear-sky'' if all criteria are satisfied, otherwise it is considered to be ``cloud-sky''.
Each criterion was applied
for a running window of \(11\) consecutive one-minute measurements, and
the characterization is assigned to the central value of the window.
Each parameter, was calculated both from the observations and the
reference clear sky model, for each comparison.
The allowable range of variation is defined by the
model-derived value of the parameter multiplied by a factor plus an
offset.
The factors and the offsets were
determined empirically, by manual inspecting each filters performance on
selected days, and adjusting them accordingly, during an iterative
process.

\begin{enumerate}
\def\labelenumi{\alph{enumi})}
\tightlist
\item
  Mean of the measured \(\overline{\text{SDR}}_i\) (Eq. \ref{eq:MeanVIP}).
  \begin{equation}
  0.91 \times \overline{\text{SDR}}_{\text{CSref},i} - 20
  < \overline{\text{SDR}}_i <
  1.095 \times \overline{\text{SDR}}_{\text{CSref},i} + 30
  \label{eq:MeanVIP}
  \end{equation}
\end{enumerate}

\begin{enumerate}
\def\labelenumi{\alph{enumi})}
\setcounter{enumi}{1}
\tightlist
\item
  Maximum measured value \(M_{\text{}}\) (Eq.~\ref{eq:MaxVIP}).
  \begin{equation}
  1 \times M_{\text{CSref},i} - 75
  < M_{\text{}i} <
  1 \times M_{\text{CSref},i} + 75
  \label{eq:MaxVIP}
  \end{equation}
\end{enumerate}

\begin{enumerate}
\def\labelenumi{\alph{enumi})}
\setcounter{enumi}{2}
\tightlist
\item
  Length \(L_i\) of the sequential line segments, connecting the points of the \(11\) SDR values (Eq. \ref{eq:VILeq}).
  \begin{equation}
  L_i = \sum_{i=1}^{n-1}\sqrt{\left ( \text{SDR}_{i+1} - \text{SDR}_{i}\right )^2 + \left ( t_{i+1} - t_i \right )^2}
  \label{eq:VILeq}
  \end{equation}
  \begin{equation}
  1 \times L_{\text{CSref},i} - 5 < L_i < 1.3 \times L_{\text{CSref},i} + 13
  \label{eq:VILcr}
  \end{equation}
  where: \(t_i\) is the time each SDR measurement has been measured
\end{enumerate}

\begin{enumerate}
\def\labelenumi{\alph{enumi})}
\setcounter{enumi}{3}
\tightlist
\item
  Standard deviation \(\sigma_i\) of the slope (\(s_i\)) between the \(11\) sequential points, normalized by the mean \(\overline{\text{SDR}}_i\) (Eq.~\ref{eq:VCT1}).
  \begin{gather}
    \sigma_i = \frac {\sqrt{\frac{1}{n-1} \sum_{i=1}^{n-1} \left( s_i - \bar{s} \right)^2}} {\overline{\text{SDR}}_i} \label{eq:VCT1} \\
    s_i = \frac{\text{SDR}_{i+1} - \text{SDR}_{i}}{t_{i+1} - t_i},\;\;   \bar{s} = \frac{1}{n-1} \sum_{i=1}^{n-1} s_i,\;\;\forall i \in \left \{ 1, 2, \ldots, n-1 \right \}\;\;
  \end{gather}
  For this criterion, \(\sigma_i\) should be below a certain threshold (Eq.~\ref{eq:VCTcr}):
  \begin{equation}
    \sigma_i < \ensuremath{1.1\times 10^{-4}} \label{eq:VCTcr}
  \end{equation}
\end{enumerate}

\begin{enumerate}
\def\labelenumi{\alph{enumi})}
\setcounter{enumi}{4}
\tightlist
\item
  Maximum difference \(X_i\) between the change in measured irradiance and the change in clear sky irradiance over each measurement interval.
  \begin{gather}
    X_i = \max{\left \{ \left | x_i - x_{\text{CSref},i} \right | \right \}} \label{eq:VSM3} \\
    x_i = \text{SDR}_{i+1} - \text{SDR}_{i} \forall i \in \left \{ 1, 2, \ldots, n-1 \right \} \label{eq:VSM1} \\
    x_{\text{CSref},i} = \text{SDR}_{\text{CSref},i+1} - \text{SDR}_{\text{CSref},i} \forall i \in \left \{ 1, 2, \ldots, n-1 \right \} \label{eq:VSM2}
  \end{gather}
  For this criterion, \(X_i\) should be below a certain threshold (Eq.~\ref{eq:VSMcr}):
  \begin{equation}
    X_i < 7.5 \label{eq:VSMcr}
  \end{equation}
\end{enumerate}

Due to a significant measurement uncertainty near the horizon, we have to exclude all measurements with SZA greater than \(85^\circ\).
Moreover, due to some obstructions around the site (hills and buildings), we excluded data with Azimuth angle between
\(35^\circ\) and \(120^\circ\) with SZA greater than \(80^\circ\).
On the latter instances, Sun is systematically, not visible by the instrument's location.
To make the measurements comparable throughout the dataset, we adjusted all one-minute radiometric values to the mean Sun - Earth distance.
Subsequently, we made all measurements relative to the Total Solar Irradiance (TSI) at \(1\,\text{au}\), in order to compensate for the Sun's intensity variability, using a time series of satellite TSI observations.
The TSI data we use are part of the ``NOAA Climate Data Record of Total Solar Irradiance'' dataset (Coddington et al. 2005).
Where the initial daily values, were interpolated to match with the time step of our measurements.
The final dataset contains
\(6589967\)
one-minute measurements, of which,
\(84.2\%\)
were identified as under clear-sky conditions and subsequently
\(15.8\%\)
as under cloud-sky conditions.

In order to investigate the SDR trends, we implemented an appropriate aggregation scheme to the 1-minute data to derive a series in coarser timescale.
To preserve the representativeness of the data we used the following criteria:
a) for the daily mean values we accept days with more than 50\% of the daytime measurements, present and valid,
b) monthly values were computed from daily means
only when at least 20 days were available.
To create the daily and monthly climatological means, we averaged the data based on the day of year and calendar month, respectively.
For the seasonal means we averaged the mean daily values in each season (Winter: December - February, Spring: March - May, etc.).
Finally, each data set was deseasonalized by subtracting the corresponding climatological annual cycle (daily or monthly) from the actual data.
To estimate SZA contribution to the SDR trends, the one-minute data were aggregated in \(1^\circ\) SZA bins, separately for the morning and afternoon hours, and then were deseasonalized as mentioned above.

\hypertarget{results}{%
\section{Results}\label{results}}

\hypertarget{long-term-sdr-trends}{%
\subsection{Long-term SDR trends}\label{long-term-sdr-trends}}

\hypertarget{version}{%
\section{Version}\label{version}}

This Rmd-skeleton uses the mdpi Latex template published 2023-03-25.
However, the official template gets more frequently updated than the \textbf{rticles}
package. Therefore, please make sure prior to paper submission, that you're
using the most recent .cls, .tex and .bst files
(available \href{http://www.mdpi.com/authors/latex}{here}).

\hypertarget{article-header-information}{%
\section{Article Header Information}\label{article-header-information}}

\hypertarget{journal-specific-yaml-variables}{%
\subsection{Journal Specific YAML variables}\label{journal-specific-yaml-variables}}

\hypertarget{introduction}{%
\section{Introduction}\label{introduction}}

The introduction should briefly place the study in a broad context and highlight
why it is important. It should define the purpose of the work and its
significance. The current state of the research field should be reviewed
carefully and key publications cited. Please highlight controversial and
diverging hypotheses when necessary. Finally, briefly mention the main aim of
the work and highlight the principal conclusions. As far as possible, please
keep the introduction comprehensible to scientists outside your particular
field of research. Citing a journal paper (\textbf{bertrand-krajewski\_distribution\_1998?}; \textbf{leutnant\_stormwater\_2016?}). And now citing a book reference (\textbf{gujer\_systems\_2008?}).
Some MDPI journals use Chicago and others use APA, this template should choose
the correct citation format for you once you specify the journal in the YAML
header.

To use endnotes, change \texttt{endnotes:\ true} in the YAML header, then use
\texttt{\textbackslash{}endnote\{This\ is\ an\ endnote.\}}.

\hypertarget{materials-and-methods}{%
\section{Materials and Methods}\label{materials-and-methods}}

Materials and Methods should be described with sufficient details to allow
others to replicate and build on published results. Please note that publication
of your manuscript implicates that you must make all materials, data, computer
code, and protocols associated with the publication available to readers. Please
disclose at the submission stage any restrictions on the availability of
materials or information. New methods and protocols should be described in
detail while well-established methods can be briefly described and appropriately
cited.

Research manuscripts reporting large datasets that are deposited in a publicly
available database should specify where the data have been deposited and provide
the relevant accession numbers. If the accession numbers have not yet been
obtained at the time of submission, please state that they will be provided
during review. They must be provided prior to publication.

Interventionary studies involving animals or humans, and other studies require
ethical approval must list the authority that provided approval and the
corresponding ethical approval code.

\hypertarget{results-1}{%
\section{Results}\label{results-1}}

This section may be divided by subheadings. It should provide a concise and
precise description of the experimental results, their interpretation as well
as the experimental conclusions that can be drawn.

\hypertarget{subsection-heading-here}{%
\subsection{Subsection Heading Here}\label{subsection-heading-here}}

Subsection text here.

\hypertarget{subsubsection-heading-here}{%
\subsubsection{Subsubsection Heading Here}\label{subsubsection-heading-here}}

Bulleted lists look like this:

\begin{itemize}
\tightlist
\item
  First bullet
\item
  Second bullet
\item
  Third bullet
\end{itemize}

Numbered lists can be added as follows:

\begin{enumerate}
\def\labelenumi{\arabic{enumi}.}
\tightlist
\item
  First item
\item
  Second item
\item
  Third item
\end{enumerate}

The text continues here.

\hypertarget{figures-tables-and-schemes}{%
\subsection{Figures, Tables and Schemes}\label{figures-tables-and-schemes}}

All figures and tables should be cited in the main text as Figure \ref{fig:fig1},
\ref{tab:tab1}, etc. To get cross-reference to figure generated by R chunks
include the \texttt{\textbackslash{}\textbackslash{}label\{\}} tag in the \texttt{fig.cap} attribute of the R chunk:
\texttt{fig.cap\ =\ "Fancy\ Caption\textbackslash{}\textbackslash{}label\{fig:plot\}"}.

\begin{figure}[h!]

{\centering \includegraphics[width=0.7\linewidth]{MDPI_article_files/figure-latex/fig1-1} 

}

\caption{A figure added with a code chunk.\label{fig:fig1}}(\#fig:fig1)
\end{figure}

When making tables using \texttt{kable}, it is suggested to use
the \texttt{format="latex"} and \texttt{tabl.envir="table"} arguments
to ensure table numbering and compatibility with the mdpi
document class.

\begin{table}[H]

\caption{(\#tab:tab1)This is a table caption. Tables should be placed in the 
             main text near to the first time they are~cited.}
\begin{tabular}[t]{lccc}
\toprule
  & mpg & cyl & disp\\
\midrule
Mazda RX4 & 21.0 & 6 & 160\\
Mazda RX4 Wag & 21.0 & 6 & 160\\
Datsun 710 & 22.8 & 4 & 108\\
Hornet 4 Drive & 21.4 & 6 & 258\\
Hornet Sportabout & 18.7 & 8 & 360\\
\bottomrule
\end{tabular}
\end{table}

For a very wide table, landscape layouts are allowed.

\startlandscape

\begin{table}[H]

\caption{(\#tab:tab2)This is a very wide table}
\begin{tabular}[t]{cccc}
\toprule
Title.1 & Title.2 & Title.3 & Title.4\\
\midrule
Entry 1 & Data & Data & This cell has some longer content that runs over
                               two lines\\
Entry 2 & Data & Data & Data\\
\bottomrule
\end{tabular}
\end{table}

\finishlandscape

\hypertarget{formatting-of-mathematical-components}{%
\subsection{Formatting of Mathematical Components}\label{formatting-of-mathematical-components}}

This is an example of an equation:

\[
a = 1.
\]

If you want numbered equations use Latex and wrap in the equation environment:

\begin{equation}
a = 1,
\end{equation}

the text following an equation need not be a new paragraph. Please punctuate
equations as regular text.

This is the example 2 of equation:

\begin{adjustwidth}{-\extralength}{0cm}
\begin{equation}
a = b + c + d + e + f + g + h + i + j + k + l + m + n + o + p + q + r + s + t + 
u + v + w + x + y + z
\end{equation}
\end{adjustwidth}

Theorem-type environments (including propositions, lemmas, corollaries etc.)
can be formatted as follows:

Example of a theorem:

\begin{Theorem}
Example text of a theorem

\end{Theorem}

The text continues here.

\hypertarget{discussion}{%
\section{Discussion}\label{discussion}}

Authors should discuss the results and how they can be interpreted in
perspective of previous studies and of the working hypotheses. The findings and
their implications should be discussed in the broadest context possible. Future
research directions may also be highlighted.

\hypertarget{conclusion}{%
\section{Conclusion}\label{conclusion}}

This section is not mandatory, but can be added to the manuscript if the
discussion is unusually long or complex.

\hypertarget{refs}{}
\begin{CSLReferences}{1}{0}
\leavevmode\vadjust pre{\hypertarget{ref-Bais2013}{}}%
Bais, A. F., T. Drosoglou, C. Meleti, K. Tourpali, and N. Kouremeti. 2013. {``Changes in Surface Shortwave Solar Irradiance from 1993 to 2011 at Thessaloniki (Greece).''} \emph{International Journal of Climatology} 33 (13): 2871--76. \url{https://doi.org/f5dzz5}.

\leavevmode\vadjust pre{\hypertarget{ref-Brent1973}{}}%
Brent, Richard P. 1973. {``Algorithms for Minimization Without Derivatives.''} \emph{PrenticeHall, Englewood Cliffs, NJ}.

\leavevmode\vadjust pre{\hypertarget{ref-Coddington2005}{}}%
Coddington, Odele, Judith L. Lean, Doug Lindholm, Peter Pilewskie, Martin Snow, and NOAA CDR Program. 2005. {``{NOAA} Climate Data Record ({CDR}) of Total Solar Irradiance ({TSI}), {NRLTSI} Version 2. {D}aily.''} 2005. \url{https://doi.org/10.7289/V55B00C1}.

\leavevmode\vadjust pre{\hypertarget{ref-Haurwitz1945}{}}%
Haurwitz, Bernhard. 1945. {``Insolation in {Relation} to {Cloudiness} and {Cloud} {Density}.''} \emph{Journal of Meteorology} 2 (September): 154--66.

\leavevmode\vadjust pre{\hypertarget{ref-Li2016}{}}%
Li, Zhanqing, W. K.‐M. Lau, V. Ramanathan, G. Wu, Y. Ding, M. G. Manoj, J. Liu, et al. 2016. {``Aerosol and Monsoon Climate Interactions over Asia.''} \emph{Reviews of Geophysics} 54 (4): 866--929. \url{https://doi.org/10.1002/2015RG000500}.

\leavevmode\vadjust pre{\hypertarget{ref-Long2000}{}}%
Long, Charles N., and Thomas P. Ackerman. 2000. {``Identification of Clear Skies from Broadband Pyranometer Measurements and Calculation of Downwelling Shortwave Cloud Effects.''} \emph{Journal of Geophysical Research: Atmospheres} 105 (D12, D12): 15609--26. \url{https://doi.org/10.1029/2000jd900077}.

\leavevmode\vadjust pre{\hypertarget{ref-Long2006}{}}%
Long, Charles N., and Y. Shi. 2006. {``The QCRad Value Added Product: Surface Radiation Measurement Quality Control Testing, Including Climatology Configurable Limits.''} DOE/SC-ARM/TR-074. Office of Science, Office of Biological; Environmental Research, U.S. Department of Energy.

\leavevmode\vadjust pre{\hypertarget{ref-Long2008a}{}}%
---------. 2008. {``An Automated Quality Assessment and Control Algorithm for Surface Radiation Measurements.''} \emph{The Open Atmospheric Science Journal}, 23--37.

\leavevmode\vadjust pre{\hypertarget{ref-Ohvril2009}{}}%
Ohvril, Hanno, Hilda Teral, Lennart Neiman, Martin Kannel, Marika Uustare, Mati Tee, Viivi Russak, et al. 2009. {``Global Dimming and Brightening Versus Atmospheric Column Transparency, Europe, 1906--2007.''} \emph{Journal of Geophysical Research} 114 (May). \url{https://doi.org/10.1029/2008JD010644}.

\leavevmode\vadjust pre{\hypertarget{ref-RCT2023}{}}%
R Core Team. 2023. \emph{R: A Language and Environment for Statistical Computing}. Vienna, Austria: R Foundation for Statistical Computing. \url{https://www.R-project.org/}.

\leavevmode\vadjust pre{\hypertarget{ref-Reno2016}{}}%
Reno, Matthew J., and Clifford W. Hansen. 2016. {``Identification of Periods of Clear Sky Irradiance in Time Series of GHI Measurements.''} \emph{Renewable Energy} 90: 520--31. \url{https://doi.org/gq3sbg}.

\leavevmode\vadjust pre{\hypertarget{ref-Reno2012}{}}%
Reno, Matthew J., Clifford W. Hansen, and Joshua S. Stein. 2012b. {``Global Horizontal Irradiance Clear Sky Models: Implementation and Analysis.''} SAND2012-2389, 1039404. \url{https://doi.org/gq5npv}.

\leavevmode\vadjust pre{\hypertarget{ref-Reno2012a}{}}%
---------. 2012a. {``{Global Horizontal Irradiance Clear Sky Models: Implementation and Analysis}.''} \emph{SANDIA REPORT SAND2012-2389 Unlimited Release Printed March 2012}, March, 1--66.

\leavevmode\vadjust pre{\hypertarget{ref-Samset2018}{}}%
Samset, B. H., M. Sand, C. J. Smith, S. E. Bauer, P. M. Forster, J. S. Fuglestvedt, S. Osprey, and C.‐F. Schleussner. 2018. {``Climate Impacts from a Removal of Anthropogenic Aerosol Emissions.''} \emph{Geophysical Research Letters} 45 (2): 1020--29. \url{https://doi.org/10.1002/2017GL076079}.

\leavevmode\vadjust pre{\hypertarget{ref-Schwarz2020}{}}%
Schwarz, M., D. Folini, S. Yang, R. P. Allan, and M. Wild. 2020. {``Changes in Atmospheric Shortwave Absorption as Important Driver of Dimming and Brightening.''} \emph{Nature Geoscience} 13 (2): 110--15. \url{https://doi.org/10.1038/s41561-019-0528-y}.

\leavevmode\vadjust pre{\hypertarget{ref-Wild2009}{}}%
Wild, Martin. 2009. {``Global Dimming and Brightening: A Review.''} \emph{Journal of Geophysical Research Atmospheres} 114 (12): 1--31. \url{https://doi.org/bcq}.

\leavevmode\vadjust pre{\hypertarget{ref-Wild2021}{}}%
Wild, Martin, Stephan Wacker, Su Yang, and Arturo Sanchez-Lorenzo. 2021. {``Evidence for Clear‐sky Dimming and Brightening in Central Europe.''} \emph{Geophysical Research Letters} 48 (6). \url{https://doi.org/10.1029/2020GL092216}.

\leavevmode\vadjust pre{\hypertarget{ref-Xia2007}{}}%
Xia, Xiangao, Hongbin Chen, Zhanqing Li, Pucai Wang, and Jiankai Wang. 2007. {``Significant Reduction of Surface Solar Irradiance Induced by Aerosols in a Suburban Region in Northeastern China.''} \emph{Journal of Geophysical Research Atmospheres} 112 (22): 1--9. \url{https://doi.org/cdtntw}.

\leavevmode\vadjust pre{\hypertarget{ref-Yamasoe2021}{}}%
Yamasoe, Marcia Akemi, Nilton Manuel Évora Rosário, Samantha Novaes Santos Martins Almeida, and Martin Wild. 2021. {``Fifty-Six Years of Surface Solar Radiation and Sunshine Duration over s{ã}o Paulo, Brazil: 1961--2016.''} \emph{Atmospheric Chemistry and Physics} 21 (9): 6593--603. \url{https://doi.org/10.5194/acp-21-6593-2021}.

\leavevmode\vadjust pre{\hypertarget{ref-Yang2021}{}}%
Yang, Su, Zijiang Zhou, Yu Yu, and Martin Wild. 2021. {``Cloud {``}Shrinking{''} and {``}Optical Thinning{''} in the {``}Dimming{''} Period and a Subsequent Recovery in the {``}Brightening{''} Period over China.''} \emph{Environmental Research Letters}, January. \url{https://doi.org/10.1088/1748-9326/abdf89}.

\leavevmode\vadjust pre{\hypertarget{ref-Yuan2021}{}}%
Yuan, Menghan, Thomas Leirvik, and Martin Wild. 2021. {``Global Trends in Downward Surface Solar Radiation from Spatial Interpolated Ground Observations During 1961-2019.''} \emph{Journal of Climate}, September, 1--56. \url{https://doi.org/10.1175/JCLI-D-21-0165.1}.

\leavevmode\vadjust pre{\hypertarget{ref-Zerefos2009}{}}%
Zerefos, C. S., K. Eleftheratos, C. Meleti, S. Kazadzis, A. Romanou, C. Ichoku, G. Tselioudis, and A. Bais. 2009. {``Solar Dimming and Brightening over Thessaloniki, Greece, and Beijing, China.''} \emph{Tellus B: Chemical and Physical Meteorology} 61 (4): 657. \url{https://doi.org/10.1111/j.1600-0889.2009.00425.x}.

\end{CSLReferences}

\end{document}
