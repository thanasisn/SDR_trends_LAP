% Options for packages loaded elsewhere
\PassOptionsToPackage{unicode}{hyperref}
\PassOptionsToPackage{hyphens}{url}
%
\documentclass[
]{article}
\usepackage{amsmath,amssymb}
\usepackage{iftex}
\ifPDFTeX
  \usepackage[T1]{fontenc}
  \usepackage[utf8]{inputenc}
  \usepackage{textcomp} % provide euro and other symbols
\else % if luatex or xetex
  \usepackage{unicode-math} % this also loads fontspec
  \defaultfontfeatures{Scale=MatchLowercase}
  \defaultfontfeatures[\rmfamily]{Ligatures=TeX,Scale=1}
\fi
\usepackage{lmodern}
\ifPDFTeX\else
  % xetex/luatex font selection
\fi
% Use upquote if available, for straight quotes in verbatim environments
\IfFileExists{upquote.sty}{\usepackage{upquote}}{}
\IfFileExists{microtype.sty}{% use microtype if available
  \usepackage[]{microtype}
  \UseMicrotypeSet[protrusion]{basicmath} % disable protrusion for tt fonts
}{}
\makeatletter
\@ifundefined{KOMAClassName}{% if non-KOMA class
  \IfFileExists{parskip.sty}{%
    \usepackage{parskip}
  }{% else
    \setlength{\parindent}{0pt}
    \setlength{\parskip}{6pt plus 2pt minus 1pt}}
}{% if KOMA class
  \KOMAoptions{parskip=half}}
\makeatother
\usepackage{xcolor}
\usepackage[margin=1in]{geometry}
\usepackage{longtable,booktabs,array}
\usepackage{calc} % for calculating minipage widths
% Correct order of tables after \paragraph or \subparagraph
\usepackage{etoolbox}
\makeatletter
\patchcmd\longtable{\par}{\if@noskipsec\mbox{}\fi\par}{}{}
\makeatother
% Allow footnotes in longtable head/foot
\IfFileExists{footnotehyper.sty}{\usepackage{footnotehyper}}{\usepackage{footnote}}
\makesavenoteenv{longtable}
\usepackage{graphicx}
\makeatletter
\def\maxwidth{\ifdim\Gin@nat@width>\linewidth\linewidth\else\Gin@nat@width\fi}
\def\maxheight{\ifdim\Gin@nat@height>\textheight\textheight\else\Gin@nat@height\fi}
\makeatother
% Scale images if necessary, so that they will not overflow the page
% margins by default, and it is still possible to overwrite the defaults
% using explicit options in \includegraphics[width, height, ...]{}
\setkeys{Gin}{width=\maxwidth,height=\maxheight,keepaspectratio}
% Set default figure placement to htbp
\makeatletter
\def\fps@figure{htbp}
\makeatother
\setlength{\emergencystretch}{3em} % prevent overfull lines
\providecommand{\tightlist}{%
  \setlength{\itemsep}{0pt}\setlength{\parskip}{0pt}}
\setcounter{secnumdepth}{5}
\newlength{\cslhangindent}
\setlength{\cslhangindent}{1.5em}
\newlength{\csllabelwidth}
\setlength{\csllabelwidth}{3em}
\newlength{\cslentryspacingunit} % times entry-spacing
\setlength{\cslentryspacingunit}{\parskip}
\newenvironment{CSLReferences}[2] % #1 hanging-ident, #2 entry spacing
 {% don't indent paragraphs
  \setlength{\parindent}{0pt}
  % turn on hanging indent if param 1 is 1
  \ifodd #1
  \let\oldpar\par
  \def\par{\hangindent=\cslhangindent\oldpar}
  \fi
  % set entry spacing
  \setlength{\parskip}{#2\cslentryspacingunit}
 }%
 {}
\usepackage{calc}
\newcommand{\CSLBlock}[1]{#1\hfill\break}
\newcommand{\CSLLeftMargin}[1]{\parbox[t]{\csllabelwidth}{#1}}
\newcommand{\CSLRightInline}[1]{\parbox[t]{\linewidth - \csllabelwidth}{#1}\break}
\newcommand{\CSLIndent}[1]{\hspace{\cslhangindent}#1}
\usepackage{booktabs}
\usepackage{longtable}
\usepackage{array}
\usepackage{multirow}
\usepackage{wrapfig}
\usepackage{float}
\usepackage{colortbl}
\usepackage{pdflscape}
\usepackage{tabu}
\usepackage{threeparttable}
\usepackage{threeparttablex}
\usepackage[normalem]{ulem}
\usepackage{makecell}
\usepackage{xcolor}
\ifLuaTeX
  \usepackage{selnolig}  % disable illegal ligatures
\fi
\IfFileExists{bookmark.sty}{\usepackage{bookmark}}{\usepackage{hyperref}}
\IfFileExists{xurl.sty}{\usepackage{xurl}}{} % add URL line breaks if available
\urlstyle{same}
\hypersetup{
  pdftitle={Full title of the paper (Capitalized)},
  pdfkeywords={GHI; SDR; Solar Brigthening/Dimming (list three to ten pertinent keywords specific
to the article, yet reasonably common within the subject discipline.).},
  hidelinks,
  pdfcreator={LaTeX via pandoc}}

\title{Full title of the paper (Capitalized)}
\author{true \and true}
\date{}

\begin{document}
\maketitle
\begin{abstract}
The shortwave downward solar irradiance (SDR) is an important factor that drives climate processes, production and can affect all living organisms.
While monitoring the long term variability, there are observations of upward and downward SDR trends on different locations around the world for different time periods.
Periods of positive tredns are refered as brightening periods and with negative tredns as dimming periods.
We studied 29 years of CHP1 data from Thessaloniki, Greece, under three sky conditions (clear sky, cloudy sky and all sky conditions), applying a cloud sky identification algorithm.
We found a positive trend for all-sky and clear-sky conditions, and also, investigated the consistency of those trends, the effect of the solar zenith angle, and the variation of the trends for the seasons of the year.
We indentified that there are some anomalies in the long term SDR trends, for all sky conditions.
\end{abstract}

{
\setcounter{tocdepth}{2}
\tableofcontents
}
\hypertarget{introduction.}{%
\section{Introduction.}\label{introduction.}}

The shortwave downward solar irradiance (SDR) at Earth's surface play a significant role, on its climate.
Changes of the SDR can be related to changes on Earth's energy budget, the mechanisms of climate change, and water and carbon cycle (\textbf{Wild2009?}).
Can also affect, solar and agricultural production, and all living organisms.
Studies of SDR variability, have identified some distinct SDR trends on different regions of the world on different time periods.
The term `brightening' is generally used to describe periods of positive SDR trend, and `dimming' for negative trend.
There are many cases on the long term records of irradiance, showing a systematic change of SDR's trend slope, occurring roughly at the last decades of the 20th century.
On multiple station in China, a dimming period was reported until about 2000, followed by a brightening period (\textbf{Yang2021?}).
A similar pattern was identified, with the breaking point around 1980, for stations in Central Europe (\textbf{Wild2021?}) and Brazil (\textbf{Yamasoe2021?}).
Also, on global scale, an AI aided continental level spatial analysis, with data from multiple station, reach similar conclusions for the above regions and for the global trend (\textbf{Yuan2021?}).

There is a consensus, among researchers, that the major factors of SDR attenuation is the interaction of Sun radiation with atmospheric aerosols and clouds.
Those interactions, among other factors, have been analysed with models (\textbf{Li2016?}; \textbf{Samset2018?}), showing the existence of feedback mechanisms between the two.
Similar finds, have been showed in observational data (\textbf{Schwarz2020?}; \textbf{Ohvril2009?}; \textbf{Zerefos2009?}; \textbf{Xia2007?} and references therein).

Due to the variability of the phenomenon, and its contributing factors, there is a constant need to investigate and monitor SDR, in different sites, to estimate its magnitude, and its relation to the local conditions.
In this study, we examine the trends of SDR, with ground-based measurements at Thessaloniki, Greece for the period 1993 to 2023, as derived from a CM-21 pyranometer.
We reevaluated and extended the dataset used by (\textbf{Bais2013?}), applying a different algorithm for the identification of clear-/cloud-sky instances (\textbf{Reno2016?}; \textbf{Reno2012a?}), and we derive the radiation trends for the period 1993 to 2023, under different sky conditions (all-sky, clear-sky and cloud-sky).

\hypertarget{version}{%
\section{Version}\label{version}}

This Rmd-skeleton uses the mdpi Latex template published 2023-03-25.
However, the official template gets more frequently updated than the \textbf{rticles}
package. Therefore, please make sure prior to paper submission, that you're
using the most recent .cls, .tex and .bst files
(available \href{http://www.mdpi.com/authors/latex}{here}).

\hypertarget{article-header-information}{%
\section{Article Header Information}\label{article-header-information}}

\hypertarget{journal-specific-yaml-variables}{%
\subsection{Journal Specific YAML variables}\label{journal-specific-yaml-variables}}

\startlandscape

\begin{longtable}[t]{llllllll}
\caption{(\#tab:mdpinames)MDPI journal names.}\\
\toprule
acoustics & biomedinformatics & dentistry & galaxies & jcp & metabolites & philosophies & socsci\\
actuators & biomimetics & dermato & games & jcs & metals & photochem & software\\
addictions & biomolecules & dermatopathology & gases & jcto & meteorology & photonics & soilsystems\\
admsci & biophysica & designs & gastroent & jdb & methane & phycology & solar\\
adolescents & biosensors & devices & gastrointestdisord & jeta & metrology & physchem & solids\\
\addlinespace
aerobiology & biotech & diabetology & gels & jfb & micro & physics & spectroscj\\
aerospace & birds & diagnostics & genealogy & jfmk & microarrays & physiologia & sports\\
agriculture & bloods & dietetics & genes & jimaging & microbiolres & plants & standards\\
agriengineering & blsf & digital & geographies & jintelligence & micromachines & plasma & stats\\
agrochemicals & brainsci & disabilities & geohazards & jlpea & microorganisms & platforms & std\\
\addlinespace
agronomy & breath & diseases & geomatics & jmmp & microplastics & pollutants & stresses\\
ai & buildings & diversity & geosciences & jmp & minerals & polymers & surfaces\\
air & businesses & dna & geotechnics & jmse & mining & polysaccharides & surgeries\\
algorithms & cancers & drones & geriatrics & jne & modelling & poultry & suschem\\
allergies & carbon & dynamics & grasses & jnt & molbank & powders & sustainability\\
\addlinespace
alloys & cardiogenetics & earth & gucdd & jof & molecules & preprints & symmetry\\
analytica & catalysts & ebj & hazardousmatters & joitmc & mps & proceedings & synbio\\
analytics & cells & ecologies & healthcare & jor & msf & processes & systems\\
anatomia & ceramics & econometrics & hearts & journalmedia & mti & prosthesis & targets\\
animals & challenges & economies & hemato & jox & muscles & proteomes & taxonomy\\
\addlinespace
antibiotics & chemengineering & education & hematolrep & jpm & nanoenergyadv & psf & technologies\\
antibodies & chemistry & ejihpe & heritage & jrfm & nanomanufacturing & psych & telecom\\
antioxidants & chemosensors & electricity & higheredu & jsan & nanomaterials & psychiatryint & test\\
applbiosci & chemproc & electrochem & highthroughput & jtaer & ncrna & psychoactives & textiles\\
appliedchem & children & electronicmat & histories & jvd & ndt & publications & thalassrep\\
\addlinespace
appliedmath & chips & electronics & horticulturae & jzbg & network & quantumrep & thermo\\
applmech & cimb & encyclopedia & hospitals & kidneydial & neuroglia & quaternary & tomography\\
applmicrobiol & civileng & endocrines & humanities & kinasesphosphatases & neurolint & qubs & tourismhosp\\
applnano & cleantechnol & energies & humans & knowledge & neurosci & radiation & toxics\\
applsci & climate & eng & hydrobiology & land & nitrogen & reactions & toxins\\
\addlinespace
aquacj & clinpract & engproc & hydrogen & languages & notspecified & receptors & transplantology\\
architecture & clockssleep & entomology & hydrology & laws & nri & recycling & transportation\\
arm & cmd & entropy & hygiene & life & nursrep & regeneration & traumacare\\
arthropoda & coasts & environments & idr & liquids & nutraceuticals & religions & traumas\\
arts & coatings & environsciproc & ijerph & literature & nutrients & remotesensing & tropicalmed\\
\addlinespace
asc & colloids & epidemiologia & ijfs & livers & obesities & reports & universe\\
asi & colorants & epigenomes & ijgi & logics & oceans & reprodmed & urbansci\\
astronomy & commodities & est & ijms & logistics & ohbm & resources & uro\\
atmosphere & compounds & fermentation & ijns & lubricants & onco & rheumato & vaccines\\
atoms & computation & fibers & ijpb & lymphatics & oncopathology & risks & vehicles\\
\addlinespace
audiolres & computers & fintech & ijtm & machines & optics & robotics & venereology\\
automation & condensedmatter & fire & ijtpp & macromol & oral & ruminants & vetsci\\
axioms & conservation & fishes & ime & magnetism & organics & safety & vibration\\
bacteria & constrmater & fluids & immuno & magnetochemistry & organoids & sci & virtualworlds\\
batteries & cosmetics & foods & informatics & make & osteology & scipharm & viruses\\
\addlinespace
bdcc & covid & forecasting & information & marinedrugs & oxygen & sclerosis & vision\\
behavsci & crops & forensicsci & infrastructures & materials & parasites & seeds & waste\\
beverages & cryptography & forests & inorganics & materproc & parasitologia & sensors & water\\
biochem & crystals & foundations & insects & mathematics & particles & separations & wem\\
bioengineering & csmf & fractalfract & instruments & mca & pathogens & sexes & wevj\\
\addlinespace
biologics & ctn & fuels & inventions & measurements & pathophysiology & signals & wind\\
biology & curroncol & future & iot & medicina & pediatrrep & sinusitis & women\\
biomass & cyber & futureinternet & j & medicines & pharmaceuticals & skins & world\\
biomechanics & dairy & futurepharmacol & jal & medsci & pharmaceutics & smartcities & youth\\
biomed & data & futurephys & jcdd & membranes & pharmacoepidemiology & sna & zoonoticdis\\
\addlinespace
biomedicines & ddc & futuretransp & jcm & merits & pharmacy & societies & \\
\bottomrule
\end{longtable}
\finishlandscape

\hypertarget{introduction}{%
\section{Introduction}\label{introduction}}

The introduction should briefly place the study in a broad context and highlight
why it is important. It should define the purpose of the work and its
significance. The current state of the research field should be reviewed
carefully and key publications cited. Please highlight controversial and
diverging hypotheses when necessary. Finally, briefly mention the main aim of
the work and highlight the principal conclusions. As far as possible, please
keep the introduction comprehensible to scientists outside your particular
field of research. Citing a journal paper (Bertrand-Krajewski, Chebbo, and Saget 1998; Leutnant, Muschalla, and Uhl 2016). And now citing a book reference Gujer (2008).
Some MDPI journals use Chicago and others use APA, this template should choose
the correct citation format for you once you specify the journal in the YAML
header.

To use endnotes, change \texttt{endnotes:\ true} in the YAML header, then use
\texttt{\textbackslash{}endnote\{This\ is\ an\ endnote.\}}.

\hypertarget{materials-and-methods}{%
\section{Materials and Methods}\label{materials-and-methods}}

Materials and Methods should be described with sufficient details to allow
others to replicate and build on published results. Please note that publication
of your manuscript implicates that you must make all materials, data, computer
code, and protocols associated with the publication available to readers. Please
disclose at the submission stage any restrictions on the availability of
materials or information. New methods and protocols should be described in
detail while well-established methods can be briefly described and appropriately
cited.

Research manuscripts reporting large datasets that are deposited in a publicly
available database should specify where the data have been deposited and provide
the relevant accession numbers. If the accession numbers have not yet been
obtained at the time of submission, please state that they will be provided
during review. They must be provided prior to publication.

Interventionary studies involving animals or humans, and other studies require
ethical approval must list the authority that provided approval and the
corresponding ethical approval code.

\hypertarget{results}{%
\section{Results}\label{results}}

This section may be divided by subheadings. It should provide a concise and
precise description of the experimental results, their interpretation as well
as the experimental conclusions that can be drawn.

\hypertarget{subsection-heading-here}{%
\subsection{Subsection Heading Here}\label{subsection-heading-here}}

Subsection text here.

\hypertarget{subsubsection-heading-here}{%
\subsubsection{Subsubsection Heading Here}\label{subsubsection-heading-here}}

Bulleted lists look like this:

\begin{itemize}
\tightlist
\item
  First bullet
\item
  Second bullet
\item
  Third bullet
\end{itemize}

Numbered lists can be added as follows:

\begin{enumerate}
\def\labelenumi{\arabic{enumi}.}
\tightlist
\item
  First item
\item
  Second item
\item
  Third item
\end{enumerate}

The text continues here.

\hypertarget{figures-tables-and-schemes}{%
\subsection{Figures, Tables and Schemes}\label{figures-tables-and-schemes}}

All figures and tables should be cited in the main text as Figure \ref{fig:fig1},
\ref{tab:tab1}, etc. To get cross-reference to figure generated by R chunks
include the \texttt{\textbackslash{}\textbackslash{}label\{\}} tag in the \texttt{fig.cap} attribute of the R chunk:
\texttt{fig.cap\ =\ "Fancy\ Caption\textbackslash{}\textbackslash{}label\{fig:plot\}"}.

\begin{figure}[h!]

{\centering \includegraphics[width=0.7\linewidth]{MDPI_article_files/figure-latex/fig1-1} 

}

\caption{A figure added with a code chunk.\label{fig:fig1}}(\#fig:fig1)
\end{figure}

When making tables using \texttt{kable}, it is suggested to use
the \texttt{format="latex"} and \texttt{tabl.envir="table"} arguments
to ensure table numbering and compatibility with the mdpi
document class.

\begin{table}[H]

\caption{(\#tab:tab1)This is a table caption. Tables should be placed in the 
             main text near to the first time they are~cited.}
\begin{tabular}[t]{lccc}
\toprule
  & mpg & cyl & disp\\
\midrule
Mazda RX4 & 21.0 & 6 & 160\\
Mazda RX4 Wag & 21.0 & 6 & 160\\
Datsun 710 & 22.8 & 4 & 108\\
Hornet 4 Drive & 21.4 & 6 & 258\\
Hornet Sportabout & 18.7 & 8 & 360\\
\bottomrule
\end{tabular}
\end{table}

For a very wide table, landscape layouts are allowed.

\startlandscape

\begin{table}[H]

\caption{(\#tab:tab2)This is a very wide table}
\begin{tabular}[t]{cccc}
\toprule
Title.1 & Title.2 & Title.3 & Title.4\\
\midrule
Entry 1 & Data & Data & This cell has some longer content that runs over
                               two lines\\
Entry 2 & Data & Data & Data\\
\bottomrule
\end{tabular}
\end{table}

\finishlandscape

\hypertarget{formatting-of-mathematical-components}{%
\subsection{Formatting of Mathematical Components}\label{formatting-of-mathematical-components}}

This is an example of an equation:

\[
a = 1.
\]

If you want numbered equations use Latex and wrap in the equation environment:

\begin{equation}
a = 1,
\end{equation}

the text following an equation need not be a new paragraph. Please punctuate
equations as regular text.

This is the example 2 of equation:

\begin{adjustwidth}{-\extralength}{0cm}
\begin{equation}
a = b + c + d + e + f + g + h + i + j + k + l + m + n + o + p + q + r + s + t + 
u + v + w + x + y + z
\end{equation}
\end{adjustwidth}

Theorem-type environments (including propositions, lemmas, corollaries etc.)
can be formatted as follows:

Example of a theorem:

\begin{Theorem}
Example text of a theorem

\end{Theorem}

The text continues here. Proofs must be formatted as follows:

Example of a proof:

\begin{proof}
Text of the proof. Note that the phrase ``of Theorem 1'\,' is optional if it is
clear which theorem is being referred to.
\end{proof}

The text continues here.

\hypertarget{discussion}{%
\section{Discussion}\label{discussion}}

Authors should discuss the results and how they can be interpreted in
perspective of previous studies and of the working hypotheses. The findings and
their implications should be discussed in the broadest context possible. Future
research directions may also be highlighted.

\hypertarget{conclusion}{%
\section{Conclusion}\label{conclusion}}

This section is not mandatory, but can be added to the manuscript if the
discussion is unusually long or complex.

\hypertarget{patents}{%
\section{Patents}\label{patents}}

This section is not mandatory, but may be added if there are patents resulting
from the work reported in this manuscript.

\hypertarget{refs}{}
\begin{CSLReferences}{1}{0}
\leavevmode\vadjust pre{\hypertarget{ref-bertrand-krajewski_distribution_1998}{}}%
Bertrand-Krajewski, J. L., G. Chebbo, and A. Saget. 1998. {``Distribution of Pollutant Mass Vs Volume in Stormwater Discharges and the First Flush Phenomenon.''} \emph{Water Research} 32 (8): 2341--56. \url{http://www.scopus.com/inward/record.url?eid=2-s2.0-0032146023\&partnerID=40\&md5=d6517b6efa014bc12f3ae70abe71977d}.

\leavevmode\vadjust pre{\hypertarget{ref-gujer_systems_2008}{}}%
Gujer, W. 2008. \emph{Systems {Analysis} for {Water} {Technology}}. Berlin, Heidelberg, Germany: Springer-Verlag.

\leavevmode\vadjust pre{\hypertarget{ref-leutnant_stormwater_2016}{}}%
Leutnant, Dominik, Dirk Muschalla, and Mathias Uhl. 2016. {``Stormwater {Pollutant} {Process} {Analysis} with {Long}-{Term} {Online} {Monitoring} {Data} at {Micro}-{Scale} {Sites}.''} \emph{Water} 8 (7): 299. \url{https://doi.org/10.3390/w8070299}.

\end{CSLReferences}

\end{document}
