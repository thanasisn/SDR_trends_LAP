%  LaTeX support: latex@mdpi.com
%DIF LATEXDIFF DIFFERENCE FILE
%DIF DEL SUBMISSION_01/MDPI_submission.tex   Sun Nov 12 17:31:09 2023
%DIF ADD MDPI_submission.tex                 Sun Dec 17 20:51:44 2023
%  For support, please attach all files needed for compiling as well as the log file, and specify your operating system, LaTeX version, and LaTeX editor.

%=================================================================
% pandoc conditionals added to preserve backwards compatibility with previous versions of rticles

\documentclass[applsci,article,submit,moreauthors,pdftex]{Definitions/mdpi}


%% Some pieces required from the pandoc template
\setlist[itemize]{leftmargin=*,labelsep=5.8mm}
\setlist[enumerate]{leftmargin=*,labelsep=4.9mm}


%--------------------
% Class Options:
%--------------------

%---------
% article
%---------
% The default type of manuscript is "article", but can be replaced by:
% abstract, addendum, article, book, bookreview, briefreport, casereport, comment, commentary, communication, conferenceproceedings, correction, conferencereport, entry, expressionofconcern, extendedabstract, datadescriptor, editorial, essay, erratum, hypothesis, interestingimage, obituary, opinion, projectreport, reply, retraction, review, perspective, protocol, shortnote, studyprotocol, systematicreview, supfile, technicalnote, viewpoint, guidelines, registeredreport, tutorial
% supfile = supplementary materials

%----------
% submit
%----------
% The class option "submit" will be changed to "accept" by the Editorial Office when the paper is accepted. This will only make changes to the frontpage (e.g., the logo of the journal will get visible), the headings, and the copyright information. Also, line numbering will be removed. Journal info and pagination for accepted papers will also be assigned by the Editorial Office.

%------------------
% moreauthors
%------------------
% If there is only one author the class option oneauthor should be used. Otherwise use the class option moreauthors.

%---------
% pdftex
%---------
% The option pdftex is for use with pdfLaTeX. Remove "pdftex" for (1) compiling with LaTeX & dvi2pdf (if eps figures are used) or for (2) compiling with XeLaTeX.

%=================================================================
% MDPI internal commands - do not modify
\firstpage{1}
\makeatletter
\setcounter{page}{\@firstpage}
\makeatother
\pubvolume{1}
\issuenum{1}
\articlenumber{0}
\pubyear{2023}
\copyrightyear{2023}
%\externaleditor{Academic Editor: Firstname Lastname}
\datereceived{ }
\daterevised{ } % Comment out if no revised date
\dateaccepted{ }
\datepublished{ }
%\datecorrected{} % For corrected papers: "Corrected: XXX" date in the original paper.
%\dateretracted{} % For corrected papers: "Retracted: XXX" date in the original paper.
\hreflink{https://doi.org/} % If needed use \linebreak
%\doinum{}
%\pdfoutput=1 % Uncommented for upload to arXiv.org

%=================================================================
% Add packages and commands here. The following packages are loaded in our class file: fontenc, inputenc, calc, indentfirst, fancyhdr, graphicx, epstopdf, lastpage, ifthen, float, amsmath, amssymb, lineno, setspace, enumitem, mathpazo, booktabs, titlesec, etoolbox, tabto, xcolor, colortbl, soul, multirow, microtype, tikz, totcount, changepage, attrib, upgreek, array, tabularx, pbox, ragged2e, tocloft, marginnote, marginfix, enotez, amsthm, natbib, hyperref, cleveref, scrextend, url, geometry, newfloat, caption, draftwatermark, seqsplit
% cleveref: load \crefname definitions after \begin{document}

%=================================================================
% Please use the following mathematics environments: Theorem, Lemma, Corollary, Proposition, Characterization, Property, Problem, Example, ExamplesandDefinitions, Hypothesis, Remark, Definition, Notation, Assumption
%% For proofs, please use the proof environment (the amsthm package is loaded by the MDPI class).

%=================================================================
% Full title of the paper (Capitalized)
\Title{Trends from 30-year observations of downward solar irradiance in
Thessaloniki, Greece}

% MDPI internal command: Title for citation in the left column
\TitleCitation{Trends from 30-year observations of downward solar
irradiance in Thessaloniki, Greece}

% Author Orchid ID: enter ID or remove command
%\newcommand{\orcidauthorA}{0000-0000-0000-000X} % Add \orcidA{} behind the author's name
%\newcommand{\orcidauthorB}{0000-0000-0000-000X} % Add \orcidB{} behind the author's name


% Authors, for the paper (add full first names)
\Author{Athanasios
Natsis$^{1}$\href{https://orcid.org/0000-0002-5199-4119}
{\orcidicon}, Alkiviadis Bais$^{1,*}$, Charikleia Meleti$^{1}$}


%\longauthorlist{yes}


% MDPI internal command: Authors, for metadata in PDF
\AuthorNames{Athanasios Natsis, Alkiviadis Bais, Charikleia Meleti}

% MDPI internal command: Authors, for citation in the left column
%\AuthorCitation{Lastname, F.; Lastname, F.; Lastname, F.}
% If this is a Chicago style journal: Lastname, Firstname, Firstname Lastname, and Firstname Lastname.
\AuthorCitation{Natsis, A.; Bais, A.; Meleti, C.}

% Affiliations / Addresses (Add [1] after \address if there is only one affiliation.)
\address{%
$^{1}$ \quad Aristotle University of Thessaloniki - Laboratory of
Atmospheric Physics, Campus Box 149, 54124 Thessaloniki,
Greece; \href{mailto:natsisphysicist@gmail.com}{\nolinkurl{natsisphysicist@gmail.com}}
(A.N.); \href{mailto:abais@auth.gr}{\nolinkurl{abais@auth.gr}} (A.B.);
\href{mailto:meleti@auth.gr}{\nolinkurl{meleti@auth.gr}} (C.M.)\\
}

% Contact information of the corresponding author
\corres{Correspondence: \href{mailto:abais@auth.gr}{\nolinkurl{abais@auth.gr}}}

% Current address and/or shared authorship








% The commands \thirdnote{} till \eighthnote{} are available for further notes

% Simple summary

%\conference{} % An extended version of a conference paper

% Abstract (Do not insert blank lines, i.e. \\)
\abstract{The shortwave downward solar irradiance (SDR) is an important
factor that drives climate processes, production and can affect all
living organisms. Observations of SDR at different locations around the
world with different environmental characteristics have been used to
investigate its long-term variability and trends at different time
scales. Periods of positive trends are referred as brightening periods
and of negative trends as dimming periods. Here we studied 30 years of
pyranometer data in Thessaloniki, Greece, under three types of sky
conditions (clear sky, cloudy sky and all sky). The clear-sky data were
identified by applying a cloud screening algorithm. We found a positive
trend of \(0.38\,\%/\text{year}\) for all-sky, \(0.35\,\%/\text{year}\)
for clear-sky conditions, and \(-0.28\,\%/\text{year}\) for cloudy
conditions. We have also investigated the consistency of these trends,
their seasonal variability, and the effect of the solar zenith angle. We
have found that for all-sky and clear-sky conditions the SDR trend is
positive in winter (\(0.7\) and \(0.8\,\%/\text{year}\), respectively)
and autumn (\(~0.4\,\%/\text{year}\)), while under cloudy skies the
trend is negative (\(-0.9\,\%/\text{year}\) in winter and
\(-0.4\,\%/\text{year}\) in autumn). In spring and summer the trend is
very close to zero, irrespective of sky conditions. The SDR trend is
increasing with increasing solar zenith angle, except under cloudy skies
where the trend is highly variable and close to zero. Finally, we
identified some anomalies in the long term SDR trends for all sky
conditions by examining the cumulative sums of monthly anomalies from
the climatological mean.}


% Keywords
\keyword{GHI; SDR; solar radiation; Solar Brigthening/Dimming; aerosols;
clouds.}

% The fields PACS, MSC, and JEL may be left empty or commented out if not applicable
%\PACS{J0101}
%\MSC{}
%\JEL{}

%%%%%%%%%%%%%%%%%%%%%%%%%%%%%%%%%%%%%%%%%%
% Only for the journal Diversity
%\LSID{\url{http://}}

%%%%%%%%%%%%%%%%%%%%%%%%%%%%%%%%%%%%%%%%%%
% Only for the journal Applied Sciences

%%%%%%%%%%%%%%%%%%%%%%%%%%%%%%%%%%%%%%%%%%

%%%%%%%%%%%%%%%%%%%%%%%%%%%%%%%%%%%%%%%%%%
% Only for the journal Data



%%%%%%%%%%%%%%%%%%%%%%%%%%%%%%%%%%%%%%%%%%
% Only for the journal Toxins


%%%%%%%%%%%%%%%%%%%%%%%%%%%%%%%%%%%%%%%%%%
% Only for the journal Encyclopedia


%%%%%%%%%%%%%%%%%%%%%%%%%%%%%%%%%%%%%%%%%%
% Only for the journal Advances in Respiratory Medicine
%\addhighlights{yes}
%\renewcommand{\addhighlights}{%

%\noindent This is an obligatory section in “Advances in Respiratory Medicine”, whose goal is to increase the discoverability and readability of the article via search engines and other scholars. Highlights should not be a copy of the abstract, but a simple text allowing the reader to quickly and simplified find out what the article is about and what can be cited from it. Each of these parts should be devoted up to 2~bullet points.\vspace{3pt}\\
%\textbf{What are the main findings?}
% \begin{itemize}[labelsep=2.5mm,topsep=-3pt]
% \item First bullet.
% \item Second bullet.
% \end{itemize}\vspace{3pt}
%\textbf{What is the implication of the main finding?}
% \begin{itemize}[labelsep=2.5mm,topsep=-3pt]
% \item First bullet.
% \item Second bullet.
% \end{itemize}
%}


%%%%%%%%%%%%%%%%%%%%%%%%%%%%%%%%%%%%%%%%%%


% tightlist command for lists without linebreak
\providecommand{\tightlist}{%
  \setlength{\itemsep}{0pt}\setlength{\parskip}{0pt}}



\usepackage{subcaption}
\captionsetup[sub]{position=bottom, labelfont={bf, small, stretch=1.17}, labelsep=space, textfont={small, stretch=1.17}, aboveskip=6pt,  belowskip=-6pt, singlelinecheck=off, justification=justified}
\usepackage{placeins}
\usepackage{longtable}
\usepackage{booktabs}
\usepackage{array}
\usepackage{multirow}
\usepackage{wrapfig}
\usepackage{float}
\usepackage{colortbl}
\usepackage{pdflscape}
\usepackage{tabu}
\usepackage{threeparttable}
\usepackage{threeparttablex}
\usepackage[normalem]{ulem}
\usepackage{makecell}
\usepackage{xcolor}
%DIF PREAMBLE EXTENSION ADDED BY LATEXDIFF
%DIF UNDERLINE PREAMBLE %DIF PREAMBLE
\RequirePackage[normalem]{ulem} %DIF PREAMBLE
\RequirePackage{color}\definecolor{RED}{rgb}{1,0,0}\definecolor{BLUE}{rgb}{0,0,1} %DIF PREAMBLE
\providecommand{\DIFadd}[1]{{\protect\color{blue}\uwave{#1}}} %DIF PREAMBLE
\providecommand{\DIFdel}[1]{{\protect\color{red}\sout{#1}}}                      %DIF PREAMBLE
%DIF SAFE PREAMBLE %DIF PREAMBLE
\providecommand{\DIFaddbegin}{} %DIF PREAMBLE
\providecommand{\DIFaddend}{} %DIF PREAMBLE
\providecommand{\DIFdelbegin}{} %DIF PREAMBLE
\providecommand{\DIFdelend}{} %DIF PREAMBLE
\providecommand{\DIFmodbegin}{} %DIF PREAMBLE
\providecommand{\DIFmodend}{} %DIF PREAMBLE
%DIF FLOATSAFE PREAMBLE %DIF PREAMBLE
\providecommand{\DIFaddFL}[1]{\DIFadd{#1}} %DIF PREAMBLE
\providecommand{\DIFdelFL}[1]{\DIFdel{#1}} %DIF PREAMBLE
\providecommand{\DIFaddbeginFL}{} %DIF PREAMBLE
\providecommand{\DIFaddendFL}{} %DIF PREAMBLE
\providecommand{\DIFdelbeginFL}{} %DIF PREAMBLE
\providecommand{\DIFdelendFL}{} %DIF PREAMBLE
%DIF COLORLISTINGS PREAMBLE %DIF PREAMBLE
\RequirePackage{listings} %DIF PREAMBLE
\RequirePackage{color} %DIF PREAMBLE
\lstdefinelanguage{DIFcode}{ %DIF PREAMBLE
%DIF DIFCODE_UNDERLINE %DIF PREAMBLE
  moredelim=[il][\color{red}\sout]{\%DIF\ <\ }, %DIF PREAMBLE
  moredelim=[il][\color{blue}\uwave]{\%DIF\ >\ } %DIF PREAMBLE
} %DIF PREAMBLE
\lstdefinestyle{DIFverbatimstyle}{ %DIF PREAMBLE
	language=DIFcode, %DIF PREAMBLE
	basicstyle=\ttfamily, %DIF PREAMBLE
	columns=fullflexible, %DIF PREAMBLE
	keepspaces=true %DIF PREAMBLE
} %DIF PREAMBLE
\lstnewenvironment{DIFverbatim}{\lstset{style=DIFverbatimstyle}}{} %DIF PREAMBLE
\lstnewenvironment{DIFverbatim*}{\lstset{style=DIFverbatimstyle,showspaces=true}}{} %DIF PREAMBLE
%DIF END PREAMBLE EXTENSION ADDED BY LATEXDIFF

\begin{document}



%%%%%%%%%%%%%%%%%%%%%%%%%%%%%%%%%%%%%%%%%%

\DIFdelbegin %DIFDELCMD < \hypertarget{introduction.}{%
%DIFDELCMD < \section{Introduction.}\label{introduction.}}
%DIFDELCMD < %%%
\DIFdelend \DIFaddbegin \hypertarget{introduction}{%
\section{Introduction}\label{introduction}}
\DIFaddend 

The shortwave downward solar irradiance (SDR) at Earth's surface plays a
significant role, on its climate. Changes of the SDR can be related to
changes on Earth's energy budget, the mechanisms of climate change, and
water and carbon cycles \citep{Wild2009}. It can also affect solar and
agricultural production, and all living organisms. Studies of SDR
variability, have identified some distinct SDR trends on different
regions of the world on different time periods. The term `brightening'
is generally used to describe periods of positive SDR trend, and
`dimming' for negative trend \citep{Wild2009}. There are many cases in
the long term records of irradiance, showing a systematic change in the
magnitude of the trend, occurring roughly in the last decades of the
20th century. On multiple stations in China, a dimming period was
reported until about 2000, followed by a brightening period
\citep{Yang2021}. A similar pattern was identified, with the breaking
point around 1980, for stations in Central Europe \citep{Wild2021} and
Brazil \citep{Yamasoe2021}. On global scale, an artificial Intelligence
aided spatial analysis on continental level with data from multiple
stations reach similar conclusions for these regions and for the global
trend \citep{Yuan2021}.

There is a consensus among researchers that the major factor affecting
the variability of SDR attenuation is the interactions of solar
radiation with atmospheric aerosols and clouds. Those interactions,
among other factors, have been analysed with models
\citep{Li2016, Samset2018}, showing the existence of feedback mechanisms
between the two. Similar findings have been shown in observational data
\citep[ and references
therein]{Schwarz2020, Ohvril2009, Zerefos2009, Xia2007}.

Due to the significant spatial and temporal variability of the trends,
and the contributing factors, there is a constant need to monitor and
investigate SDR in different sites in order to estimate the degree of
variability, and its relation to the local conditions. In this study, we
examine the trends of SDR, using ground-based measurements at
Thessaloniki, Greece, for the period 1993 to 2023, as derived from a
CM-21 pyranometer. We reevaluated and extended the dataset used by
\citet{Bais2013}, we applied a different algorithm for the
identification of clear-/cloud-sky instances
\citep{Reno2016, Reno2012a}, and we derived the SDR trends for the
period of study, under different sky conditions (all-sky, clear-sky and
cloud-sky). Finally, we investigated the dependence of the trends on
solar zenith angle and season.

\DIFdelbegin %DIFDELCMD < \hypertarget{observational-data-and-methodology.}{%
%DIFDELCMD < \section{Observational data and
%DIFDELCMD < methodology.}\label{observational-data-and-methodology.}}
%DIFDELCMD < %%%
\DIFdelend \DIFaddbegin \hypertarget{data-and-methodology}{%
\section{Data and methodology}\label{data-and-methodology}}
\DIFaddend 

The SDR data were measured with a Kipp \& Zonen CM-21 pyranometer
operating continuously at the Laboratory of Atmospheric Physics of the
Aristotle University of Thessaloniki (\(40^\circ\,38'\,\)N,
\(22^\circ\,57'\,\)E, \(80\,\)m~a.s.l.) \DIFdelbegin \DIFdel{in }\DIFdelend \DIFaddbegin \DIFadd{since 1993. Here we use data for
}\DIFaddend the period from 1993-04-13 to \DIFdelbegin \DIFdel{2023-04-13}\DIFdelend \DIFaddbegin \DIFadd{2023-04-12}\DIFaddend . The monitoring site is located
near the city centre \DIFdelbegin \DIFdel{, and we expect to be }\DIFdelend \DIFaddbegin \DIFadd{thus we expect that measurements are }\DIFaddend affected by
the urban environment. During the study period, the pyranometer has been
independently calibrated three times at the Meteorologisches
Observatorium Lindenberg, DWD, \DIFdelbegin \DIFdel{when it was verified the
}\DIFdelend \DIFaddbegin \DIFadd{verifying that }\DIFaddend stability of the
instrument\DIFdelbegin \DIFdel{to within }\DIFdelend \DIFaddbegin \DIFadd{'s sensitivity is }\DIFaddend better than \(0.7\%\) relative to the
initial calibration by the manufacturer. Along with SDR, the direct beam
radiation (DNI) \DIFdelbegin \DIFdel{was also measured by }\DIFdelend \DIFaddbegin \DIFadd{is also measured with }\DIFaddend a collocated Kipp \& Zonen CHP-1
pyrheliometer \DIFdelbegin \DIFdel{, for the period }\DIFdelend \DIFaddbegin \DIFadd{since }\DIFaddend 2016-04-01\DIFdelbegin \DIFdel{to 2023-04-13. Although,
we have performed a similar analysis to the DNI data, the results are
not presented here, as they lack the appropriate statistical
significance, due to the sorter duration of the data. However, the }\DIFdelend \DIFaddbegin \DIFadd{. The }\DIFaddend DNI data were used as auxiliary data
\DIFdelbegin \DIFdel{, }\DIFdelend \DIFaddbegin \DIFadd{to support the selection of appropriate thresholds }\DIFaddend in the clear sky
identification algorithm (CSid), which is discussed later\DIFdelbegin \DIFdel{, for the selection of the
appropriate thresholds}\DIFdelend . It is noted
that \DIFdelbegin \DIFdel{despite the capability of }\DIFdelend the \DIFdelbegin \DIFdel{CSid algorithm to use the DNI as a characterization parameter, we
haven't utilized it here, }\DIFdelend \DIFaddbegin \DIFadd{limited dataset of DNI was not used for the identification of
clear sky cases in the entire SDR series }\DIFaddend to avoid any selection bias \DIFdelbegin \DIFdel{, }\DIFdelend due
to unequal length of the two datasets. There are four distinct steps in
the creation of the dataset \DIFdelbegin \DIFdel{analysed }\DIFdelend \DIFaddbegin \DIFadd{analyzed }\DIFaddend here: a)~the acquisition of
radiation measurements from the sensors, b)~the data quality check,
c)~the identification of ``clear sky'' conditions from the radiometric
data, and d)~the aggregation of data and trend analysis.

For the acquisition of radiometric data, the signal of the pyranometer
is sampled \DIFdelbegin \DIFdel{with }\DIFdelend \DIFaddbegin \DIFadd{at }\DIFaddend a rate of \(1\,\text{Hz}\). The mean and the standard
deviation of these samples are \DIFaddbegin \DIFadd{calculated and }\DIFaddend recorded every minute. The
measurements are corrected for the zero offset (``dark signal'' in
volts)\DIFdelbegin \DIFdel{. The ``dark
signal'' }\DIFdelend \DIFaddbegin \DIFadd{, which }\DIFaddend is calculated by averaging all measurements recorded for a
period of \(3\,\text{h}\), before (morning) or after (evening) the Sun
reaches an elevation angle of \(-10^\circ\). The signal is converted to
irradiance using a ramped value of the instrument's sensitivity between
\DIFaddbegin \DIFadd{subsequent }\DIFaddend calibrations.

A manual screening was performed, to remove inconsistent and erroneous
recordings that can occur stochastically or systematically, during the
continuous operation of the instruments. The manual screening is aided
by a radiation data quality assurance procedure, adjusted for the site,
which is based on the methods of Long and
Shi~\citetext{\citeyear{Long2008a}; \citeyear{Long2006}}. Thus,
problematic recordings have been excluded from further processing.
Although it is impossible to detect all false data, the large number of
available data, and the aggregation scheme we used, ensures the \DIFdelbegin \DIFdel{good
}\DIFdelend quality
of the radiation measurements used in this study.

\DIFdelbegin \DIFdel{In order to be able }\DIFdelend \DIFaddbegin \DIFadd{Due to the significant measurement uncertainty when the Sun is near the
horizon, we have excluded all measurements with SZA greater than
\(85^\circ\). Moreover, due to obstructions around the site (hills and
buildings) which block the direct irradiance, we excluded data with
azimuth angle in the range \(58^{\circ}\) - \(120^{\circ}\) and with SZA
greater than \(78^{\circ}\). To make the measurements comparable
throughout the dataset, we adjusted all one-minute data to the mean Sun
- Earth distance. Subsequently, we adjusted all measurements to the
Total Solar Irradiance (TSI) at \(1\,\text{au}\), in order to compensate
for the Sun's intensity variability, using a time series of satellite
TSI observations. The TSI data we used are part of the ``NOAA Climate
Data Record of Total Solar Irradiance'' dataset \mbox{%DIFAUXCMD
\citep{Coddington2005}}\hskip0pt%DIFAUXCMD
.
The initial daily values of this dataset were interpolated to match the
time step of our measurements.
}

\DIFadd{In order }\DIFaddend to estimate the effect of the sky conditions on the long term
variability of SDR, we created three datasets \DIFdelbegin \DIFdel{, }\DIFdelend by characterizing each
one-minute measurement with a corresponding \DIFdelbegin \DIFdel{sky
condition }\DIFdelend \DIFaddbegin \DIFadd{sky-condition flag }\DIFaddend (i.e.,
all-sky, clear-sky and cloudy-sky). To identify the \DIFdelbegin \DIFdel{clear-sky conditions }\DIFdelend \DIFaddbegin \DIFadd{clear-cases }\DIFaddend we used
a method proposed by \citet{Long2000} and by \citet{Reno2016}, which
were adapted and configured for the site \DIFdelbegin \DIFdel{, as
the authors suggest. }%DIFDELCMD < 

%DIFDELCMD < %%%
\DIFdel{We have to note , that
}\DIFdelend \DIFaddbegin \DIFadd{of Thessaloniki. We note that
all methods have some subjectivity in }\DIFaddend the definition of clear or cloudy
sky \DIFdelbegin \DIFdel{, has some
subjectivity, in any method of characterization}\DIFdelend \DIFaddbegin \DIFadd{cases}\DIFaddend . As a result, the details of the definition are site specific
\DIFdelbegin \DIFdel{, it relies }\DIFdelend \DIFaddbegin \DIFadd{and they rely }\DIFaddend on a combination of thresholds and comparisons with ideal
radiation models and statistical analysis of different signal metrics.
The CSid algorithm was calibrated with the main focus, to identify the
presence of clouds\DIFdelbegin \DIFdel{on
the sky}\DIFdelend . Despite the fine-tuning of the procedure, \DIFaddbegin \DIFadd{in }\DIFaddend a few
marginal cases \DIFdelbegin \DIFdel{exist, that have been identified manually as }\DIFdelend false positive or false negative \DIFdelbegin \DIFdel{but }\DIFdelend \DIFaddbegin \DIFadd{results were identified
by manual inspection. However, due to their small number, they }\DIFaddend cannot
affect the final results of the study. \DIFdelbegin %DIFDELCMD < 

%DIFDELCMD < %%%
\DIFdelend For completeness, we \DIFdelbegin \DIFdel{will }\DIFdelend provide
below a brief overview of the \DIFdelbegin \DIFdel{clear
sky }\DIFdelend \DIFaddbegin \DIFadd{clear-sky }\DIFaddend identification algorithm (CSid),
along with the site specific thresholds.
\DIFaddbegin 

\hypertarget{the-clear-sky-identification-algorithm}{%
\subsection{The clear sky identification
algorithm}\label{the-clear-sky-identification-algorithm}}

\DIFaddend To calculate the reference \DIFdelbegin \DIFdel{clear sky
}\DIFdelend \DIFaddbegin \DIFadd{clear-sky }\DIFaddend \(\text{SDR}_\text{CSref}\) we used
the \(\text{SDR}_\text{Haurwitz}\) derived by the radiation model of
\citet{Haurwitz1945}, adjusted for our site with a factor \(a\)
(Eq.~\ref{eq:ahau}), \DIFdelbegin \DIFdel{resulted by }\DIFdelend \DIFaddbegin \DIFadd{estimated through }\DIFaddend an iterative optimization
process, as described by \citet{Long2000} and \citet{Reno2016}. The
target of the optimization was the minimization of a function \(f(a)\)
(Eq.~\ref{eq:minf}) and was accomplished with the algorithmic function
``optimise'', which is an implementation based on the work of
\citet{Brent1973}, from the library ``stats'' of the R programming
language \citep{RCT2023}. \begin{equation}
f(a) = \frac{1}{n}\sum_{i=1}^{n} ( \text{SDR}_{\text{CSid},i} - a \times \text{SDR}_{\text{testCSref},i} )^2 \label{eq:minf}
\end{equation} where: \(n\) is the total number of daylight \DIFdelbegin \DIFdel{records}\DIFdelend \DIFaddbegin \DIFadd{data}\DIFaddend ,
\(\text{SDR}_{\text{CSid},i}\) are the \DIFdelbegin \DIFdel{records }\DIFdelend \DIFaddbegin \DIFadd{data }\DIFaddend identified as clear sky by
CSid, \(a\) is a \DIFdelbegin \DIFdel{hypothetical }\DIFdelend \DIFaddbegin \DIFadd{site-specific }\DIFaddend adjustment factor, and
\(\text{SDR}_{\text{testCSref},i}\) is \DIFaddbegin \DIFadd{the SDR derived by }\DIFaddend any of the
tested \DIFdelbegin \DIFdel{clear sky
}\DIFdelend \DIFaddbegin \DIFadd{clear-sky }\DIFaddend radiation models.

The optimization and the selection of the clear sky reference model, was
performed on SDR observations for the period 2016 - 2021. During the
optimization, eight simple clear sky radiation models were tested
(namely, Daneshyar-Paltridge-Proctor, Kasten-Czeplak, Haurwitz,
Berger-Duffie, Adnot-Bourges-Campana-Gicquel, Robledo-Soler, Kasten and
Ineichen-Perez), with a wide range of factors. These models are
described in more details by \citet{Reno2012} and evaluated by
\citet{Reno2016}. We found, that Haurwitz's model, adjusted with the
factor \(a = 0.965\) yields one of the lowest root mean squared errors
(RMSE), while the procedure \DIFdelbegin \DIFdel{, }\DIFdelend manages to characterize \DIFaddbegin \DIFadd{successfully }\DIFaddend the
majority of the data. Thus, our clear sky reference is derived by the
Eq.~\ref{eq:ahau}. \begin{equation}
\text{SDR}_\text{CSref} = a \times \text{SDR}_\text{Haurwitz} = 0.965 \times 1098 \times \cos(\theta) \times \exp \left( \frac{ - 0.057}{\cos(\theta)} \right) \label{eq:ahau}
\end{equation} where \DIFdelbegin \DIFdel{: \(\text{SDR}_\text{CSref}\) is the reference clear
sky SDR, in \(\text{W}\,\text{m}^{-2}\) and }\DIFdelend \(\theta\) is the solar zenith angle (SZA).

The criteria that were used to identify whether a measurement was taken
under clear-sky conditions are presented below. A data point is flagged
as ``clear-sky'' if all criteria are satisfied\DIFdelbegin \DIFdel{, }\DIFdelend \DIFaddbegin \DIFadd{; }\DIFaddend otherwise it is
considered \DIFdelbegin \DIFdel{to be }\DIFdelend \DIFaddbegin \DIFadd{as }\DIFaddend ``cloud-sky''. Each criterion was applied for a running
window of \(11\) consecutive one-minute measurements, and the
characterization \DIFdelbegin \DIFdel{is }\DIFdelend \DIFaddbegin \DIFadd{was }\DIFaddend assigned to the central \DIFdelbegin \DIFdel{value }\DIFdelend \DIFaddbegin \DIFadd{datum }\DIFaddend of the window. Each
parameter, was calculated both from the observations and the reference
clear sky model, for each comparison. The allowable range of variation
is defined by the model-derived value of the parameter multiplied by a
factor plus an offset. The factors and the offsets were determined
empirically, by manually inspecting each filter's performance on
selected days and adjusting them accordingly during an iterative
process. The criteria are \DIFaddbegin \DIFadd{listed below, together with the range of
values within which the respective parameter should fall in order to
raise the clear-sky flag}\DIFaddend :

\begin{enumerate}
\def\labelenumi{\alph{enumi})}
\tightlist
\item
  Mean of the measured \(\overline{\text{SDR}}_i\) (Eq.
  \ref{eq:MeanVIP}). \begin{equation}
  0.91 \times \overline{\text{SDR}}_{\text{CSref},i} - 20\,Wm^{-2}
  < \overline{\text{SDR}}_i <
  1.095 \times \overline{\text{SDR}}_{\text{CSref},i} + 30\,Wm^{-2}
  \label{eq:MeanVIP}
  \end{equation}
\item
  Maximum measured value \(M_{i}\) (Eq.~\ref{eq:MaxVIP}).
  \begin{equation}
  1 \times M_{\text{CSref},i} - 75\,Wm^{-2}
  < M_{i} <
  1 \times M_{\text{CSref},i} + 75\,Wm^{-2}
  \label{eq:MaxVIP}
  \end{equation}
\item
  Length \(L_i\) of the sequential line segments, connecting the points
  of the \(11\) SDR values (Eq. \ref{eq:VILeq}). \begin{equation}
  L_i = \sum_{i=1}^{n-1}\sqrt{\left ( \text{SDR}_{i+1} - \text{SDR}_{i}\right )^2 + \left ( t_{i+1} - t_i \right )^2}
  \label{eq:VILeq}
  \end{equation} \begin{equation}
  1 \times L_{\text{CSref},i} - 5 < L_i < 1.3 \times L_{\text{CSref},i} + 13
  \label{eq:VILcr}
  \end{equation} where: \(t_i\) is the time stamp of each SDR
  measurement.
\item
  Standard deviation \(\sigma_i\) of the slope (\(s_i\)) between the
  \(11\) sequential points, normalized by the mean
  \(\overline{\text{SDR}}_i\) (Eq.~\ref{eq:VCT1}). \begin{gather}
    \sigma_i = \frac{\sqrt{\frac{1}{n-1} \sum_{i=1}^{n-1} \left( s_i - \bar{s} \right)^2}}{\overline{\text{SDR}}_i} \label{eq:VCT1} \\
    s_i = \frac{\text{SDR}_{i+1} - \text{SDR}_{i}}{t_{i+1} - t_i},\;\;   \bar{s} = \frac{1}{n-1} \sum_{i=1}^{n-1} s_i,\;\;\forall i \in \left \{ 1, 2, \ldots, n-1 \right \}\;\;
  \end{gather} For this criterion, \(\sigma_i\) should be below a
  certain threshold (Eq.~\ref{eq:VCTcr}): \begin{equation}
    \sigma_i < \ensuremath{1.1\times 10^{-4}} \label{eq:VCTcr}
  \end{equation}
\item
  Maximum difference \(X_i\) between the change in measured irradiance
  and the change in clear sky irradiance over each measurement interval.
  \begin{gather}
    X_i = \max{\left \{ \left | x_i - x_{\text{CSref},i} \right | \right \}} \label{eq:VSM3} \\
    x_i = \text{SDR}_{i+1} - \text{SDR}_{i} \forall i \in \left \{ 1, 2, \ldots, n-1 \right \} \label{eq:VSM1} \\
    x_{\text{CSref},i} = \text{SDR}_{\text{CSref},i+1} - \text{SDR}_{\text{CSref},i} \forall i \in \left \{ 1, 2, \ldots, n-1 \right \} \label{eq:VSM2}
  \end{gather} For this criterion, \(X_i\) should be below a certain
  threshold (Eq.~\ref{eq:VSMcr}): \begin{equation}
    X_i < 7.5\,Wm^{-2} \label{eq:VSMcr}
  \end{equation}
\end{enumerate}

\DIFdelbegin \DIFdel{Due to the significant measurement uncertainty when the Sun is near the
horizon, we have excluded all measurements with SZA greater than
\(85^\circ\). Moreover, due to some obstructions around the site (hills
and buildings), we excluded data with azimuth angle between \(35^\circ\)
and \(120^\circ\) with SZA greater than \(80^\circ\). On the latter
instances, Sun is systematically not visible from the instrument's
location. To make the measurements comparable throughout the dataset, we
adjusted all one-minute radiometric values to the mean Sun - Earth
distance. Subsequently, we made all measurements relative to the Total
Solar Irradiance (TSI) at \(1\,\text{au}\), in order to compensate for
the Sun's intensity variability, using a time series of satellite TSI
observations. The TSI data we used are part of the ``NOAA Climate Data
Record of Total Solar Irradiance'' dataset \mbox{%DIFAUXCMD
\citep{Coddington2005}}\hskip0pt%DIFAUXCMD
. The
initial daily values of this dataset were interpolated to match the time
step of our measurements. }\DIFdelend In the final dataset \DIFdelbegin \DIFdel{\(84.2\%\) of the }\DIFdelend \DIFaddbegin \DIFadd{\(23.3\%\) of the 1-minute }\DIFaddend data were identified as
under clear-sky conditions and \DIFdelbegin \DIFdel{\(15.8\%\) }\DIFdelend \DIFaddbegin \DIFadd{\(43\%\) }\DIFaddend as under cloud-sky conditions.

\DIFaddbegin \hypertarget{aggregation-of-data-and-statistical-approach}{%
\subsection{Aggregation of data and statistical
approach}\label{aggregation-of-data-and-statistical-approach}}

\DIFaddend In order to investigate the SDR trends \DIFaddbegin \DIFadd{which are the main focus of the
study}\DIFaddend , we implemented an appropriate aggregation scheme to the
\DIFdelbegin \DIFdel{1-minute }\DIFdelend \DIFaddbegin \DIFadd{one-minute }\DIFaddend data to derive a series in coarser time-scales. To preserve
the representativeness of the data we used the following criteria: a) \DIFdelbegin \DIFdel{for the daily mean values we
accept days with more }\DIFdelend \DIFaddbegin \DIFadd{we
excluded all days with less }\DIFaddend than 50\% of the \DIFdelbegin \DIFdel{daytime measurements present and valid}\DIFdelend \DIFaddbegin \DIFadd{expected daytime
measurements}\DIFaddend , b) \DIFdelbegin \DIFdel{monthly
values }\DIFdelend \DIFaddbegin \DIFadd{daily means for the clear-sky and cloudy-sky datasets
were calculated only for days with more than 60\% of the expected
daytime measurements identified as clear or cloudy respectively, c)
monthly means }\DIFaddend were computed from daily means only when at least 20 days
\DIFaddbegin \DIFadd{of the month }\DIFaddend were available. To create the daily and monthly
climatological means, we averaged the data based on the day of year and
calendar month, respectively. For the seasonal means we averaged the
mean daily values in each season (Winter: December - February, Spring:
March - May, etc.). Finally, each data set was deseasonalized by
subtracting the corresponding climatological annual cycle (daily or
monthly) from the actual data. \DIFdelbegin \DIFdel{To }\DIFdelend \DIFaddbegin \DIFadd{Finally, to }\DIFaddend estimate the SZA effect on
the SDR trends, the one-minute data were aggregated in \(1^\circ\) SZA
bins, separately for the morning and afternoon hours, and then were
deseasonalized as mentioned above.

\hypertarget{results}{%
\section{Results}\label{results}}

\hypertarget{long-term-sdr-trends}{%
\subsection{Long-term SDR trends}\label{long-term-sdr-trends}}

We calculated the linear \DIFdelbegin \DIFdel{SDR trends }\DIFdelend \DIFaddbegin \DIFadd{trends of SDR}\DIFaddend , from the departures of the mean
daily values from the daily climatology and for the three sky conditions
(Table~\ref{tab:trendtable}). In Figure~\ref{fig:trendALL} we present
only the time series under all-sky conditions; the plots for clear-sky
and cloud-sky conditions, are \DIFdelbegin \DIFdel{very similar and are }\DIFdelend shown in the Appendix
(Figures~\ref{fig:trendCLEAR} and~ \ref{fig:trendCLOUD}). \DIFaddbegin \DIFadd{In the studied
period, there is no significant break or change in the variability
pattern of the time series. }\DIFaddend We observe a positive trend for all-sky
conditions (\(0.38\,\%/y\)), a very close but smaller trend for
clear-skies (\DIFdelbegin \DIFdel{\(0.35\,\%/y\)}\DIFdelend \DIFaddbegin \DIFadd{\(0.097\,\%/y\)}\DIFaddend ) and a negative weaker trend for
cloudy-skies (\DIFdelbegin \DIFdel{\(-0.28\,\%/y\)}\DIFdelend \DIFaddbegin \DIFadd{\(0.41\,\%/y\)}\DIFaddend ). In the studied period, there is no
significant break or change in the variability of the time series. Other
studies for the European region reported a change of the SDR slope,
around 1980 \citep{Wild2021, Yuan2021, Ohmura2009}, a few years before
the start of our records. It is interesting to note, that for the
observations period, the trend of the TSI is \DIFdelbegin \DIFdel{\(-0.0002\,\%/y\)}\DIFdelend \DIFaddbegin \DIFadd{\(-0.00024\,\%/y\)}\DIFaddend , and
thus we can not attribute any major effect on SDR trend to Solar
variability.

\begin{table}[H]

\caption{\label{tab:trendtable}Trends in SDR daily means for different sky conditions for the period 1993 - 2023.}
\begin{tabu} to \linewidth {>{\centering\arraybackslash}p{8em}>{\raggedleft}X>{\raggedleft}X>{\raggedleft}X\DIFaddbeginFL \DIFaddFL{>}{\raggedleft}\DIFaddFL{X>}{\centering}\DIFaddFL{X>}{\raggedleft}\DIFaddFL{X>}{\raggedleft}\DIFaddFL{X>}{\raggedleft}\DIFaddFL{X>}{\raggedleft}\DIFaddFL{X>}{\centering}\DIFaddFL{X}\DIFaddendFL }
\toprule
\DIFdelbeginFL \DIFdelFL{Sky conditions }%DIFDELCMD < & %%%
\DIFdelendFL Trend [\%/year] & \DIFdelbeginFL \DIFdelFL{Statistical signif. }%DIFDELCMD < [%%%
\DIFdelFL{\%}%DIFDELCMD < ] %%%
\DIFdelendFL \DIFaddbeginFL \DIFaddFL{Rsqrd }& \DIFaddFL{Pearson cor. }& \DIFaddFL{Sky conditions }\DIFaddendFL & Days \DIFdelbeginFL \DIFdelFL{with data}\DIFdelendFL \DIFaddbeginFL & \DIFaddFL{t\_eff }& \DIFaddFL{t\_eff\_cri }& \DIFaddFL{conf\_2.5 }& \DIFaddFL{conf\_97.5 }& \DIFaddFL{mean\_clima }& \DIFaddFL{ChangeWpY}\DIFaddendFL \\
\midrule
\DIFaddbeginFL \DIFaddFL{0.3756 }& \DIFaddFL{0.00830 }& \DIFaddFL{0.0911 }& \DIFaddendFL All skies & \DIFdelbeginFL \DIFdelFL{0.376 }\DIFdelendFL \DIFaddbeginFL \DIFaddFL{10250 }\DIFaddendFL & \DIFdelbeginFL \DIFdelFL{100.00 }\DIFdelendFL \DIFaddbeginFL \DIFaddFL{4.727 }\DIFaddendFL & \DIFdelbeginFL \DIFdelFL{10256}\DIFdelendFL \DIFaddbeginFL \DIFaddFL{1.96 }& \DIFaddFL{0.000811 }& \DIFaddFL{0.001246 }& \DIFaddFL{387 }& \DIFaddFL{1.452}\DIFaddendFL \\
\DIFaddbeginFL \DIFaddFL{0.0972 }& \DIFaddFL{0.01912 }& \DIFaddFL{0.1383 }& \DIFaddendFL Clear skies & \DIFdelbeginFL \DIFdelFL{0.349 }\DIFdelendFL \DIFaddbeginFL \DIFaddFL{2684 }\DIFaddendFL & \DIFdelbeginFL \DIFdelFL{100.00 }\DIFdelendFL \DIFaddbeginFL \DIFaddFL{0.476 }\DIFaddendFL & \DIFdelbeginFL \DIFdelFL{10256}\DIFdelendFL \DIFaddbeginFL \DIFaddFL{1.96 }& \DIFaddFL{0.000194 }& \DIFaddFL{0.000338 }& \DIFaddFL{515 }& \DIFaddFL{0.501}\DIFaddendFL \\
\DIFaddbeginFL \DIFaddFL{0.4077 }& \DIFaddFL{0.00657 }& \DIFaddFL{0.0810 }& \DIFaddendFL Cloudy skies & \DIFdelbeginFL \DIFdelFL{-0.276 }\DIFdelendFL \DIFaddbeginFL \DIFaddFL{4937 }\DIFaddendFL & \DIFdelbeginFL \DIFdelFL{99.97 }\DIFdelendFL \DIFaddbeginFL \DIFaddFL{3.566 }\DIFaddendFL & \DIFdelbeginFL \DIFdelFL{5067}\DIFdelendFL \DIFaddbeginFL \DIFaddFL{1.96 }& \DIFaddFL{0.000733 }& \DIFaddFL{0.001499 }& \DIFaddFL{290 }& \DIFaddFL{1.181}\DIFaddendFL \\
\bottomrule
\end{tabu}
\end{table}

\begin{figure}[h!]

{\centering \includegraphics[width=.70\linewidth]{./images/LongtermTrends-2} 

}

\caption{Anomalies (\%) of the daily all-sky SDR \DIFdelbeginFL \DIFdelFL{, relative to }\DIFdelendFL \DIFaddbeginFL \DIFaddFL{from the }\DIFaddendFL climatological \DIFdelbeginFL \DIFdelFL{values }\DIFdelendFL \DIFaddbeginFL \DIFaddFL{mean }\DIFaddendFL for \DIFdelbeginFL \DIFdelFL{1993 }\DIFdelendFL \DIFaddbeginFL \DIFaddFL{the period1993 }\DIFaddendFL - 2023. The black line \DIFdelbeginFL \DIFdelFL{shows }\DIFdelendFL \DIFaddbeginFL \DIFaddFL{is }\DIFaddendFL the long term linear trend.}\label{fig:trendALL}
\end{figure}

Although the year-to-year variability of the anomalies (Figure
\ref{fig:trendALL} and Figures \ref{fig:trendCLEAR},
\ref{fig:trendCLOUD} in Appendix), shows a rather homogeneous behaviour,
plots of the cumulative sums (CUSUM) \citep{Regier2019} of the anomalies
can reveal different structures in the records of all three sky
conditions. In the cases of all-sky and clear-sky conditions (Figures
\ref{fig:cusummonth-1} and \ref{fig:cusummonth-2}), we observe three
macroscopic periods. A downward part from the start until about 2005, a
relatively steady part until about 2016 and, finally, a steep upward
part until the present. For cloud-sky (Figure~\ref{fig:cusummonth-3}),
we have a different pattern; it begins with a relatively steady part
until 1997, followed by an upward part until 2005, and a long decline
until 2020, with a small positive slope until the present. For a uniform
trend, we would expect the CUSUMs of the anomalies to have a symmetric
`V' shape. This would indicate that the anomalies are evenly distributed
around the climatological mean, and for a positive uniform trend, the
first half to be below and the other half above the climatological mean.
In our case, there is a more complex evolution of the anomalies. Another
distinct feature of the CUSUMs, is the different pattern of the
cloudy-sky dataset which peaks around the middle of the period.
Although, there seems to exist a complementary relation to the CUSUMs of
the clear- and all-sky cases, we can not assert that clouds are the main
driver for this relation due to the great difference in the number of
observational data between the two datasets
(Table~\ref{tab:trendtable}).

\begin{figure}[h!]
    \begin{adjustwidth}{-\extralength}{0cm}
        {\centering 
        \subfloat[All skies.\label{fig:cusummonth-1}]
            {\includegraphics[width=.32\linewidth]{./images/CumulativeMonthlyCuSum-1}}\hfill
        \subfloat[Clear skies.\label{fig:cusummonth-2}]
            {\includegraphics[width=.32\linewidth]{./images/CumulativeMonthlyCuSum-5}}\hfill
        \subfloat[Cloudy skies.\label{fig:cusummonth-3}]
            {\includegraphics[width=.32\linewidth]{./images/CumulativeMonthlyCuSum-9}}\hfill
        }
\caption{Cumulative sum plots of the monthly SDR anomalies in (\%) for different sky conditions.}\label{fig:cusummonth}
\end{adjustwidth}
\end{figure}

In order to investigate further the features of the CUSUMs, we created
another set of CUSUM plots by subtracting the corresponding long term
trend from the SDR anomaly data, prior to the CUSUM calculation
(Figure~\ref{fig:cusumnotrendmonthly}). With this approach periods when
the CUSUMs diverge from zero can be interpreted as a systematic
variation of SDR from the climatological mean. When the CUSUM is
increasing, the added values are above the climatological values of the
SDR trend and vice versa. Overall, for all- and clear-sky conditions
(Figures~\ref{fig:cusumnotrendmonthly-1}
and~\ref{fig:cusumnotrendmonthly-2}) we observe periods when the
anomalies diverge from the climatological value, each lasting for
several years. The pattern in both datasets is very similar, suggesting
prevalence in clear skies over Thessaloniki. It is interesting that in
the period 1993 - 2016 the anomalies have a high variability around
zero, while after 2016, the range of the variability is decreased to
about one third of the prior period. For cloudy-sky conditions
(Figure~\ref{fig:cusumnotrendmonthly-3}) the period 1997 - 2008 is
dominated by positive CUSUMs, suggesting a reduced effect of clouds on
SDR. From 1997 to mid-2000s CUSUMs are increasing, likely due to a
continuous decrease in the optical thickness of clouds, followed by a
period of rapid increase (within 3 years) in cloud optical thickness
lasting up to 2008. The following stable period spans for about 15 years
up to 2021 when CUSUMs start increasing again.

\begin{figure}[h!]
    \begin{adjustwidth}{-\extralength}{0cm}
        {\centering 
            \subfloat[All skies.\label{fig:cusumnotrendmonthly-1}]
                {\includegraphics[width=.32\linewidth]{./images/CumulativeMonthlyCuSumNOtrend-1} }\hfill
            \subfloat[Clear skies.\label{fig:cusumnotrendmonthly-2}]
                {\includegraphics[width=.32\linewidth]{./images/CumulativeMonthlyCuSumNOtrend-5} }\hfill
            \subfloat[Cloudy skies.\label{fig:cusumnotrendmonthly-3}]
                {\includegraphics[width=.32\linewidth]{./images/CumulativeMonthlyCuSumNOtrend-9} }
        }
        \caption{Cumulative sum plots of monthly SDR anomalies in (\%) for different sky conditions after removing the long-term linear trend.}\label{fig:cusumnotrendmonthly}
\end{adjustwidth}
\end{figure}

\DIFdelbegin %DIFDELCMD < \hypertarget{effects-of-the-solar-zenith-angle-on-sdr.}{%
%DIFDELCMD < \subsection{Effects of the solar zenith angle on
%DIFDELCMD < SDR.}\label{effects-of-the-solar-zenith-angle-on-sdr.}}
%DIFDELCMD < %%%
\DIFdelend \DIFaddbegin \hypertarget{effects-of-the-solar-zenith-angle-on-sdr}{%
\subsection{Effects of the solar zenith angle on
SDR}\label{effects-of-the-solar-zenith-angle-on-sdr}}
\DIFaddend 

The solar zenith angle is a major factor \DIFdelbegin \DIFdel{of SDR reaching the ground, due
to the }\DIFdelend \DIFaddbegin \DIFadd{affecting the SDR, since
increases in SZA leads to }\DIFaddend enhancement of the radiation path in the
atmosphere, especially in urban environments with human activities
emitting aerosols \citep{Wang2021}. In order to estimate the effect of
\DIFdelbegin \DIFdel{the }\DIFdelend SZA on the SDR trends, we grouped the \DIFdelbegin \DIFdel{anomaly }\DIFdelend data in bins of \(1^\circ\) SZA,
and calculated the overall trend for each bin\DIFaddbegin \DIFadd{, separately for the daily
periods }\DIFaddend before noon and after noon (Figure~\ref{fig:szatrends}).
Although there are seasonal dependencies of the minimum SZA (see
Appendix, Figure~\ref{fig:SZAtrendSeason}), these dependencies \DIFdelbegin \DIFdel{would not
be furtherexamined here.
}\DIFdelend \DIFaddbegin \DIFadd{are not
discussed further.
}

\DIFaddend For all-sky \DIFdelbegin \DIFdel{and
clear-sky }\DIFdelend conditions the brightening effect of SDR (positive trend) is
stronger for large SZAs (Figures~\ref{fig:szatrends-1} and
\ref{fig:szatrends-2}) \DIFaddbegin \DIFadd{ranging from about \(0.1\,\%/y\) to about
\(0.7\,\%/y\)}\DIFaddend . The trends in the morning and afternoon hours are more or
less consistent with small differences, which can be attributed to
systematic diurnal variations of aerosols, particularly during the warm
period of the year \citep{Wang2021}.
\DIFaddbegin 

\DIFaddend For cloudy-sky conditions (Figure~\ref{fig:szatrends-3}), we can not
discern any significant dependence of the SDR trend with SZA. For SZAs
\(16^\circ\) - \(50^\circ\), the trends range within about
\(\pm 0.2\,\%/y\), with a weak statistical significance. Between
\(50^\circ\) and \(75^\circ\) SZA the trends for the period before noon
are stronger and negative, possibly associated with stronger attenuation
by clouds under oblique incidence angles.

\begin{figure}[h!]
    \begin{adjustwidth}{-\extralength}{0cm}
        {\centering 
            \subfloat[All skies.\label{fig:szatrends-1}]
                {\includegraphics[width=.32\linewidth]{./images/SzaTrends-1}}\hfill
            \subfloat[Clear skies.\label{fig:szatrends-2}]
                {\includegraphics[width=.32\linewidth]{./images/SzaTrends-4}}\hfill
            \subfloat[Cloudy skies.\label{fig:szatrends-3}]
                {\includegraphics[width=.32\linewidth]{./images/SzaTrends-7}}
        }
        \caption{Long term trends of SDR as a function of SZA separately \DIFdelbeginFL \DIFdelFL{form }\DIFdelendFL \DIFaddbeginFL \DIFaddFL{from }\DIFaddendFL morning and afternoon periods. Solid shapes  represent statistically significant trends ($p < 0.005$).}\label{fig:szatrends}
    \end{adjustwidth}
\end{figure}

\hypertarget{long-term-trends-by-season}{%
\subsection{Long term trends by
season}\label{long-term-trends-by-season}}

Similarly to the long term trends \DIFaddbegin \DIFadd{from daily means of SDR }\DIFaddend discussed
above, we have calculated the trend of the anomalies for the three
different sky conditions, and for each season of the year, using the
corresponding mean monthly values (Figure~\ref{fig:seasonalALL} and
Table~\ref{tab:trendseasontable}). For all-sky conditions the trend in
SDR in winter is the largest (\(0.69\,\%/y\)), followed by the trend in
autumn (\(0.43\,\%/y\), a value close to the long term trend) both
statistically significant above the \(99\,\%\) confidence level. In
spring and summer, the trends are much smaller and of lesser statistical
significance. These seasonal differences indicate a possible relation of
the trends in SDR to trends of clouds during winter and autumn. For
clear-skies, the trend in winter is \DIFdelbegin \DIFdel{\(0.83\,\%/y\)}\DIFdelend \DIFaddbegin \DIFadd{\(0.36\,\%/y\)}\DIFaddend , larger than for
all-skies (\(0.69\,\%/y\)), which is another indication of a decreasing
trend in cloud optical thickness. Moreover, the trends under clear- and
cloudy-sky conditions are almost complementary to each other,
particularly for winter and autumn, where the signal is stronger. During
spring and summer the statistical significance is very low and the
actual trend too small for a meaningful comparison.

\begin{figure}[h!]
    \begin{adjustwidth}{-\extralength}{0cm}
        {\centering 
            \includegraphics[width=1\linewidth]{./images/SeasonalTrendsTogether3-2} 
        }
        \caption{Linear trends (black lines) of monthly mean anomalies of SDR by season (rows of plots) for the three sky conditions (columns of plots).}\label{fig:seasonalALL}
    \end{adjustwidth}
\end{figure}

\begin{table}[!h]

\caption{\label{tab:trendseasontable}SDR linear trends of monthly anomalies for each season of the year.}
\begin{tabu} to \linewidth {>{\centering\arraybackslash}p{8em}>{\centering}X>{\raggedleft}X>{\raggedleft}X\DIFaddbeginFL \DIFaddFL{>}{\centering}\DIFaddFL{X>}{\centering}\DIFaddFL{X>}{\raggedleft}\DIFaddFL{X>}{\raggedleft}\DIFaddFL{X>}{\centering}\DIFaddFL{X>}{\centering}\DIFaddFL{X>}{\raggedleft}\DIFaddFL{X}\DIFaddendFL }
\toprule
\DIFdelbeginFL \DIFdelFL{Sky condition }%DIFDELCMD < & %%%
\DIFdelFL{Season }%DIFDELCMD < & %%%
\DIFdelendFL Trend [\%/year] & \DIFdelbeginFL \DIFdelFL{Statistical signif.}%DIFDELCMD < [%%%
\DIFdelFL{\%}%DIFDELCMD < ]%%%
\DIFdelendFL \DIFaddbeginFL \DIFaddFL{slope.p }& \DIFaddFL{Rsqrd }& \DIFaddFL{cor.estimate }& \DIFaddFL{Sky condition }& \DIFaddFL{Season }& \DIFaddFL{N\_eff }& \DIFaddFL{t\_eff }& \DIFaddFL{t\_eff\_cri }& \DIFaddFL{conf\_2.5 }& \DIFaddFL{conf\_97.5}\DIFaddendFL \\
\midrule
\DIFdelbeginFL %DIFDELCMD < \cellcolor{gray!6}{} %%%
\DIFdelendFL \DIFaddbeginFL \cellcolor{gray!6}{0.6870} & \cellcolor{gray!6}{0.001330} & \cellcolor{gray!6}{0.31200} & \cellcolor{gray!6}{0.5587342} & \cellcolor{gray!6}{All skies} \DIFaddendFL & \cellcolor{gray!6}{Winter} & \DIFdelbeginFL %DIFDELCMD < \cellcolor{gray!6}{0.6860} %%%
\DIFdelendFL \DIFaddbeginFL \cellcolor{gray!6}{24.08252} \DIFaddendFL & \DIFdelbeginFL %DIFDELCMD < \cellcolor{gray!6}{99.9}%%%
\DIFdelendFL \DIFaddbeginFL \cellcolor{gray!6}{1e-07} & \cellcolor{gray!6}{2.063524} & \cellcolor{gray!6}{-0.0011} & \cellcolor{gray!6}{0.0015253}\DIFaddendFL \\
\DIFdelbeginFL %DIFDELCMD < 

%DIFDELCMD <  %%%
\DIFdelendFL \DIFaddbeginFL \cmidrule{1-11}
\DIFaddFL{0.1360 }& \DIFaddFL{0.219000 }& \DIFaddFL{0.05150 }& \DIFaddFL{0.2270307 }& \DIFaddFL{All skies }\DIFaddendFL & Spring & \DIFdelbeginFL \DIFdelFL{0.1450 }\DIFdelendFL \DIFaddbeginFL \DIFaddFL{44.44554 }\DIFaddendFL & \DIFdelbeginFL \DIFdelFL{81.8}\DIFdelendFL \DIFaddbeginFL \DIFaddFL{1e-07 }& \DIFaddFL{2.014797 }& \DIFaddFL{-0.0011 }& \DIFaddFL{0.0015253}\DIFaddendFL \\
\DIFdelbeginFL %DIFDELCMD < 

%DIFDELCMD < \cellcolor{gray!6}{} %%%
\DIFdelendFL \DIFaddbeginFL \cmidrule{1-11}
\cellcolor{gray!6}{0.1230} & \cellcolor{gray!6}{0.096100} & \cellcolor{gray!6}{0.09580} & \cellcolor{gray!6}{0.3094390} & \cellcolor{gray!6}{All skies} \DIFaddendFL & \cellcolor{gray!6}{Summer} & \DIFdelbeginFL %DIFDELCMD < \cellcolor{gray!6}{0.1200} %%%
\DIFdelendFL \DIFaddbeginFL \cellcolor{gray!6}{23.37502} \DIFaddendFL & \DIFdelbeginFL %DIFDELCMD < \cellcolor{gray!6}{89.4}%%%
\DIFdelendFL \DIFaddbeginFL \cellcolor{gray!6}{0e+00} & \cellcolor{gray!6}{2.066823} & \cellcolor{gray!6}{-0.0011} & \cellcolor{gray!6}{0.0015253}\DIFaddendFL \\
\DIFdelbeginFL %DIFDELCMD < 

%DIFDELCMD < \multirow{-4}{*}{\centering\arraybackslash All skies} %%%
\DIFdelendFL \DIFaddbeginFL \cmidrule{1-11}
\DIFaddFL{0.4300 }& \DIFaddFL{0.006670 }& \DIFaddFL{0.23500 }& \DIFaddFL{0.4844650 }& \DIFaddFL{All skies }\DIFaddendFL & Autumn & \DIFdelbeginFL \DIFdelFL{0.4310 }\DIFdelendFL \DIFaddbeginFL \DIFaddFL{21.43236 }\DIFaddendFL & \DIFdelbeginFL \DIFdelFL{99.3}\DIFdelendFL \DIFaddbeginFL \DIFaddFL{1e-07 }& \DIFaddFL{2.077062 }& \DIFaddFL{-0.0011 }& \DIFaddFL{0.0015253}\DIFaddendFL \\
\DIFdelbeginFL %DIFDELCMD < \cmidrule{1-4}
%DIFDELCMD < \cellcolor{gray!6}{} %%%
\DIFdelendFL \DIFaddbeginFL \cmidrule{1-11}
\cellcolor{gray!6}{0.3570} & \cellcolor{gray!6}{0.007260} & \cellcolor{gray!6}{0.23000} & \cellcolor{gray!6}{0.4800278} & \cellcolor{gray!6}{Clear skies} \DIFaddendFL & \cellcolor{gray!6}{Winter} & \DIFdelbeginFL %DIFDELCMD < \cellcolor{gray!6}{0.8260} %%%
\DIFdelendFL \DIFaddbeginFL \cellcolor{gray!6}{18.55573} \DIFaddendFL & \DIFdelbeginFL %DIFDELCMD < \cellcolor{gray!6}{100.0}%%%
\DIFdelendFL \DIFaddbeginFL \cellcolor{gray!6}{0e+00} & \cellcolor{gray!6}{2.096421} & \cellcolor{gray!6}{-0.0011} & \cellcolor{gray!6}{0.0015253}\DIFaddendFL \\
\DIFdelbeginFL %DIFDELCMD < 

%DIFDELCMD <  %%%
\DIFdelendFL \DIFaddbeginFL \cmidrule{1-11}
\DIFaddFL{0.0317 }& \DIFaddFL{0.721000 }& \DIFaddFL{0.00445 }& \DIFaddFL{0.0667338 }& \DIFaddFL{Clear skies }\DIFaddendFL & Spring & \DIFdelbeginFL \DIFdelFL{0.0613 }\DIFdelendFL \DIFaddbeginFL \DIFaddFL{21.33345 }\DIFaddendFL & \DIFdelbeginFL \DIFdelFL{38.6}\DIFdelendFL \DIFaddbeginFL \DIFaddFL{0e+00 }& \DIFaddFL{2.077636 }& \DIFaddFL{-0.0011 }& \DIFaddFL{0.0015253}\DIFaddendFL \\
\DIFdelbeginFL %DIFDELCMD < 

%DIFDELCMD < \cellcolor{gray!6}{} %%%
\DIFdelendFL \DIFaddbeginFL \cmidrule{1-11}
\cellcolor{gray!6}{-0.0931} & \cellcolor{gray!6}{0.007770} & \cellcolor{gray!6}{0.22700} & \cellcolor{gray!6}{-0.4764646} & \cellcolor{gray!6}{Clear skies} \DIFaddendFL & \cellcolor{gray!6}{Summer} & \DIFdelbeginFL %DIFDELCMD < \cellcolor{gray!6}{-0.0307} %%%
\DIFdelendFL \DIFaddbeginFL \cellcolor{gray!6}{25.12298} \DIFaddendFL & \DIFdelbeginFL %DIFDELCMD < \cellcolor{gray!6}{25.9}%%%
\DIFdelendFL \DIFaddbeginFL \cellcolor{gray!6}{0e+00} & \cellcolor{gray!6}{2.059027} & \cellcolor{gray!6}{-0.0011} & \cellcolor{gray!6}{0.0015253}\DIFaddendFL \\
\DIFdelbeginFL %DIFDELCMD < 

%DIFDELCMD < \multirow{-4}{*}{\centering\arraybackslash Clear skies} %%%
\DIFdelendFL \DIFaddbeginFL \cmidrule{1-11}
\DIFaddFL{0.1100 }& \DIFaddFL{0.308000 }& \DIFaddFL{0.03710 }& \DIFaddFL{0.1926085 }& \DIFaddFL{Clear skies }\DIFaddendFL & Autumn & \DIFdelbeginFL \DIFdelFL{0.3670 }\DIFdelendFL \DIFaddbeginFL \DIFaddFL{23.96832 }\DIFaddendFL & \DIFdelbeginFL \DIFdelFL{97.2}\DIFdelendFL \DIFaddbeginFL \DIFaddFL{0e+00 }& \DIFaddFL{2.064043 }& \DIFaddFL{-0.0011 }& \DIFaddFL{0.0015253}\DIFaddendFL \\
\DIFdelbeginFL %DIFDELCMD < \cmidrule{1-4}
%DIFDELCMD < \cellcolor{gray!6}{} %%%
\DIFdelendFL \DIFaddbeginFL \cmidrule{1-11}
\cellcolor{gray!6}{0.8250} & \cellcolor{gray!6}{0.000123} & \cellcolor{gray!6}{0.41500} & \cellcolor{gray!6}{0.6439160} & \cellcolor{gray!6}{Cloudy skies} \DIFaddendFL & \cellcolor{gray!6}{Winter} & \DIFdelbeginFL %DIFDELCMD < \cellcolor{gray!6}{-0.8820} %%%
\DIFdelendFL \DIFaddbeginFL \cellcolor{gray!6}{14.68318} \DIFaddendFL & \DIFdelbeginFL %DIFDELCMD < \cellcolor{gray!6}{98.9}%%%
\DIFdelendFL \DIFaddbeginFL \cellcolor{gray!6}{1e-07} & \cellcolor{gray!6}{2.135462} & \cellcolor{gray!6}{-0.0011} & \cellcolor{gray!6}{0.0015253}\DIFaddendFL \\
\DIFdelbeginFL %DIFDELCMD < 

%DIFDELCMD <  %%%
\DIFdelendFL \DIFaddbeginFL \cmidrule{1-11}
\DIFaddFL{0.0942 }& \DIFaddFL{0.425000 }& \DIFaddFL{0.02210 }& \DIFaddFL{0.1485860 }& \DIFaddFL{Cloudy skies }\DIFaddendFL & Spring & \DIFdelbeginFL \DIFdelFL{-0.0991 }\DIFdelendFL \DIFaddbeginFL \DIFaddFL{35.40650 }\DIFaddendFL & \DIFdelbeginFL \DIFdelFL{37.2}\DIFdelendFL \DIFaddbeginFL \DIFaddFL{1e-07 }& \DIFaddFL{2.029275 }& \DIFaddFL{-0.0011 }& \DIFaddFL{0.0015253}\DIFaddendFL \\
\DIFdelbeginFL %DIFDELCMD < 

%DIFDELCMD < \cellcolor{gray!6}{} %%%
\DIFdelendFL \DIFaddbeginFL \cmidrule{1-11}
\cellcolor{gray!6}{-0.0983} & \cellcolor{gray!6}{0.414000} & \cellcolor{gray!6}{0.02400} & \cellcolor{gray!6}{-0.1548507} & \cellcolor{gray!6}{Cloudy skies} \DIFaddendFL & \cellcolor{gray!6}{Summer} & \DIFdelbeginFL %DIFDELCMD < \cellcolor{gray!6}{0.0444} %%%
\DIFdelendFL \DIFaddbeginFL \cellcolor{gray!6}{37.66778} \DIFaddendFL & \DIFdelbeginFL %DIFDELCMD < \cellcolor{gray!6}{21.5}%%%
\DIFdelendFL \DIFaddbeginFL \cellcolor{gray!6}{1e-07} & \cellcolor{gray!6}{2.024981} & \cellcolor{gray!6}{-0.0011} & \cellcolor{gray!6}{0.0015253}\DIFaddendFL \\
\DIFdelbeginFL %DIFDELCMD < 

%DIFDELCMD < \multirow{-4}{*}{\centering\arraybackslash Cloudy skies} %%%
\DIFdelendFL \DIFaddbeginFL \cmidrule{1-11}
\DIFaddFL{0.1950 }& \DIFaddFL{0.460000 }& \DIFaddFL{0.01960 }& \DIFaddFL{0.1401398 }& \DIFaddFL{Cloudy skies }\DIFaddendFL & Autumn & \DIFdelbeginFL \DIFdelFL{-0.4000 }\DIFdelendFL \DIFaddbeginFL \DIFaddFL{28.41812 }\DIFaddendFL & \DIFdelbeginFL \DIFdelFL{89.5}\DIFdelendFL \DIFaddbeginFL \DIFaddFL{2e-07 }& \DIFaddFL{2.047050 }& \DIFaddFL{-0.0011 }& \DIFaddFL{0.0015253}\DIFaddendFL \\
\bottomrule
\end{tabu}
\end{table}

\hypertarget{conclusions}{%
\section{Conclusions}\label{conclusions}}

We have demonstrated that in the period 1993 - 2023, there is a positive
trend in SDR of (\(0.38\,\%/y\)) (brightening) (positive trend) in
Thessaloniki, Greece, under all-sky conditions. A previous study
\citep{Bais2013} for the period 1993 - 2011 found also a positive trend
of \(0.33\,\%/y\). The increase of this trend indicates that the
brightening of SDR continues and is probably caused by continuing
decreases in aerosol optical depth and the optical thickness of clouds
over the area. Moreover, we found a similar trend under clear-sky
conditions (\DIFdelbegin \DIFdel{\(0.35\,\%/y\)}\DIFdelend \DIFaddbegin \DIFadd{\(0.097\,\%/y\)}\DIFaddend ) that further supports the assumption that
the brightening is caused mainly by decreasing aerosols. Unfortunately,
for the entire period there is no available data for the aerosols, in
order to quantify their effect on SDR. However, \citet{Siomos2020} have
shown that aerosol optical depth over Thessaloniki is decreasing
constantly at least up to 2018. The attenuation of SDR by aerosols over
Europe have been proposed as major factor by \citet{Wild2021}. The
dimming effect on SDR under cloudy-sky conditions (\DIFdelbegin \DIFdel{\(-0.28\,\%/y\)}\DIFdelend \DIFaddbegin \DIFadd{\(0.41\,\%/y\)}\DIFaddend ),
suggests that cloud optical thickness is decreasing during this period.
Because we have no adequate data to investigate the long term changes of
cloud thickness in the region, we cannot verify if the negative SDR
trend we observe under under cloudy-skies can be attributed solely to
changes in clouds.

The observed brightening on SDR over Thessaloniki is dependent on SZA
(larger SZAs lead to stronger brightening). The trend is also dependent
on season, with winter showing the strongest statistically significant
trend of \(0.69\) and \DIFdelbegin \DIFdel{\(0.83\,\%/y\) }\DIFdelend \DIFaddbegin \DIFadd{\(0.36\,\%/y\) }\DIFaddend for all- and clear-skies,
respectively, in contrast to spring and summer. The trends for autumn
are also significant but smaller ( \(0.43\) and \DIFdelbegin \DIFdel{\(0.37\,\%/y\) }\DIFdelend \DIFaddbegin \DIFadd{\(0.11\,\%/y\) }\DIFaddend for all-
and clear-skies, respectively). Our findings are in agreement with other
studies for the region.

Using the CUSUMs of the monthly departures for all- and clear-skies, we
observed periods where the CUSUMs remain relatively stable, with a steep
decline before and a steep increase after. This is an indication that
the whole brightening effect does not follow a smooth development over
time.

Continued observations with a collocated pyrheliometer, which started in
2016, will allow us to further investigate the variability of solar
radiation at ground level in Thessaloniki. Also, additional data of
cloudiness, aerosols, atmospheric water vapour, etc., will allow better
attribution and quantification of the effects of these factors on SRD.

%%%%%%%%%%%%%%%%%%%%%%%%%%%%%%%%%%%%%%%%%%

\vspace{6pt}

%%%%%%%%%%%%%%%%%%%%%%%%%%%%%%%%%%%%%%%%%%
%% optional

% Only for the journal Methods and Protocols:
% If you wish to submit a video article, please do so with any other supplementary material.
% \supplementary{The following supporting information can be downloaded at: \linksupplementary{s1}, Figure S1: title; Table S1: title; Video S1: title. A supporting video article is available at doi: link.}

%%%%%%%%%%%%%%%%%%%%%%%%%%%%%%%%%%%%%%%%%%

\funding{This research received no external funding.}



\dataavailability{Data as daily sums are available through the WRDC
database \url{http://wrdc.mgo.rssi.ru}. One minute data are available on
request from the corresponding author. The data are not publicly
available for protection against unmonitored commercial use.}



%%%%%%%%%%%%%%%%%%%%%%%%%%%%%%%%%%%%%%%%%%
%% Optional

%% Only for journal Encyclopedia

\abbreviations{Abbreviations}{
The following abbreviations are used in this manuscript:\\

\noindent
\begin{tabular}{@{}ll}
DNI & Direct beam/normal irradiance \\
CSid & Clear sky identification algorithm \\
CUSUM & Cumulative sum \\
SDR & Solar downward radiation \\
SZA & Solar zenith angle \\
\end{tabular}}

%%%%%%%%%%%%%%%%%%%%%%%%%%%%%%%%%%%%%%%%%%
%% Optional
\input{"appendix.tex"}
%%%%%%%%%%%%%%%%%%%%%%%%%%%%%%%%%%%%%%%%%%
\begin{adjustwidth}{-\extralength}{0cm}

%\printendnotes[custom] % Un-comment to print a list of endnotes


\reftitle{References}
\bibliography{manualreferences.bib}

% If authors have biography, please use the format below
%\section*{Short Biography of Authors}
%\bio
%{\raisebox{-0.35cm}{\includegraphics[width=3.5cm,height=5.3cm,clip,keepaspectratio]{Definitions/author1.pdf}}}
%{\textbf{Firstname Lastname} Biography of first author}
%
%\bio
%{\raisebox{-0.35cm}{\includegraphics[width=3.5cm,height=5.3cm,clip,keepaspectratio]{Definitions/author2.jpg}}}
%{\textbf{Firstname Lastname} Biography of second author}

%%%%%%%%%%%%%%%%%%%%%%%%%%%%%%%%%%%%%%%%%%
%% for journal Sci
%\reviewreports{\\
%Reviewer 1 comments and authors’ response\\
%Reviewer 2 comments and authors’ response\\
%Reviewer 3 comments and authors’ response
%}
%%%%%%%%%%%%%%%%%%%%%%%%%%%%%%%%%%%%%%%%%%
\PublishersNote{}
\end{adjustwidth}


\end{document}
