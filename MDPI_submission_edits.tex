%  LaTeX support: latex@mdpi.com
%DIF LATEXDIFF DIFFERENCE FILE
%DIF DEL SUBMISSION_01/MDPI_submission.tex   Sun Nov 12 17:31:09 2023
%DIF ADD MDPI_submission.tex                 Wed Dec 27 12:26:23 2023
%  For support, please attach all files needed for compiling as well as the log file, and specify your operating system, LaTeX version, and LaTeX editor.

%=================================================================
% pandoc conditionals added to preserve backwards compatibility with previous versions of rticles

\documentclass[applsci,article,submit,moreauthors,pdftex]{Definitions/mdpi}


%% Some pieces required from the pandoc template
\setlist[itemize]{leftmargin=*,labelsep=5.8mm}
\setlist[enumerate]{leftmargin=*,labelsep=4.9mm}


%--------------------
% Class Options:
%--------------------

%---------
% article
%---------
% The default type of manuscript is "article", but can be replaced by:
% abstract, addendum, article, book, bookreview, briefreport, casereport, comment, commentary, communication, conferenceproceedings, correction, conferencereport, entry, expressionofconcern, extendedabstract, datadescriptor, editorial, essay, erratum, hypothesis, interestingimage, obituary, opinion, projectreport, reply, retraction, review, perspective, protocol, shortnote, studyprotocol, systematicreview, supfile, technicalnote, viewpoint, guidelines, registeredreport, tutorial
% supfile = supplementary materials

%----------
% submit
%----------
% The class option "submit" will be changed to "accept" by the Editorial Office when the paper is accepted. This will only make changes to the frontpage (e.g., the logo of the journal will get visible), the headings, and the copyright information. Also, line numbering will be removed. Journal info and pagination for accepted papers will also be assigned by the Editorial Office.

%------------------
% moreauthors
%------------------
% If there is only one author the class option oneauthor should be used. Otherwise use the class option moreauthors.

%---------
% pdftex
%---------
% The option pdftex is for use with pdfLaTeX. Remove "pdftex" for (1) compiling with LaTeX & dvi2pdf (if eps figures are used) or for (2) compiling with XeLaTeX.

%=================================================================
% MDPI internal commands - do not modify
\firstpage{1}
\makeatletter
\setcounter{page}{\@firstpage}
\makeatother
\pubvolume{1}
\issuenum{1}
\articlenumber{0}
\pubyear{2023}
\copyrightyear{2023}
%\externaleditor{Academic Editor: Firstname Lastname}
\datereceived{ }
\daterevised{ } % Comment out if no revised date
\dateaccepted{ }
\datepublished{ }
%\datecorrected{} % For corrected papers: "Corrected: XXX" date in the original paper.
%\dateretracted{} % For corrected papers: "Retracted: XXX" date in the original paper.
\hreflink{https://doi.org/} % If needed use \linebreak
%\doinum{}
%\pdfoutput=1 % Uncommented for upload to arXiv.org

%=================================================================
% Add packages and commands here. The following packages are loaded in our class file: fontenc, inputenc, calc, indentfirst, fancyhdr, graphicx, epstopdf, lastpage, ifthen, float, amsmath, amssymb, lineno, setspace, enumitem, mathpazo, booktabs, titlesec, etoolbox, tabto, xcolor, colortbl, soul, multirow, microtype, tikz, totcount, changepage, attrib, upgreek, array, tabularx, pbox, ragged2e, tocloft, marginnote, marginfix, enotez, amsthm, natbib, hyperref, cleveref, scrextend, url, geometry, newfloat, caption, draftwatermark, seqsplit
% cleveref: load \crefname definitions after \begin{document}

%=================================================================
% Please use the following mathematics environments: Theorem, Lemma, Corollary, Proposition, Characterization, Property, Problem, Example, ExamplesandDefinitions, Hypothesis, Remark, Definition, Notation, Assumption
%% For proofs, please use the proof environment (the amsthm package is loaded by the MDPI class).

%=================================================================
% Full title of the paper (Capitalized)
%DIF 73c73
%DIF < \Title{Trends from 30-year observations of downward solar irradiance in
%DIF -------
\Title{Trends from 30-Year Observations of Downward Solar Irradiance in %DIF > 
%DIF -------
Thessaloniki, Greece}

% MDPI internal command: Title for citation in the left column
%DIF 77-78c77-78
%DIF < \TitleCitation{Trends from 30-year observations of downward solar
%DIF < irradiance in Thessaloniki, Greece}
%DIF -------
\TitleCitation{Trends from 30-Year Observations of Downward Solar %DIF > 
Irradiance in Thessaloniki, Greece} %DIF > 
%DIF -------

% Author Orchid ID: enter ID or remove command
%\newcommand{\orcidauthorA}{0000-0000-0000-000X} % Add \orcidA{} behind the author's name
%\newcommand{\orcidauthorB}{0000-0000-0000-000X} % Add \orcidB{} behind the author's name


% Authors, for the paper (add full first names)
\Author{Athanasios
Natsis$^{1}$\href{https://orcid.org/0000-0002-5199-4119}
{\orcidicon}, Alkiviadis Bais$^{1,*}$, Charikleia Meleti$^{1}$}


%\longauthorlist{yes}


% MDPI internal command: Authors, for metadata in PDF
\AuthorNames{Athanasios Natsis, Alkiviadis Bais, Charikleia Meleti}

% MDPI internal command: Authors, for citation in the left column
%\AuthorCitation{Lastname, F.; Lastname, F.; Lastname, F.}
% If this is a Chicago style journal: Lastname, Firstname, Firstname Lastname, and Firstname Lastname.
\AuthorCitation{Natsis, A.; Bais, A.; Meleti, C.}

% Affiliations / Addresses (Add [1] after \address if there is only one affiliation.)
\address{%
$^{1}$ \quad Aristotle University of Thessaloniki - Laboratory of
Atmospheric Physics, Campus Box 149, 54124 Thessaloniki,
Greece; \href{mailto:natsisphysicist@gmail.com}{\nolinkurl{natsisphysicist@gmail.com}}
(A.N.); \href{mailto:abais@auth.gr}{\nolinkurl{abais@auth.gr}} (A.B.);
\href{mailto:meleti@auth.gr}{\nolinkurl{meleti@auth.gr}} (C.M.)\\
}

% Contact information of the corresponding author
\corres{Correspondence: \href{mailto:abais@auth.gr}{\nolinkurl{abais@auth.gr}}}

% Current address and/or shared authorship








% The commands \thirdnote{} till \eighthnote{} are available for further notes

% Simple summary

%\conference{} % An extended version of a conference paper

% Abstract (Do not insert blank lines, i.e. \\)
\abstract{The shortwave downward solar irradiance (SDR) is an important
%DIF 131-154c131-156
%DIF < factor that drives climate processes, production and can affect all
%DIF < living organisms. Observations of SDR at different locations around the
%DIF < world with different environmental characteristics have been used to
%DIF < investigate its long-term variability and trends at different time
%DIF < scales. Periods of positive trends are referred as brightening periods
%DIF < and of negative trends as dimming periods. Here we studied 30 years of
%DIF < pyranometer data in Thessaloniki, Greece, under three types of sky
%DIF < conditions (clear sky, cloudy sky and all sky). The clear-sky data were
%DIF < identified by applying a cloud screening algorithm. We found a positive
%DIF < trend of \(0.38\,\%/\text{year}\) for all-sky, \(0.35\,\%/\text{year}\)
%DIF < for clear-sky conditions, and \(-0.28\,\%/\text{year}\) for cloudy
%DIF < conditions. We have also investigated the consistency of these trends,
%DIF < their seasonal variability, and the effect of the solar zenith angle. We
%DIF < have found that for all-sky and clear-sky conditions the SDR trend is
%DIF < positive in winter (\(0.7\) and \(0.8\,\%/\text{year}\), respectively)
%DIF < and autumn (\(~0.4\,\%/\text{year}\)), while under cloudy skies the
%DIF < trend is negative (\(-0.9\,\%/\text{year}\) in winter and
%DIF < \(-0.4\,\%/\text{year}\) in autumn). In spring and summer the trend is
%DIF < very close to zero, irrespective of sky conditions. The SDR trend is
%DIF < increasing with increasing solar zenith angle, except under cloudy skies
%DIF < where the trend is highly variable and close to zero. Finally, we
%DIF < identified some anomalies in the long term SDR trends for all sky
%DIF < conditions by examining the cumulative sums of monthly anomalies from
%DIF < the climatological mean.}
%DIF -------
factor that drives climate processes and energy production and can %DIF > 
affect all living organisms. Observations of SDR at different locations %DIF > 
around the world with different environmental characteristics have been %DIF > 
used to investigate its long-term variability and trends at different %DIF > 
time scales. Periods of positive trends are referred to as brightening %DIF > 
periods and of negative trends as dimming periods. In this study we have %DIF > 
used 30 years of pyranometer data in Thessaloniki, Greece, to %DIF > 
investigate the variability of SDR under three types of sky conditions %DIF > 
(clear-, cloudy- and all-sky). The clear-sky data were identified by %DIF > 
applying a cloud screening algorithm. We have found a positive trend of %DIF > 
\(0.38\,\%/\text{year}\) for all-sky, \(\sim 0.1\,\%/\text{year}\) for %DIF > 
clear-sky, and \(0.41\,\%/\text{year}\) for cloudy conditions. The %DIF > 
consistency of these trends, their seasonal variability, and the effect %DIF > 
of the solar zenith angle have also been investigated. Under all three %DIF > 
sky categories, the SDR trend is stronger in winter, with \(0.7\), %DIF > 
\(0.4\), and \(0.76\,\%/\text{year}\), respectively, for all-, clear-, %DIF > 
and cloudy-sky conditions. The next larger seasonal trends are in autumn %DIF > 
\(0.42\) and \(0.19\,\%/\text{year}\), for all and cloudy skies, %DIF > 
respectively. The rest of the seasonal trends are significant smaller, %DIF > 
close to zero, with a negative values in summer, for clear and cloudy %DIF > 
skies. The SDR trend is increasing with increasing solar zenith angle, %DIF > 
except under cloudy skies, where the trend is highly variable and close %DIF > 
to zero. Finally, we discuss shorter-term variations in SDR anomalies by %DIF > 
examining the patterns of the cumulative sums of monthly anomalies from %DIF > 
the climatological mean, both before and after removing the long-term %DIF > 
trend.} %DIF > 
%DIF -------


% Keywords
%DIF 158-159c160-161
%DIF < \keyword{GHI; SDR; solar radiation; Solar Brigthening/Dimming; aerosols;
%DIF < clouds.}
%DIF -------
\keyword{GHI; SDR; solar radiation; solar brigthening/dimming; aerosols; %DIF > 
clouds} %DIF > 
%DIF -------

% The fields PACS, MSC, and JEL may be left empty or commented out if not applicable
%\PACS{J0101}
%\MSC{}
%\JEL{}

%%%%%%%%%%%%%%%%%%%%%%%%%%%%%%%%%%%%%%%%%%
% Only for the journal Diversity
%\LSID{\url{http://}}

%%%%%%%%%%%%%%%%%%%%%%%%%%%%%%%%%%%%%%%%%%
% Only for the journal Applied Sciences

%%%%%%%%%%%%%%%%%%%%%%%%%%%%%%%%%%%%%%%%%%

%%%%%%%%%%%%%%%%%%%%%%%%%%%%%%%%%%%%%%%%%%
% Only for the journal Data



%%%%%%%%%%%%%%%%%%%%%%%%%%%%%%%%%%%%%%%%%%
% Only for the journal Toxins


%%%%%%%%%%%%%%%%%%%%%%%%%%%%%%%%%%%%%%%%%%
% Only for the journal Encyclopedia


%%%%%%%%%%%%%%%%%%%%%%%%%%%%%%%%%%%%%%%%%%
% Only for the journal Advances in Respiratory Medicine
%\addhighlights{yes}
%\renewcommand{\addhighlights}{%

%\noindent This is an obligatory section in “Advances in Respiratory Medicine”, whose goal is to increase the discoverability and readability of the article via search engines and other scholars. Highlights should not be a copy of the abstract, but a simple text allowing the reader to quickly and simplified find out what the article is about and what can be cited from it. Each of these parts should be devoted up to 2~bullet points.\vspace{3pt}\\
%\textbf{What are the main findings?}
% \begin{itemize}[labelsep=2.5mm,topsep=-3pt]
% \item First bullet.
% \item Second bullet.
% \end{itemize}\vspace{3pt}
%\textbf{What is the implication of the main finding?}
% \begin{itemize}[labelsep=2.5mm,topsep=-3pt]
% \item First bullet.
% \item Second bullet.
% \end{itemize}
%}


%%%%%%%%%%%%%%%%%%%%%%%%%%%%%%%%%%%%%%%%%%


% tightlist command for lists without linebreak
\providecommand{\tightlist}{%
  \setlength{\itemsep}{0pt}\setlength{\parskip}{0pt}}



\usepackage{subcaption}
\captionsetup[sub]{position=bottom, labelfont={bf, small, stretch=1.17}, labelsep=space, textfont={small, stretch=1.17}, aboveskip=6pt,  belowskip=-6pt, singlelinecheck=off, justification=justified}
\usepackage{placeins}
\usepackage{longtable}
\usepackage{booktabs}
\usepackage{array}
\usepackage{multirow}
\usepackage{wrapfig}
\usepackage{float}
\usepackage{colortbl}
\usepackage{pdflscape}
\usepackage{tabu}
\usepackage{threeparttable}
\usepackage{threeparttablex}
\usepackage[normalem]{ulem}
\usepackage{makecell}
\usepackage{xcolor}
%DIF PREAMBLE EXTENSION ADDED BY LATEXDIFF
%DIF UNDERLINE PREAMBLE %DIF PREAMBLE
\RequirePackage[normalem]{ulem} %DIF PREAMBLE
\RequirePackage{color}\definecolor{RED}{rgb}{1,0,0}\definecolor{BLUE}{rgb}{0,0,1} %DIF PREAMBLE
\providecommand{\DIFadd}[1]{{\protect\color{blue}\uwave{#1}}} %DIF PREAMBLE
\providecommand{\DIFdel}[1]{{\protect\color{red}\sout{#1}}}                      %DIF PREAMBLE
%DIF SAFE PREAMBLE %DIF PREAMBLE
\providecommand{\DIFaddbegin}{} %DIF PREAMBLE
\providecommand{\DIFaddend}{} %DIF PREAMBLE
\providecommand{\DIFdelbegin}{} %DIF PREAMBLE
\providecommand{\DIFdelend}{} %DIF PREAMBLE
\providecommand{\DIFmodbegin}{} %DIF PREAMBLE
\providecommand{\DIFmodend}{} %DIF PREAMBLE
%DIF FLOATSAFE PREAMBLE %DIF PREAMBLE
\providecommand{\DIFaddFL}[1]{\DIFadd{#1}} %DIF PREAMBLE
\providecommand{\DIFdelFL}[1]{\DIFdel{#1}} %DIF PREAMBLE
\providecommand{\DIFaddbeginFL}{} %DIF PREAMBLE
\providecommand{\DIFaddendFL}{} %DIF PREAMBLE
\providecommand{\DIFdelbeginFL}{} %DIF PREAMBLE
\providecommand{\DIFdelendFL}{} %DIF PREAMBLE
%DIF COLORLISTINGS PREAMBLE %DIF PREAMBLE
\RequirePackage{listings} %DIF PREAMBLE
\RequirePackage{color} %DIF PREAMBLE
\lstdefinelanguage{DIFcode}{ %DIF PREAMBLE
%DIF DIFCODE_UNDERLINE %DIF PREAMBLE
  moredelim=[il][\color{red}\sout]{\%DIF\ <\ }, %DIF PREAMBLE
  moredelim=[il][\color{blue}\uwave]{\%DIF\ >\ } %DIF PREAMBLE
} %DIF PREAMBLE
\lstdefinestyle{DIFverbatimstyle}{ %DIF PREAMBLE
	language=DIFcode, %DIF PREAMBLE
	basicstyle=\ttfamily, %DIF PREAMBLE
	columns=fullflexible, %DIF PREAMBLE
	keepspaces=true %DIF PREAMBLE
} %DIF PREAMBLE
\lstnewenvironment{DIFverbatim}{\lstset{style=DIFverbatimstyle}}{} %DIF PREAMBLE
\lstnewenvironment{DIFverbatim*}{\lstset{style=DIFverbatimstyle,showspaces=true}}{} %DIF PREAMBLE
%DIF END PREAMBLE EXTENSION ADDED BY LATEXDIFF

\begin{document}



%%%%%%%%%%%%%%%%%%%%%%%%%%%%%%%%%%%%%%%%%%

\DIFdelbegin %DIFDELCMD < \hypertarget{introduction.}{%
%DIFDELCMD < \section{Introduction.}\label{introduction.}}
%DIFDELCMD < %%%
\DIFdelend \DIFaddbegin \hypertarget{introduction}{%
\section{Introduction}\label{introduction}}
\DIFaddend 

The shortwave downward solar irradiance (SDR) at Earth's surface plays a
significant role \DIFdelbegin \DIFdel{, }\DIFdelend on its climate. Changes \DIFdelbegin \DIFdel{of }\DIFdelend \DIFaddbegin \DIFadd{in }\DIFaddend the SDR can be related to
changes \DIFdelbegin \DIFdel{on }\DIFdelend \DIFaddbegin \DIFadd{in }\DIFaddend Earth's energy budget, the mechanisms of climate change, and
water and carbon cycles \citep{Wild2009}. It can also affect solar and
agricultural production \DIFdelbegin \DIFdel{, }\DIFdelend and all living organisms. Studies of SDR
variability \DIFdelbegin \DIFdel{, }\DIFdelend have identified some distinct SDR trends \DIFdelbegin \DIFdel{on }\DIFdelend \DIFaddbegin \DIFadd{in }\DIFaddend different
regions of the world \DIFdelbegin \DIFdel{on }\DIFdelend \DIFaddbegin \DIFadd{in }\DIFaddend different time periods. The term `brightening'
is generally used to describe periods of positive SDR trend, and
`dimming' for \DIFaddbegin \DIFadd{periods of }\DIFaddend negative trend \citep{Wild2009}. There are many
cases in the \DIFdelbegin \DIFdel{long term }\DIFdelend \DIFaddbegin \DIFadd{long-term }\DIFaddend records of irradiance \DIFdelbegin \DIFdel{, }\DIFdelend showing a systematic change
in the magnitude of the trend, occurring roughly in the last decades of
the 20th century. \DIFdelbegin \DIFdel{On }\DIFdelend \DIFaddbegin \DIFadd{At }\DIFaddend multiple stations in China, a dimming period was
reported until about 2000, followed by a brightening period
\citep{Yang2021}. A similar pattern was identified, with \DIFdelbegin \DIFdel{the }\DIFdelend \DIFaddbegin \DIFadd{a }\DIFaddend breaking
point around 1980, for stations in Central Europe \citep{Wild2021} and
Brazil \citep{Yamasoe2021}. On global scale, an artificial \DIFdelbegin \DIFdel{Intelligence
}\DIFdelend \DIFaddbegin \DIFadd{intelligence
}\DIFaddend aided spatial analysis on \DIFaddbegin \DIFadd{the }\DIFaddend continental level with data from multiple
stations \DIFdelbegin \DIFdel{reach }\DIFdelend \DIFaddbegin \DIFadd{reached }\DIFaddend similar conclusions for these regions and for the
global trend \citep{Yuan2021}.

There is a consensus among researchers that the major factor affecting
the variability of SDR attenuation is the interactions of solar
radiation with atmospheric aerosols and clouds. Those interactions,
among other factors, have been \DIFdelbegin \DIFdel{analysed }\DIFdelend \DIFaddbegin \DIFadd{analyzed }\DIFaddend with models
\citep{Li2016, Samset2018}, showing the existence of feedback mechanisms
between the two. Similar findings have been shown \DIFdelbegin \DIFdel{in observational data
\mbox{%DIFAUXCMD
\citep[ and references
therein]{Schwarz2020, Ohvril2009, Zerefos2009, Xia2007}}\hskip0pt%DIFAUXCMD
}\DIFdelend \DIFaddbegin \DIFadd{from the analysis of
observations at other locations
\mbox{%DIFAUXCMD
\citep{Schwarz2020, Ohvril2009, Zerefos2009, Xia2007} }\hskip0pt%DIFAUXCMD
}{[}\DIFadd{and references
therein}{]}\DIFadd{. In the Mediterranean region aerosols have been recognized as
an important factor affecting the penetration of solar radiation at the
surface \mbox{%DIFAUXCMD
\citep{Fountoulakis2016, Siomos2018, Gkikas2013, Lozano2021}}\hskip0pt%DIFAUXCMD
.
These studies investigated the long-term trend in aerosol optical depth,
which has been found to decrease in the last three decades, the
transport and composition of aerosols, and their radiative effects}\DIFaddend .

Due to the significant spatial and temporal variability of the trends
\DIFdelbegin \DIFdel{,
}\DIFdelend and the contributing factors, there is a constant need to monitor and
investigate SDR \DIFdelbegin \DIFdel{in }\DIFdelend \DIFaddbegin \DIFadd{at }\DIFaddend different sites in order to estimate the degree of
variability, and its relation to the local conditions. In this study, we
examine the trends of SDR, using ground-based measurements at
Thessaloniki, Greece, for the period \DIFaddbegin \DIFadd{from }\DIFaddend 1993 to \DIFdelbegin \DIFdel{2023, as derived from a
CM-21 pyranometer. We reevaluated }\DIFdelend \DIFaddbegin \DIFadd{2023. We re-evaluated
}\DIFaddend and extended the dataset used by \citet{Bais2013}, we applied a
different algorithm for the identification of clear-/cloud-sky instances
\DIFdelbegin \DIFdel{\mbox{%DIFAUXCMD
\citep{Reno2016, Reno2012a}}\hskip0pt%DIFAUXCMD
}\DIFdelend \DIFaddbegin \DIFadd{\mbox{%DIFAUXCMD
\citep{Reno2016, Reno2012}}\hskip0pt%DIFAUXCMD
}\DIFaddend , and we derived the SDR trends for the period
of study, under different sky conditions (all-sky, clear-sky\DIFdelbegin \DIFdel{and
cloud-sky}\DIFdelend \DIFaddbegin \DIFadd{, and
cloudy-sky}\DIFaddend ). Finally, we investigated the dependence of the trends on
solar zenith angle and season.

\DIFdelbegin %DIFDELCMD < \hypertarget{observational-data-and-methodology.}{%
%DIFDELCMD < \section{Observational data and
%DIFDELCMD < methodology.}\label{observational-data-and-methodology.}}
%DIFDELCMD < %%%
\DIFdelend \DIFaddbegin \hypertarget{data-and-methodology}{%
\section{Data and Methodology}\label{data-and-methodology}}
\DIFaddend 

The SDR data were measured with a Kipp \& Zonen CM-21 pyranometer
operating continuously at the Laboratory of Atmospheric Physics of the
Aristotle University of Thessaloniki (\(40^\circ\,38'\,\)N,
\(22^\circ\,57'\,\)E, \(80\,\)m~a.s.l.)\DIFdelbegin \DIFdel{in }\DIFdelend \DIFaddbegin \DIFadd{. Here, we used data for }\DIFaddend the
period from \DIFdelbegin \DIFdel{1993-04-13 to 2023-04-13. }\DIFdelend \DIFaddbegin \DIFadd{13 April 1993 to 13 April 2023. }\DIFaddend The monitoring site \DIFdelbegin \DIFdel{is }\DIFdelend \DIFaddbegin \DIFadd{was
}\DIFaddend located near the city \DIFdelbegin \DIFdel{centre, and we expect to be }\DIFdelend \DIFaddbegin \DIFadd{center, thus we expect that measurements were
}\DIFaddend affected by the urban environment\DIFaddbegin \DIFadd{, mainly by aerosols}\DIFaddend . During the study
period, the pyranometer \DIFdelbegin \DIFdel{has been }\DIFdelend \DIFaddbegin \DIFadd{was }\DIFaddend independently calibrated three times at the
Meteorologisches Observatorium Lindenberg, DWD, \DIFdelbegin \DIFdel{when it was verified }\DIFdelend \DIFaddbegin \DIFadd{verifying that }\DIFaddend the
stability of the instrument\DIFdelbegin \DIFdel{to within }\DIFdelend \DIFaddbegin \DIFadd{'s sensitivity was }\DIFaddend better than \(0.7\%\)
relative to the initial calibration by the manufacturer. Along with SDR,
the direct beam radiation (DNI) was also measured \DIFdelbegin \DIFdel{by }\DIFdelend \DIFaddbegin \DIFadd{with }\DIFaddend a collocated Kipp
\& Zonen CHP-1 pyrheliometer \DIFdelbegin \DIFdel{, for the period 2016-04-01 to 2023-04-13. Although,
we have performed a similar analysis to the DNI data, the results are
not presented here, as they lack the appropriate statistical
significance, due to the sorter duration of the data. However, the }\DIFdelend \DIFaddbegin \DIFadd{since 1 April 2016. The }\DIFaddend DNI data were used
as auxiliary data \DIFdelbegin \DIFdel{, in
the clear sky }\DIFdelend \DIFaddbegin \DIFadd{to support the selection of appropriate thresholds in
the clear-sky }\DIFaddend identification algorithm (CSid), which is discussed \DIFdelbegin \DIFdel{later, for the selection of the
appropriate thresholds}\DIFdelend \DIFaddbegin \DIFadd{in
Section \ref{CDIDalgorithm}}\DIFaddend . It is noted that \DIFdelbegin \DIFdel{despite the capability of }\DIFdelend the \DIFdelbegin \DIFdel{CSid algorithm to use the DNI as a characterization parameter, we
haven't utilized it here, }\DIFdelend \DIFaddbegin \DIFadd{limited dataset of DNI
was not used for the identification of clear-sky cases in the entire SDR
series }\DIFaddend to avoid any selection bias \DIFdelbegin \DIFdel{, due to }\DIFdelend \DIFaddbegin \DIFadd{due to the }\DIFaddend unequal length of the two
datasets. There are four distinct steps in the creation of the dataset
\DIFdelbegin \DIFdel{analysed here: }\DIFdelend \DIFaddbegin \DIFadd{analyzed here: (}\DIFaddend a)~the acquisition of radiation measurements from the
sensors, \DIFaddbegin \DIFadd{(}\DIFaddend b)~the data quality check, \DIFaddbegin \DIFadd{(}\DIFaddend c)~the identification of ``clear
sky'' conditions from the \DIFdelbegin \DIFdel{radiometric }\DIFdelend \DIFaddbegin \DIFadd{SDR }\DIFaddend data, and \DIFaddbegin \DIFadd{(}\DIFaddend d)~the aggregation of data and
trend analysis.

For the acquisition of radiometric data, the signal of the pyranometer
\DIFdelbegin \DIFdel{is sampled with }\DIFdelend \DIFaddbegin \DIFadd{was sampled at }\DIFaddend a rate of \(1\,\text{Hz}\). The mean and the standard
deviation of these samples \DIFdelbegin \DIFdel{are }\DIFdelend \DIFaddbegin \DIFadd{were calculated and }\DIFaddend recorded every minute.
The measurements \DIFdelbegin \DIFdel{are }\DIFdelend \DIFaddbegin \DIFadd{were }\DIFaddend corrected for the zero offset (``dark signal'' in
volts)\DIFdelbegin \DIFdel{. The ``dark
signal'' is }\DIFdelend \DIFaddbegin \DIFadd{, which was }\DIFaddend calculated by averaging all measurements recorded for
a period of \(3\,\text{h}\), before (morning) or after (evening) the Sun
reaches an elevation angle of \(-10^\circ\). The signal \DIFdelbegin \DIFdel{is }\DIFdelend \DIFaddbegin \DIFadd{was }\DIFaddend converted to
irradiance using a ramped value of the instrument's sensitivity between
\DIFaddbegin \DIFadd{subsequent }\DIFaddend calibrations.

A manual screening was performed, to remove inconsistent and erroneous
recordings that can occur stochastically or systematically, during the
continuous operation of the instruments. The manual screening \DIFdelbegin \DIFdel{is }\DIFdelend \DIFaddbegin \DIFadd{was }\DIFaddend aided
by a radiation data quality assurance procedure, adjusted for the site,
which \DIFdelbegin \DIFdel{is }\DIFdelend \DIFaddbegin \DIFadd{was }\DIFaddend based on the methods of Long and
Shi~\DIFdelbegin \DIFdel{\mbox{%DIFAUXCMD
\citetext{\citeyear{Long2008a}; \citeyear{Long2006}}}\hskip0pt%DIFAUXCMD
}\DIFdelend \DIFaddbegin \DIFadd{\mbox{%DIFAUXCMD
\citep{Long2006, Long2008a}}\hskip0pt%DIFAUXCMD
}\DIFaddend . Thus, problematic recordings have been
excluded from further processing. Although it is impossible to detect
all false data, the large number of available data, and the aggregation
scheme we used, ensures the \DIFdelbegin \DIFdel{good
}\DIFdelend quality of the radiation measurements used
in this study.

\DIFdelbegin \DIFdel{In order to be able }\DIFdelend \DIFaddbegin \DIFadd{Due to the significant measurement uncertainty when the Sun is near the
horizon, we have excluded all measurements with solar zenith angle (SZA)
greater than \(85^\circ\). Moreover, due to obstructions around the site
(hills and buildings) that block the direct irradiance, we excluded data
with azimuth angle in the range \(58^{\circ}\)--\(120^{\circ}\) and with
SZA greater than \(78^{\circ}\). To make the measurements comparable
throughout the dataset, we adjusted all one-minute data to the mean
Sun--Earth distance. Subsequently, we adjusted all measurements to the
Total Solar Irradiance (TSI) at \(1\,\text{au}\), in order to compensate
for the Sun's intensity variability, using a time series of satellite
TSI observations. The TSI data we used are part of the `NOAA Climate
Data Record of Total Solar Irradiance' dataset \mbox{%DIFAUXCMD
\citep{Coddington2005}}\hskip0pt%DIFAUXCMD
.
The initial daily values of this dataset were interpolated to match the
time step of our measurements.
}

\DIFadd{In order }\DIFaddend to estimate the effect of the sky conditions on the long term
variability of SDR, we created three datasets \DIFdelbegin \DIFdel{, }\DIFdelend by characterizing each
one-minute measurement with a corresponding \DIFdelbegin \DIFdel{sky
condition }\DIFdelend \DIFaddbegin \DIFadd{sky-condition flag }\DIFaddend (i.e.,
all-sky, clear-sky\DIFaddbegin \DIFadd{, }\DIFaddend and cloudy-sky). To identify the \DIFdelbegin \DIFdel{clear-sky conditions we used
a }\DIFdelend \DIFaddbegin \DIFadd{clear cases we used
the }\DIFaddend method proposed by \DIFdelbegin \DIFdel{\mbox{%DIFAUXCMD
\citet{Long2000} }\hskip0pt%DIFAUXCMD
and
by }\DIFdelend \citet{Reno2016}, which \DIFdelbegin \DIFdel{were adapted and configured for the site}\DIFdelend \DIFaddbegin \DIFadd{requires the definition
of some site specific parameters. These parameters were determined by an
iterative process}\DIFaddend , as the \DIFdelbegin \DIFdel{authors suggest}\DIFdelend \DIFaddbegin \DIFadd{original authors proposed and are discussed in
the next section}\DIFaddend .

We \DIFdelbegin \DIFdel{have to note , that }\DIFdelend \DIFaddbegin \DIFadd{note that all methods have some subjectivity in }\DIFaddend the definition of
clear or cloudy sky \DIFdelbegin \DIFdel{, has some
subjectivity, in any method of characterization}\DIFdelend \DIFaddbegin \DIFadd{cases}\DIFaddend . As a result, the details of the definition
are site specific, \DIFdelbegin \DIFdel{it relies }\DIFdelend \DIFaddbegin \DIFadd{and they rely }\DIFaddend on a combination of thresholds and
comparisons with ideal radiation models and statistical analysis of
different signal metrics. The CSid algorithm was calibrated with the
main focus \DIFdelbegin \DIFdel{, }\DIFdelend to identify the presence of clouds\DIFdelbegin \DIFdel{on
the sky}\DIFdelend . Despite the fine-tuning
of the procedure, \DIFaddbegin \DIFadd{in }\DIFaddend a few marginal cases \DIFdelbegin \DIFdel{exist, that have been identified manually as }\DIFdelend false positive or false
negative \DIFdelbegin \DIFdel{but cannot }\DIFdelend \DIFaddbegin \DIFadd{results were identified by manual inspection. However, due to
their small number, they did not }\DIFaddend affect the final results of the study.
\DIFdelbegin %DIFDELCMD < 

%DIFDELCMD < %%%
\DIFdelend For completeness, we \DIFdelbegin \DIFdel{will }\DIFdelend provide below a brief overview of the \DIFdelbegin \DIFdel{clear
sky identification algorithm (CSid
)}\DIFdelend \DIFaddbegin \DIFadd{CSid
algorithm}\DIFaddend , along with the site specific thresholds.
\DIFaddbegin 

\hypertarget{CDIDalgorithm}{%
\subsection{The clear sky identification
algorithm}\label{CDIDalgorithm}}

\DIFaddend To calculate the reference \DIFdelbegin \DIFdel{clear sky
}\DIFdelend \DIFaddbegin \DIFadd{clear-sky }\DIFaddend \(\text{SDR}_\text{CSref}\) we used
the \(\text{SDR}_\text{Haurwitz}\) derived by the radiation model of
\citet{Haurwitz1945} \DIFaddbegin \DIFadd{(Eq.~\ref{eq:hau})}\DIFaddend , adjusted for our site\DIFaddbegin \DIFadd{:
}\begin{equation}
\DIFadd{\text{SDR}_\text{Haurwitz} = 1098 \times \cos(\theta) \times \exp \left( \frac{ - 0.059}{\cos(\theta)} \right) \label{eq:hau}
}\end{equation} \DIFadd{where \(\theta\) is the SZA.
}

\DIFadd{The adjustment was made }\DIFaddend with a factor \(a\) (Eq.~\ref{eq:ahau}), \DIFdelbegin \DIFdel{resulted by }\DIFdelend \DIFaddbegin \DIFadd{which
was estimated through }\DIFaddend an iterative optimization process, as described by
\citet{Long2000} and \citet{Reno2016}. The target of the optimization
was the minimization of a function \(f(a)\) (Eq.~\ref{eq:minf}) and was
accomplished with the algorithmic function ``optimise'', which is an
implementation based on the work of \citet{Brent1973}, from the library
``stats'' of the R programming language \citep{RCT2023}.
\begin{equation}
f(a) = \frac{1}{n}\sum_{i=1}^{n} ( \text{SDR}_{\text{CSid},i} - a \times \text{SDR}_{\text{testCSref},i} )^2 \label{eq:minf}
\end{equation} where: \(n\) is the total number of daylight \DIFdelbegin \DIFdel{records}\DIFdelend \DIFaddbegin \DIFadd{data}\DIFaddend ,
\(\text{SDR}_{\text{CSid},i}\) are the \DIFdelbegin \DIFdel{records identified as clear sky
}\DIFdelend \DIFaddbegin \DIFadd{data identified as clear-sky }\DIFaddend by
CSid, \(a\) is a \DIFdelbegin \DIFdel{hypothetical }\DIFdelend \DIFaddbegin \DIFadd{site-specific }\DIFaddend adjustment factor, and
\(\text{SDR}_{\text{testCSref},i}\) is \DIFaddbegin \DIFadd{the SDR derived by }\DIFaddend any of the
tested \DIFdelbegin \DIFdel{clear sky
}\DIFdelend \DIFaddbegin \DIFadd{clear-sky }\DIFaddend radiation models.

The optimization and the selection of the clear sky reference model, was
performed on SDR observations for the period 2016 - 2021. During the
optimization, eight simple clear sky radiation models were tested
(namely, Daneshyar-Paltridge-Proctor, Kasten-Czeplak, Haurwitz,
Berger-Duffie, Adnot-Bourges-Campana-Gicquel, Robledo-Soler, Kasten and
Ineichen-Perez), with a wide range of factors. These models are
described in more \DIFdelbegin \DIFdel{details }\DIFdelend \DIFaddbegin \DIFadd{detail }\DIFaddend by \citet{Reno2012} and \DIFaddbegin \DIFadd{are }\DIFaddend evaluated by
\citet{Reno2016}. We found, that Haurwitz's model, adjusted with the
factor \(a = 0.965\) yields one of the lowest root mean squared errors
(RMSE), while the procedure \DIFdelbegin \DIFdel{, }\DIFdelend manages to characterize \DIFaddbegin \DIFadd{successfully }\DIFaddend the
majority of the data. Thus, our clear sky reference is derived by the
Eq.~\ref{eq:ahau}\DIFdelbegin \DIFdel{.
}\DIFdelend \DIFaddbegin \DIFadd{: }\DIFaddend \begin{equation}
\text{SDR}_\text{CSref} = a \times \text{SDR}_\text{Haurwitz} = 0.965 \times 1098 \times \cos(\theta) \times \exp \left( \frac{ - 0.057}{\cos(\theta)} \right) \label{eq:ahau}
\end{equation}
\DIFdelbegin \DIFdel{where: \(\text{SDR}_\text{CSref}\) is the reference clear
sky SDR, in \(\text{W}\,\text{m}^{-2}\) and \(\theta\) is the solar
zenith angle (SZA).
}\DIFdelend 

The criteria that were used to identify whether a measurement was taken
under clear-sky conditions are presented below. A data point is flagged
as ``clear-sky'' if all criteria are satisfied\DIFdelbegin \DIFdel{, }\DIFdelend \DIFaddbegin \DIFadd{; }\DIFaddend otherwise it is
considered \DIFdelbegin \DIFdel{to be }\DIFdelend \DIFaddbegin \DIFadd{as }\DIFaddend ``cloud-sky''. Each criterion was applied for a running
window of \(11\) consecutive one-minute measurements, and the
characterization \DIFdelbegin \DIFdel{is }\DIFdelend \DIFaddbegin \DIFadd{was }\DIFaddend assigned to the central \DIFdelbegin \DIFdel{value }\DIFdelend \DIFaddbegin \DIFadd{datum }\DIFaddend of the window. Each
parameter \DIFdelbegin \DIFdel{, was calculated both }\DIFdelend \DIFaddbegin \DIFadd{was calculated }\DIFaddend from the observations \DIFdelbegin \DIFdel{and }\DIFdelend \DIFaddbegin \DIFadd{in comparison to }\DIFaddend the
reference clear sky model\DIFdelbegin \DIFdel{, for each comparison}\DIFdelend . The allowable range of variation is defined
by the model-derived value of the parameter multiplied by a factor plus
an offset. The factors and the offsets were determined empirically, by
manually inspecting each filter's performance on selected days and
adjusting them accordingly during an iterative process. The criteria are
\DIFaddbegin \DIFadd{listed below, together with the range of values within which the
respective parameter should fall in order to raise the clear-sky flag}\DIFaddend :

\begin{enumerate}
\def\labelenumi{\alph{enumi})}
\tightlist
\item
  Mean of the measured \(\overline{\text{SDR}}_i\) (Eq.
  \ref{eq:MeanVIP}). \begin{equation}
  0.91 \times \overline{\text{SDR}}_{\text{CSref},i} - 20\,Wm^{-2}
  < \overline{\text{SDR}}_i <
  1.095 \times \overline{\text{SDR}}_{\text{CSref},i} + 30\,Wm^{-2}
  \label{eq:MeanVIP}
  \end{equation}
\item
  Maximum measured value \(M_{i}\) (Eq.~\ref{eq:MaxVIP}).
  \begin{equation}
  1 \times M_{\text{CSref},i} - 75\,Wm^{-2}
  < M_{i} <
  1 \times M_{\text{CSref},i} + 75\,Wm^{-2}
  \label{eq:MaxVIP}
  \end{equation}
\item
  Length \(L_i\) of the sequential line segments, connecting the points
  of the \(11\) SDR values (Eq. \ref{eq:VILeq}). \begin{equation}
  L_i = \sum_{i=1}^{n-1}\sqrt{\left ( \text{SDR}_{i+1} - \text{SDR}_{i}\right )^2 + \left ( t_{i+1} - t_i \right )^2}
  \label{eq:VILeq}
  \end{equation} \begin{equation}
  1 \times L_{\text{CSref},i} - 5 < L_i < 1.3 \times L_{\text{CSref},i} + 13
  \label{eq:VILcr}
  \end{equation} where: \(t_i\) is the time stamp of each SDR
  measurement.
\item
  Standard deviation \(\sigma_i\) of the slope (\(s_i\)) between the
  \(11\) sequential points, normalized by the mean
  \(\overline{\text{SDR}}_i\) (Eq.~\ref{eq:VCT1}). \begin{gather}
    \sigma_i = \frac{\sqrt{\frac{1}{n-1} \sum_{i=1}^{n-1} \left( s_i - \bar{s} \right)^2}}{\overline{\text{SDR}}_i} \label{eq:VCT1} \\
    s_i = \frac{\text{SDR}_{i+1} - \text{SDR}_{i}}{t_{i+1} - t_i},\;\;   \bar{s} = \frac{1}{n-1} \sum_{i=1}^{n-1} s_i,\;\;\forall i \in \left \{ 1, 2, \ldots, n-1 \right \}\;\;
  \end{gather} For this criterion, \(\sigma_i\) should be below a
  certain threshold (Eq.~\ref{eq:VCTcr}): \begin{equation}
    \sigma_i < \ensuremath{1.1\times 10^{-4}} \label{eq:VCTcr}
  \end{equation}
\item
  Maximum difference \(X_i\) between the change in measured irradiance
  and the change in clear sky irradiance over each measurement interval.
  \begin{gather}
    X_i = \max{\left \{ \left | x_i - x_{\text{CSref},i} \right | \right \}} \label{eq:VSM3} \\
    x_i = \text{SDR}_{i+1} - \text{SDR}_{i} \forall i \in \left \{ 1, 2, \ldots, n-1 \right \} \label{eq:VSM1} \\
    x_{\text{CSref},i} = \text{SDR}_{\text{CSref},i+1} - \text{SDR}_{\text{CSref},i} \forall i \in \left \{ 1, 2, \ldots, n-1 \right \} \label{eq:VSM2}
  \end{gather} For this criterion, \(X_i\) should be below a certain
  threshold (Eq.~\ref{eq:VSMcr}): \begin{equation}
    X_i < 7.5\,Wm^{-2} \label{eq:VSMcr}
  \end{equation}
\end{enumerate}

\DIFdelbegin \DIFdel{Due to the significant measurement uncertainty when the Sun is near the
horizon, we have excluded all measurements with SZA greater than
\(85^\circ\). Moreover, due to some obstructions around the site (hills
and buildings), we excluded data with azimuth angle between \(35^\circ\)
and \(120^\circ\) with SZA greater than \(80^\circ\). On the latter
instances, Sun is systematically not visible from the instrument's
location. To make the measurements comparable throughout the dataset, we
adjusted all one-minute radiometric values to the mean Sun - Earth
distance. Subsequently, we made all measurements relative to the Total
Solar Irradiance (TSI) at \(1\,\text{au}\), in order to compensate for
the Sun's intensity variability, using a time series of satellite TSI
observations. The TSI data we used are part of the ``NOAA Climate Data
Record of Total Solar Irradiance'' dataset \mbox{%DIFAUXCMD
\citep{Coddington2005}}\hskip0pt%DIFAUXCMD
. The
initial daily values of this dataset were interpolated to match the time
step of our measurements. }\DIFdelend In the final dataset \DIFdelbegin \DIFdel{\(84.2\%\) of the data
}\DIFdelend \DIFaddbegin \DIFadd{\(26\,\%\) of the days }\DIFaddend were identified as under
clear-sky conditions and \DIFdelbegin \DIFdel{\(15.8\%\) }\DIFdelend \DIFaddbegin \DIFadd{\(48\,\%\) }\DIFaddend as under cloud-sky conditions. \DIFaddbegin \DIFadd{The
remaining \(26\,\%\) of the data correspond to mixed cases and were not
analyzed as a separate group.
}

\hypertarget{aggregationstatistical}{%
\subsection{Aggregation of data and statistical
approach}\label{aggregationstatistical}}
\DIFaddend 

In order to investigate the SDR trends \DIFaddbegin \DIFadd{which are the main focus of the
study}\DIFaddend , we implemented an \DIFdelbegin \DIFdel{appropriate
}\DIFdelend aggregation scheme to the \DIFdelbegin \DIFdel{1-minute }\DIFdelend \DIFaddbegin \DIFadd{one-minute }\DIFaddend data to
derive \DIFdelbegin \DIFdel{a }\DIFdelend series in coarser time-scales. To preserve the representativeness
of the data we used the following criteria: a) \DIFdelbegin \DIFdel{for the daily mean values we accept days with
more }\DIFdelend \DIFaddbegin \DIFadd{we excluded all days with
less }\DIFaddend than 50\% of the \DIFdelbegin \DIFdel{daytime measurements present and valid}\DIFdelend \DIFaddbegin \DIFadd{expected daytime measurements}\DIFaddend , b) \DIFdelbegin \DIFdel{monthly
values }\DIFdelend \DIFaddbegin \DIFadd{daily means for
the clear-sky and cloudy-sky datasets were calculated only for days with
more than 60\% of the expected daytime measurements identified as clear
or cloudy respectively, c) monthly means }\DIFaddend were computed from daily means\DIFaddbegin \DIFadd{.
For the all-skies dataset monthly means were computed }\DIFaddend only when at least
20 days were available. \DIFdelbegin \DIFdel{To create the daily and monthly climatological means, we
averaged the data based on the day of year and calendar month,
respectively. For the seasonal means we averaged the
mean daily }\DIFdelend \DIFaddbegin \DIFadd{Seasonal means were derived by averaging the
monthly mean }\DIFaddend values in each season (\DIFdelbegin \DIFdel{Winter}\DIFdelend \DIFaddbegin \DIFadd{winter}\DIFaddend : December - February, \DIFdelbegin \DIFdel{Spring}\DIFdelend \DIFaddbegin \DIFadd{spring}\DIFaddend :
March - May, etc.). \DIFdelbegin \DIFdel{Finally, each data set was }\DIFdelend \DIFaddbegin \DIFadd{The daily and monthly climatological means were
derived by averaging the data for each day of year and calendar month,
respectively. The daily and monthly datasets were }\DIFaddend deseasonalized by
subtracting the corresponding climatological annual cycle (daily or
monthly) from the actual data. \DIFdelbegin \DIFdel{To }\DIFdelend \DIFaddbegin \DIFadd{Finally, to }\DIFaddend estimate the SZA effect on
the SDR trends, the one-minute data were aggregated in \DIFdelbegin \DIFdel{\(1^\circ\) }\DIFdelend \DIFaddbegin \DIFadd{\(1^{\circ}\) }\DIFaddend SZA
bins, separately for the morning and afternoon hours\DIFdelbegin \DIFdel{, and then were deseasonalized as
mentioned above}\DIFdelend \DIFaddbegin \DIFadd{.
}

\DIFadd{The linear trends were calculated using a first order autoregressive
model with lag of 1 day using the ``maximum likelihood'' fitting method
\mbox{%DIFAUXCMD
\citep{Gardner1980, Jones1980}}\hskip0pt%DIFAUXCMD
, by implementing the function ``arima''
from the library ``stats'' of the R programming language
\mbox{%DIFAUXCMD
\citep{RCT2023}}\hskip0pt%DIFAUXCMD
. The trends were reported together with the \(2\sigma\)
errors}\DIFaddend .

\hypertarget{results}{%
\section{Results}\label{results}}

\hypertarget{long-term-sdr-trends}{%
\subsection{Long-term SDR trends}\label{long-term-sdr-trends}}

We calculated the linear \DIFdelbegin \DIFdel{SDR trends }\DIFdelend \DIFaddbegin \DIFadd{trends of SDR}\DIFaddend , from the departures of the mean
daily values from the daily climatology and for the three sky
conditions\DIFdelbegin \DIFdel{(}\DIFdelend \DIFaddbegin \DIFadd{. These are presented in }\DIFaddend Table~\ref{tab:trendtable} \DIFdelbegin \DIFdel{)}\DIFdelend \DIFaddbegin \DIFadd{which
contains also the \(2\sigma\) standard error, the Pearson's correlation
coefficient R and the trend in absolute units}\DIFaddend . In
Figure~\ref{fig:trendALL} we present only the time series under all-sky
conditions; the plots for clear-sky and cloud-sky conditions, are \DIFdelbegin \DIFdel{very similar and are }\DIFdelend shown
in the Appendix (Figures~\ref{fig:trendCLEAR} and~
\ref{fig:trendCLOUD}). \DIFdelbegin \DIFdel{We observe a
positive trend }\DIFdelend \DIFaddbegin \DIFadd{In the studied period, there is no significant
break or change in the variability pattern of the time series. The
linear trends in all three datasets are positive and around
\(0.4\,\%/y\) }\DIFaddend for all-sky \DIFdelbegin \DIFdel{conditions (\(0.38\,\%/y\)), a very close but
smaller trend }\DIFdelend \DIFaddbegin \DIFadd{and cloudy-sky conditions, while }\DIFaddend for
clear-skies \DIFdelbegin \DIFdel{(\(0.35\,\%/y\))}\DIFdelend \DIFaddbegin \DIFadd{the trend is much smaller (\textasciitilde{}\(0.1\,\%/y\)).
The linear trends were calculated taking into account the
autocorrelation of the time series and all three are statistically
significant at least at the \(95\,\%\) confidence level, as they are
larger than the corresponding \(2\sigma\) errors, despite the small
values of R, which is due to the large variability of the daily values.
The clear-sky trend is very small suggesting a small effect from
aerosols and water vapor which are the dominant factors of the SDR
variability \mbox{%DIFAUXCMD
\citep{Fountoulakis2016, Siomos2018, Yu2022}}\hskip0pt%DIFAUXCMD
. In contrast,
the large positive trend of SDR under cloudy skies can be attributed to
reduction in cloud cover and/or cloud optical depth. Lack of continuous
observations of cloud optical thickness that could support these
findings does not allow drawing firm conclusions. However, there are
indications that the total cloud-cover as inferred from the ERA5
analysis for the grid point of Thessaloniki is decreasing over the
period of study. From the difference between all-sky }\DIFaddend and \DIFdelbegin \DIFdel{a negative weaker
trend for cloudy-skies (\(-0.28\,\%/y\)). In the studied period, there is no significant break or change in the variability of the
time series}\DIFdelend \DIFaddbegin \DIFadd{clear-sky SDR
trends, expressed in \(W/m^2/y\) using the long-term mean of the
respective datasets, the radiative effect of clouds is estimated to
\(0.96\,W/m^2/y\). This estimate is similar to the cloud radiative
forcing of \(1.22\,W/m^2/y\) reported for Granada, Spain
\mbox{%DIFAUXCMD
\citep{Lozano2023}}\hskip0pt%DIFAUXCMD
.
}

\DIFadd{The all-sky trend is similar to the one reported in \mbox{%DIFAUXCMD
\citet{Bais2013}
}\hskip0pt%DIFAUXCMD
from a ten-year shorter dataset suggesting that the tendency of SDR in
Thessaloniki is systematic}\DIFaddend . Other studies for the European region
reported a change \DIFdelbegin \DIFdel{of the SDR slope, }\DIFdelend \DIFaddbegin \DIFadd{in the SDR trend }\DIFaddend around 1980 \DIFaddbegin \DIFadd{from negative to positive
with comparable magnitude }\DIFaddend \citep{Wild2021, Yuan2021, Ohmura2009}, \DIFdelbegin \DIFdel{a few years
}\DIFdelend \DIFaddbegin \DIFadd{well
}\DIFaddend before the start of our records. \DIFdelbegin \DIFdel{It is interesting to note, that for
the observations period , the trend of }\DIFdelend \DIFaddbegin \DIFadd{However, the trends reported here for
the three datasets are in accordance with the widely accepted solar
radiation brightening over Europe. For the period of our observations
the trend in }\DIFaddend the TSI is \DIFdelbegin \DIFdel{\(-0.0002\,\%/y\)}\DIFdelend \DIFaddbegin \DIFadd{negligible (\(-0.00022\,\%/y\))}\DIFaddend , and thus we
\DIFdelbegin \DIFdel{can not attribute any major effect on }\DIFdelend \DIFaddbegin \DIFadd{cannot attribute any significant effect on the }\DIFaddend SDR trend to \DIFdelbegin \DIFdel{Solar }\DIFdelend \DIFaddbegin \DIFadd{solar
}\DIFaddend variability.

\begin{table}[H]

\caption{\label{tab:trendtable}Trends in SDR daily means for different sky conditions for the period 1993 - 2023.}
\begin{tabu} to \linewidth {>{\centering\arraybackslash}p{8em}>{\raggedleft}X>{\raggedleft}X>{\raggedleft}X\DIFaddbeginFL \DIFaddFL{>}{\raggedleft}\DIFaddFL{X>}{\raggedleft}\DIFaddFL{X}\DIFaddendFL }
\toprule
Sky conditions & Trend [\%/year] & \DIFdelbeginFL \DIFdelFL{Statistical signif. }\DIFdelendFL \DIFaddbeginFL \DIFaddFL{Trend S.E. ($2\sigma$) }& \DIFaddFL{Pearson correl. }& \DIFaddFL{Trend }\DIFaddendFL [\DIFdelbeginFL \DIFdelFL{\%}\DIFdelendFL \DIFaddbeginFL \DIFaddFL{W/m\textsuperscript{2}/year}\DIFaddendFL ] & Days\DIFdelbeginFL \DIFdelFL{with data}\DIFdelendFL \\
\midrule
All skies & \DIFdelbeginFL \DIFdelFL{0.376 }\DIFdelendFL \DIFaddbeginFL \DIFaddFL{0.380 }\DIFaddendFL & \DIFdelbeginFL \DIFdelFL{100.00 }\DIFdelendFL \DIFaddbeginFL \DIFaddFL{0.120 }\DIFaddendFL & \DIFdelbeginFL \DIFdelFL{10256}\DIFdelendFL \DIFaddbeginFL \DIFaddFL{0.091 }& \DIFaddFL{1.460 }& \DIFaddFL{10251}\DIFaddendFL \\
Clear skies & \DIFdelbeginFL \DIFdelFL{0.349 }\DIFdelendFL \DIFaddbeginFL \DIFaddFL{0.097 }\DIFaddendFL & \DIFdelbeginFL \DIFdelFL{100.00 }\DIFdelendFL \DIFaddbeginFL \DIFaddFL{0.033 }\DIFaddendFL & \DIFdelbeginFL \DIFdelFL{10256}\DIFdelendFL \DIFaddbeginFL \DIFaddFL{0.140 }& \DIFaddFL{0.501 }& \DIFaddFL{2684}\DIFaddendFL \\
Cloudy skies & \DIFdelbeginFL \DIFdelFL{-0.276 }\DIFdelendFL \DIFaddbeginFL \DIFaddFL{0.410 }\DIFaddendFL & \DIFdelbeginFL \DIFdelFL{99.97 }\DIFdelendFL \DIFaddbeginFL \DIFaddFL{0.180 }\DIFaddendFL & \DIFdelbeginFL \DIFdelFL{5067}\DIFdelendFL \DIFaddbeginFL \DIFaddFL{0.081 }& \DIFaddFL{1.180 }& \DIFaddFL{4937}\DIFaddendFL \\
\bottomrule
\end{tabu}
\end{table}

\begin{figure}[h!]

{\centering \DIFdelbeginFL %DIFDELCMD < \includegraphics[width=.70\linewidth]{./images/LongtermTrends-2} 
%DIFDELCMD < %%%
\DIFdelendFL \DIFaddbeginFL \includegraphics[width=.75\linewidth]{./images/LongtermTrends-1} 
\DIFaddendFL 

}

\caption{Anomalies (\%) of the daily all-sky SDR \DIFdelbeginFL \DIFdelFL{, relative to }\DIFdelendFL \DIFaddbeginFL \DIFaddFL{from the }\DIFaddendFL climatological \DIFdelbeginFL \DIFdelFL{values }\DIFdelendFL \DIFaddbeginFL \DIFaddFL{mean }\DIFaddendFL for \DIFaddbeginFL \DIFaddFL{the period }\DIFaddendFL 1993 \DIFdelbeginFL \DIFdelFL{- }\DIFdelendFL \DIFaddbeginFL \DIFaddFL{-- }\DIFaddendFL 2023. The black line \DIFdelbeginFL \DIFdelFL{shows }\DIFdelendFL \DIFaddbeginFL \DIFaddFL{is }\DIFaddendFL the long term linear trend.}\label{fig:trendALL}
\end{figure}

Although the year-to-year variability of the anomalies (Figure
\ref{fig:trendALL} and Figures\DIFaddbegin \DIFadd{~}\DIFaddend \ref{fig:trendCLEAR},
\ref{fig:trendCLOUD} in Appendix), shows a rather homogeneous \DIFdelbegin \DIFdel{behaviour}\DIFdelend \DIFaddbegin \DIFadd{behavior}\DIFaddend ,
plots of the cumulative sums (CUSUM) \citep{Regier2019} of the anomalies
can reveal different structures in the records of all three sky
conditions. \DIFdelbegin \DIFdel{In the cases of all-sky and clear-sky conditions (Figures
\ref{fig:cusummonth-1} and \ref{fig:cusummonth-2}), we observe three
macroscopic periods. A downward part from the start until about 2005, a
relatively steady part until about 2016 and, finally, a steep upward
part until the present. For cloud-sky (Figure~\ref{fig:cusummonth-3}),
we have a different pattern; it begins with a relatively steady part
until 1997, followed by an upward part until 2005, and a long decline
until 2020, with a small positive slope until the present. For a }\DIFdelend \DIFaddbegin \DIFadd{For time series with a }\DIFaddend uniform trend, we would expect the
CUSUMs of the anomalies to have a symmetric `V' shape \DIFaddbegin \DIFadd{centered around
the middle of the series}\DIFaddend . This would indicate that the anomalies are
evenly distributed around the climatological mean, and for a positive
uniform trend, the first half \DIFdelbegin \DIFdel{to be }\DIFdelend \DIFaddbegin \DIFadd{is }\DIFaddend below and the \DIFdelbegin \DIFdel{other }\DIFdelend \DIFaddbegin \DIFadd{second }\DIFaddend half above the
climatological mean. In our case, there is a more complex evolution of
the anomalies. \DIFdelbegin \DIFdel{Another
distinct feature of the CUSUMs, is the different pattern of the
cloudy-sky dataset which peaks around the middle of the period.
Although, there seems to exist acomplementary relation to the CUSUMs of
the clear- and all-sky cases, we can not assert that clouds are the main
driver for
this relation due to the great difference in the number of
observational data between the
two datasets
(Table~\ref{tab:trendtable})}\DIFdelend \DIFaddbegin \DIFadd{For all-skies (Figure~\ref{fig:cusummonth-1}), we observe
three rather distinct periods: a) a downward part between the start of
the datasets and about 2000, denoting that all anomalies are negative
thus below the climatology; b) a relatively steady part lasting for
almost 20 years suggesting little variability in SDR anomalies; and c) a
steep upward part to present indicating anomalies above the climatology.
The CUSUMs for cloudy-skies (Figure~\ref{fig:cusummonth-3}), show a
similar behavior with some short-term differences that do not change the
overall pattern. For clear skies (Figure~\ref{fig:cusummonth-2}), a
monotonic downward tendency is evident until 2004, suggesting that the
anomalies are all negative. After 2004 the anomalies turn to positive at
a fast rate for about five years and at a slower rate thereafter}\DIFaddend .

\begin{figure}[h!]
    \begin{adjustwidth}{-\extralength}{0cm}
        {\centering 
        \subfloat[All skies.\label{fig:cusummonth-1}]
            {\includegraphics[width=.32\linewidth]{./images/CumulativeMonthlyCuSum-1}}\hfill
        \subfloat[Clear skies.\label{fig:cusummonth-2}]
            {\includegraphics[width=.32\linewidth]{./images/CumulativeMonthlyCuSum-5}}\hfill
        \subfloat[Cloudy skies.\label{fig:cusummonth-3}]
            {\includegraphics[width=.32\linewidth]{./images/CumulativeMonthlyCuSum-9}}\hfill
        }
\caption{Cumulative sum plots of the monthly SDR anomalies in (\%) for different sky conditions.}\label{fig:cusummonth}
\end{adjustwidth}
\end{figure}

In order to \DIFdelbegin \DIFdel{investigate }\DIFdelend \DIFaddbegin \DIFadd{unveil }\DIFaddend further the features of the \DIFdelbegin \DIFdel{CUSUMs, we created
}\DIFdelend \DIFaddbegin \DIFadd{variability of the three
datasets, Figure~\ref{fig:cusumnotrendmonthly} presents }\DIFaddend another set of
CUSUM plots \DIFdelbegin \DIFdel{by subtracting the corresponding long term
trend from the SDR anomaly data, prior to the CUSUM calculation
(Figure~\ref{fig:cusumnotrendmonthly})}\DIFdelend \DIFaddbegin \DIFadd{using anomalies after the long-term linear trend is removed}\DIFaddend .
With this approach\DIFaddbegin \DIFadd{, }\DIFaddend periods when the CUSUMs diverge from zero can be
interpreted as a systematic variation of SDR from the climatological
mean. When the CUSUM is increasing, the \DIFdelbegin \DIFdel{added }\DIFdelend \DIFaddbegin \DIFadd{anomalies }\DIFaddend values are above the
\DIFdelbegin \DIFdel{climatological values of the
SDR trend }\DIFdelend \DIFaddbegin \DIFadd{climatology }\DIFaddend and vice versa. Overall, for all- and \DIFdelbegin \DIFdel{clear-sky }\DIFdelend \DIFaddbegin \DIFadd{cloudy-sky }\DIFaddend conditions
(Figures~\ref{fig:cusumnotrendmonthly-1}
and~\DIFdelbegin \DIFdel{\ref{fig:cusumnotrendmonthly-2}}\DIFdelend \DIFaddbegin \DIFadd{\ref{fig:cusumnotrendmonthly-3}}\DIFaddend ) we observe periods \DIFdelbegin \DIFdel{when the
anomalies
diverge }\DIFdelend \DIFaddbegin \DIFadd{with anomalies
diverging }\DIFaddend from the climatological \DIFdelbegin \DIFdel{value}\DIFdelend \DIFaddbegin \DIFadd{values}\DIFaddend , each lasting for several
years. \DIFaddbegin \DIFadd{These fluctuations are probably within the natural variability
and no distinct changes are identified. }\DIFaddend The pattern in both datasets is
\DIFdelbegin \DIFdel{very }\DIFdelend similar, suggesting prevalence in \DIFdelbegin \DIFdel{clear }\DIFdelend \DIFaddbegin \DIFadd{cloudy }\DIFaddend skies over Thessaloniki. \DIFdelbegin \DIFdel{It is interesting that in
the period 1993 - 2016 the anomalies have a high variability around
zero, while after 2016, the range of the variability is decreased to
about one third of the prior period. For
cloudy-sky conditions
}\DIFdelend \DIFaddbegin \DIFadd{For
clear skies }\DIFaddend (Figure~\DIFdelbegin \DIFdel{\ref{fig:cusumnotrendmonthly-3}) the period 1997 - 2008 is dominated by positive CUSUMs, suggesting a reduced effect of clouds on
SDR. From 1997 to mid-2000s CUSUMs are increasing, likely due to a
continuous decrease in the
optical thickness of clouds, followed by a
period of rapid increase (within 3 years) in cloud optical thickness
lasting up to 2008. The following stable period spans for about 15 years
up to 2021 when CUSUMs start increasing again }\DIFdelend \DIFaddbegin \DIFadd{\ref{fig:cusumnotrendmonthly-2}) the distinct change
in 2004 is now clearer. The most likely reason for this change is the
monotonic reduction of aerosols in Thessaloniki. At that year there a
change in the rate of decrease in aerosol optical depth as illustrated
in Figure 7 of \mbox{%DIFAUXCMD
\citet{Siomos2020}}\hskip0pt%DIFAUXCMD
. This abrupt change in CUSUMs lasts
until about 2010 when the anomalies become again variable}\DIFaddend .

\begin{figure}[h!]
    \begin{adjustwidth}{-\extralength}{0cm}
        {\centering 
            \subfloat[All skies.\label{fig:cusumnotrendmonthly-1}]
                {\includegraphics[width=.32\linewidth]{./images/CumulativeMonthlyCuSumNOtrend-1} }\hfill
            \subfloat[Clear skies.\label{fig:cusumnotrendmonthly-2}]
                {\includegraphics[width=.32\linewidth]{./images/CumulativeMonthlyCuSumNOtrend-5} }\hfill
            \subfloat[Cloudy skies.\label{fig:cusumnotrendmonthly-3}]
                {\includegraphics[width=.32\linewidth]{./images/CumulativeMonthlyCuSumNOtrend-9} }
        }
        \caption{Cumulative sum plots of monthly SDR anomalies in (\%) for different sky conditions after removing the long-term linear trend.}\label{fig:cusumnotrendmonthly}
\end{adjustwidth}
\end{figure}

\DIFdelbegin %DIFDELCMD < \hypertarget{effects-of-the-solar-zenith-angle-on-sdr.}{%
%DIFDELCMD < \subsection{Effects of the solar zenith angle on
%DIFDELCMD < SDR.}\label{effects-of-the-solar-zenith-angle-on-sdr.}}
%DIFDELCMD < %%%
\DIFdelend \DIFaddbegin \hypertarget{effects-of-the-solar-zenith-angle-on-sdr}{%
\subsection{Effects of the solar zenith angle on
SDR}\label{effects-of-the-solar-zenith-angle-on-sdr}}
\DIFaddend 

The solar zenith angle is a major factor \DIFdelbegin \DIFdel{of SDR reaching the ground, due
to the }\DIFdelend \DIFaddbegin \DIFadd{affecting the SDR, since
increases in SZA leads to }\DIFaddend enhancement of the radiation path in the
atmosphere, especially in urban environments with human activities
emitting aerosols \citep{Wang2021}. In order to estimate the effect of
\DIFdelbegin \DIFdel{the }\DIFdelend SZA on the SDR trends, we grouped the \DIFdelbegin \DIFdel{anomaly }\DIFdelend data in bins of \(1^\circ\) SZA,
and calculated the overall trend for each bin\DIFaddbegin \DIFadd{, separately for the daily
periods }\DIFaddend before noon and after noon (Figure~\ref{fig:szatrends}).
Although there are seasonal dependencies of the minimum SZA (see
Appendix, Figure~\ref{fig:SZAtrendSeason}), these dependencies \DIFdelbegin \DIFdel{would not
be furtherexamined here.
}\DIFdelend \DIFaddbegin \DIFadd{are not
discussed further.
}

\DIFaddend For all-sky \DIFdelbegin \DIFdel{and
clear-sky }\DIFdelend conditions the brightening effect of SDR (positive trend)
\DIFdelbegin \DIFdel{is
stronger for large }\DIFdelend \DIFaddbegin \DIFadd{increases with }\DIFaddend SZAs (Figures~\ref{fig:szatrends-1}\DIFdelbegin \DIFdel{and
\ref{fig:szatrends-2}) }\DIFdelend \DIFaddbegin \DIFadd{) ranging from about
\(0.1\,\%/y\) to about \(0.7\,\%/y\) for the statistically significant
trends}\DIFaddend . The trends in the morning and afternoon hours are more or less
consistent with small differences \DIFdelbegin \DIFdel{, }\DIFdelend \DIFaddbegin \DIFadd{at small SZAs }\DIFaddend which can be attributed
to \DIFdelbegin \DIFdel{systematic diurnal variations of aerosols
, particularly
}\DIFdelend \DIFaddbegin \DIFadd{effects on clear sky SDR from systematic diurnal patterns of aerosols
during the warm period of the year, consistently with the results
reported for China by \mbox{%DIFAUXCMD
\citet{Wang2021}}\hskip0pt%DIFAUXCMD
. Note that SZAs less than
\(25^\circ\) can only occur }\DIFaddend during the warm period of the year \DIFdelbegin \DIFdel{\mbox{%DIFAUXCMD
\citep{Wang2021}}\hskip0pt%DIFAUXCMD
.
}\DIFdelend \DIFaddbegin \DIFadd{around
noon when clear skies are more frequent. The increasing trend with SZA
is likely caused by the increased attenuation of SDR with SZA. The
effect is larger when aerosol and/or cloud layers are optically thicker,
therefore, decreases in aerosol and clouds through the study period will
result in larger positive trends of SDR at larger SZAs.
}

\DIFadd{Under clear skies (Figures~\ref{fig:szatrends-2}), the trends are
smaller and less variable, ranging between \(0.1\) and \(0.15\,\%/y\) up
to \(77^\circ\) SZA. At higher SZAs and in the afternoon hours there a
sharp increase in the trend up to \(0.3\,\%/y\), which may have been
caused by the long path length of radiation through the atmosphere as
discussed above for the all-sky conditions. The small differences in the
trend between morning and afternoon between \(35^\circ\) and
\(60^\circ\) SZA is likely a result of less attenuation of SDR in the
morning hours due to lesser amounts of aerosols and shallower boundary
layer.
}

\DIFaddend For cloudy-sky conditions (Figure~\ref{fig:szatrends-3}), we \DIFdelbegin \DIFdel{can not }\DIFdelend \DIFaddbegin \DIFadd{cannot
}\DIFaddend discern any significant dependence of the SDR trend with SZA \DIFdelbegin \DIFdel{. For SZAs \(16^\circ\)
- \(50^\circ\), the trends range within about \(\pm 0.2\,\%/y\), with a
weak statistical significance. Between \(50^\circ\) and \(75^\circ\) SZA
the trends for the period before noon are stronger and negative, possibly }\DIFdelend \DIFaddbegin \DIFadd{as the
variability of irradiance is dominated by the cloud effects leading to
insignificant trends. Statistically significant trends appear only in
the afternoon and for SZAs larger than \(60^\circ\). The sharp increase
of the trend at SZAs larger than \(\sim{75}^{\circ}\), observed also for
clear skies, is probably }\DIFaddend associated with stronger attenuation by clouds
under oblique incidence angles\DIFaddbegin \DIFadd{, which result also in smaller
variability}\DIFaddend .

\begin{figure}[h!]
    \begin{adjustwidth}{-\extralength}{0cm}
        {\centering 
            \subfloat[All skies.\label{fig:szatrends-1}]
                {\includegraphics[width=.32\linewidth]{./images/SzaTrends-1}}\hfill
            \subfloat[Clear skies.\label{fig:szatrends-2}]
                {\includegraphics[width=.32\linewidth]{./images/SzaTrends-4}}\hfill
            \subfloat[Cloudy skies.\label{fig:szatrends-3}]
                {\includegraphics[width=.32\linewidth]{./images/SzaTrends-7}}
        }
        \caption{Long term trends of \DIFaddbeginFL \DIFaddFL{daily }\DIFaddendFL SDR as a function of SZA \DIFaddbeginFL \DIFaddFL{for (a) all-sky, (b) clear-sky and (c) cloudy-sky conditions, }\DIFaddendFL separately \DIFdelbeginFL \DIFdelFL{form }\DIFdelendFL \DIFaddbeginFL \DIFaddFL{for }\DIFaddendFL morning and afternoon periods. Solid shapes represent statistically significant trends ($p < 0.005$).}\label{fig:szatrends}
    \end{adjustwidth}
\end{figure}

\DIFdelbegin %DIFDELCMD < \hypertarget{long-term-trends-by-season}{%
%DIFDELCMD < \subsection{Long term trends by
%DIFDELCMD < season}\label{long-term-trends-by-season}}
%DIFDELCMD < %%%
\DIFdelend \DIFaddbegin \hypertarget{long-term-sdr-trends-by-season}{%
\subsection{Long term SDR trends by
season}\label{long-term-sdr-trends-by-season}}
\DIFaddend 

Similarly to the long term trends \DIFaddbegin \DIFadd{from daily means of SDR }\DIFaddend discussed
above, we have calculated the trend \DIFdelbegin \DIFdel{of the anomalies }\DIFdelend for the three \DIFdelbegin \DIFdel{different sky conditions , }\DIFdelend \DIFaddbegin \DIFadd{sky conditions }\DIFaddend and for
each season of the year, using the corresponding mean monthly \DIFdelbegin \DIFdel{values
}\DIFdelend \DIFaddbegin \DIFadd{anomalies
}\DIFaddend (Figure~\ref{fig:seasonalALL} and Table~\ref{tab:trendseasontable}).
\DIFaddbegin \DIFadd{Table~\ref{tab:trendseasontable} contains also the \(2\sigma\) standard
error, the Pearson's correlation coefficient R and the corresponding
p-value. The winter linear trends exhibit generally the largest R values
ranging between \(0.54\) and \(0.60\,\%/y\). }\DIFaddend For all-sky conditions the
trend in SDR in winter is the largest (\DIFdelbegin \DIFdel{\(0.69\,\%/y\)}\DIFdelend \DIFaddbegin \DIFadd{\(0.7\,\%/y\)}\DIFaddend ), followed by the
trend in autumn (\DIFdelbegin \DIFdel{\(0.43\,\%/y\)}\DIFdelend \DIFaddbegin \DIFadd{\(0.42\,\%/y\)}\DIFaddend , a value close to the long term trend)
both statistically significant \DIFdelbegin \DIFdel{above
the \(99\,\%\) }\DIFdelend \DIFaddbegin \DIFadd{at the \(95\,\%\) }\DIFaddend confidence level. In
spring and summer, the trends are much smaller and of lesser statistical
significance. These seasonal differences indicate a possible relation of
the trends in SDR to trends \DIFdelbegin \DIFdel{of }\DIFdelend \DIFaddbegin \DIFadd{in }\DIFaddend clouds during winter and autumn. For
clear-skies, the trend in winter is \DIFdelbegin \DIFdel{\(0.83\,\%/y\), larger than }\DIFdelend \DIFaddbegin \DIFadd{\(0.4\,\%/y\) and is associated with
the decreasing trend in aerosol optical depth \mbox{%DIFAUXCMD
\citep{Siomos2020}}\hskip0pt%DIFAUXCMD
.
Moreover, it is almost half of that }\DIFaddend for all-skies\DIFdelbegin \DIFdel{(\(0.69\,\%/y\))}\DIFdelend , which is another
indication of a decreasing trend in cloud optical thickness. \DIFdelbegin \DIFdel{Moreover, the }\DIFdelend \DIFaddbegin \DIFadd{In other
seasons the clear-sky trend is very small (below \(0.1\,\%/y\)).
Finally, for cloudy-skies the winter trend is the largest
(\(0.76\,\%/y\)) and greater than for all-skies, followed by a much
smaller trend in autumn (\(0.19\,\%/y\)).
}

\DIFadd{The }\DIFaddend trends under clear- and cloudy-sky conditions are \DIFdelbegin \DIFdel{almost
complementary to each other, particularly for winterand autumn, where
the signal is stronger. During spring and summer the statistical
significance is very low and }\DIFdelend \DIFaddbegin \DIFadd{of the same
direction, and it would be expected that their sum is similar to }\DIFaddend the
\DIFdelbegin \DIFdel{actual trend too small for a meaningful
comparison}\DIFdelend \DIFaddbegin \DIFadd{all-sky trend. This does not happen, especially for winter, likely due
to the way the monthly means for clear and cloudy skies were calculated.
Daily means were calculated only when at least \(60\,\%\) of the clear-
or cloudy-sky data were available (see sect.
\ref{aggregationstatistical})}\DIFaddend .

\begin{figure}[h!]
    \begin{adjustwidth}{-\extralength}{0cm}
        {\centering 
            \DIFdelbeginFL %DIFDELCMD < \includegraphics[width=1\linewidth]{./images/SeasonalTrendsTogether3-2} 
%DIFDELCMD <         %%%
\DIFdelendFL \DIFaddbeginFL \includegraphics[width=1\linewidth]{./images/SeasonalMTrendsTogether3-2}   %DIF >  Seasonal from Monthly
        \DIFaddendFL }
        \caption{\DIFdelbeginFL \DIFdelFL{Linear trends (black lines) of monthly }\DIFdelendFL \DIFaddbeginFL \DIFaddFL{Monthly }\DIFaddendFL mean anomalies of SDR by season (rows of plots) for the three sky conditions (columns of plots). \DIFaddbeginFL \DIFaddFL{The black lines represent the linear trends.}\DIFaddendFL }\label{fig:seasonalALL}
    \end{adjustwidth}
\end{figure}

\begin{table}[!h]

\caption{\label{tab:trendseasontable}SDR linear trends of monthly anomalies for each season of the year \DIFaddbeginFL \DIFaddFL{and related statistical parameters}\DIFaddendFL .}
\begin{tabu} to \linewidth {>{\centering\arraybackslash}p{8em}>{\centering}X>{\raggedleft}X>{\raggedleft}X\DIFaddbeginFL \DIFaddFL{>}{\raggedleft}\DIFaddFL{X>}{\raggedleft}\DIFaddFL{X}\DIFaddendFL }
\toprule
Sky condition & Season & Trend [\%/year] & \DIFdelbeginFL \DIFdelFL{Statistical signif.}%DIFDELCMD < [%%%
\DIFdelFL{\%}%DIFDELCMD < ]%%%
\DIFdelendFL \DIFaddbeginFL \DIFaddFL{Trend S.E. ($2\sigma$) }& \DIFaddFL{Pearson correl. }& \DIFaddFL{Trend p-value}\DIFaddendFL \\
\midrule
\cellcolor{gray!6}{} & \cellcolor{gray!6}{Winter} & \DIFdelbeginFL %DIFDELCMD < \cellcolor{gray!6}{0.6860} %%%
\DIFdelendFL \DIFaddbeginFL \cellcolor{gray!6}{0.70} \DIFaddendFL & \DIFdelbeginFL %DIFDELCMD < \cellcolor{gray!6}{99.9}%%%
\DIFdelendFL \DIFaddbeginFL \cellcolor{gray!6}{0.43} & \cellcolor{gray!6}{0.54} & \cellcolor{gray!6}{0.003}\DIFaddendFL \\

 & Spring & \DIFdelbeginFL \DIFdelFL{0.1450 }\DIFdelendFL \DIFaddbeginFL \DIFaddFL{0.11 }\DIFaddendFL & \DIFdelbeginFL \DIFdelFL{81.8}\DIFdelendFL \DIFaddbeginFL \DIFaddFL{0.24 }& \DIFaddFL{0.17 }& \DIFaddFL{0.371}\DIFaddendFL \\

\cellcolor{gray!6}{} & \cellcolor{gray!6}{Summer} & \DIFdelbeginFL %DIFDELCMD < \cellcolor{gray!6}{0.1200} %%%
\DIFdelendFL \DIFaddbeginFL \cellcolor{gray!6}{0.11} \DIFaddendFL & \DIFdelbeginFL %DIFDELCMD < \cellcolor{gray!6}{89.4}%%%
\DIFdelendFL \DIFaddbeginFL \cellcolor{gray!6}{0.15} & \cellcolor{gray!6}{0.25} & \cellcolor{gray!6}{0.175}\DIFaddendFL \\

\multirow{-4}{*}{\centering\arraybackslash All skies} & Autumn & \DIFdelbeginFL \DIFdelFL{0.4310 }\DIFdelendFL \DIFaddbeginFL \DIFaddFL{0.42 }\DIFaddendFL & \DIFdelbeginFL \DIFdelFL{99.3}\DIFdelendFL \DIFaddbeginFL \DIFaddFL{0.30 }& \DIFaddFL{0.47 }& \DIFaddFL{0.009}\DIFaddendFL \\
\DIFdelbeginFL %DIFDELCMD < \cmidrule{1-4}
%DIFDELCMD < %%%
\DIFdelendFL \DIFaddbeginFL \cmidrule{1-6}
\DIFaddendFL \cellcolor{gray!6}{} & \cellcolor{gray!6}{Winter} & \DIFdelbeginFL %DIFDELCMD < \cellcolor{gray!6}{0.8260} %%%
\DIFdelendFL \DIFaddbeginFL \cellcolor{gray!6}{0.40} \DIFaddendFL & \DIFdelbeginFL %DIFDELCMD < \cellcolor{gray!6}{100.0}%%%
\DIFdelendFL \DIFaddbeginFL \cellcolor{gray!6}{0.20} & \cellcolor{gray!6}{0.60} & \cellcolor{gray!6}{0.001}\DIFaddendFL \\

 & Spring & \DIFdelbeginFL \DIFdelFL{0.0613 }\DIFdelendFL \DIFaddbeginFL \DIFaddFL{0.06 }\DIFaddendFL & \DIFdelbeginFL \DIFdelFL{38.6}\DIFdelendFL \DIFaddbeginFL \DIFaddFL{0.17 }& \DIFaddFL{0.13 }& \DIFaddFL{0.497}\DIFaddendFL \\

\cellcolor{gray!6}{} & \cellcolor{gray!6}{Summer} & \DIFdelbeginFL %DIFDELCMD < \cellcolor{gray!6}{-0.0307} %%%
\DIFdelendFL \DIFaddbeginFL \cellcolor{gray!6}{-0.05} \DIFaddendFL & \DIFdelbeginFL %DIFDELCMD < \cellcolor{gray!6}{25.9}%%%
\DIFdelendFL \DIFaddbeginFL \cellcolor{gray!6}{0.06} & \cellcolor{gray!6}{-0.30} & \cellcolor{gray!6}{0.106}\DIFaddendFL \\

\multirow{-4}{*}{\centering\arraybackslash Clear skies} & Autumn & \DIFdelbeginFL \DIFdelFL{0.3670 }\DIFdelendFL \DIFaddbeginFL \DIFaddFL{0.05 }\DIFaddendFL & \DIFdelbeginFL \DIFdelFL{97.2}\DIFdelendFL \DIFaddbeginFL \DIFaddFL{0.12 }& \DIFaddFL{0.17 }& \DIFaddFL{0.366}\DIFaddendFL \\
\DIFdelbeginFL %DIFDELCMD < \cmidrule{1-4}
%DIFDELCMD < %%%
\DIFdelendFL \DIFaddbeginFL \cmidrule{1-6}
\DIFaddendFL \cellcolor{gray!6}{} & \cellcolor{gray!6}{Winter} & \DIFdelbeginFL %DIFDELCMD < \cellcolor{gray!6}{-0.8820} %%%
\DIFdelendFL \DIFaddbeginFL \cellcolor{gray!6}{0.76} \DIFaddendFL & \DIFdelbeginFL %DIFDELCMD < \cellcolor{gray!6}{98.9}%%%
\DIFdelendFL \DIFaddbeginFL \cellcolor{gray!6}{0.40} & \cellcolor{gray!6}{0.59} & \cellcolor{gray!6}{0.001}\DIFaddendFL \\

 & Spring & \DIFdelbeginFL \DIFdelFL{-0.0991 }\DIFdelendFL \DIFaddbeginFL \DIFaddFL{0.06 }\DIFaddendFL & \DIFdelbeginFL \DIFdelFL{37.2}\DIFdelendFL \DIFaddbeginFL \DIFaddFL{0.23 }& \DIFaddFL{0.10 }& \DIFaddFL{0.593}\DIFaddendFL \\

\cellcolor{gray!6}{} & \cellcolor{gray!6}{Summer} & \DIFdelbeginFL %DIFDELCMD < \cellcolor{gray!6}{0.0444} %%%
\DIFdelendFL \DIFaddbeginFL \cellcolor{gray!6}{-0.08} \DIFaddendFL & \DIFdelbeginFL %DIFDELCMD < \cellcolor{gray!6}{21.5}%%%
\DIFdelendFL \DIFaddbeginFL \cellcolor{gray!6}{0.27} & \cellcolor{gray!6}{-0.11} & \cellcolor{gray!6}{0.560}\DIFaddendFL \\

\multirow{-4}{*}{\centering\arraybackslash Cloudy skies} & Autumn & \DIFdelbeginFL \DIFdelFL{-0.4000 }\DIFdelendFL \DIFaddbeginFL \DIFaddFL{0.19 }\DIFaddendFL & \DIFdelbeginFL \DIFdelFL{89.5}\DIFdelendFL \DIFaddbeginFL \DIFaddFL{0.43 }& \DIFaddFL{0.16 }& \DIFaddFL{0.384}\DIFaddendFL \\
\bottomrule
\end{tabu}
\end{table}

\hypertarget{conclusions}{%
\section{Conclusions}\label{conclusions}}

We have \DIFdelbegin \DIFdel{demonstrated that in the period }\DIFdelend \DIFaddbegin \DIFadd{analyzed a 30-year dataset of SRD measurements in Thessaloniki
Greece (}\DIFaddend 1993 \DIFdelbegin \DIFdel{- }\DIFdelend \DIFaddbegin \DIFadd{-- }\DIFaddend 2023\DIFdelbegin \DIFdel{, }\DIFdelend \DIFaddbegin \DIFadd{) aiming to identify the long term variability of
solar irradiance under different sky conditions. Under all-sky
conditions }\DIFaddend there is a positive trend in SDR of \DIFdelbegin \DIFdel{(}\DIFdelend \(0.38\,\%/y\)
\DIFdelbegin \DIFdel{) }\DIFdelend (brightening)\DIFdelbegin \DIFdel{(positive trend) in
Thessaloniki, Greece, under all-sky conditions}\DIFdelend . A previous study \citep{Bais2013} for the period 1993 \DIFdelbegin \DIFdel{- }\DIFdelend \DIFaddbegin \DIFadd{--
}\DIFaddend 2011 \DIFdelbegin \DIFdel{found }\DIFdelend \DIFaddbegin \DIFadd{reported }\DIFaddend also a positive trend of \(0.33\,\%/y\). The \DIFaddbegin \DIFadd{slight
}\DIFaddend increase of this trend indicates that the brightening of SDR continues
and is \DIFdelbegin \DIFdel{probably }\DIFdelend \DIFaddbegin \DIFadd{likely }\DIFaddend caused by continuing decreases in aerosol optical depth
and the optical thickness of clouds over the area. \DIFdelbegin \DIFdel{Moreover, we found a similar trend }\DIFdelend \DIFaddbegin \DIFadd{A smaller trend has
been found }\DIFaddend under clear-sky conditions (\DIFdelbegin \DIFdel{\(0.35\,\%/y\)) that further supports
the assumption that }\DIFdelend \DIFaddbegin \DIFadd{\(0.097\,\%/y\)) which supports
the notion that part of }\DIFaddend the brightening is caused \DIFdelbegin \DIFdel{mainly }\DIFdelend by decreasing
aerosols. \DIFdelbegin \DIFdel{Unfortunately,
for the entire period there is no available data for the aerosols, in
order to quantify their effect on SDR. However, }\DIFdelend \citet{Siomos2020} have shown that aerosol optical depth over
Thessaloniki is decreasing constantly at least up to 2018. The
attenuation of SDR by aerosols over Europe \DIFdelbegin \DIFdel{have }\DIFdelend \DIFaddbegin \DIFadd{has }\DIFaddend been proposed as major
factor by \citet{Wild2021}. \DIFdelbegin \DIFdel{The
dimming }\DIFdelend \DIFaddbegin \DIFadd{Unfortunately, for this study aerosol data
for the entire period were not available in order to quantify their
effect on SDR. The brightening }\DIFaddend effect on SDR under cloudy-sky conditions
(\DIFdelbegin \DIFdel{\(-0.28\,\%/y\)}\DIFdelend \DIFaddbegin \DIFadd{\(0.41\,\%/y\)}\DIFaddend ), suggests that cloud optical thickness is \DIFaddbegin \DIFadd{also
}\DIFaddend decreasing during this period. \DIFdelbegin \DIFdel{Because we have no adequate data to investigate the long term changes of cloud thickness in }\DIFdelend \DIFaddbegin \DIFadd{As long term data of cloud optical
thickness are also not available for }\DIFaddend the region, we cannot \DIFdelbegin \DIFdel{verify if the negative SDR
trend we observe under under cloudy-skies can be attributed solely to
changes in clouds}\DIFdelend \DIFaddbegin \DIFadd{draw
quantitative conclusions}\DIFaddend .

The observed brightening on SDR over Thessaloniki is dependent on SZA
(larger SZAs lead to stronger brightening). The trend is also dependent
on season, with winter showing the strongest statistically significant
trend of \DIFdelbegin \DIFdel{\(0.69\) and \(0.83\,\%/y\) }\DIFdelend \DIFaddbegin \DIFadd{\(0.7\) and \(0.76\,\%/y\) }\DIFaddend for all- and \DIFdelbegin \DIFdel{clear-skies}\DIFdelend \DIFaddbegin \DIFadd{cloudy-skies}\DIFaddend ,
respectively, in contrast to spring and summer. The trends for autumn
are also significant but smaller (\DIFdelbegin \DIFdel{\(0.43\) and \(0.37\,\%/y\) }\DIFdelend \DIFaddbegin \DIFadd{\(0.42\) and \(0.19\,\%/y\) }\DIFaddend for all-
and \DIFdelbegin \DIFdel{clear-skies}\DIFdelend \DIFaddbegin \DIFadd{cloudy-skies}\DIFaddend , respectively). \DIFdelbegin \DIFdel{Our findings are in agreement with other
studies for the region}\DIFdelend \DIFaddbegin \DIFadd{The trend for clear skies is largest in
winter (\(0.4\,\%/y\)) and negligible in spring, summer and autumn}\DIFaddend .

Using the CUSUMs of the monthly departures for all- and \DIFdelbegin \DIFdel{clear-skies}\DIFdelend \DIFaddbegin \DIFadd{cloudy-skies}\DIFaddend , we
observed \DIFdelbegin \DIFdel{periods }\DIFdelend \DIFaddbegin \DIFadd{a 20-year period starting around 2000 }\DIFaddend where the CUSUMs remain
relatively stable, with a steep decline before and a steep increase
after. \DIFdelbegin \DIFdel{This is an indication that the whole brightening effect does not follow a smooth development over
time}\DIFdelend \DIFaddbegin \DIFadd{The rather smooth course of the CUSUMs suggests that no important
change in the SDR pattern has occurred in the entire record}\DIFaddend .

Continued observations with a collocated pyrheliometer, which started in
2016, will allow us to further investigate the variability of solar
radiation at ground level in Thessaloniki. Also, additional data of
cloudiness, aerosols, atmospheric water vapour, etc., will allow better
attribution and quantification of the effects of these factors on SRD.

%%%%%%%%%%%%%%%%%%%%%%%%%%%%%%%%%%%%%%%%%%

\vspace{6pt}

%%%%%%%%%%%%%%%%%%%%%%%%%%%%%%%%%%%%%%%%%%
%% optional

% Only for the journal Methods and Protocols:
% If you wish to submit a video article, please do so with any other supplementary material.
% \supplementary{The following supporting information can be downloaded at: \linksupplementary{s1}, Figure S1: title; Table S1: title; Video S1: title. A supporting video article is available at doi: link.}

%%%%%%%%%%%%%%%%%%%%%%%%%%%%%%%%%%%%%%%%%%

\funding{This research received no external funding.}



\dataavailability{Data as daily sums are available through the WRDC
database \url{http://wrdc.mgo.rssi.ru}. One minute data are available on
request from the corresponding author. The data are not publicly
available for protection against unmonitored commercial use.}



%%%%%%%%%%%%%%%%%%%%%%%%%%%%%%%%%%%%%%%%%%
%% Optional

%% Only for journal Encyclopedia

\DIFdelbegin %DIFDELCMD < \abbreviations{Abbreviations}{
%DIFDELCMD < The following abbreviations are used in this manuscript:\\
%DIFDELCMD < 

%DIFDELCMD < \noindent
%DIFDELCMD < \begin{tabular}{@{}ll}
%DIFDELCMD < DNI & Direct beam/normal irradiance \\
%DIFDELCMD < CSid & Clear sky identification algorithm \\
%DIFDELCMD < CUSUM & Cumulative sum \\
%DIFDELCMD < SDR & Solar downward radiation \\
%DIFDELCMD < SZA & Solar zenith angle \\
%DIFDELCMD < \end{tabular}}
%DIFDELCMD < %%%
\DIFdelend \DIFaddbegin \abbreviations{Abbreviations}{
The following abbreviations are used in this manuscript:\\

\noindent
\begin{tabular}{@{}ll}
DNI & Direct beam/normal irradiance \\
ERA5 & ECMWF Reanalysis v5 \\
CSid & Clear sky identification algorithm \\
CUSUM & Cumulative sum \\
SDR & Solar downward radiation \\
SZA & Solar zenith angle \\
\end{tabular}}
\DIFaddend 

%%%%%%%%%%%%%%%%%%%%%%%%%%%%%%%%%%%%%%%%%%
%% Optional
\input{"appendix.tex"}
%%%%%%%%%%%%%%%%%%%%%%%%%%%%%%%%%%%%%%%%%%
\begin{adjustwidth}{-\extralength}{0cm}

%\printendnotes[custom] % Un-comment to print a list of endnotes


\reftitle{References}
\bibliography{manualreferences.bib}

% If authors have biography, please use the format below
%\section*{Short Biography of Authors}
%\bio
%{\raisebox{-0.35cm}{\includegraphics[width=3.5cm,height=5.3cm,clip,keepaspectratio]{Definitions/author1.pdf}}}
%{\textbf{Firstname Lastname} Biography of first author}
%
%\bio
%{\raisebox{-0.35cm}{\includegraphics[width=3.5cm,height=5.3cm,clip,keepaspectratio]{Definitions/author2.jpg}}}
%{\textbf{Firstname Lastname} Biography of second author}

%%%%%%%%%%%%%%%%%%%%%%%%%%%%%%%%%%%%%%%%%%
%% for journal Sci
%\reviewreports{\\
%Reviewer 1 comments and authors’ response\\
%Reviewer 2 comments and authors’ response\\
%Reviewer 3 comments and authors’ response
%}
%%%%%%%%%%%%%%%%%%%%%%%%%%%%%%%%%%%%%%%%%%
\PublishersNote{}
\end{adjustwidth}


\end{document}
