%  LaTeX support: latex@mdpi.com
%DIF LATEXDIFF DIFFERENCE FILE
%DIF DEL ../SUBMISSION_02/Submission/MDPI_submission.tex   Wed Dec 20 16:51:38 2023
%DIF ADD ./applsci-2743235.tex                             Tue Dec 26 16:22:30 2023
%  For support, please attach all files needed for compiling as well as the log file, and specify your operating system, LaTeX version, and LaTeX editor.

%=================================================================
% pandoc conditionals added to preserve backwards compatibility with previous versions of rticles

%DIF 7c7
%DIF < \documentclass[applsci,article,submit,moreauthors,pdftex]{Definitions/mdpi}
%DIF -------
\documentclass[applsci,article,accept,moreauthors,pdftex]{Definitions/mdpi} %DIF > 
%DIF -------


%% Some pieces required from the pandoc template
\setlist[itemize]{leftmargin=*,labelsep=5.8mm}
\setlist[enumerate]{leftmargin=*,labelsep=4.9mm}


%--------------------
% Class Options:
%--------------------

%---------
% article
%---------
% The default type of manuscript is "article", but can be replaced by:
% abstract, addendum, article, book, bookreview, briefreport, casereport, comment, commentary, communication, conferenceproceedings, correction, conferencereport, entry, expressionofconcern, extendedabstract, datadescriptor, editorial, essay, erratum, hypothesis, interestingimage, obituary, opinion, projectreport, reply, retraction, review, perspective, protocol, shortnote, studyprotocol, systematicreview, supfile, technicalnote, viewpoint, guidelines, registeredreport, tutorial
% supfile = supplementary materials

%----------
% submit
%----------
% The class option "submit" will be changed to "accept" by the Editorial Office when the paper is accepted. This will only make changes to the frontpage (e.g., the logo of the journal will get visible), the headings, and the copyright information. Also, line numbering will be removed. Journal info and pagination for accepted papers will also be assigned by the Editorial Office.

%DIF 31a31
 %DIF > 
%DIF -------
%------------------
% moreauthors
%------------------
% If there is only one author the class option oneauthor should be used. Otherwise use the class option moreauthors.

%---------
% pdftex
%---------
% The option pdftex is for use with pdfLaTeX. Remove "pdftex" for (1) compiling with LaTeX & dvi2pdf (if eps figures are used) or for (2) compiling with XeLaTeX.

%=================================================================
% MDPI internal commands - do not modify
\firstpage{1}
\makeatletter
\setcounter{page}{\@firstpage}
\makeatother
\pubvolume{1}
\issuenum{1}
\articlenumber{0}
%DIF 50c51
%DIF < \pubyear{2023}
%DIF -------
\pubyear{2024} %DIF > 
%DIF -------
\copyrightyear{2023}
%DIF 52-54c53-55
%DIF < %\externaleditor{Academic Editor: Firstname Lastname}
%DIF < \datereceived{ }
%DIF < \daterevised{ } % Comment out if no revised date
%DIF -------
\externaleditor{Academic Editor: } %DIF > 
\datereceived{13 November 2023 } %DIF > 
\daterevised{20 December 2023 } % Comment out if no revised date %DIF > 
%DIF -------
\dateaccepted{ }
\datepublished{ }
%\datecorrected{} % For corrected papers: "Corrected: XXX" date in the original paper.
%\dateretracted{} % For corrected papers: "Retracted: XXX" date in the original paper.
\hreflink{https://doi.org/} % If needed use \linebreak
%\doinum{}
%\pdfoutput=1 % Uncommented for upload to arXiv.org

%=================================================================
% Add packages and commands here. The following packages are loaded in our class file: fontenc, inputenc, calc, indentfirst, fancyhdr, graphicx, epstopdf, lastpage, ifthen, float, amsmath, amssymb, lineno, setspace, enumitem, mathpazo, booktabs, titlesec, etoolbox, tabto, xcolor, colortbl, soul, multirow, microtype, tikz, totcount, changepage, attrib, upgreek, array, tabularx, pbox, ragged2e, tocloft, marginnote, marginfix, enotez, amsthm, natbib, hyperref, cleveref, scrextend, url, geometry, newfloat, caption, draftwatermark, seqsplit
% cleveref: load \crefname definitions after \begin{document}

%=================================================================
% Please use the following mathematics environments: Theorem, Lemma, Corollary, Proposition, Characterization, Property, Problem, Example, ExamplesandDefinitions, Hypothesis, Remark, Definition, Notation, Assumption
%% For proofs, please use the proof environment (the amsthm package is loaded by the MDPI class).

%=================================================================
% Full title of the paper (Capitalized)
%DIF 73-74c74-83
%DIF < \Title{Trends from 30-year observations of downward solar irradiance in
%DIF < Thessaloniki, Greece}
%DIF -------
\Title{Trends from 30-Year Observations of Downward Solar Irradiance in %DIF > 
Thessaloniki, \colorbox{green}{Greece}}%MDPI: Important note: %DIF > 
%1. The paper was edited by our English editor, please check the whole text and confirm if your meaning is retained. %DIF > 
%2. Do not delete any comment we left for you and reply to each comment so that we can understand your meaning clearly. %DIF > 
%3. Please directly correct on this version. If you need to revise somewhere in your paper, please highlight the revisions and track changes to make us known. %DIF > 
%4. Please finish the proofreading based on this version. %DIF > 
%5.Please confirm and revise all the comments with “Confirmed”, “OK”, “Revised”, “It should be italic”; “I confirm”; “I confirm xx is correct”; “I have checked and revised all.” , etc.“ %DIF > 
%6. Please note that at this stage (the manuscript has been accepted in the current form), we will not accept authorship or content changes to the main text. %DIF > 
%(Thank you for your cooperation in advance.) %DIF > 
 %DIF > 
%DIF -------

% MDPI internal command: Title for citation in the left column
%DIF 77-78c86-87
%DIF < \TitleCitation{Trends from 30-year observations of downward solar
%DIF < irradiance in Thessaloniki, Greece}
%DIF -------
\TitleCitation{Trends from 30-Year Observations of Downward Solar Irradiance in %DIF > 
Thessaloniki, Greece} %DIF > 
%DIF -------

% Author Orchid ID: enter ID or remove command
%\newcommand{\orcidauthorA}{0000-0000-0000-000X} % Add \orcidA{} behind the author's name
%\newcommand{\orcidauthorB}{0000-0000-0000-000X} % Add \orcidB{} behind the author's name


% Authors, for the paper (add full first names)
\Author{Athanasios
%DIF 87-88c96-97
%DIF < Natsis$^{1}$\href{https://orcid.org/0000-0002-5199-4119}
%DIF < {\orcidicon}, Alkiviadis Bais$^{1,*}$, Charikleia Meleti$^{1}$}
%DIF -------
Natsis \href{https://orcid.org/0000-0002-5199-4119} %DIF > 
{\orcidicon}, Alkiviadis Bais * and Charikleia Meleti} %DIF > 
%DIF -------


%\longauthorlist{yes}


% MDPI internal command: Authors, for metadata in PDF
\AuthorNames{Athanasios Natsis, Alkiviadis Bais, Charikleia Meleti}

% MDPI internal command: Authors, for citation in the left column
%\AuthorCitation{Lastname, F.; Lastname, F.; Lastname, F.}
% If this is a Chicago style journal: Lastname, Firstname, Firstname Lastname, and Firstname Lastname.
\AuthorCitation{Natsis, A.; Bais, A.; Meleti, C.}

% Affiliations / Addresses (Add [1] after \address if there is only one affiliation.)
%DIF 103-108c112-116
%DIF < \address{%
%DIF < $^{1}$ \quad Aristotle University of Thessaloniki - Laboratory of
%DIF < Atmospheric Physics, Campus Box 149, 54124 Thessaloniki,
%DIF < Greece; \href{mailto:natsisphysicist@gmail.com}{\nolinkurl{natsisphysicist@gmail.com}}
%DIF < (A.N.); \href{mailto:abais@auth.gr}{\nolinkurl{abais@auth.gr}} (A.B.);
%DIF < \href{mailto:meleti@auth.gr}{\nolinkurl{meleti@auth.gr}} (C.M.)\\
%DIF -------
\address[1]{Laboratory of %DIF > 
Atmospheric Physics, Aristotle University of Thessaloniki, Campus Box 149, \mbox{54124 Thessaloniki, %DIF > 
Greece;} {{natsisphysicist@gmail.com}} %DIF > 
(A.N.);  %DIF > 
{{meleti@auth.gr}} (C.M.)\\ %DIF > 
%DIF -------
}

% Contact information of the corresponding author
%DIF 112c120
%DIF < \corres{Correspondence: \href{mailto:abais@auth.gr}{\nolinkurl{abais@auth.gr}}}
%DIF -------
\corres{\hangafter=1 \hangindent=1.0em \hspace{-1em}Correspondence: {{abais@auth.gr}}} %DIF > 
%DIF -------

% Current address and/or shared authorship








% The commands \thirdnote{} till \eighthnote{} are available for further notes

% Simple summary

%\conference{} % An extended version of a conference paper

% Abstract (Do not insert blank lines, i.e. \\)
\abstract{The shortwave downward solar irradiance (SDR) is an important
%DIF 131c139
%DIF < factor that drives climate processes, energy production and can affect
%DIF -------
factor that drives climate processes and energy production and can affect %DIF > 
%DIF -------
all living organisms. Observations of SDR at different locations around
the world with different environmental characteristics have been used to
investigate its long-term variability and trends at different time
%DIF 135c143
%DIF < scales. Periods of positive trends are referred as brightening periods
%DIF -------
scales. Periods of positive trends are referred to as brightening periods %DIF > 
%DIF -------
and of negative trends as dimming periods. In this study we have used 30
years of pyranometer data in Thessaloniki, Greece, to investigate the
variability of SDR under three types of sky conditions (clear-, cloudy-
and all-sky). The clear-sky data were identified by applying a cloud
screening algorithm. We have found a positive trend of
%DIF 141-142c149-150
%DIF < \(0.38\,\%/\text{year}\) for all-sky, \(\sim 0.1\,\%/\text{year}\) for
%DIF < clear-sky, and \(0.41\,\%/\text{year}\) for cloudy conditions. The
%DIF -------
\(0.38\%/\text{year}\) for all-sky, $\sim$0.1\%/{year} for %DIF > 
clear-sky, and \(0.41\%/\text{year}\) for cloudy conditions. The %DIF > 
%DIF -------
consistency of these trends, their seasonal variability, and the effect
of the solar zenith angle have also been investigated. Under all three
%DIF 145-153c153-161
%DIF < sky categories the SDR trend is stronger in winter with \(0.7\), \(0.4\)
%DIF < and \(0.76\,\%/\text{year}\), respectively for all-, clear- and
%DIF < cloudy-skies. The next larger seasonal trends are in automn \(0.42\) and
%DIF < \(0.19\,\%/\text{year}\), for all- and cloudy-skies respectively. The
%DIF < rest seasonal trends are significant smaller, close to zero, with a
%DIF < negative values in summer, for clear- and cloudy-skies. The SDR trend is
%DIF < increasing with increasing solar zenith angle, except under cloudy skies
%DIF < where the trend is highly variable and close to zero. Finally, we are
%DIF < discussing shorter-term variations in SDR anomalies by examining the
%DIF -------
sky categories, the SDR trend is stronger in winter, with \(0.7\), \(0.4\), %DIF > 
and \(0.76\%/\text{year}\), respectively, for all-, clear-, and %DIF > 
cloudy-sky conditions. The next larger seasonal trends are in autumn---\(0.42\) and %DIF > 
\(0.19\%/\text{year}\), for all and cloudy skies, respectively. The %DIF > 
rest of the seasonal trends are significant smaller, close to zero, with a %DIF > 
negative values in summer, for clear and cloudy skies. The SDR trend is %DIF > 
increasing with increasing solar zenith angle, except under cloudy skies, %DIF > 
where the trend is highly variable and close to zero. Finally, we %DIF > 
discuss shorter-term variations in SDR anomalies by examining the %DIF > 
%DIF -------
patterns of the cumulative sums of monthly anomalies from the
climatological mean, both before and after removing the long-term
trend.}


% Keywords
%DIF 160-161c168-169
%DIF < \keyword{GHI; SDR; solar radiation; Solar Brigthening/Dimming; aerosols;
%DIF < clouds.}
%DIF -------
\keyword{GHI; SDR; solar radiation; solar brigthening/dimming; aerosols; %DIF > 
clouds} %DIF > 
%DIF -------

% The fields PACS, MSC, and JEL may be left empty or commented out if not applicable
%\PACS{J0101}
%\MSC{}
%\JEL{}

%%%%%%%%%%%%%%%%%%%%%%%%%%%%%%%%%%%%%%%%%%
% Only for the journal Diversity
%\LSID{\url{http://}}

%%%%%%%%%%%%%%%%%%%%%%%%%%%%%%%%%%%%%%%%%%
% Only for the journal Applied Sciences

%%%%%%%%%%%%%%%%%%%%%%%%%%%%%%%%%%%%%%%%%%

%%%%%%%%%%%%%%%%%%%%%%%%%%%%%%%%%%%%%%%%%%
% Only for the journal Data



%%%%%%%%%%%%%%%%%%%%%%%%%%%%%%%%%%%%%%%%%%
% Only for the journal Toxins


%%%%%%%%%%%%%%%%%%%%%%%%%%%%%%%%%%%%%%%%%%
% Only for the journal Encyclopedia


%%%%%%%%%%%%%%%%%%%%%%%%%%%%%%%%%%%%%%%%%%
% Only for the journal Advances in Respiratory Medicine
%\addhighlights{yes}
%\renewcommand{\addhighlights}{%

%\noindent This is an obligatory section in “Advances in Respiratory Medicine”, whose goal is to increase the discoverability and readability of the article via search engines and other scholars. Highlights should not be a copy of the abstract, but a simple text allowing the reader to quickly and simplified find out what the article is about and what can be cited from it. Each of these parts should be devoted up to 2~bullet points.\vspace{3pt}\\
%\textbf{What are the main findings?}
% \begin{itemize}[labelsep=2.5mm,topsep=-3pt]
% \item First bullet.
% \item Second bullet.
% \end{itemize}\vspace{3pt}
%\textbf{What is the implication of the main finding?}
% \begin{itemize}[labelsep=2.5mm,topsep=-3pt]
% \item First bullet.
% \item Second bullet.
% \end{itemize}
%}


%%%%%%%%%%%%%%%%%%%%%%%%%%%%%%%%%%%%%%%%%%


% tightlist command for lists without linebreak
\providecommand{\tightlist}{%
  \setlength{\itemsep}{0pt}\setlength{\parskip}{0pt}}



\usepackage{subcaption}
\captionsetup[sub]{position=bottom, labelfont={bf, small, stretch=1.17}, labelsep=space, textfont={small, stretch=1.17}, aboveskip=6pt,  belowskip=-6pt, singlelinecheck=off, justification=justified}
\usepackage{placeins}
\usepackage{longtable}
\usepackage{booktabs}
\usepackage{array}
\usepackage{multirow}
\usepackage{wrapfig}
\usepackage{float}
\usepackage{colortbl}
\usepackage{pdflscape}
\usepackage{tabu}
\usepackage{threeparttable}
\usepackage{threeparttablex}
\usepackage[normalem]{ulem}
\usepackage{makecell}
\usepackage{xcolor}
%DIF PREAMBLE EXTENSION ADDED BY LATEXDIFF
%DIF UNDERLINE PREAMBLE %DIF PREAMBLE
\RequirePackage[normalem]{ulem} %DIF PREAMBLE
\RequirePackage{color}\definecolor{RED}{rgb}{1,0,0}\definecolor{BLUE}{rgb}{0,0,1} %DIF PREAMBLE
\providecommand{\DIFadd}[1]{{\protect\color{blue}\uwave{#1}}} %DIF PREAMBLE
\providecommand{\DIFdel}[1]{{\protect\color{red}\sout{#1}}}                      %DIF PREAMBLE
%DIF SAFE PREAMBLE %DIF PREAMBLE
\providecommand{\DIFaddbegin}{} %DIF PREAMBLE
\providecommand{\DIFaddend}{} %DIF PREAMBLE
\providecommand{\DIFdelbegin}{} %DIF PREAMBLE
\providecommand{\DIFdelend}{} %DIF PREAMBLE
\providecommand{\DIFmodbegin}{} %DIF PREAMBLE
\providecommand{\DIFmodend}{} %DIF PREAMBLE
%DIF FLOATSAFE PREAMBLE %DIF PREAMBLE
\providecommand{\DIFaddFL}[1]{\DIFadd{#1}} %DIF PREAMBLE
\providecommand{\DIFdelFL}[1]{\DIFdel{#1}} %DIF PREAMBLE
\providecommand{\DIFaddbeginFL}{} %DIF PREAMBLE
\providecommand{\DIFaddendFL}{} %DIF PREAMBLE
\providecommand{\DIFdelbeginFL}{} %DIF PREAMBLE
\providecommand{\DIFdelendFL}{} %DIF PREAMBLE
%DIF COLORLISTINGS PREAMBLE %DIF PREAMBLE
\RequirePackage{listings} %DIF PREAMBLE
\RequirePackage{color} %DIF PREAMBLE
\lstdefinelanguage{DIFcode}{ %DIF PREAMBLE
%DIF DIFCODE_UNDERLINE %DIF PREAMBLE
  moredelim=[il][\color{red}\sout]{\%DIF\ <\ }, %DIF PREAMBLE
  moredelim=[il][\color{blue}\uwave]{\%DIF\ >\ } %DIF PREAMBLE
} %DIF PREAMBLE
\lstdefinestyle{DIFverbatimstyle}{ %DIF PREAMBLE
	language=DIFcode, %DIF PREAMBLE
	basicstyle=\ttfamily, %DIF PREAMBLE
	columns=fullflexible, %DIF PREAMBLE
	keepspaces=true %DIF PREAMBLE
} %DIF PREAMBLE
\lstnewenvironment{DIFverbatim}{\lstset{style=DIFverbatimstyle}}{} %DIF PREAMBLE
\lstnewenvironment{DIFverbatim*}{\lstset{style=DIFverbatimstyle,showspaces=true}}{} %DIF PREAMBLE
%DIF END PREAMBLE EXTENSION ADDED BY LATEXDIFF

\begin{document}



%%%%%%%%%%%%%%%%%%%%%%%%%%%%%%%%%%%%%%%%%%

\hypertarget{introduction}{%
\section{Introduction}\label{introduction}}

The shortwave downward solar irradiance (SDR) at Earth's surface plays a
significant role \DIFdelbegin \DIFdel{, }\DIFdelend on its climate. Changes \DIFdelbegin \DIFdel{of }\DIFdelend \DIFaddbegin \DIFadd{in }\DIFaddend the SDR can be related to
changes \DIFdelbegin \DIFdel{on }\DIFdelend \DIFaddbegin \DIFadd{in }\DIFaddend Earth's energy budget, the mechanisms of climate change, and
water and carbon cycles \citep{Wild2009}. It can also affect solar and
agricultural production \DIFdelbegin \DIFdel{, }\DIFdelend and all living organisms. Studies of SDR
variability have identified some distinct SDR trends \DIFdelbegin \DIFdel{on }\DIFdelend \DIFaddbegin \DIFadd{in }\DIFaddend different
regions of the world \DIFdelbegin \DIFdel{on }\DIFdelend \DIFaddbegin \DIFadd{in }\DIFaddend different time periods. The term `brightening'
is generally used to describe periods of positive SDR trend, and
`dimming' for \DIFaddbegin \DIFadd{periods of }\DIFaddend negative trend \citep{Wild2009}. There are many cases in
the \DIFdelbegin \DIFdel{long term }\DIFdelend \DIFaddbegin \DIFadd{long-term }\DIFaddend records of irradiance \DIFdelbegin \DIFdel{, }\DIFdelend showing a systematic change in the
magnitude of the trend, occurring roughly in the last decades of the
20th century. \DIFdelbegin \DIFdel{On }\DIFdelend \DIFaddbegin \DIFadd{At }\DIFaddend multiple stations in China, a dimming period was
reported until about 2000, followed by a brightening period
\citep{Yang2021}. A similar pattern was identified, with a breaking
point around 1980, for stations in Central Europe \citep{Wild2021} and
Brazil\DIFaddbegin \DIFadd{~}\DIFaddend \citep{Yamasoe2021}. On global scale, an \DIFdelbegin \DIFdel{Artificial Intelligence
}\DIFdelend \DIFaddbegin \DIFadd{artificial intelligence
}\DIFaddend aided spatial analysis on \DIFaddbegin \DIFadd{the }\DIFaddend continental level with data from multiple
stations reached similar conclusions for these regions and for the
global trend \citep{Yuan2021}.

There is a consensus among researchers that the major factor affecting
the variability of SDR attenuation is the interactions of solar
radiation with atmospheric aerosols and clouds. Those interactions,
among other factors, have been analyzed with models
\citep{Li2016, Samset2018}, showing the existence of feedback mechanisms
between the two. Similar findings have been shown from the analysis of
observations at other \DIFdelbegin \DIFdel{locations \mbox{%DIFAUXCMD
\citep[ and references
therein]{Schwarz2020, Ohvril2009, Zerefos2009, Xia2007}}\hskip0pt%DIFAUXCMD
}\DIFdelend \DIFaddbegin \hl{locations} %DIF > MDPI: We revised citation. Please confirm.
 \DIFadd{\mbox{%DIFAUXCMD
\citep{Schwarz2020, Ohvril2009, Zerefos2009, Xia2007} }\hskip0pt%DIFAUXCMD
}[\DIFadd{and references
therein}]\DIFaddend . In the
Mediterranean region\DIFaddbegin \DIFadd{, }\DIFaddend aerosols have been recognized as an important
factor affecting the penetration of solar radiation \DIFdelbegin \DIFdel{to }\DIFdelend \DIFaddbegin \DIFadd{at }\DIFaddend the surface
\citep{Fountoulakis2016, Siomos2018, Gkikas2013, Lozano2021}. These
studies investigated the long-term trend in aerosol optical depth, which
has been found to decrease in the last three decades, the transport and
composition of aerosols, and their radiative effects.

Due to the significant spatial and temporal variability of the trends
\DIFdelbegin \DIFdel{,
}\DIFdelend and the contributing factors, there is a constant need to monitor and
investigate SDR \DIFdelbegin \DIFdel{in }\DIFdelend \DIFaddbegin \DIFadd{at }\DIFaddend different sites in order to estimate the degree of
variability \DIFdelbegin \DIFdel{, }\DIFdelend and its relation to the local conditions. In this study, we
examine the trends of SDR \DIFdelbegin \DIFdel{, }\DIFdelend using ground-based measurements at
Thessaloniki, Greece, for the period \DIFaddbegin \DIFadd{from }\DIFaddend 1993 to 2023. We re-evaluated and
extended the dataset used by \citet{Bais2013}, we applied a different
algorithm for the identification of clear-/cloud-sky instances
\citep{Reno2016, Reno2012a}, and we derived the SDR trends for the
period of study \DIFdelbegin \DIFdel{, }\DIFdelend under different sky conditions (all-sky, clear-sky\DIFdelbegin \DIFdel{and
cloud-sky}\DIFdelend \DIFaddbegin \DIFadd{, and
cloudy-sky}\DIFaddend ). Finally, we investigated the dependence of the trends on
solar zenith angle and season.

\DIFdelbegin %DIFDELCMD < \hypertarget{data-and-methodology}{%
%DIFDELCMD < \section{Data and methodology}\label{data-and-methodology}}
%DIFDELCMD < %%%
\DIFdelend \DIFaddbegin \hypertarget{data-and-methodology}{%
\section{Data and Methodology}\label{data-and-methodology}}
\DIFaddend 

The SDR data were measured with a Kipp \& Zonen CM-21 pyranometer
operating continuously at the Laboratory of Atmospheric Physics of the
Aristotle University of Thessaloniki (\(40^\circ\,38'\,\)N,
\(22^\circ\,57'\,\)E, \(80\,\)m~a.s.l.). Here\DIFdelbegin \DIFdel{we use }\DIFdelend \DIFaddbegin \DIFadd{, we used }\DIFaddend data for the period
from \DIFdelbegin \DIFdel{1993-04-13 to 2023-04-13. }\DIFdelend \DIFaddbegin \DIFadd{13~April 1993 to 13 April 2023. }\DIFaddend The monitoring site \DIFdelbegin \DIFdel{is }\DIFdelend \DIFaddbegin \DIFadd{was }\DIFaddend located near the
city center\DIFaddbegin \DIFadd{, }\DIFaddend thus we expect that measurements \DIFdelbegin \DIFdel{are }\DIFdelend \DIFaddbegin \DIFadd{were }\DIFaddend affected by the urban
environment, mainly by aerosols. During the study period, the
pyranometer \DIFdelbegin \DIFdel{has been }\DIFdelend \DIFaddbegin \DIFadd{was }\DIFaddend independently calibrated three times at the
Meteorologisches Observatorium Lindenberg, DWD, verifying that the
stability of the instrument's sensitivity \DIFdelbegin \DIFdel{is }\DIFdelend \DIFaddbegin \DIFadd{was }\DIFaddend better than \(0.7\%\)
relative to the initial calibration by the manufacturer. Along with SDR,
the direct beam radiation (DNI) \DIFdelbegin \DIFdel{is }\DIFdelend \DIFaddbegin \DIFadd{was }\DIFaddend also measured with a collocated Kipp
\& Zonen CHP-1 pyrheliometer since \DIFdelbegin \DIFdel{2016-04-01. }\DIFdelend \DIFaddbegin \DIFadd{1 April 2016. }\DIFaddend The DNI data were used as
auxiliary data to support the selection of appropriate thresholds in the
clear-sky identification algorithm (CSid), which is discussed in \DIFdelbegin \DIFdel{sect.
}\DIFdelend \DIFaddbegin \DIFadd{Section
}\DIFaddend \ref{CDIDalgorithm}. It is noted that the limited dataset of DNI was not
used for the identification of \DIFdelbegin \DIFdel{clear sky }\DIFdelend \DIFaddbegin \DIFadd{clear-sky }\DIFaddend cases in the entire SDR series
to avoid any selection bias due to \DIFaddbegin \DIFadd{the }\DIFaddend unequal length of the two datasets.
There are four distinct steps in the creation of the dataset analyzed
here: \DIFaddbegin \DIFadd{(}\DIFaddend a)~the acquisition of radiation measurements from the sensors,
\DIFaddbegin \DIFadd{(}\DIFaddend b)~the data quality check, \DIFaddbegin \DIFadd{(}\DIFaddend c)~the identification of ``clear sky''
conditions from the SDR data, and \DIFaddbegin \DIFadd{(}\DIFaddend d)~the aggregation of data and trend
analysis.

For the acquisition of radiometric data, the signal of the pyranometer
\DIFdelbegin \DIFdel{is }\DIFdelend \DIFaddbegin \DIFadd{was }\DIFaddend sampled at a rate of \(1\,\text{Hz}\). The mean and the standard
deviation of these samples \DIFdelbegin \DIFdel{are }\DIFdelend \DIFaddbegin \DIFadd{were }\DIFaddend calculated and recorded every minute. The
measurements \DIFdelbegin \DIFdel{are }\DIFdelend \DIFaddbegin \DIFadd{were }\DIFaddend corrected for the zero offset (``dark signal'' in
volts), which \DIFdelbegin \DIFdel{is }\DIFdelend \DIFaddbegin \DIFadd{was }\DIFaddend calculated by averaging all measurements recorded for a
period of \(3\,\text{h}\), before (morning) or after (evening) the Sun
reaches an elevation angle of \(-10^\circ\). The signal \DIFdelbegin \DIFdel{is }\DIFdelend \DIFaddbegin \DIFadd{was }\DIFaddend converted to
irradiance using a ramped value of the instrument's sensitivity between
subsequent calibrations.

A manual screening was performed \DIFdelbegin \DIFdel{, }\DIFdelend to remove inconsistent and erroneous
recordings that can occur stochastically or systematically \DIFdelbegin \DIFdel{, }\DIFdelend during the
continuous operation of the instruments. The manual screening \DIFdelbegin \DIFdel{is }\DIFdelend \DIFaddbegin \DIFadd{was }\DIFaddend aided
by a radiation data quality assurance procedure, adjusted for the site,
which \DIFdelbegin \DIFdel{is }\DIFdelend \DIFaddbegin \DIFadd{was }\DIFaddend based on the methods of Long and
Shi~\citep{Long2006, Long2008a}. Thus, problematic recordings have been
excluded from further processing. Although it is impossible to detect
all false data, the large number of available data, and the aggregation
scheme we used, ensures the quality of the radiation measurements used
in this study.

Due to the significant measurement uncertainty when the Sun is near the
horizon, we have excluded all measurements with solar zenith angle (SZA)
greater than \(85^\circ\). Moreover, due to obstructions around the site
(hills and buildings) \DIFdelbegin \DIFdel{which }\DIFdelend \DIFaddbegin \DIFadd{that }\DIFaddend block the direct irradiance, we excluded
data with azimuth angle in the range \(58^{\circ}\)\DIFdelbegin \DIFdel{- }\DIFdelend \DIFaddbegin \DIFadd{--}\DIFaddend \(120^{\circ}\)
and with SZA greater than \(78^{\circ}\). To make the measurements
comparable throughout the dataset, we adjusted all one-minute data to
the mean \DIFdelbegin \DIFdel{Sun - Earth }\DIFdelend \DIFaddbegin \DIFadd{Sun--Earth }\DIFaddend distance. Subsequently, we adjusted all
measurements to the \DIFdelbegin \DIFdel{Total Solar Irradiance }\DIFdelend \DIFaddbegin \DIFadd{total solar irradiance }\DIFaddend (TSI) at \(1\,\text{au}\) \DIFdelbegin \DIFdel{, }\DIFdelend in
order to compensate for the Sun's intensity variability \DIFdelbegin \DIFdel{, }\DIFdelend using a time
series of satellite TSI observations. The TSI data we used are part of
the \DIFdelbegin \DIFdel{``}\DIFdelend \DIFaddbegin \DIFadd{`}\DIFaddend NOAA Climate Data Record of Total Solar Irradiance\DIFdelbegin \DIFdel{'' }\DIFdelend \DIFaddbegin \DIFadd{' }\DIFaddend dataset
\citep{Coddington2005}. The initial daily values of this dataset were
interpolated to match the time step of our measurements.

In order to estimate the effect of the sky conditions on the \DIFdelbegin \DIFdel{long term
}\DIFdelend \DIFaddbegin \DIFadd{long-term
}\DIFaddend variability of SDR, we created three datasets by characterizing each
one-minute measurement with a corresponding sky-condition flag (i.e.,
all-sky, clear-sky\DIFaddbegin \DIFadd{, }\DIFaddend and cloudy-sky). To identify the \DIFdelbegin \DIFdel{clear-cases }\DIFdelend \DIFaddbegin \DIFadd{clear cases, }\DIFaddend we used
the method proposed by \citet{Reno2016}, which requires the definition
of some site specific parameters. These parameters were determined by an
iterative process, as the original authors proposed\DIFaddbegin \DIFadd{, }\DIFaddend and are discussed in
the next section.

We note that all methods have some subjectivity in the definition of
clear or cloudy sky cases. As a result, the details of the definition
are site specific, and they rely on a combination of thresholds and
comparisons with ideal radiation models and statistical analysis of
different signal metrics. The CSid algorithm was calibrated with the
main focus \DIFdelbegin \DIFdel{, }\DIFdelend to identify the presence of clouds. Despite the fine-tuning
of the procedure, in a few marginal cases\DIFaddbegin \DIFadd{, }\DIFaddend false positive or false
negative results were identified by manual inspection. However, due to
their small number, they \DIFdelbegin \DIFdel{cannot }\DIFdelend \DIFaddbegin \DIFadd{did not }\DIFaddend affect the final results of the study.
For completeness, we provide below a brief overview of the CSid
algorithm, along with the \DIFdelbegin \DIFdel{site specific }\DIFdelend \DIFaddbegin \DIFadd{site-specific }\DIFaddend thresholds.

\DIFdelbegin %DIFDELCMD < \hypertarget{CDIDalgorithm}{%
%DIFDELCMD < \subsection{The clear sky identification
%DIFDELCMD < algorithm}\label{CDIDalgorithm}}
%DIFDELCMD < %%%
\DIFdelend \DIFaddbegin \hypertarget{CDIDalgorithm}{%
\subsection{The Clear Sky Identification
Algorithm}\label{CDIDalgorithm}}
\DIFaddend 

To calculate the reference clear-sky \(\text{SDR}_\text{CSref}\)\DIFaddbegin \DIFadd{, }\DIFaddend we used
the \(\text{SDR}_\text{Haurwitz}\) derived by the radiation model of
\citet{Haurwitz1945} (\DIFdelbegin \DIFdel{Eq.~\ref{eq:hau}}\DIFdelend \DIFaddbegin \DIFadd{Equation~\eqref{eq:hau}}\DIFaddend ), adjusted for our site:
\begin{equation}
\text{SDR}_\text{Haurwitz} = 1098 \times \cos(\theta) \times \exp \left( \frac{ - 0.059}{\cos(\theta)} \right) \label{eq:hau}
\end{equation} where \(\theta\) is the SZA.

The adjustment was made with a factor \(a\) (\DIFdelbegin \DIFdel{Eq.~\ref{eq:ahau}}\DIFdelend \DIFaddbegin \DIFadd{Equation~\eqref{eq:ahau}}\DIFaddend ), which
was estimated through an iterative optimization process, as described by
\citet{Long2000} and \citet{Reno2016}. The target of the optimization
was the minimization of a function \(f(a)\) (\DIFdelbegin \DIFdel{Eq.~\ref{eq:minf}}\DIFdelend \DIFaddbegin \DIFadd{Equation~\eqref{eq:minf}}\DIFaddend ) and was
accomplished with the algorithmic function \DIFdelbegin \DIFdel{``optimise''}\DIFdelend \DIFaddbegin \DIFadd{`optimise'}\DIFaddend , which is an
implementation based on the work of \citet{Brent1973}, from the library
\DIFdelbegin \DIFdel{``stats'' }\DIFdelend \DIFaddbegin \DIFadd{`stats' }\DIFaddend of the R programming language \citep{RCT2023}.
\begin{equation}
f(a) = \frac{1}{n}\sum_{i=1}^{n} ( \text{SDR}_{\text{CSid},i} - a \times \text{SDR}_{\text{testCSref},i} )^2 \label{eq:minf}
\end{equation} where \DIFdelbegin \DIFdel{: }\DIFdelend \(n\) is the total number of daylight data,
\(\text{SDR}_{\text{CSid},i}\) are the data identified as clear-sky by
CSid, \(a\) is a site-specific adjustment factor, and
\(\text{SDR}_{\text{testCSref},i}\) is the SDR derived by any of the
tested clear-sky radiation models.

The optimization and the selection of the \DIFdelbegin \DIFdel{clear sky reference model , }\DIFdelend \DIFaddbegin \DIFadd{clear-sky reference model }\DIFaddend was
performed on SDR observations for the period \DIFdelbegin \DIFdel{2016 - 2021. }\DIFdelend \DIFaddbegin \DIFadd{2016--2021. }\DIFaddend During the
optimization, eight simple \DIFdelbegin \DIFdel{clear sky }\DIFdelend \DIFaddbegin \DIFadd{clear-sky }\DIFaddend radiation models were tested
(namely, \DIFdelbegin \DIFdel{Daneshyar-Paltridge-Proctor, Kasten-Czeplak}\DIFdelend \DIFaddbegin \DIFadd{Daneshyar--Paltridge--Proctor, Kasten--Czeplak}\DIFaddend , Haurwitz,
\DIFdelbegin \DIFdel{Berger-Duffie, Adnot-Bourges-Campana-Gicquel, Robledo-Soler, Kastenand
Ineichen-Perez) , }\DIFdelend \DIFaddbegin \DIFadd{Berger--Duffie, Adnot--Bourges--Campana--Gicquel, Robledo--Soler, Kasten, and
Ineichen--Perez) }\DIFaddend with a wide range of factors. These models are
described in more detail by \citet{Reno2012} and are evaluated by
\citet{Reno2016}. We found \DIFdelbegin \DIFdel{, }\DIFdelend that Haurwitz's model, adjusted with the
factor \(a = 0.965\)\DIFaddbegin \DIFadd{, }\DIFaddend yields one of the lowest root mean squared errors
(RMSE), \DIFdelbegin \DIFdel{while }\DIFdelend \DIFaddbegin \DIFadd{and }\DIFaddend the procedure manages to \DIFdelbegin \DIFdel{characterize successfully }\DIFdelend \DIFaddbegin \DIFadd{successfully characterize }\DIFaddend the
majority of the data. Thus, our clear sky reference is derived by
\DIFdelbegin \DIFdel{the
Eq.~\ref{eq:ahau}}\DIFdelend \DIFaddbegin \DIFadd{Equation~\eqref{eq:ahau}}\DIFaddend : \begin{equation}
\text{SDR}_\text{CSref} = a \times \text{SDR}_\text{Haurwitz} = 0.965 \times 1098 \times \cos(\theta) \times \exp \left( \frac{ - 0.057}{\cos(\theta)} \right) \label{eq:ahau}
\end{equation}

The criteria that were used to identify whether a measurement was taken
under clear-sky conditions are presented below. A data point is flagged
as \DIFdelbegin \DIFdel{``}\DIFdelend \DIFaddbegin \DIFadd{`}\DIFaddend clear-sky\DIFdelbegin \DIFdel{'' }\DIFdelend \DIFaddbegin \DIFadd{' }\DIFaddend if all criteria are satisfied; otherwise\DIFaddbegin \DIFadd{, }\DIFaddend it is
considered as \DIFdelbegin \DIFdel{``}\DIFdelend \DIFaddbegin \DIFadd{`}\DIFaddend cloud-sky\DIFdelbegin \DIFdel{''}\DIFdelend \DIFaddbegin \DIFadd{'}\DIFaddend . Each criterion was applied for a running
window of \(11\) consecutive one-minute measurements, and the
characterization was assigned to the central datum of the window. Each
parameter was calculated from the observations in comparison to the
reference \DIFdelbegin \DIFdel{clear sky }\DIFdelend \DIFaddbegin \DIFadd{clear-sky }\DIFaddend model. The allowable range of variation is defined
by the model-derived value of the parameter multiplied by a factor plus
an offset. The factors and the offsets were determined empirically \DIFdelbegin \DIFdel{, }\DIFdelend by
manually inspecting each filter's performance on selected days and
adjusting them accordingly during an iterative process. The criteria are
listed below, together with the range of values within which the
respective parameter should fall in order to raise the clear-sky flag:

\begin{enumerate}
\def\DIFdelbegin %DIFDELCMD < \labelenumi{\alph{enumi})}
%DIFDELCMD < %%%
\DIFdelend \DIFaddbegin \labelenumi{(\alph{enumi})}
\DIFaddend \tightlist
\item
  \DIFdelbegin \DIFdel{Mean }\DIFdelend \DIFaddbegin \hl{Mean} %DIF > MDPI: We removed italics of units. Please confirm.
 \DIFaddend of the measured \(\overline{\text{SDR}}_i\) (\DIFdelbegin \DIFdel{Eq.
  \ref{eq:MeanVIP}). }\DIFdelend \DIFaddbegin \DIFadd{Equation~\eqref{eq:MeanVIP}): }\DIFaddend \begin{equation}
  0.91 \times \overline{\text{SDR}}_{\text{CSref},i} - 20\,\DIFdelbegin \DIFdel{Wm}\DIFdelend \DIFaddbegin \DIFadd{\text{Wm}}\DIFaddend ^{-2}
  < \overline{\text{SDR}}_i <
  1.095 \times \overline{\text{SDR}}_{\text{CSref},i} + 30\,\DIFdelbegin \DIFdel{Wm}\DIFdelend \DIFaddbegin \DIFadd{\text{Wm}}\DIFaddend ^{-2}
  \label{eq:MeanVIP}
  \end{equation}
\item
  Maximum measured value \(M_{i}\) (\DIFdelbegin \DIFdel{Eq.~\ref{eq:MaxVIP}).
  }\DIFdelend \DIFaddbegin \DIFadd{Equation~\eqref{eq:MaxVIP}):
  }\DIFaddend \begin{equation}
  1 \times M_{\text{CSref},i} - 75\,\DIFdelbegin \DIFdel{Wm}\DIFdelend \DIFaddbegin \DIFadd{\text{Wm}}\DIFaddend ^{-2}
  < M_{i} <
  1 \times M_{\text{CSref},i} + 75\,\DIFdelbegin \DIFdel{Wm}\DIFdelend \DIFaddbegin \DIFadd{\text{Wm}}\DIFaddend ^{-2}
  \label{eq:MaxVIP}
  \end{equation}
\item
  Length \(L_i\) of the sequential line segments, connecting the points
  of the \(11\) SDR values (\DIFdelbegin \DIFdel{Eq. \ref{eq:VILeq}). }\DIFdelend \DIFaddbegin \DIFadd{Equation~\eqref{eq:VILeq}): }\DIFaddend \begin{equation}
  L_i = \sum_{i=1}^{n-1}\sqrt{\left ( \text{SDR}_{i+1} - \text{SDR}_{i}\right )^2 + \left ( t_{i+1} - t_i \right )^2}
  \label{eq:VILeq}
  \end{equation} \begin{equation}
  1 \times L_{\text{CSref},i} - 5 < L_i < 1.3 \times L_{\text{CSref},i} + 13
  \label{eq:VILcr}
  \end{equation} where \DIFdelbegin \DIFdel{: }\DIFdelend \(t_i\) is the time stamp of each SDR
  measurement.
\item
  Standard deviation \(\sigma_i\) of the slope (\(s_i\)) between the
  \(11\) sequential points, normalized by the mean
  \(\overline{\text{SDR}}_i\) (\DIFdelbegin \DIFdel{Eq.~\ref{eq:VCT1}). }\DIFdelend \DIFaddbegin \DIFadd{Equation~\eqref{eq:VCT1}): }\DIFaddend \begin{gather}
    \sigma_i = \frac{\sqrt{\frac{1}{n-1} \sum_{i=1}^{n-1} \left( s_i - \bar{s} \right)^2}}{\overline{\text{SDR}}_i} \label{eq:VCT1} \\
    s_i = \frac{\text{SDR}_{i+1} - \text{SDR}_{i}}{t_{i+1} - t_i},\;\;   \bar{s} = \frac{1}{n-1} \sum_{i=1}^{n-1} s_i,\;\;\forall i \in \left \{ 1, 2, \ldots, n-1 \right \}\;\;
  \end{gather} For this criterion, \(\sigma_i\) should be below a
  certain threshold (\DIFdelbegin \DIFdel{Eq.~\ref{eq:VCTcr}}\DIFdelend \DIFaddbegin \DIFadd{Equation~\eqref{eq:VCTcr}}\DIFaddend ): \begin{equation}
    \sigma_i < \ensuremath{1.1\times 10^{-4}} \label{eq:VCTcr}
  \end{equation}
\item
  Maximum difference \(X_i\) between the change in measured irradiance
  and the change in clear sky irradiance over each measurement interval\DIFdelbegin \DIFdel{.
  }\DIFdelend \DIFaddbegin \DIFadd{:
  }\DIFaddend \begin{gather}
    X_i = \max{\left \{ \left | x_i - x_{\text{CSref},i} \right | \right \}} \label{eq:VSM3} \\
    x_i = \text{SDR}_{i+1} - \text{SDR}_{i} \forall i \in \left \{ 1, 2, \ldots, n-1 \right \} \label{eq:VSM1} \\
    x_{\text{CSref},i} = \text{SDR}_{\text{CSref},i+1} - \text{SDR}_{\text{CSref},i} \forall i \in \left \{ 1, 2, \ldots, n-1 \right \} \label{eq:VSM2}
  \end{gather} For this criterion, \(X_i\) should be below a certain
  threshold (\DIFdelbegin \DIFdel{Eq.~\ref{eq:VSMcr}): }\DIFdelend \DIFaddbegin \DIFadd{Equation~\eqref{eq:VSMcr}): }\DIFaddend \begin{equation}
    X_i < 7.5\,\DIFdelbegin \DIFdel{Wm}\DIFdelend \DIFaddbegin \DIFadd{\text{Wm}}\DIFaddend ^{-2} \label{eq:VSMcr}
  \end{equation}
\end{enumerate}

In the final dataset\DIFdelbegin \DIFdel{\(26\,\%\) }\DIFdelend \DIFaddbegin \DIFadd{, \(26\%\) }\DIFaddend of the days were identified as under
clear-sky conditions and \DIFdelbegin \DIFdel{\(48\,\%\) }\DIFdelend \DIFaddbegin \DIFadd{\(48\%\) }\DIFaddend as under cloud-sky conditions. The
remaining \DIFdelbegin \DIFdel{\(26\,\%\) }\DIFdelend \DIFaddbegin \DIFadd{\(26\%\) }\DIFaddend of the data correspond to mixed cases and were not
analyzed as a separate group.

\DIFdelbegin %DIFDELCMD < \hypertarget{aggregationstatistical}{%
%DIFDELCMD < \subsection{Aggregation of data and statistical
%DIFDELCMD < approach}\label{aggregationstatistical}}
%DIFDELCMD < %%%
\DIFdelend \DIFaddbegin \hypertarget{aggregationstatistical}{%
\subsection{Aggregation of Data and Statistical
Approach}\label{aggregationstatistical}}
\DIFaddend 

In order to investigate the SDR trends \DIFdelbegin \DIFdel{which }\DIFdelend \DIFaddbegin \DIFadd{that }\DIFaddend are the main focus of the
study, we implemented an aggregation scheme to the one-minute data to
derive series in coarser \DIFdelbegin \DIFdel{time-scales}\DIFdelend \DIFaddbegin \DIFadd{time scales}\DIFaddend . To preserve the representativeness
of the data\DIFaddbegin \DIFadd{, }\DIFaddend we used the following criteria: \DIFdelbegin \DIFdel{a) }\DIFdelend \DIFaddbegin \DIFadd{(a)~}\DIFaddend we excluded all days with
less than 50\% of the expected daytime measurements, \DIFdelbegin \DIFdel{b) }\DIFdelend \DIFaddbegin \DIFadd{(b)~}\DIFaddend daily means for
the clear-sky and cloudy-sky datasets were calculated only for days with
more than 60\% of the expected daytime measurements identified as clear
or cloudy\DIFdelbegin \DIFdel{respectively, }\DIFdelend \DIFaddbegin \DIFadd{, respectively, and (}\DIFaddend c) monthly means were computed from daily means.
For the all-skies dataset\DIFaddbegin \DIFadd{, }\DIFaddend monthly means were computed only when at least
20 days were available. Seasonal means were derived by averaging the
monthly mean values in each season (winter: \DIFdelbegin \DIFdel{December - February}\DIFdelend \DIFaddbegin \DIFadd{December--February}\DIFaddend , spring:
\DIFdelbegin \DIFdel{March - May}\DIFdelend \DIFaddbegin \DIFadd{March--May}\DIFaddend , etc.). The daily and monthly climatological means were
derived by averaging the data for each day of \DIFaddbegin \DIFadd{the }\DIFaddend year and calendar month,
respectively. The daily and monthly datasets were deseasonalized by
subtracting the corresponding climatological annual cycle (daily or
monthly) from the actual data. Finally, to estimate the SZA effect on
the SDR trends, the one-minute data were aggregated in \(1^{\circ}\) SZA
bins, separately for the morning and afternoon hours.

The linear trends were calculated using a \DIFdelbegin \DIFdel{first order }\DIFdelend \DIFaddbegin \DIFadd{first-order }\DIFaddend autoregressive
model with lag of 1\DIFaddbegin \DIFadd{~}\DIFaddend day using the \DIFdelbegin \DIFdel{``maximum likelihood'' }\DIFdelend \DIFaddbegin \DIFadd{`maximum likelihood' }\DIFaddend fitting method
\citep{Gardner1980, Jones1980} \DIFdelbegin \DIFdel{, }\DIFdelend by implementing the function \DIFdelbegin \DIFdel{``arima''
}\DIFdelend \DIFaddbegin \DIFadd{`arima'
}\DIFaddend from the library \DIFdelbegin \DIFdel{``stats'' }\DIFdelend \DIFaddbegin \DIFadd{`stats' }\DIFaddend of the R programming language
\citep{RCT2023}. The trends were reported together with the \(2\sigma\)
errors.

\hypertarget{results}{%
\section{Results}\label{results}}

\DIFdelbegin %DIFDELCMD < \hypertarget{long-term-sdr-trends}{%
%DIFDELCMD < \subsection{Long-term SDR trends}\label{long-term-sdr-trends}}
%DIFDELCMD < %%%
\DIFdelend \DIFaddbegin \hypertarget{long-term-sdr-trends}{%
\subsection{Long-Term SDR Trends}\label{long-term-sdr-trends}}
\DIFaddend 

We calculated the linear trends of SDR \DIFdelbegin \DIFdel{, }\DIFdelend from the departures of the mean
daily values from the daily climatology and for the three sky
conditions. These are presented in Table~\ref{tab:trendtable}\DIFdelbegin \DIFdel{which
contains also }\DIFdelend \DIFaddbegin \DIFadd{, which also contains }\DIFaddend the \(2\sigma\) standard error, the Pearson's correlation
coefficient R\DIFaddbegin \DIFadd{, }\DIFaddend and the trend in absolute units. In
Figure~\ref{fig:trendALL}\DIFaddbegin \DIFadd{, }\DIFaddend we present only the time series under all-sky
conditions; the plots for clear-sky and cloud-sky conditions, are shown
in \DIFdelbegin \DIFdel{the Appendix (Figures~\ref{fig:trendCLEAR} and~
\ref{fig:trendCLOUD}). }\DIFdelend \DIFaddbegin \DIFadd{\mbox{\hl{Appendix} %MDPI: We revised citation. Please confirm.
 \ref{app1}} \mbox{(Figures~\ref{fig:trendCLEAR} and~
\ref{fig:trendCLOUD}).} }\DIFaddend In the studied period, there is no significant
break or change in the variability pattern of the time series. The
linear trends in all three datasets are positive and \DIFdelbegin \DIFdel{around
\(0.4\,\%/y\) }\DIFdelend \DIFaddbegin \hl{around} %DIF > MDPI: We removed italics of units. Please confirm. The same below.
\DIFadd{\(0.4\%/\)y }\DIFaddend for all-sky and cloudy-sky conditions, \DIFdelbegin \DIFdel{while }\DIFdelend \DIFaddbegin \DIFadd{whereas }\DIFaddend for
clear-skies the trend is much smaller (\textasciitilde{}\DIFdelbegin \DIFdel{\(0.1\,\%/y\)}\DIFdelend \DIFaddbegin \DIFadd{\(0.1\%/\)y}\DIFaddend ).
The linear trends were calculated taking into account the
autocorrelation of the time series\DIFaddbegin \DIFadd{, }\DIFaddend and all three are statistically
significant at least at the \DIFdelbegin \DIFdel{\(95\,\%\) }\DIFdelend \DIFaddbegin \DIFadd{\(95\%\) }\DIFaddend confidence level, as they are
larger than the corresponding \(2\sigma\) errors, despite the small
values of R, which is due to the large variability of the daily values.
The clear-sky trend is very small\DIFaddbegin \DIFadd{, }\DIFaddend suggesting a small effect from
aerosols and water vapor\DIFaddbegin \DIFadd{, }\DIFaddend which are the dominant factors of the SDR
variability \citep{Fountoulakis2016, Siomos2018, Yu2022}. In contrast,
the large positive trend of SDR under cloudy skies can be attributed to
reduction in cloud cover and/or cloud optical depth. Lack of continuous
observations of cloud optical thickness that could support these
findings does not allow drawing firm conclusions. However, there are
indications that the total \DIFdelbegin \DIFdel{cloud-cover }\DIFdelend \DIFaddbegin \DIFadd{cloud cover }\DIFaddend as inferred from the ERA5
analysis for the grid point of Thessaloniki is decreasing over the
period of study. From the difference between all-sky and clear-sky SDR
trends, expressed in \DIFdelbegin \DIFdel{\(W/m^2/y\) }\DIFdelend \DIFaddbegin \DIFadd{W/m$^2$/y }\DIFaddend using the long-term mean of the
respective datasets, the radiative effect of clouds is estimated to
\DIFdelbegin \DIFdel{\(0.96\,W/m^2/y\)}\DIFdelend \DIFaddbegin \DIFadd{0.96 W/m$^2$/y}\DIFaddend . This estimate is similar to the cloud radiative
forcing of \DIFdelbegin \DIFdel{\(1.22\,W/m^2/y\) }\DIFdelend \DIFaddbegin \DIFadd{1.22 W/m$^2$/y  }\DIFaddend reported for Granada, Spain
\citep{Lozano2023}.

The all-sky trend is similar to the one reported in \citet{Bais2013}
from a ten-year shorter dataset\DIFaddbegin \DIFadd{, }\DIFaddend suggesting that the tendency of SDR in
Thessaloniki is systematic. Other studies for the European region
reported a change in the SDR trend around 1980 from negative to positive
with comparable magnitude \citep{Wild2021, Yuan2021, Ohmura2009}, well
before the start of our records. However, the trends reported here for
the three datasets are in accordance with the widely accepted solar
radiation brightening over Europe. For the period of our observations\DIFaddbegin \DIFadd{,
}\DIFaddend the trend in the TSI is negligible (\DIFdelbegin \DIFdel{\(-0.00022\,\%/y\)}\DIFdelend \DIFaddbegin \DIFadd{\(-0.00022\%/\)y}\DIFaddend ), and thus we
cannot attribute any significant effect on the SDR trend to solar
variability.

\begin{table}[H]

\caption{\label{tab:trendtable}Trends in SDR daily means for different sky conditions for the period \DIFdelbeginFL \DIFdelFL{1993 - 2023.}\DIFdelendFL \DIFaddbeginFL \DIFaddFL{1993--2023.}\DIFaddendFL }
\begin{tabu} to \linewidth {>{\centering\arraybackslash}p{8em}>{\raggedleft}X>{\raggedleft}X>{\raggedleft}X>{\raggedleft}X>{\raggedleft}X}
\toprule
\DIFdelbeginFL \DIFdelFL{Sky conditions }\DIFdelendFL \DIFaddbeginFL \multirow{-1}{*}{\textbf{Sky Conditions}} \DIFaddendFL & \DIFdelbeginFL \DIFdelFL{Trend }%DIFDELCMD < [%%%
\DIFdelFL{\%/year}%DIFDELCMD < ] %%%
\DIFdelendFL \DIFaddbeginFL \textbf{\DIFaddFL{Trend }[\DIFaddFL{\%/year}]} \DIFaddendFL & \DIFdelbeginFL \DIFdelFL{Trend S.E. ($2\sigma$) }\DIFdelendFL \DIFaddbeginFL \textbf{\DIFaddFL{Trend S.E. (}\boldmath\DIFaddFL{$2\sigma$)}} \DIFaddendFL & \DIFdelbeginFL \DIFdelFL{Pearson correl. }\DIFdelendFL \DIFaddbeginFL \textbf{\DIFaddFL{Pearson Correl.}} \DIFaddendFL & \DIFdelbeginFL \DIFdelFL{Trend }%DIFDELCMD < [%%%
\DIFdelFL{W/m\textsuperscript{2}/year}%DIFDELCMD < ] %%%
\DIFdelendFL \DIFaddbeginFL \textbf{\DIFaddFL{Trend }[\DIFaddFL{W/m\textsuperscript{2}/year}]} \DIFaddendFL & \DIFdelbeginFL \DIFdelFL{Days}\DIFdelendFL \DIFaddbeginFL \multirow{-1}{*}{\textbf{Days}}\DIFaddendFL \\
\midrule
All skies & 0.380 & 0.120 & 0.091 & 1.460 & 10251\\
Clear skies & 0.097 & 0.033 & 0.140 & 0.501 & 2684\\
Cloudy skies & 0.410 & 0.180 & 0.081 & 1.180 & 4937\\
\bottomrule
\end{tabu}
\end{table}

\DIFdelbegin %DIFDELCMD < \begin{figure}[h!]
%DIFDELCMD < %%%
\DIFdelendFL \DIFaddbeginFL \vspace{-6pt}

\begin{figure}[H]
\DIFaddendFL 

{ \DIFdelbeginFL %DIFDELCMD < \centering %%%
\DIFdelendFL \includegraphics[width=.75\linewidth]{./images/LongtermTrends-1} 

}

\caption{Anomalies (\%) of the daily all-sky SDR from the climatological mean for the period \DIFdelbeginFL \DIFdelFL{1993 -- 2023. }\DIFdelendFL \DIFaddbeginFL \DIFaddFL{1993--2023. }\DIFaddendFL The black line is the \DIFdelbeginFL \DIFdelFL{long term }\DIFdelendFL \DIFaddbeginFL \DIFaddFL{long-term }\DIFaddendFL linear trend.}\label{fig:trendALL}
\end{figure}

Although the year-to-year variability of the anomalies (Figure
\ref{fig:trendALL} and \DIFdelbegin \DIFdel{Figures~\ref{fig:trendCLEAR},
\ref{fig:trendCLOUD} in Appendix), }\DIFdelend \DIFaddbegin \DIFadd{\mbox{Figures~\ref{fig:trendCLEAR} and
\ref{fig:trendCLOUD}} in \mbox{Appendix \ref{app1})} }\DIFaddend shows a rather homogeneous behavior,
plots of the cumulative sums (CUSUM) \citep{Regier2019} of the anomalies
can reveal different structures in the records of all three sky
conditions. For time series with a uniform trend, we would expect the
CUSUMs of the anomalies to have a symmetric `V' shape centered around
the middle of the series. This would indicate that the anomalies are
evenly distributed around the climatological mean, and\DIFaddbegin \DIFadd{, }\DIFaddend for a positive
uniform trend, the first half is below and the second half above the
climatological mean. In our case, there is a more complex evolution of
the anomalies. For \DIFdelbegin \DIFdel{all-skies }\DIFdelend \DIFaddbegin \DIFadd{all skies }\DIFaddend (Figure~\DIFdelbegin \DIFdel{\ref{fig:cusummonth-1}}\DIFdelend \DIFaddbegin \DIFadd{\ref{fig:cusummonth}a}\DIFaddend ), we observe
three rather distinct periods: \DIFaddbegin \DIFadd{(}\DIFaddend a) a downward part between the start of
the datasets and about 2000, denoting that all anomalies are negative\DIFaddbegin \DIFadd{,
}\DIFaddend thus below the climatology; \DIFaddbegin \DIFadd{(}\DIFaddend b) a relatively steady part lasting for
almost 20 years\DIFaddbegin \DIFadd{, }\DIFaddend suggesting little variability in SDR anomalies; and \DIFaddbegin \DIFadd{(}\DIFaddend c) a
steep upward part to \DIFdelbegin \DIFdel{present }\DIFdelend \DIFaddbegin \DIFadd{the present, }\DIFaddend indicating anomalies above the climatology.
The CUSUMs for \DIFdelbegin \DIFdel{cloudy-skies }\DIFdelend \DIFaddbegin \DIFadd{cloudy skies }\DIFaddend (Figure~\DIFdelbegin \DIFdel{\ref{fig:cusummonth-3}) , }\DIFdelend \DIFaddbegin \DIFadd{\ref{fig:cusummonth}c) }\DIFaddend show a
similar behavior with some short-term differences that do not change the
overall pattern. For clear skies (Figure~\DIFdelbegin \DIFdel{\ref{fig:cusummonth-2}}\DIFdelend \DIFaddbegin \DIFadd{\ref{fig:cusummonth}b}\DIFaddend ), a
monotonic downward tendency is evident until 2004, suggesting that the
anomalies are all negative. After 2004\DIFaddbegin \DIFadd{, }\DIFaddend the anomalies turn \DIFdelbegin \DIFdel{to }\DIFdelend positive at
a fast rate for about five years and at a slower rate thereafter.

\DIFdelbegin %DIFDELCMD < \begin{figure}[h!]
%DIFDELCMD <     %%%
\DIFdelendFL \DIFaddbeginFL \begin{figure}[H]
    \DIFaddendFL \begin{adjustwidth}{-\extralength}{0cm}
        {\centering 
       \DIFdelbeginFL %DIFDELCMD < \subfloat[All skies.\label{fig:cusummonth-1}]
%DIFDELCMD <             %%%
\DIFdelendFL %DIF >  \subfloat[\label{fig:cusummonth-1}]
            {\includegraphics[width=.32\linewidth]{./images/CumulativeMonthlyCuSum-1}}\hfill
       \DIFdelbeginFL %DIFDELCMD < \subfloat[Clear skies.\label{fig:cusummonth-2}]
%DIFDELCMD <             %%%
\DIFdelendFL %DIF >  \subfloat[\label{fig:cusummonth-2}]
            {\includegraphics[width=.32\linewidth]{./images/CumulativeMonthlyCuSum-5}}\hfill
        \DIFdelbeginFL %DIFDELCMD < \subfloat[Cloudy skies.\label{fig:cusummonth-3}]
%DIFDELCMD <             %%%
\DIFdelendFL \DIFaddbeginFL \subfloat[\label{fig:cusummonth-3}]
            \DIFaddendFL {\includegraphics[width=.32\linewidth]{./images/CumulativeMonthlyCuSum-9}}\hfill
        }
\DIFdelbeginFL %DIFDELCMD < \caption{%
{%DIFAUXCMD
\DIFdelFL{Cumulative sum plots of the monthly SDR anomalies in (\%) for different sky conditions.}}%DIFAUXCMD
%DIFDELCMD < \label{fig:cusummonth}
%DIFDELCMD < %%%
\DIFdelendFL \end{adjustwidth}
\DIFaddbeginFL \caption{\hl{Cumulative} %DIF > MDPI: We moved the subfigure explanations into the figure caption. Please confirm.
 \DIFaddFL{sum plots of the monthly SDR anomalies in (\%) for different sky conditions: (}\textbf{\DIFaddFL{a}}\DIFaddFL{)~all skies; (}\textbf{\DIFaddFL{b}}\DIFaddFL{) clear skies; (}\textbf{\DIFaddFL{c}}\DIFaddFL{) cloudy skies.}}\label{fig:cusummonth}

\DIFaddendFL \end{figure}

In order to unveil further the features of the variability of the three
datasets, Figure~\ref{fig:cusumnotrendmonthly} presents another set of
CUSUM plots using anomalies after the long-term linear trend is removed.
With this approach, periods when the CUSUMs diverge from zero can be
interpreted as a systematic variation of SDR from the climatological
mean. When the CUSUM is increasing, the anomalies values are above the
climatology\DIFaddbegin \DIFadd{, }\DIFaddend and vice versa. Overall, for all- and cloudy-sky conditions
(\DIFdelbegin \DIFdel{Figures~\ref{fig:cusumnotrendmonthly-1}
and~\ref{fig:cusumnotrendmonthly-3})}\DIFdelend \DIFaddbegin \DIFadd{Figure~\ref{fig:cusumnotrendmonthly}a,c), }\DIFaddend we observe periods with anomalies
diverging from the climatological values, each lasting for several
years. These fluctuations are probably within the natural variability\DIFaddbegin \DIFadd{,
}\DIFaddend and no distinct changes are identified. The pattern in both datasets is
similar, suggesting prevalence in cloudy skies over Thessaloniki. For
clear skies (Figure~\DIFdelbegin \DIFdel{\ref{fig:cusumnotrendmonthly-2})}\DIFdelend \DIFaddbegin \DIFadd{\ref{fig:cusumnotrendmonthly}b), }\DIFaddend the distinct change
in 2004 is now clearer. The most likely reason for this change is the
monotonic reduction of aerosols in Thessaloniki. \DIFdelbegin \DIFdel{At that yearthere }\DIFdelend \DIFaddbegin \DIFadd{In that year, there was }\DIFaddend a
change in the rate of decrease in aerosol optical depth\DIFaddbegin \DIFadd{, }\DIFaddend as illustrated
in Figure 7 of \citet{Siomos2020}. This abrupt change in CUSUMs \DIFdelbegin \DIFdel{lasts
}\DIFdelend \DIFaddbegin \DIFadd{lasted
}\DIFaddend until about 2010\DIFaddbegin \DIFadd{, }\DIFaddend when the anomalies become again variable.

\DIFdelbegin %DIFDELCMD < \begin{figure}[h!]
%DIFDELCMD <     %%%
\DIFdelendFL \DIFaddbeginFL \begin{figure}[H]
    \DIFaddendFL \begin{adjustwidth}{-\extralength}{0cm}
        {\centering 
          \DIFdelbeginFL %DIFDELCMD < \subfloat[All skies.\label{fig:cusumnotrendmonthly-1}]
%DIFDELCMD <                 %%%
\DIFdelendFL %DIF >   \subfloat[All skies.\label{fig:cusumnotrendmonthly-1}]
                {\includegraphics[width=.32\linewidth]{./images/CumulativeMonthlyCuSumNOtrend-1} }\hfill
           \DIFdelbeginFL %DIFDELCMD < \subfloat[Clear skies.\label{fig:cusumnotrendmonthly-2}]
%DIFDELCMD <                 %%%
\DIFdelendFL %DIF >  \subfloat[Clear skies.\label{fig:cusumnotrendmonthly-2}]
                {\includegraphics[width=.32\linewidth]{./images/CumulativeMonthlyCuSumNOtrend-5} }\hfill
           \DIFdelbeginFL %DIFDELCMD < \subfloat[Cloudy skies.\label{fig:cusumnotrendmonthly-3}]
%DIFDELCMD <                 %%%
\DIFdelendFL %DIF >  \subfloat[Cloudy skies.\label{fig:cusumnotrendmonthly-3}]
                {\includegraphics[width=.32\linewidth]{./images/CumulativeMonthlyCuSumNOtrend-9} }
        }
\DIFdelbeginFL %DIFDELCMD < \caption{%
{%DIFAUXCMD
\DIFdelFL{Cumulative sum plots of monthly SDR anomalies in (\%) for different sky conditions after removing the long-term linear trend.}}%DIFAUXCMD
%DIFDELCMD < \label{fig:cusumnotrendmonthly}
%DIFDELCMD < %%%
\DIFdelendFL \DIFaddbeginFL 

\DIFaddendFL \end{adjustwidth}
        \DIFaddbeginFL \caption{\hl{Cumulative} %DIF > MDPI: We moved the subfigure explanations into the figure caption. Please confirm.
 \DIFaddFL{sum plots of monthly SDR anomalies in (\%) for different sky conditions after removing the long-term linear trend: (}\textbf{\DIFaddFL{a}}\DIFaddFL{)~all skies; (}\textbf{\DIFaddFL{b}}\DIFaddFL{) clear skies; (}\textbf{\DIFaddFL{c}}\DIFaddFL{) cloudy skies.}}\label{fig:cusumnotrendmonthly}
\DIFaddendFL \end{figure}

\DIFdelbegin %DIFDELCMD < \hypertarget{effects-of-the-solar-zenith-angle-on-sdr}{%
%DIFDELCMD < \subsection{Effects of the solar zenith angle on
%DIFDELCMD < SDR}\label{effects-of-the-solar-zenith-angle-on-sdr}}
%DIFDELCMD < %%%
\DIFdelend \DIFaddbegin \hypertarget{effects-of-the-solar-zenith-angle-on-sdr}{%
\subsection{Effects of the Solar Zenith Angle on
SDR}\label{effects-of-the-solar-zenith-angle-on-sdr}}
\DIFaddend 

The solar zenith angle is a major factor affecting the SDR, \DIFdelbegin \DIFdel{since
}\DIFdelend \DIFaddbegin \DIFadd{as
}\DIFaddend increases in SZA leads to enhancement of the radiation path in the
atmosphere, especially in urban environments with human activities
emitting aerosols \citep{Wang2021}. In order to estimate the effect of
SZA on the SDR trends, we grouped the data in bins of \(1^\circ\) SZA
\DIFdelbegin \DIFdel{,
}\DIFdelend and calculated the overall trend for each bin \DIFdelbegin \DIFdel{, }\DIFdelend separately for the daily
periods before noon and after noon (Figure~\ref{fig:szatrends}).
Although there are seasonal dependencies of the minimum SZA (see
Appendix \DIFaddbegin \DIFadd{\ref{app1}}\DIFaddend , Figure~\ref{fig:SZAtrendSeason}), these dependencies are not
discussed further.

For all-sky conditions\DIFaddbegin \DIFadd{, }\DIFaddend the brightening effect of SDR (positive trend)
increases with SZAs (\DIFdelbegin \DIFdel{Figures~\ref{fig:szatrends-1})}\DIFdelend \DIFaddbegin \DIFadd{Figure~\ref{fig:szatrends}a), }\DIFaddend ranging from about
\DIFdelbegin \DIFdel{\(0.1\,\%/y\) to about \(0.7\,\%/y\) }\DIFdelend \DIFaddbegin \DIFadd{\(0.1\%/\)y to about \(0.7\%/\)y }\DIFaddend for the statistically significant
trends. The trends in the morning and afternoon hours are more or less
consistent with small differences at small SZAs\DIFaddbegin \DIFadd{, }\DIFaddend which can be attributed
to effects on clear sky SDR from systematic diurnal patterns of aerosols
during the warm period of the year, \DIFdelbegin \DIFdel{consistently }\DIFdelend \DIFaddbegin \DIFadd{consistent }\DIFaddend with the results
reported for China by \citet{Wang2021}. Note that SZAs less than
\(25^\circ\) can only occur during the warm period of the year around
noon when clear skies are more frequent. The increasing trend with SZA
is likely caused by the increased attenuation of SDR with SZA. The
effect is larger when aerosol and/or cloud layers are optically thicker\DIFdelbegin \DIFdel{,
}\DIFdelend \DIFaddbegin \DIFadd{;
}\DIFaddend therefore, decreases in aerosol and clouds through the study period will
result in larger positive trends of SDR at larger SZAs.

Under clear skies (\DIFdelbegin \DIFdel{Figures~\ref{fig:szatrends-2}}\DIFdelend \DIFaddbegin \DIFadd{Figure~\ref{fig:szatrends}b}\DIFaddend ), the trends are
smaller and less variable, ranging between \(0.1\) and \DIFdelbegin \DIFdel{\(0.15\,\%/y\) }\DIFdelend \DIFaddbegin \DIFadd{\(0.15\%/\)y }\DIFaddend up
to \(77^\circ\) SZA. At higher SZAs and in the afternoon hours\DIFdelbegin \DIFdel{there }\DIFdelend \DIFaddbegin \DIFadd{, there is }\DIFaddend a
sharp increase in the trend up to \DIFdelbegin \DIFdel{\(0.3\,\%/y\)}\DIFdelend \DIFaddbegin \DIFadd{\(0.3\%/\)y}\DIFaddend , which may have been
caused by the long path length of radiation through the atmosphere as
discussed above for the all-sky conditions. The small differences in the
trend between morning and afternoon between \(35^\circ\) and
\(60^\circ\) SZA is likely a result of less attenuation of SDR in the
morning hours due to lesser amounts of aerosols and \DIFaddbegin \DIFadd{a }\DIFaddend shallower boundary
layer.

For cloudy-sky conditions (Figure~\DIFdelbegin \DIFdel{\ref{fig:szatrends-3}}\DIFdelend \DIFaddbegin \DIFadd{\ref{fig:szatrends}c}\DIFaddend ), we cannot
discern any significant dependence of the SDR trend with SZA\DIFaddbegin \DIFadd{, }\DIFaddend as the
variability of irradiance is dominated by the cloud effects leading to
insignificant trends. Statistically significant trends appear only in
the afternoon and for SZAs larger than \(60^\circ\). The sharp increase
of the trend at SZAs larger than \DIFdelbegin \DIFdel{\(\sim{75}^{\circ}\)}\DIFdelend \DIFaddbegin \DIFadd{$\sim$}{\DIFadd{75}}\DIFadd{$^{\circ}$}\DIFaddend , observed also for
clear skies, is probably associated with stronger attenuation by clouds
under oblique incidence angles, which \DIFdelbegin \DIFdel{result also }\DIFdelend \DIFaddbegin \DIFadd{also result }\DIFaddend in smaller
variability.

\DIFdelbegin %DIFDELCMD < \begin{figure}[h!]
%DIFDELCMD <     %%%
\DIFdelendFL \DIFaddbeginFL \begin{figure}[H]
    \DIFaddendFL \begin{adjustwidth}{-\extralength}{0cm}
        {\centering 
          \DIFdelbeginFL %DIFDELCMD < \subfloat[All skies.\label{fig:szatrends-1}]
%DIFDELCMD <                 %%%
\DIFdelendFL %DIF >   \subfloat[All skies.\label{fig:szatrends-1}]
                {\includegraphics[width=.32\linewidth]{./images/SzaTrends-1}}\hfill
         \DIFdelbeginFL %DIFDELCMD < \subfloat[Clear skies.\label{fig:szatrends-2}]
%DIFDELCMD <                 %%%
\DIFdelendFL %DIF >    \subfloat[Clear skies.\label{fig:szatrends-2}]
                {\includegraphics[width=.32\linewidth]{./images/SzaTrends-4}}\hfill
           \DIFdelbeginFL %DIFDELCMD < \subfloat[Cloudy skies.\label{fig:szatrends-3}]
%DIFDELCMD <                 %%%
\DIFdelendFL %DIF >  \subfloat[Cloudy skies.\label{fig:szatrends-3}]
                {\includegraphics[width=.32\linewidth]{./images/SzaTrends-7}}
        }
\DIFdelbeginFL %DIFDELCMD < \caption{%
{%DIFAUXCMD
\DIFdelFL{Long term trends of daily SDR as a function of SZA for (a) all-sky, (b) clear-sky and (c) cloudy-sky conditions, separately for morning and afternoon periods. Solid shapes represent statistically significant trends ($p < 0.005$).}}%DIFAUXCMD
%DIFDELCMD < \label{fig:szatrends}
%DIFDELCMD <     %%%
\DIFdelendFL \DIFaddbeginFL 

    \DIFaddendFL \end{adjustwidth}
        \DIFaddbeginFL \caption{\hl{Long} %DIF > MDPI: We removed duplicate explanation in figure.
 \DIFaddFL{-term trends of daily SDR as a function of SZA for (}\textbf{\DIFaddFL{a}}\DIFaddFL{) all-sky, (}\textbf{\DIFaddFL{b}}\DIFaddFL{) clear-sky and (}\textbf{\DIFaddFL{c}}\DIFaddFL{) cloudy-sky conditions, separately for morning and afternoon periods. Solid shapes represent statistically significant trends ($p < 0.005$).}}\label{fig:szatrends}
\DIFaddendFL \end{figure}

\DIFdelbegin %DIFDELCMD < \hypertarget{long-term-sdr-trends-by-season}{%
%DIFDELCMD < \subsection{Long term SDR trends by
%DIFDELCMD < season}\label{long-term-sdr-trends-by-season}}
%DIFDELCMD < %%%
\DIFdelend \DIFaddbegin \hypertarget{long-term-sdr-trends-by-season}{%
\subsection{Long-Term SDR Trends by
Season}\label{long-term-sdr-trends-by-season}}
\DIFaddend 

Similarly to the long term trends from daily means of SDR discussed
above, we have calculated the trend for the three sky conditions and for
each season of the year \DIFdelbegin \DIFdel{, }\DIFdelend using the corresponding mean monthly anomalies
(Figure~\ref{fig:seasonalALL} and Table~\ref{tab:trendseasontable}).
Table~\ref{tab:trendseasontable} \DIFdelbegin \DIFdel{contains also }\DIFdelend \DIFaddbegin \DIFadd{also contains }\DIFaddend the \(2\sigma\) standard
error, the Pearson's correlation coefficient R\DIFaddbegin \DIFadd{, }\DIFaddend and the corresponding
p-value. The winter linear trends \DIFdelbegin \DIFdel{exhibit generally }\DIFdelend \DIFaddbegin \DIFadd{generally exhibit }\DIFaddend the largest R values\DIFaddbegin \DIFadd{,
}\DIFaddend ranging between \(0.54\) and \DIFdelbegin \DIFdel{\(0.60\,\%/y\)}\DIFdelend \DIFaddbegin \DIFadd{\(0.60\%/\)y}\DIFaddend . For all-sky conditions\DIFaddbegin \DIFadd{, }\DIFaddend the
trend in SDR in winter is the largest (\DIFdelbegin \DIFdel{\(0.7\,\%/y\)}\DIFdelend \DIFaddbegin \DIFadd{\(0.7\%/\)y}\DIFaddend ), followed by the
trend in autumn (\DIFdelbegin \DIFdel{\(0.42\,\%/y\)}\DIFdelend \DIFaddbegin \DIFadd{\(0.42\%/\)y}\DIFaddend , a value close to the \DIFdelbegin \DIFdel{long term trend)}\DIFdelend \DIFaddbegin \DIFadd{long-term trend),
}\DIFaddend both statistically significant at the \DIFdelbegin \DIFdel{\(95\,\%\) }\DIFdelend \DIFaddbegin \DIFadd{\(95\%\) }\DIFaddend confidence level. In
spring and summer, the trends are much smaller and of lesser statistical
significance. These seasonal differences indicate a possible relation of
the trends in SDR to trends in clouds during winter and autumn. For
\DIFdelbegin \DIFdel{clear-skies}\DIFdelend \DIFaddbegin \DIFadd{clear skies}\DIFaddend , the trend in winter is \DIFdelbegin \DIFdel{\(0.4\,\%/y\) }\DIFdelend \DIFaddbegin \DIFadd{\(0.4\%/\)y }\DIFaddend and is associated with
the decreasing trend in aerosol optical depth \citep{Siomos2020}.
Moreover, it is almost half of that for \DIFdelbegin \DIFdel{all-skies}\DIFdelend \DIFaddbegin \DIFadd{all skies}\DIFaddend , which is another
indication of a decreasing trend in cloud optical thickness. In other
seasons\DIFaddbegin \DIFadd{, }\DIFaddend the clear-sky trend is very small (below \DIFdelbegin \DIFdel{\(0.1\,\%/y\)}\DIFdelend \DIFaddbegin \DIFadd{\(0.1\%/\)y}\DIFaddend ).
Finally, for \DIFdelbegin \DIFdel{cloudy-skies }\DIFdelend \DIFaddbegin \DIFadd{cloudy skies, }\DIFaddend the winter trend is the largest
(\DIFdelbegin \DIFdel{\(0.76\,\%/y\)}\DIFdelend \DIFaddbegin \DIFadd{\(0.76\%/\)y}\DIFaddend ) and greater than for \DIFdelbegin \DIFdel{all-skies}\DIFdelend \DIFaddbegin \DIFadd{all skies}\DIFaddend , followed by a much
smaller trend in autumn (\DIFdelbegin \DIFdel{\(0.19\,\%/y\)}\DIFdelend \DIFaddbegin \DIFadd{\(0.19\%/\)y}\DIFaddend ).

The trends under clear- and cloudy-sky conditions are \DIFdelbegin \DIFdel{of }\DIFdelend \DIFaddbegin \DIFadd{in }\DIFaddend the same
direction, and it would be expected that their sum is similar to the
all-sky trend. This does not happen, especially for winter, likely due
to the way the monthly means for clear and cloudy skies were calculated.
Daily means were calculated only when at least \DIFdelbegin \DIFdel{\(60\,\%\) }\DIFdelend \DIFaddbegin \DIFadd{\(60\%\) }\DIFaddend of the clear-
or cloudy-sky data were available (see \DIFdelbegin \DIFdel{sect.
}\DIFdelend \DIFaddbegin \DIFadd{Section }\DIFaddend \ref{aggregationstatistical}).



\DIFdelbegin %DIFDELCMD < \begin{figure}[h!]
%DIFDELCMD <     \begin{adjustwidth}{-\extralength}{0cm}
%DIFDELCMD <         {\centering 
%DIFDELCMD <             \includegraphics[width=1\linewidth]{./images/SeasonalMTrendsTogether3-2}   %%%
%DIF <  Seasonal from Monthly
        %DIFDELCMD < }
%DIFDELCMD <         %%%
%DIFDELCMD < \caption{%
{%DIFAUXCMD
\DIFdelFL{Monthly mean anomalies of SDR by season (rows of plots) for the three sky conditions (columns of plots). The black lines represent the linear trends.}}%DIFAUXCMD
%DIFDELCMD < \label{fig:seasonalALL}
%DIFDELCMD <     \end{adjustwidth}
%DIFDELCMD < \end{figure}
%DIFDELCMD < 

%DIFDELCMD < \begin{table}[!h]
%DIFDELCMD < %%%
\DIFdelendFL \DIFaddbeginFL \begin{table}[H]
\DIFaddendFL 

\caption{\label{tab:trendseasontable}SDR linear trends of monthly anomalies for each season of the year and related statistical parameters.}
\begin{tabu} to \linewidth {>{\centering\arraybackslash}p{8em}>{\centering}X>{\raggedleft}X>{\raggedleft}X>{\raggedleft}X>{\raggedleft}X}
\toprule
\DIFdelbeginFL \DIFdelFL{Sky condition }\DIFdelendFL \DIFaddbeginFL \multirow{-1}{*}{\textbf{Sky Condition}} \DIFaddendFL & \DIFdelbeginFL \DIFdelFL{Season }\DIFdelendFL \DIFaddbeginFL \multirow{-1}{*}{\textbf{Season}} \DIFaddendFL & \DIFdelbeginFL \DIFdelFL{Trend }%DIFDELCMD < [%%%
\DIFdelFL{\%/year}%DIFDELCMD < ] %%%
\DIFdelendFL \DIFaddbeginFL \textbf{\DIFaddFL{Trend }[\DIFaddFL{\%/year}]} \DIFaddendFL & \DIFdelbeginFL \DIFdelFL{Trend S.E. ($2\sigma$) }\DIFdelendFL \DIFaddbeginFL \textbf{\DIFaddFL{Trend S.E. (}\boldmath\DIFaddFL{$2\sigma$)}} \DIFaddendFL & \DIFdelbeginFL \DIFdelFL{Pearson correl. }\DIFdelendFL \DIFaddbeginFL \textbf{\DIFaddFL{Pearson Correl.}} \DIFaddendFL & \DIFdelbeginFL \DIFdelFL{Trend p-value}\DIFdelendFL \DIFaddbeginFL \textbf{\DIFaddFL{Trend }\emph{\DIFaddFL{p}}\DIFaddFL{-Value}}\DIFaddendFL \\
\midrule
\cellcolor{gray!6}{} & \cellcolor{gray!6}{Winter} & \cellcolor{gray!6}{0.70} & \cellcolor{gray!6}{0.43} & \cellcolor{gray!6}{0.54} & \cellcolor{gray!6}{0.003}\\

 & Spring & 0.11 & 0.24 & 0.17 & 0.371\\

\cellcolor{gray!6}{} & \cellcolor{gray!6}{Summer} & \cellcolor{gray!6}{0.11} & \cellcolor{gray!6}{0.15} & \cellcolor{gray!6}{0.25} & \cellcolor{gray!6}{0.175}\\

\multirow{-4}{*}{\centering\arraybackslash All skies} & Autumn & 0.42 & 0.30 & 0.47 & 0.009\\
\cmidrule{1-6}
\cellcolor{gray!6}{} & \cellcolor{gray!6}{Winter} & \cellcolor{gray!6}{0.40} & \cellcolor{gray!6}{0.20} & \cellcolor{gray!6}{0.60} & \cellcolor{gray!6}{0.001}\\

 & Spring & 0.06 & 0.17 & 0.13 & 0.497\\

\cellcolor{gray!6}{} & \cellcolor{gray!6}{Summer} & \DIFdelbeginFL %DIFDELCMD < \cellcolor{gray!6}{-0.05} %%%
\DIFdelendFL \DIFaddbeginFL \cellcolor{gray!6}{$-$0.05} \DIFaddendFL & \cellcolor{gray!6}{0.06} & \DIFdelbeginFL %DIFDELCMD < \cellcolor{gray!6}{-0.30} %%%
\DIFdelendFL \DIFaddbeginFL \cellcolor{gray!6}{$-$0.30} \DIFaddendFL & \cellcolor{gray!6}{0.106}\\

\multirow{-4}{*}{\centering\arraybackslash Clear skies} & Autumn & 0.05 & 0.12 & 0.17 & 0.366\\
\DIFdelbeginFL %DIFDELCMD < \cmidrule{1-6}
%DIFDELCMD < %%%
\DIFdelendFL %DIF > \cmidrule{1-6}
\DIFaddbeginFL \bottomrule
\end{tabu}
\end{table}

\begin{table}[H]\ContinuedFloat

\caption{\label{tab:trendseasontable}\emph{\DIFaddFL{Cont.}}}
\begin{tabu} \DIFaddFL{to }\linewidth {\DIFaddFL{>}{\centering\arraybackslash}\DIFaddFL{p}{\DIFaddFL{8em}}\DIFaddFL{>}{\centering}\DIFaddFL{X>}{\raggedleft}\DIFaddFL{X>}{\raggedleft}\DIFaddFL{X>}{\raggedleft}\DIFaddFL{X>}{\raggedleft}\DIFaddFL{X}}
\toprule
\multirow{-1}{*}{\textbf{Sky Condition}} & \multirow{-1}{*}{\textbf{Season}} & \textbf{\DIFaddFL{Trend }[\DIFaddFL{\%/year}]} & \textbf{\DIFaddFL{Trend S.E. (}\boldmath\DIFaddFL{$2\sigma$)}} & \textbf{\DIFaddFL{Pearson Correl.}} & \textbf{\DIFaddFL{Trend }\emph{\DIFaddFL{p}}\DIFaddFL{-Value}}\\
\midrule
\DIFaddendFL \cellcolor{gray!6}{} & \cellcolor{gray!6}{Winter} & \cellcolor{gray!6}{0.76} & \cellcolor{gray!6}{0.40} & \cellcolor{gray!6}{0.59} & \cellcolor{gray!6}{0.001}\\

 & Spring & 0.06 & 0.23 & 0.10 & 0.593\\

\cellcolor{gray!6}{} & \cellcolor{gray!6}{Summer} & \DIFdelbeginFL %DIFDELCMD < \cellcolor{gray!6}{-0.08} %%%
\DIFdelendFL \DIFaddbeginFL \cellcolor{gray!6}{$-$0.08} \DIFaddendFL & \cellcolor{gray!6}{0.27} & \DIFdelbeginFL %DIFDELCMD < \cellcolor{gray!6}{-0.11} %%%
\DIFdelendFL \DIFaddbeginFL \cellcolor{gray!6}{$-$0.11} \DIFaddendFL & \cellcolor{gray!6}{0.560}\\

\multirow{-4}{*}{\centering\arraybackslash Cloudy skies} & Autumn & 0.19 & 0.43 & 0.16 & 0.384\\
\bottomrule
\end{tabu}
\end{table}


\DIFdelbegin %DIFDELCMD < \hypertarget{conclusions}{%
%DIFDELCMD < \section{Conclusions}\label{conclusions}}
%DIFDELCMD < %%%
\DIFdelend \DIFaddbegin \vspace{-7pt}

\begin{figure}[H]
    \begin{adjustwidth}{-\extralength}{0cm}
        {\centering 
            \includegraphics[width=1\linewidth]{./images/SeasonalMTrendsTogether3-2}   %DIF >  Seasonal from Monthly
        }

    \end{adjustwidth}
 \caption{\colorbox{red}{Monthly} %DIF > MDPI: Please add the explanation for subfifgures in the figure.
 \DIFaddFL{mean anomalies of SDR by season (rows of plots) for the three sky conditions (columns of plots). The black lines represent the linear trends.}}\label{fig:seasonalALL}
\end{figure}


\hypertarget{conclusions}{%
\section{\colorbox{green}{Conclusions}\label{conclusions}}} %DIF > MDPI: Please make sure no figure in paper is repeated
\DIFaddend 

We have analyzed a 30-year dataset of SRD measurements in Thessaloniki\DIFdelbegin \DIFdel{Greece (1993 -- 2023)}\DIFdelend \DIFaddbegin \DIFadd{,
Greece (1993--2023), }\DIFaddend aiming to identify the \DIFdelbegin \DIFdel{long term }\DIFdelend \DIFaddbegin \DIFadd{long-term }\DIFaddend variability of
solar irradiance under different sky conditions. Under all-sky
conditions\DIFaddbegin \DIFadd{, }\DIFaddend there is a positive trend in SDR of \DIFdelbegin \DIFdel{\(0.38\,\%/y\)
}\DIFdelend \DIFaddbegin \DIFadd{\(0.38\%/\)y
}\DIFaddend (brightening). A previous study \citep{Bais2013} for the period \DIFdelbegin \DIFdel{1993 --
2011 reported also }\DIFdelend \DIFaddbegin \DIFadd{1993--2011 also reported }\DIFaddend a positive trend of \DIFdelbegin \DIFdel{\(0.33\,\%/y\)}\DIFdelend \DIFaddbegin \DIFadd{\(0.33\%/\)y}\DIFaddend . The slight
increase \DIFdelbegin \DIFdel{of }\DIFdelend \DIFaddbegin \DIFadd{in }\DIFaddend this trend indicates that the brightening of SDR continues
and is likely caused by continuing decreases in aerosol optical depth
and the optical thickness of clouds over the area. A smaller trend has
been found under clear-sky conditions (\DIFdelbegin \DIFdel{\(0.097\,\%/y\))}\DIFdelend \DIFaddbegin \DIFadd{\(0.097\%/\)y), }\DIFaddend which supports
the notion that part of the brightening is caused by decreasing
aerosols. \citet{Siomos2020} \DIFdelbegin \DIFdel{have shown }\DIFdelend \DIFaddbegin \DIFadd{showed }\DIFaddend that aerosol optical depth over
Thessaloniki \DIFdelbegin \DIFdel{is decreasing constantly}\DIFdelend \DIFaddbegin \DIFadd{was decreasing constantly, }\DIFaddend at least up \DIFdelbegin \DIFdel{to }\DIFdelend \DIFaddbegin \DIFadd{until }\DIFaddend 2018. The
attenuation of SDR by aerosols over Europe has been proposed as major
factor by \citet{Wild2021}. Unfortunately, for this study\DIFaddbegin \DIFadd{, }\DIFaddend aerosol data
for the entire period were not available in order to quantify their
effect on SDR. The brightening effect on SDR under cloudy-sky conditions
(\DIFdelbegin \DIFdel{\(0.41\,\%/y\)) , }\DIFdelend \DIFaddbegin \DIFadd{\(0.41\%/\)y) }\DIFaddend suggests that cloud optical thickness is also
decreasing during this period. As \DIFdelbegin \DIFdel{long term }\DIFdelend \DIFaddbegin \DIFadd{long-term }\DIFaddend data of cloud optical
thickness are also not available for the region, we cannot draw
quantitative\DIFaddbegin \DIFadd{~}\DIFaddend conclusions.

The observed brightening on SDR over Thessaloniki is dependent on SZA
(larger SZAs lead to stronger brightening). The trend is also dependent
on season, with winter showing the strongest statistically significant
trend of \(0.7\) and \DIFdelbegin \DIFdel{\(0.76\,\%/y\) }\DIFdelend \DIFaddbegin \DIFadd{\(0.76\%/\)y }\DIFaddend for all- and cloudy-skies,
respectively, in contrast to spring and summer. The trends for autumn
are also significant but smaller (\(0.42\) and \DIFdelbegin \DIFdel{\(0.19\,\%/y\) }\DIFdelend \DIFaddbegin \DIFadd{\(0.19\%/\)y }\DIFaddend for all-
and cloudy-skies, respectively). The trend for clear skies is largest in
winter (\DIFdelbegin \DIFdel{\(0.4\,\%/y\)}\DIFdelend \DIFaddbegin \DIFadd{\(0.4\%/\)y}\DIFaddend ) and negligible in spring, summer\DIFaddbegin \DIFadd{, }\DIFaddend and autumn.

Using the CUSUMs of the monthly departures for \DIFdelbegin \DIFdel{all- and cloudy-skies}\DIFdelend \DIFaddbegin \DIFadd{all and cloudy skies}\DIFaddend , we
observed a 20-year period starting around 2000 where the CUSUMs remain
relatively stable, with a steep decline before and a steep increase
after. The rather smooth course of the CUSUMs suggests that no important
change in the SDR pattern has occurred in the entire record.

Continued observations with a collocated pyrheliometer, which started in
2016, will allow us to further investigate the variability of solar
radiation at ground level in Thessaloniki. Also, additional data of
cloudiness, aerosols, atmospheric water \DIFdelbegin \DIFdel{vapour}\DIFdelend \DIFaddbegin \DIFadd{vapor}\DIFaddend , etc., will allow better
attribution and quantification of the effects of these factors on SRD.

%%%%%%%%%%%%%%%%%%%%%%%%%%%%%%%%%%%%%%%%%%

\vspace{6pt}

%%%%%%%%%%%%%%%%%%%%%%%%%%%%%%%%%%%%%%%%%%
%% optional

% Only for the journal Methods and Protocols:
% If you wish to submit a video article, please do so with any other supplementary material.
% \supplementary{The following supporting information can be downloaded at: \linksupplementary{s1}, Figure S1: title; Table S1: title; Video S1: title. A supporting video article is available at doi: link.}
\DIFdelbegin %DIFDELCMD < 

%DIFDELCMD < %%%
\DIFdelend \DIFaddbegin \authorcontributions{\hl{  } %MDPI: For research articles with several authors, a short paragraph specifying their individual contributions must be provided. The following statements should be used ``Conceptualization, X.X. and Y.Y.; methodology, X.X.; software, X.X.; validation, X.X., Y.Y. and Z.Z.; formal analysis, X.X.; investigation, X.X.; resources, X.X.; data curation, X.X.; writing---original draft preparation, X.X.; writing---review and editing, X.X.; visualization, X.X.; supervision, X.X.; project administration, X.X.; funding acquisition, Y.Y. All authors have read and agreed to the published version of the manuscript.'', please turn to the  \href{http://img.mdpi.org/data/contributor-role-instruction.pdf}{CRediT taxonomy} for the term explanation. Authorship must be limited to those who have contributed substantially to the work~reported.
}
\DIFaddend %%%%%%%%%%%%%%%%%%%%%%%%%%%%%%%%%%%%%%%%%%

\funding{This research received no external funding.}

\DIFdelbegin %DIFDELCMD < \dataavailability{Data as daily sums are available through the WRDC
%DIFDELCMD < database \url{http://wrdc.mgo.rssi.ru}. One minute data are available on
%DIFDELCMD < request from the corresponding author. The data are not publicly
%DIFDELCMD < available for protection against unmonitored commercial use.}
%DIFDELCMD < %%%
\DIFdelend \DIFaddbegin \institutionalreview{\hl{  } %MDPI: In this section, you should add the Institutional Review Board Statement and approval number, if relevant to your study. You might choose to exclude this statement if the study did not require ethical approval. Please note that the Editorial Office might ask you for further information. Please add “The study was conducted in accordance with the Declaration of Helsinki, and approved by the Institutional Review Board (or Ethics Committee) of NAME OF INSTITUTE (protocol code XXX and date of approval).” for studies involving humans. OR “The animal study protocol was approved by the Institutional Review Board (or Ethics Committee) of NAME OF INSTITUTE (protocol code XXX and date of approval).” for studies involving animals. OR “Ethical review and approval were waived for this study due to REASON (please provide a detailed justification).” OR “Not applicable” for studies not involving humans or animals.
}

\informedconsent{\hl{  } %MDPI: Any research article describing a study involving humans should contain this statement. Please add ``Informed consent was obtained from all subjects involved in the study.'' OR ``Patient consent was waived due to REASON (please provide a detailed justification).'' OR ``Not applicable'' for studies not involving humans. You might also choose to exclude this statement if the study did not involve humans.

%Written informed consent for publication must be obtained from participating patients who can be identified (including by the patients themselves). Please state ``Written informed consent has been obtained from the patient(s) to publish this paper'' if applicable.
}
\DIFaddend 

\DIFaddbegin \dataavailability{Data as daily sums are available through the WRDC
database, \url{http://wrdc.mgo.rssi.ru}. One-minute data are available on
request from the corresponding author. The data are not publicly
available for protection against unmonitored commercial use.}


\conflictsofinterest{\hl{  } %MDPI: Declare conflicts of interest or state ``The authors declare no conflict of interest.'' Authors must identify and declare any personal circumstances or interest that may be perceived as inappropriately influencing the representation or interpretation of reported research results. Any role of the funders in the design of the study; in the collection, analyses or interpretation of data; in the writing of the manuscript; or in the decision to publish the results must be declared in this section. If there is no role, please state ``The funders had no role in the design of the study; in the collection, analyses, or interpretation of data; in the writing of the manuscript; or in the decision to publish the results''.
} 
\DIFaddend %%%%%%%%%%%%%%%%%%%%%%%%%%%%%%%%%%%%%%%%%%
%% Optional

%% Only for journal Encyclopedia

\abbreviations{Abbreviations}{
The following abbreviations are used in this manuscript:\\

\noindent
\begin{tabular}{@{}ll}
DNI & Direct beam/normal irradiance \\
ERA5 & ECMWF Reanalysis v5 \\
CSid & Clear sky identification algorithm \\
CUSUM & Cumulative sum \\
SDR & Solar downward radiation \\
SZA & Solar zenith angle \\
\end{tabular}}

%%%%%%%%%%%%%%%%%%%%%%%%%%%%%%%%%%%%%%%%%%
%% Optional
\DIFdelbegin %DIFDELCMD < \input{"appendix.tex"}
%DIFDELCMD < %%%
\DIFdelend %DIF > \input{"appendix.tex"}
\DIFaddbegin 

%DIF >  % \renewcommand\thefigure{\thesection.\arabic{figure}} %% my
%DIF >  % \renewcommand\thefigure{\thesection.\arabic{figure}}
%DIF > 
%DIF >  %% From R rticle MDPI template
%DIF >  %% optional
%DIF >  \appendixtitles{no} %Leave argument "no" if all appendix headings stay EMPTY (then no dot is printed after "Appendix A"). If the appendix sections contain a heading then change the argument to "yes".
%DIF >  \appendixsections{one} %Leave argument "multiple" if there are multiple sections. Then a counter is printed ("Appendix A"). If there is only one appendix section then change the argument to "one" and no counter is printed ("Appendix").
%DIF > 
%DIF >  % \appendix
%DIF >  % \section{}
%DIF > 
%DIF >  \appendix
%DIF >  \setcounter{secnumdepth}{0}
%DIF >  \section{Appendix}
%DIF > 
%DIF >  \setcounter{figure}{0}    %% my



%DIF > % From raw MDPI template
%DIF > % Optional
\appendixtitles{no} %DIF >  Leave argument "no" if all appendix headings stay EMPTY (then no dot is printed after "Appendix A"). If the appendix sections contain a heading then change the argument to "yes".
\appendixstart
\appendix
\section[\appendixname~\thesection]{}

\label{app1}

%DIF >  copy output from Rmd

\begin{figure}[H]
    {
        \includegraphics[width=0.75\linewidth]{./images/LongtermTrends-2}

    }
    \caption{\DIFaddFL{Anomalies (\%) of the daily clear-sky SDR, relative to climatological values for 1993--2023. The black line shows the long-term linear trend for clear-sky conditions.}}\label{fig:trendCLEAR}
\end{figure}



\begin{figure}[H]
    {
        \includegraphics[width=0.75\linewidth]{./images/LongtermTrends-3}

    }
    \caption{\DIFaddFL{Anomalies (\%) of the daily cloud-sky SDR, relative to climatological values for 1993--2023. The black line shows the long-term linear trend for cloud-sky conditions.}}\label{fig:trendCLOUD}
\end{figure}

\vspace{-6pt}


\begin{figure}[H]
    \begin{adjustwidth}{-\extralength}{0cm}
        {\centering
            \includegraphics[width=1.0\linewidth]{./images/SzaTrendsSeasTogether-2}
        }
   \end{adjustwidth}
        \caption{\colorbox{red}{Long} %DIF > MDPI: Please add the explanation for subfifgures in the figure.
 \DIFaddFL{-term trends of SDR as a function of SZA separately form morning
               and afternoon periods, by season (rows of plots) for the three sky
               conditions (columns of plots).
               Solid shapes represent statistical significant trends ($p<0.005$).
               Cases where $p<0.005$ or with less than $85$~observations may be
               missing from view.}}\label{fig:SZAtrendSeason}

\end{figure}




\FloatBarrier

%DIF >  \subsection{}
%DIF >  The appendix is an optional section that can contain details and data supplemental to the main text. For example, explanations of experimental details that would disrupt the flow of the main text, but nonetheless remain crucial to understanding and reproducing the research shown; figures of replicates for experiments of which representative data is shown in the main text can be added here if brief, or as Supplementary data. Mathematical proofs of results not central to the paper can be added as an appendix.
%DIF > 
%DIF >  \section{}
%DIF >  All appendix sections must be cited in the main text. In the appendixes, Figures, Tables, etc. should be labeled starting with `A', e.g., Figure A1, Figure A2, etc.

\DIFaddend %%%%%%%%%%%%%%%%%%%%%%%%%%%%%%%%%%%%%%%%%%
\begin{adjustwidth}{-\extralength}{0cm}

%\printendnotes[custom] % Un-comment to print a list of endnotes


\reftitle{References}
\DIFdelbegin %DIFDELCMD < \bibliography{manualreferences.bib}
%DIFDELCMD < %%%
\DIFdelend %DIF > \bibliography{manualreferences.bib}
\DIFaddbegin \begin{thebibliography}{999}

\bibitem[Wild(2009)]{Wild2009}
\DIFadd{Wild, M.
}\newblock \DIFadd{Global dimming and brightening: A review.
}\newblock {\em \DIFadd{J. Geophys. Res. Atmos.}} {\bf \DIFadd{2009}}\DIFadd{, }{\em
  \DIFadd{114}}\DIFadd{,~1--31.
}\newblock {\url{https://doi.org/10/bcq}}\DIFadd{.
}

\bibitem[Yang et~al.(2021)Yang, Zhou, Yu, and Wild]{Yang2021}
\DIFadd{Yang, S.; Zhou, Z.; Yu, Y.; Wild, M.
}\newblock \DIFadd{Cloud }{\DIFadd{\textquotedblleft}}\DIFadd{shrinking}{\DIFadd{\textquotedblright}} \DIFadd{and
  }{\DIFadd{\textquotedblleft}}\DIFadd{optical thinning}{\DIFadd{\textquotedblright}} \DIFadd{in the
  }{\DIFadd{\textquotedblleft}}\DIFadd{dimming}{\DIFadd{\textquotedblright}} \DIFadd{period and a subsequent
  recovery in the }{\DIFadd{\textquotedblleft}}\DIFadd{brightening}{\DIFadd{\textquotedblright}} \DIFadd{period
  over China.
}\newblock {\em \DIFadd{Environ. Res. Lett.}} {\bf \DIFadd{2021}}\DIFadd{, }\emph{\DIFadd{16}}\DIFadd{, 034013.
}\newblock {\url{https://doi.org/10.1088/1748-9326/abdf89}}\DIFadd{.
}

\bibitem[Wild et~al.(2021)Wild, Wacker, Yang, and Sanchez-Lorenzo]{Wild2021}
\DIFadd{Wild, M.; Wacker, S.; Yang, S.; Sanchez-Lorenzo, A.
}\newblock \DIFadd{Evidence for Clear‐Sky Dimming and Brightening in Central Europe.
}\newblock {\em \DIFadd{Geophys. Res. Lett.}} {\bf \DIFadd{2021}}\DIFadd{, }{\em \DIFadd{48}}\DIFadd{, e2020GL092216.
}\newblock {\url{https://doi.org/10.1029/2020GL092216}}\DIFadd{.
}

\bibitem[Yamasoe et~al.(2021)Yamasoe, Ros{\'{a}}rio, Almeida, and
  Wild]{Yamasoe2021}
\DIFadd{Yamasoe, M.A.; Ros}{\DIFadd{\'{a}}}\DIFadd{rio, N.M.}{\DIFadd{\'{E}}}\DIFadd{.; Almeida, S.N.S.M.; Wild, M.
}\newblock \DIFadd{Fifty-six years of surface solar radiation and sunshine duration over
  S}{\DIFadd{\~{a}}}\DIFadd{o Paulo, Brazil: 1961--2016.
}\newblock {\em \DIFadd{Atmos. Chem. Phys.}} {\bf \DIFadd{2021}}\DIFadd{, }{\em
  \DIFadd{21}}\DIFadd{,~6593--6603.
}\newblock {\url{https://doi.org/10.5194/acp-21-6593-2021}}\DIFadd{.
}

\bibitem[Yuan et~al.(2021)Yuan, Leirvik, and Wild]{Yuan2021}
\DIFadd{Yuan, M.; Leirvik, T.; Wild, M.
}\newblock \DIFadd{Global trends in downward surface solar radiation from spatial
  interpolated ground observations during 1961--2019.
}\newblock {\em \DIFadd{J. Clim.}} {\bf \DIFadd{2021}}\DIFadd{, }\emph{\DIFadd{34}}\DIFadd{, 9501--9521.
}\newblock {\url{https://doi.org/10.1175/JCLI-D-21-0165.1}}\DIFadd{.
}

\bibitem[Li et~al.(2016)Li, Lau, Ramanathan, Wu, Ding, Manoj, Liu, Qian, Li,
  Zhou, Fan, Rosenfeld, Ming, Wang, Huang, Wang, Xu, Lee, Cribb, Zhang, Yang,
  Zhao, Takemura, Wang, Xia, Yin, Zhang, Guo, Zhai, Sugimoto, Babu, and
  Brasseur]{Li2016}
\DIFadd{Li, Z.; Lau, W.K.; Ramanathan, V.; Wu, G.; Ding, Y.; Manoj, M.G.; Liu, J.;
  Qian, Y.; Li, J.; Zhou, T.;  et~al.
}\newblock \DIFadd{Aerosol and monsoon climate interactions over Asia.
}\newblock {\em \DIFadd{Rev. Geophys.}} {\bf \DIFadd{2016}}\DIFadd{, }{\em \DIFadd{54}}\DIFadd{,~866--929.
}\newblock {\url{https://doi.org/10.1002/2015RG000500}}\DIFadd{.
}

\bibitem[Samset et~al.(2018)Samset, Sand, Smith, Bauer, Forster, Fuglestvedt,
  Osprey, and Schleussner]{Samset2018}
\DIFadd{Samset, B.H.; Sand, M.; Smith, C.J.; Bauer, S.E.; Forster, P.M.; Fuglestvedt,
  J.S.; Osprey, S.; Schleussner, C.
}\newblock \DIFadd{Climate Impacts From a Removal of Anthropogenic Aerosol Emissions.
}\newblock {\em \DIFadd{Geophys. Res. Lett.}} {\bf \DIFadd{2018}}\DIFadd{, }{\em \DIFadd{45}}\DIFadd{,~1020--1029.
}\newblock {\url{https://doi.org/10.1002/2017GL076079}}\DIFadd{.
}

\bibitem[Schwarz et~al.(2020)Schwarz, Folini, Yang, Allan, and
  Wild]{Schwarz2020}
\DIFadd{Schwarz, M.; Folini, D.; Yang, S.; Allan, R.P.; Wild, M.
}\newblock \DIFadd{Changes in atmospheric shortwave absorption as important driver of
  dimming and brightening.
}\newblock {\em \DIFadd{Nat. Geosci.}} {\bf \DIFadd{2020}}\DIFadd{, }{\em \DIFadd{13}}\DIFadd{,~110--115.
}\newblock {\url{https://doi.org/10.1038/s41561-019-0528-y}}\DIFadd{.
}

\bibitem[Ohvril et~al.(2009)Ohvril, Teral, Neiman, Kannel, Uustare, Tee,
  Russak, Okulov, Jõeveer, Kallis, Ohvril, Terez, Terez, Gushchin, Abakumova,
  Gorbarenko, Tsvetkov, and Laulainen]{Ohvril2009}
\DIFadd{Ohvril, H.; Teral, H.; Neiman, L.; Kannel, M.; Uustare, M.; Tee, M.; Russak,
  V.; Okulov, O.; Jõeveer, A.; Kallis, A.;  et~al.
}\newblock \DIFadd{Global dimming and brightening versus atmospheric column
  transparency, Europe, 1906–2007.
}\newblock {\em \DIFadd{J. Geophys. Res.}} {\bf \DIFadd{2009}}\DIFadd{, }{\em \DIFadd{114}}\DIFadd{.
}\newblock {\url{https://doi.org/10.1029/2008JD010644}}\DIFadd{.
}

\bibitem[Zerefos et~al.(2009)Zerefos, Eleftheratos, Meleti, Kazadzis, Romanou,
  Ichoku, Tselioudis, and Bais]{Zerefos2009}
\DIFadd{Zerefos, C.S.; Eleftheratos, K.; Meleti, C.; Kazadzis, S.; Romanou, A.; Ichoku,
  C.; Tselioudis, G.; Bais, A.
}\newblock \DIFadd{Solar dimming and brightening over Thessaloniki, Greece, and Beijing,
  China.
}\newblock {\em \DIFadd{Tellus B Chem. Phys. Meteorol.}} {\bf \DIFadd{2009}}\DIFadd{, }{\em
  \DIFadd{61}}\DIFadd{,~657.
}\newblock {\url{https://doi.org/10.1111/j.1600-0889.2009.00425.x}}\DIFadd{.
}

\bibitem[Xia et~al.(2007)Xia, Chen, Li, Wang, and Wang]{Xia2007}
\DIFadd{Xia, X.; Chen, H.; Li, Z.; Wang, P.; Wang, J.
}\newblock \DIFadd{Significant reduction of surface solar irradiance induced by aerosols
  in a suburban region in northeastern China.
}\newblock {\em \DIFadd{J. Geophys. Res. Atmos.}} {\bf \DIFadd{2007}}\DIFadd{, }{\em
  \DIFadd{112}}\DIFadd{,~1--9.
}\newblock {\url{https://doi.org/10/cdtntw}}\DIFadd{.
}

\bibitem[Fountoulakis et~al.(2016)Fountoulakis, Redondas, Bais,
  Rodriguez-Franco, Fragkos, and Cede]{Fountoulakis2016}
\DIFadd{Fountoulakis, I.; Redondas, A.; Bais, A.F.; Rodriguez-Franco, J.J.; Fragkos,
  K.; Cede, A.
}\newblock \DIFadd{Dead time effect on the Brewer measurements: Correction and estimated
  uncertainties.
}\newblock {\em \DIFadd{Atmos. Meas. Tech.}} {\bf \DIFadd{2016}}\DIFadd{, }{\em
  \DIFadd{9}}\DIFadd{,~1799--1816.
}\newblock {\url{https://doi.org/10/gcc32t}}\DIFadd{.
}

\bibitem[Siomos et~al.(2018)Siomos, Balis, Voudouri, Giannakaki, Filioglou,
  Amiridis, Papayannis, and Fragkos]{Siomos2018}
\DIFadd{Siomos, N.; Balis, D.S.; Voudouri, K.A.; Giannakaki, E.; Filioglou, M.;
  Amiridis, V.; Papayannis, A.; Fragkos, K.
}\newblock \DIFadd{Are }{\DIFadd{EARLINET}} \DIFadd{and }{\DIFadd{AERONET}} \DIFadd{climatologies consistent? The case of
  Thessaloniki, Greece.
}\newblock {\em \DIFadd{Atmos. Chem. Phys.}} {\bf \DIFadd{2018}}\DIFadd{, }{\em
  \DIFadd{18}}\DIFadd{,~11885--11903.
}\newblock {\url{https://doi.org/10.5194/acp-18-11885-2018}}\DIFadd{.
}

\bibitem[Gkikas et~al.(2013)Gkikas, Hatzianastassiou, Mihalopoulos, Katsoulis,
  Kazadzis, Pey, Querol, and Torres]{Gkikas2013}
\DIFadd{Gkikas, A.; Hatzianastassiou, N.; Mihalopoulos, N.; Katsoulis, V.; Kazadzis,
  S.; Pey, J.; Querol, X.; Torres, O.
}\newblock \DIFadd{The regime of intense desert dust episodes in the Mediterranean based
  on contemporary satellite observations and ground measurements.
}\newblock {\em \DIFadd{Atmos. Chem. Phys.}} {\bf \DIFadd{2013}}\DIFadd{, }{\em
  \DIFadd{13}}\DIFadd{,~12135--12154.
}\newblock {\url{https://doi.org/10.5194/acp-13-12135-2013}}\DIFadd{.
}

\bibitem[Lozano et~al.(2021)Lozano, Sánchez-Hernández, Guerrero-Rascado,
  Alados, and Foyo-Moreno]{Lozano2021}
\DIFadd{Lozano, I.L.; Sánchez-Hernández, G.; Guerrero-Rascado, J.L.; Alados, I.;
  Foyo-Moreno, I.
}\newblock \DIFadd{Aerosol radiative effects in photosynthetically active radiation and
  total irradiance at a Mediterranean site from an 11-year database.
}\newblock {\em \DIFadd{Atmos. Res.}} {\bf \DIFadd{2021}}\DIFadd{, }{\em \DIFadd{255}}\DIFadd{,~105538.
}\newblock {\url{https://doi.org/10.1016/j.atmosres.2021.105538}}\DIFadd{.
}

\bibitem[Bais et~al.(2013)Bais, Drosoglou, Meleti, Tourpali, and
  Kouremeti]{Bais2013}
\DIFadd{Bais, A.F.; Drosoglou, T.; Meleti, C.; Tourpali, K.; Kouremeti, N.
}\newblock \DIFadd{Changes in surface shortwave solar irradiance from 1993 to 2011 at
  Thessaloniki (Greece).
}\newblock {\em \DIFadd{Int. J. Climatol.}} {\bf \DIFadd{2013}}\DIFadd{, }{\em
  \DIFadd{33}}\DIFadd{,~2871--2876.
}\newblock {\url{https://doi.org/10/f5dzz5}}\DIFadd{.
}

\bibitem[Reno and Hansen(2016)]{Reno2016}
\DIFadd{Reno, M.J.; Hansen, C.W.
}\newblock \DIFadd{Identification of periods of clear sky irradiance in time series of
  GHI measurements.
}\newblock {\em \DIFadd{Renew. Energy}} {\bf \DIFadd{2016}}\DIFadd{, }{\em \DIFadd{90}}\DIFadd{,~520--531.
}\newblock {\url{https://doi.org/10/gq3sbg}}\DIFadd{.
}

\bibitem[Reno et~al.(2012)Reno, Hansen, and Stein]{Reno2012a}
\colorbox{green}{Reno,} %DIF > MDPI: Refs. 18 and 26 are duplicated. Please remove duplicated references and rearrange all the references to appear in numerical order. Please ensure that there are no duplicated references.
 \DIFadd{M.J.; Hansen, C.W.; Stein, J.S.
}\newblock {\emph{\DIFadd{Global Horizontal Irradiance Clear Sky Models: Implementation and
  Analysis}}}\DIFadd{;
}\newblock \hl{Technical report;}  %DIF > MDPI: Please confirm if it can be removed. Please add the name of the publisher and their location.
 \DIFadd{2012.
}

\bibitem[Long and Shi(2006)]{Long2006}
\DIFadd{Long, C.N.; Shi, Y.
}\newblock \emph{\DIFadd{The QCRad Value Added Product: Surface Radiation Measurement Quality
  Control Testing, Including Climatology Configurable Limits}}\DIFadd{;
}\newblock \hl{Technical Report DOE/SC-ARM/TR-074;} %DIF > MDPI: Please confirm if it can be removed. Please add the name of the publisher and their location.
  \DIFadd{2006.
}

\bibitem[Long and Shi(2008)]{Long2008a}
\DIFadd{Long, C.N.; Shi, Y.
}\newblock \DIFadd{An Automated Quality Assessment and Control Algorithm for Surface
  Radiation Measurements.
}\newblock {\em  \DIFadd{Open Atmos. Sci. J.}} {\bf \DIFadd{2008}}\DIFadd{, }\emph{\DIFadd{2}}\DIFadd{, 23--37.
}

\bibitem[Coddington et~al.(2005)Coddington, Lean, Lindholm, Pilewskie, Snow,
  and {NOAA CDR Program}]{Coddington2005}
\DIFadd{Coddington, O.; Lean, J.L.; Lindholm, D.; Pilewskie, P.; Snow, M.; }{\DIFadd{NOAA CDR
  Program}}\DIFadd{.
}\newblock \emph{{\DIFadd{NOAA}} \DIFadd{Climate Data Record (}{\DIFadd{CDR}}\DIFadd{) of Total Solar Irradiance (}{\DIFadd{TSI}}\DIFadd{),
  }{\DIFadd{NRLTSI}} \DIFadd{Version 2}}\DIFadd{; }\hl{{D}aily;} %DIF > MDPI: Please confirm if it can be removed. Please add the name of the publisher and their location.
  \DIFadd{2005.
}\newblock {\url{https://doi.org/10.7289/V55B00C1}}\DIFadd{.
}

\bibitem[Haurwitz(1945)]{Haurwitz1945}
\DIFadd{Haurwitz, B.
}\newblock \DIFadd{Insolation in }{\DIFadd{Relation}} \DIFadd{to }{\DIFadd{Cloudiness}} \DIFadd{and }{\DIFadd{Cloud}} {\DIFadd{Density}}\DIFadd{.
}\newblock {\em \DIFadd{J. Meteorol.}} {\bf \DIFadd{1945}}\DIFadd{, }{\em \DIFadd{2}}\DIFadd{,~154--166.
}

\bibitem[Long and Ackerman(2000)]{Long2000}
\DIFadd{Long, C.N.; Ackerman, T.P.
}\newblock \DIFadd{Identification of clear skies from broadband pyranometer measurements
  and calculation of downwelling shortwave cloud effects.
}\newblock {\em \DIFadd{J. Geophys. Res. Atmos.}} {\bf \DIFadd{2000}}\DIFadd{, }{\em
  \DIFadd{105}}\DIFadd{,~15609--15626.
}\newblock {\url{https://doi.org/10.1029/2000jd900077}}\DIFadd{.
}

\bibitem[Brent(1973)]{Brent1973}
\DIFadd{Brent, R.P.
}\newblock \emph{\DIFadd{Algorithms for Minimization without Derivatives}}\DIFadd{;
}\newblock {\em \DIFadd{PrenticeHall: Englewood Cliffs, NJ, USA,}} {\DIFadd{1973}}\DIFadd{.
}

\bibitem[{R Core Team}(2023)]{RCT2023}
{\DIFadd{R Core Team}}\DIFadd{.
}\newblock {\em \DIFadd{R: A Language and Environment for Statistical Computing}}\DIFadd{;
}\newblock \DIFadd{R Foundation for Statistical Computing: Vienna, Austria,  2023.
}

\bibitem[Reno et~al.(2012)Reno, Hansen, and Stein]{Reno2012}
\colorbox{red}{Reno,} 
 \DIFadd{M.J.; Hansen, C.W.; Stein, J.S.
}\newblock \emph{\DIFadd{Global Horizontal Irradiance Clear Sky Models: Implementation and
  Analysis}}\DIFadd{;
}\newblock \hl{Technical Report SAND2012-2389, 1039404;} %DIF > MDPI: Please confirm if it can be removed. Please add the name of the publisher and their location.
  \DIFadd{2012.
}\newblock {\url{https://doi.org/10/gq5npv}}\DIFadd{.
}

\bibitem[Gardner et~al.(1980)Gardner, Harvey, and Phillips]{Gardner1980}
\DIFadd{Gardner, G.; Harvey, A.C.; Phillips, G.D.A.
}\newblock \DIFadd{Algorithm }{\DIFadd{AS}} \DIFadd{154: An Algorithm for Exact Maximum Likelihood
  Estimation of Autoregressive-Moving Average Models by Means of Kalman
  Filtering.
}\newblock {\em \DIFadd{Appl. Stat.}} {\bf \DIFadd{1980}}\DIFadd{, }{\em \DIFadd{29}}\DIFadd{,~311.
}\newblock {\url{https://doi.org/10.2307/2346910}}\DIFadd{.
}

\bibitem[Jones(1980)]{Jones1980}
\DIFadd{Jones, R.H.
}\newblock \DIFadd{Maximum Likelihood Fitting of }{\DIFadd{ARMA}} \DIFadd{Models to Time Series With
  Missing Observations.
}\newblock {\em \DIFadd{Technometrics}} {\bf \DIFadd{1980}}\DIFadd{, }{\em \DIFadd{22}}\DIFadd{,~389--395.
}\newblock {\url{https://doi.org/10.1080/00401706.1980.10486171}}\DIFadd{.
}

\bibitem[Yu et~al.(2022)Yu, Zhang, Wang, Qin, Jiang, and Li]{Yu2022}
\DIFadd{Yu, L.; Zhang, M.; Wang, L.; Qin, W.; Jiang, D.; Li, J.
}\newblock \DIFadd{Variability of surface solar radiation under clear skies over
  Qinghai-Tibet Plateau: Role of aerosols and water vapor.
}\newblock {\em \DIFadd{Atmos. Environ.}} {\bf \DIFadd{2022}}\DIFadd{, }{\em \DIFadd{287}}\DIFadd{,~119286.
}\newblock {\url{https://doi.org/10.1016/j.atmosenv.2022.119286}}\DIFadd{.
}

\bibitem[Lozano et~al.(2023)Lozano, Alados, and Foyo-Moreno]{Lozano2023}
\DIFadd{Lozano, I.L.; Alados, I.; Foyo-Moreno, I.
}\newblock \DIFadd{Analysis of the solar radiation/atmosphere interaction at a
  Mediterranean site: The role of clouds.
}\newblock {\em \DIFadd{Atmos. Res.}} {\bf \DIFadd{2023}}\DIFadd{, }{\em \DIFadd{296}}\DIFadd{,~107072.
}\newblock {\url{https://doi.org/10.1016/j.atmosres.2023.107072}}\DIFadd{.
}

\bibitem[Ohmura(2009)]{Ohmura2009}
\DIFadd{Ohmura, A.
}\newblock \DIFadd{Observed decadal variations in surface solar radiation and their
  causes.
}\newblock {\em \DIFadd{J. Geophys. Res. Atmos.}} {\bf \DIFadd{2009}}\DIFadd{, }{\em \DIFadd{114}}\DIFadd{.
}\newblock {\url{https://doi.org/10.1029/2008JD011290}}\DIFadd{.
}

\bibitem[Regier et~al.(2019)Regier, Brice{\~{n}}o, and Boyer]{Regier2019}
\DIFadd{Regier, P.; Brice}{\DIFadd{\~{n}}}\DIFadd{o, H.; Boyer, J.N.
}\newblock \DIFadd{Analyzing and comparing complex environmental time series using a
  cumulative sums approach.
}\newblock {\em {\DIFadd{MethodsX}}} {\bf \DIFadd{2019}}\DIFadd{, }{\em \DIFadd{6}}\DIFadd{,~779--787.
}\newblock {\url{https://doi.org/10.1016/j.mex.2019.03.014}}\DIFadd{.
}

\bibitem[Siomos et~al.(2020)Siomos, Fountoulakis, Natsis, Drosoglou, and
  Bais]{Siomos2020}
\DIFadd{Siomos, N.; Fountoulakis, I.; Natsis, A.; Drosoglou, T.; Bais, A.
}\newblock \DIFadd{Automated Aerosol Classification from Spectral }{\DIFadd{UV}} \DIFadd{Measurements
  Using Machine Learning Clustering.
}\newblock {\em \DIFadd{Remote Sens.}} {\bf \DIFadd{2020}}\DIFadd{, }{\em \DIFadd{12}}\DIFadd{,~965.
}\newblock {\url{https://doi.org/10.3390/rs12060965}}\DIFadd{.
}

\bibitem[Wang et~al.(2021)Wang, Zhang, Sanchez‐Lorenzo, Tanaka, Trentmann,
  Yuan, and Wild]{Wang2021}
\DIFadd{Wang, Y.; Zhang, J.; Sanchez‐Lorenzo, A.; Tanaka, K.; Trentmann, J.; Yuan,
  W.; Wild, M.
}\newblock \DIFadd{Hourly Surface Observations Suggest Stronger Solar Dimming and
  Brightening at Sunrise and Sunset Over China.
}\newblock {\em \DIFadd{Geophys. Res. Lett.}} {\bf \DIFadd{2021}}\DIFadd{, }{\em \DIFadd{48}}\DIFadd{, e2020GL091422.
}\newblock {\url{https://doi.org/10.1029/2020GL091422}}\DIFadd{.
}

\end{thebibliography}
\DIFaddend 

% If authors have biography, please use the format below
%\section*{Short Biography of Authors}
%\bio
%{\raisebox{-0.35cm}{\includegraphics[width=3.5cm,height=5.3cm,clip,keepaspectratio]{Definitions/author1.pdf}}}
%{\textbf{Firstname Lastname} Biography of first author}
%
%\bio
%{\raisebox{-0.35cm}{\includegraphics[width=3.5cm,height=5.3cm,clip,keepaspectratio]{Definitions/author2.jpg}}}
%{\textbf{Firstname Lastname} Biography of second author}

%%%%%%%%%%%%%%%%%%%%%%%%%%%%%%%%%%%%%%%%%%
%% for journal Sci
%\reviewreports{\\
%Reviewer 1 comments and authors’ response\\
%Reviewer 2 comments and authors’ response\\
%Reviewer 3 comments and authors’ response
%}
%%%%%%%%%%%%%%%%%%%%%%%%%%%%%%%%%%%%%%%%%%
\PublishersNote{}
\end{adjustwidth}


\end{document}
