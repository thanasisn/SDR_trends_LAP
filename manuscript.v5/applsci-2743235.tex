%  LaTeX support: latex@mdpi.com
%  For support, please attach all files needed for compiling as well as the log file, and specify your operating system, LaTeX version, and LaTeX editor.

%=================================================================
% pandoc conditionals added to preserve backwards compatibility with previous versions of rticles

\documentclass[applsci,article,accept,moreauthors,pdftex]{Definitions/mdpi}


%% Some pieces required from the pandoc template
\setlist[itemize]{leftmargin=*,labelsep=5.8mm}
\setlist[enumerate]{leftmargin=*,labelsep=4.9mm}


%--------------------
% Class Options:
%--------------------

%---------
% article
%---------
% The default type of manuscript is "article", but can be replaced by:
% abstract, addendum, article, book, bookreview, briefreport, casereport, comment, commentary, communication, conferenceproceedings, correction, conferencereport, entry, expressionofconcern, extendedabstract, datadescriptor, editorial, essay, erratum, hypothesis, interestingimage, obituary, opinion, projectreport, reply, retraction, review, perspective, protocol, shortnote, studyprotocol, systematicreview, supfile, technicalnote, viewpoint, guidelines, registeredreport, tutorial
% supfile = supplementary materials

%----------
% submit
%----------
% The class option "submit" will be changed to "accept" by the Editorial Office when the paper is accepted. This will only make changes to the frontpage (e.g., the logo of the journal will get visible), the headings, and the copyright information. Also, line numbering will be removed. Journal info and pagination for accepted papers will also be assigned by the Editorial Office.


%------------------
% moreauthors
%------------------
% If there is only one author the class option oneauthor should be used. Otherwise use the class option moreauthors.

%---------
% pdftex
%---------
% The option pdftex is for use with pdfLaTeX. Remove "pdftex" for (1) compiling with LaTeX & dvi2pdf (if eps figures are used) or for (2) compiling with XeLaTeX.

%=================================================================
% MDPI internal commands - do not modify
\firstpage{1}
\makeatletter
\setcounter{page}{\@firstpage}
\makeatother
\pubvolume{1}
\issuenum{1}
\articlenumber{0}
\pubyear{2024}
\copyrightyear{2023}
\externaleditor{Academic Editor: }
\datereceived{13 November 2023 }
\daterevised{20 December 2023 } % Comment out if no revised date
\dateaccepted{ }
\datepublished{ }
%\datecorrected{} % For corrected papers: "Corrected: XXX" date in the original paper.
%\dateretracted{} % For corrected papers: "Retracted: XXX" date in the original paper.
\hreflink{https://doi.org/} % If needed use \linebreak
%\doinum{}
%\pdfoutput=1 % Uncommented for upload to arXiv.org

%=================================================================
% Add packages and commands here. The following packages are loaded in our class file: fontenc, inputenc, calc, indentfirst, fancyhdr, graphicx, epstopdf, lastpage, ifthen, float, amsmath, amssymb, lineno, setspace, enumitem, mathpazo, booktabs, titlesec, etoolbox, tabto, xcolor, colortbl, soul, multirow, microtype, tikz, totcount, changepage, attrib, upgreek, array, tabularx, pbox, ragged2e, tocloft, marginnote, marginfix, enotez, amsthm, natbib, hyperref, cleveref, scrextend, url, geometry, newfloat, caption, draftwatermark, seqsplit
% cleveref: load \crefname definitions after \begin{document}

%=================================================================
% Please use the following mathematics environments: Theorem, Lemma, Corollary, Proposition, Characterization, Property, Problem, Example, ExamplesandDefinitions, Hypothesis, Remark, Definition, Notation, Assumption
%% For proofs, please use the proof environment (the amsthm package is loaded by the MDPI class).

%=================================================================
% Full title of the paper (Capitalized)
\Title{Trends from 30-Year Observations of Downward Solar Irradiance in
Thessaloniki, \colorbox{green}{Greece}}%MDPI: Important note:
%1. The paper was edited by our English editor, please check the whole text and confirm if your meaning is retained.
%2. Do not delete any comment we left for you and reply to each comment so that we can understand your meaning clearly.
%3. Please directly correct on this version. If you need to revise somewhere in your paper, please highlight the revisions and track changes to make us known.
%4. Please finish the proofreading based on this version.
%5.Please confirm and revise all the comments with “Confirmed”, “OK”, “Revised”, “It should be italic”; “I confirm”; “I confirm xx is correct”; “I have checked and revised all.” , etc.“
%6. Please note that at this stage (the manuscript has been accepted in the current form), we will not accept authorship or content changes to the main text.
%(Thank you for your cooperation in advance.)


% MDPI internal command: Title for citation in the left column
\TitleCitation{Trends from 30-Year Observations of Downward Solar Irradiance in
Thessaloniki, Greece}

% Author Orchid ID: enter ID or remove command
%\newcommand{\orcidauthorA}{0000-0000-0000-000X} % Add \orcidA{} behind the author's name
%\newcommand{\orcidauthorB}{0000-0000-0000-000X} % Add \orcidB{} behind the author's name


% Authors, for the paper (add full first names)
\Author{Athanasios
Natsis \href{https://orcid.org/0000-0002-5199-4119}
{\orcidicon}, \colorbox{magenta}{Alkiviadis Bais} \href{https://orcid.org/0000-0003-3899-2001} %Natsis: added orcid  
{\orcidicon}* and Charikleia Meleti}


%\longauthorlist{yes}


% MDPI internal command: Authors, for metadata in PDF
\AuthorNames{Athanasios Natsis, Alkiviadis Bais, Charikleia Meleti}

% MDPI internal command: Authors, for citation in the left column
%\AuthorCitation{Lastname, F.; Lastname, F.; Lastname, F.}
% If this is a Chicago style journal: Lastname, Firstname, Firstname Lastname, and Firstname Lastname.
\AuthorCitation{Natsis, A.; Bais, A.; Meleti, C.}

% Affiliations / Addresses (Add [1] after \address if there is only one affiliation.)
\address[1]{Laboratory of
Atmospheric Physics, Aristotle University of Thessaloniki, Campus Box 149, \mbox{54124 Thessaloniki,
Greece;} {{natsisphysicist@gmail.com}}
(A.N.); 
{{meleti@auth.gr}} (C.M.)\\
}

% Contact information of the corresponding author
\corres{\hangafter=1 \hangindent=1.0em \hspace{-1em}Correspondence: {{abais@auth.gr}}}

% Current address and/or shared authorship








% The commands \thirdnote{} till \eighthnote{} are available for further notes

% Simple summary

%\conference{} % An extended version of a conference paper

% Abstract (Do not insert blank lines, i.e. \\)
\abstract{The shortwave downward solar irradiance (SDR) is an important
factor that drives climate processes and energy production and can affect
all living organisms. Observations of SDR at different locations around
the world with different environmental characteristics have been used to
investigate its long-term variability and trends at different time
scales. Periods of positive trends are referred to as brightening periods
and of negative trends as dimming periods. In this study we have used 30
years of pyranometer data in Thessaloniki, Greece, to investigate the
variability of SDR under three types of sky conditions (clear-, cloudy-
and all-sky). The clear-sky data were identified by applying a cloud
screening algorithm. We have found a positive trend of
\(0.38\%/\text{year}\) for all-sky, $\sim$0.1\%/{year} for
clear-sky, and \(0.41\%/\text{year}\) for cloudy conditions. The
consistency of these trends, their seasonal variability, and the effect
of the solar zenith angle have also been investigated. Under all three
sky categories, the SDR trend is stronger in winter, with \(0.7\), \(0.4\),
and \(0.76\%/\text{year}\), respectively, for all-, clear-, and
cloudy-sky conditions. The next larger seasonal trends are in autumn---\(0.42\) and
\(0.19\%/\text{year}\), for all and cloudy skies, respectively. The
rest of the seasonal trends are significant smaller, close to zero, with a
negative values in summer, for clear and cloudy skies. The SDR trend is
increasing with increasing solar zenith angle, except under cloudy skies,
where the trend is highly variable and close to zero. Finally, we
discuss shorter-term variations in SDR anomalies by examining the
patterns of the cumulative sums of monthly anomalies from the
climatological mean, both before and after removing the long-term
trend.}


% Keywords
\keyword{GHI; SDR; solar radiation; solar brigthening/dimming; aerosols;
clouds}

% The fields PACS, MSC, and JEL may be left empty or commented out if not applicable
%\PACS{J0101}
%\MSC{}
%\JEL{}

%%%%%%%%%%%%%%%%%%%%%%%%%%%%%%%%%%%%%%%%%%
% Only for the journal Diversity
%\LSID{\url{http://}}

%%%%%%%%%%%%%%%%%%%%%%%%%%%%%%%%%%%%%%%%%%
% Only for the journal Applied Sciences

%%%%%%%%%%%%%%%%%%%%%%%%%%%%%%%%%%%%%%%%%%

%%%%%%%%%%%%%%%%%%%%%%%%%%%%%%%%%%%%%%%%%%
% Only for the journal Data



%%%%%%%%%%%%%%%%%%%%%%%%%%%%%%%%%%%%%%%%%%
% Only for the journal Toxins


%%%%%%%%%%%%%%%%%%%%%%%%%%%%%%%%%%%%%%%%%%
% Only for the journal Encyclopedia


%%%%%%%%%%%%%%%%%%%%%%%%%%%%%%%%%%%%%%%%%%
% Only for the journal Advances in Respiratory Medicine
%\addhighlights{yes}
%\renewcommand{\addhighlights}{%

%\noindent This is an obligatory section in “Advances in Respiratory Medicine”, whose goal is to increase the discoverability and readability of the article via search engines and other scholars. Highlights should not be a copy of the abstract, but a simple text allowing the reader to quickly and simplified find out what the article is about and what can be cited from it. Each of these parts should be devoted up to 2~bullet points.\vspace{3pt}\\
%\textbf{What are the main findings?}
% \begin{itemize}[labelsep=2.5mm,topsep=-3pt]
% \item First bullet.
% \item Second bullet.
% \end{itemize}\vspace{3pt}
%\textbf{What is the implication of the main finding?}
% \begin{itemize}[labelsep=2.5mm,topsep=-3pt]
% \item First bullet.
% \item Second bullet.
% \end{itemize}
%}


%%%%%%%%%%%%%%%%%%%%%%%%%%%%%%%%%%%%%%%%%%


% tightlist command for lists without linebreak
\providecommand{\tightlist}{%
  \setlength{\itemsep}{0pt}\setlength{\parskip}{0pt}}



\usepackage{subcaption}
\captionsetup[sub]{position=bottom, labelfont={bf, small, stretch=1.17}, labelsep=space, textfont={small, stretch=1.17}, aboveskip=6pt,  belowskip=-6pt, singlelinecheck=off, justification=justified}
\usepackage{placeins}
\usepackage{longtable}
\usepackage{booktabs}
\usepackage{array}
\usepackage{multirow}
\usepackage{wrapfig}
\usepackage{float}
\usepackage{colortbl}
\usepackage{pdflscape}
\usepackage{tabu}
\usepackage{threeparttable}
\usepackage{threeparttablex}
\usepackage[normalem]{ulem}
\usepackage{makecell}
\usepackage{xcolor}

\begin{document}



%%%%%%%%%%%%%%%%%%%%%%%%%%%%%%%%%%%%%%%%%%

\hypertarget{introduction}{%
\section{Introduction}\label{introduction}}

The shortwave downward solar irradiance (SDR) at Earth's surface plays a
significant role on its climate. Changes in the SDR can be related to
changes in Earth's energy budget, the mechanisms of climate change, and
water and carbon cycles \citep{Wild2009}. It can also affect solar and
agricultural production and all living organisms. Studies of SDR
variability have identified some distinct SDR trends in different
regions of the world in different time periods. The term `brightening'
is generally used to describe periods of positive SDR trend, and
`dimming' for periods of negative trend \citep{Wild2009}. There are many cases in
the long-term records of irradiance showing a systematic change in the
magnitude of the trend, occurring roughly in the last decades of the
20th century. At multiple stations in China, a dimming period was
reported until about 2000, followed by a brightening period
\citep{Yang2021}. A similar pattern was identified, with a breaking
point around 1980, for stations in Central Europe \citep{Wild2021} and
Brazil~\citep{Yamasoe2021}. On global scale, an artificial intelligence
aided spatial analysis on the continental level with data from multiple
stations reached similar conclusions for these regions and for the
global trend \citep{Yuan2021}.

There is a consensus among researchers that the major factor affecting
the variability of SDR attenuation is the interactions of solar
radiation with atmospheric aerosols and clouds. Those interactions,
among other factors, have been analyzed with models
\citep{Li2016, Samset2018}, showing the existence of feedback mechanisms
between the two. Similar findings have been shown from the analysis of
observations at other \hl{locations} %MDPI: We revised citation. Please confirm. 
%Natsis: Confirmed
 \citep{Schwarz2020, Ohvril2009, Zerefos2009, Xia2007} [and references
therein]. In the
Mediterranean region, aerosols have been recognized as an important
factor affecting the penetration of solar radiation at the surface
\citep{Fountoulakis2016, Siomos2018, Gkikas2013, Lozano2021}. These
studies investigated the long-term trend in aerosol optical depth, which
has been found to decrease in the last three decades, the transport and
composition of aerosols, and their radiative effects.

Due to the significant spatial and temporal variability of the trends
and the contributing factors, there is a constant need to monitor and
investigate SDR at different sites in order to estimate the degree of
variability and its relation to the local conditions. In this study, we
examine the trends of SDR using ground-based measurements at
Thessaloniki, Greece, for the period from 1993 to 2023. We re-evaluated and
extended the dataset used by \citet{Bais2013}, we applied a different
algorithm for the identification of clear-/cloud-sky instances
\citep{Reno2016, Reno2012}, %Natsis: citation edited
and we derived the SDR trends for the
period of study under different sky conditions (all-sky, clear-sky, and
cloudy-sky). Finally, we investigated the dependence of the trends on
solar zenith angle and season.

\hypertarget{data-and-methodology}{%
\section{Data and Methodology}\label{data-and-methodology}}

The SDR data were measured with a Kipp \& Zonen CM-21 pyranometer
operating continuously at the Laboratory of Atmospheric Physics of the
Aristotle University of Thessaloniki (\(40^\circ\,38'\,\)N,
\(22^\circ\,57'\,\)E, \(80\,\)m~a.s.l.). Here, we used data for the period
from 13~April 1993 to 13 April 2023. The monitoring site was located near the
city center, thus we expect that measurements were affected by the urban
environment, mainly by aerosols. During the study period, the
pyranometer was independently calibrated three times at the
Meteorologisches Observatorium Lindenberg, DWD, verifying that the
stability of the instrument's sensitivity was better than \(0.7\%\)
relative to the initial calibration by the manufacturer. Along with SDR,
the direct beam radiation (DNI) was also measured with a collocated Kipp
\& Zonen CHP-1 pyrheliometer since 1 April 2016. The DNI data were used as
auxiliary data to support the selection of appropriate thresholds in the
clear-sky identification algorithm (CSid), which is discussed in Section
\ref{CDIDalgorithm}. It is noted that the limited dataset of DNI was not
used for the identification of clear-sky cases in the entire SDR series
to avoid any selection bias due to the unequal length of the two datasets.
There are four distinct steps in the creation of the dataset analyzed
here: (a)~the acquisition of radiation measurements from the sensors,
(b)~the data quality check, (c)~the identification of ``clear sky''
conditions from the SDR data, and (d)~the aggregation of data and trend
analysis.

For the acquisition of radiometric data, the signal of the pyranometer
was sampled at a rate of \(1\,\text{Hz}\). The mean and the standard
deviation of these samples were calculated and recorded every minute. The
measurements were corrected for the zero offset (``dark signal'' in
volts), which was calculated by averaging all measurements recorded for a
period of \(3\,\text{h}\), before (morning) or after (evening) the Sun
reaches an elevation angle of \(-10^\circ\). The signal was converted to
irradiance using a ramped value of the instrument's sensitivity between
subsequent calibrations.

A manual screening was performed to remove inconsistent and erroneous
recordings that can occur stochastically or systematically during the
continuous operation of the instruments. The manual screening was aided
by a radiation data quality assurance procedure, adjusted for the site,
which was based on the methods of Long and
Shi~\citep{Long2006, Long2008a}. Thus, problematic recordings have been
excluded from further processing. Although it is impossible to detect
all false data, the large number of available data, and the aggregation
scheme we used, ensures the quality of the radiation measurements used
in this study.

Due to the significant measurement uncertainty when the Sun is near the
horizon, we have excluded all measurements with solar zenith angle (SZA)
greater than \(85^\circ\). Moreover, due to obstructions around the site
(hills and buildings) that block the direct irradiance, we excluded
data with azimuth angle in the range \(58^{\circ}\)--\(120^{\circ}\)
and with SZA greater than \(78^{\circ}\). To make the measurements
comparable throughout the dataset, we adjusted all one-minute data to
the mean Sun--Earth distance. Subsequently, we adjusted all
measurements to the total solar irradiance (TSI) at \(1\,\text{au}\) in
order to compensate for the Sun's intensity variability using a time
series of satellite TSI observations. The TSI data we used are part of
the `NOAA Climate Data Record of Total Solar Irradiance' dataset
\citep{Coddington2005}. The initial daily values of this dataset were
interpolated to match the time step of our measurements.

In order to estimate the effect of the sky conditions on the long-term
variability of SDR, we created three datasets by characterizing each
one-minute measurement with a corresponding sky-condition flag (i.e.,
all-sky, clear-sky, and cloudy-sky). To identify the clear cases, we used
the method proposed by \citet{Reno2016}, which requires the definition
of some site specific parameters. These parameters were determined by an
iterative process, as the original authors proposed, and are discussed in
the next section.

We note that all methods have some subjectivity in the definition of
clear or cloudy sky cases. As a result, the details of the definition
are site specific, and they rely on a combination of thresholds and
comparisons with ideal radiation models and statistical analysis of
different signal metrics. The CSid algorithm was calibrated with the
main focus to identify the presence of clouds. Despite the fine-tuning
of the procedure, in a few marginal cases, false positive or false
negative results were identified by manual inspection. However, due to
their small number, they did not affect the final results of the study.
For completeness, we provide below a brief overview of the CSid
algorithm, along with the site-specific thresholds.

\hypertarget{CDIDalgorithm}{%
\subsection{The Clear Sky Identification
Algorithm}\label{CDIDalgorithm}}

To calculate the reference clear-sky \(\text{SDR}_\text{CSref}\), we used
the \(\text{SDR}_\text{Haurwitz}\) derived by the radiation model of
\citet{Haurwitz1945} (Equation~\eqref{eq:hau}), adjusted for our site:
\begin{equation}
\text{SDR}_\text{Haurwitz} = 1098 \times \cos(\theta) \times \exp \left( \frac{ - 0.059}{\cos(\theta)} \right) \label{eq:hau}
\end{equation} where \(\theta\) is the SZA.

The adjustment was made with a factor \(a\) (Equation~\eqref{eq:ahau}), which
was estimated through an iterative optimization process, as described by
\citet{Long2000} and \citet{Reno2016}. The target of the optimization
was the minimization of a function \(f(a)\) (Equation~\eqref{eq:minf}) and was
accomplished with the algorithmic function `optimise', which is an
implementation based on the work of \citet{Brent1973}, from the library
`stats' of the R programming language \citep{RCT2023}.
\begin{equation}
f(a) = \frac{1}{n}\sum_{i=1}^{n} ( \text{SDR}_{\text{CSid},i} - a \times \text{SDR}_{\text{testCSref},i} )^2 \label{eq:minf}
\end{equation} where \(n\) is the total number of daylight data,
\(\text{SDR}_{\text{CSid},i}\) are the data identified as clear-sky by
CSid, \(a\) is a site-specific adjustment factor, and
\(\text{SDR}_{\text{testCSref},i}\) is the SDR derived by any of the
tested clear-sky radiation models.

The optimization and the selection of the clear-sky reference model was
performed on SDR observations for the period 2016--2021. During the
optimization, eight simple clear-sky radiation models were tested
(namely, Daneshyar--Paltridge--Proctor, Kasten--Czeplak, Haurwitz,
Berger--Duffie, Adnot--Bourges--Campana--Gicquel, Robledo--Soler, Kasten, and
Ineichen--Perez) with a wide range of factors. These models are
described in more detail by \citet{Reno2012} and are evaluated by
\citet{Reno2016}. We found that Haurwitz's model, adjusted with the
factor \(a = 0.965\), yields one of the lowest root mean squared errors
(RMSE), and the procedure manages to successfully characterize the
majority of the data. Thus, our clear sky reference is derived by
Equation~\eqref{eq:ahau}: \begin{equation}
\text{SDR}_\text{CSref} = a \times \text{SDR}_\text{Haurwitz} = 0.965 \times 1098 \times \cos(\theta) \times \exp \left( \frac{ - 0.057}{\cos(\theta)} \right) \label{eq:ahau}
\end{equation}

The criteria that were used to identify whether a measurement was taken
under clear-sky conditions are presented below. A data point is flagged
as `clear-sky' if all criteria are satisfied; otherwise, it is
considered as `cloud-sky'. Each criterion was applied for a running
window of \(11\) consecutive one-minute measurements, and the
characterization was assigned to the central datum of the window. Each
parameter was calculated from the observations in comparison to the
reference clear-sky model. The allowable range of variation is defined
by the model-derived value of the parameter multiplied by a factor plus
an offset. The factors and the offsets were determined empirically by
manually inspecting each filter's performance on selected days and
adjusting them accordingly during an iterative process. The criteria are
listed below, together with the range of values within which the
respective parameter should fall in order to raise the clear-sky flag:

\begin{enumerate}
\def\labelenumi{(\alph{enumi})}
\tightlist
\item
  \hl{Mean} %MDPI: We removed italics of units. Please confirm. 
  %Natsis: Confirmed
 of the measured \(\overline{\text{SDR}}_i\) (Equation~\eqref{eq:MeanVIP}): \begin{equation}
  0.91 \times \overline{\text{SDR}}_{\text{CSref},i} - 20\,\text{Wm}^{-2}
  < \overline{\text{SDR}}_i <
  1.095 \times \overline{\text{SDR}}_{\text{CSref},i} + 30\,\text{Wm}^{-2}
  \label{eq:MeanVIP}
  \end{equation}
\item
  Maximum measured value \(M_{i}\) (Equation~\eqref{eq:MaxVIP}):
  \begin{equation}
  1 \times M_{\text{CSref},i} - 75\,\text{Wm}^{-2}
  < M_{i} <
  1 \times M_{\text{CSref},i} + 75\,\text{Wm}^{-2}
  \label{eq:MaxVIP}
  \end{equation}
\item
  Length \(L_i\) of the sequential line segments, connecting the points
  of the \(11\) SDR values (Equation~\eqref{eq:VILeq}): \begin{equation}
  L_i = \sum_{i=1}^{n-1}\sqrt{\left ( \text{SDR}_{i+1} - \text{SDR}_{i}\right )^2 + \left ( t_{i+1} - t_i \right )^2}
  \label{eq:VILeq}
  \end{equation} \begin{equation}
  1 \times L_{\text{CSref},i} - 5 < L_i < 1.3 \times L_{\text{CSref},i} + 13
  \label{eq:VILcr}
  \end{equation} where \(t_i\) is the time stamp of each SDR
  measurement.
\item
  Standard deviation \(\sigma_i\) of the slope (\(s_i\)) between the
  \(11\) sequential points, normalized by the mean
  \(\overline{\text{SDR}}_i\) (Equation~\eqref{eq:VCT1}): \begin{gather}
    \sigma_i = \frac {\sqrt{\frac{1}{n-1} \sum_{i=1}^{n-1} \left( s_i - \bar{s} \right)^2}} {\overline{\text{SDR}}_i} \label{eq:VCT1} \\
    s_i = \frac{\text{SDR}_{i+1} - \text{SDR}_{i}}{t_{i+1} - t_i},\;\;   \bar{s} = \frac{1}{n-1} \sum_{i=1}^{n-1} s_i,\;\;\forall i \in \left \{ 1, 2, \ldots, n-1 \right \}\;\;
  \end{gather} For this criterion, \(\sigma_i\) should be below a
  certain threshold (Equation~\eqref{eq:VCTcr}): \begin{equation}
    \sigma_i < \ensuremath{1.1\times 10^{-4}} \label{eq:VCTcr}
  \end{equation}
\item
  Maximum difference \(X_i\) between the change in measured irradiance
  and the change in clear sky irradiance over each measurement interval:
  \begin{gather}
    X_i = \max{\left \{ \left | x_i - x_{\text{CSref},i} \right | \right \}} \label{eq:VSM3} \\
    x_i = \text{SDR}_{i+1} - \text{SDR}_{i} \forall i \in \left \{ 1, 2, \ldots, n-1 \right \} \label{eq:VSM1} \\
    x_{\text{CSref},i} = \text{SDR}_{\text{CSref},i+1} - \text{SDR}_{\text{CSref},i} \forall i \in \left \{ 1, 2, \ldots, n-1 \right \} \label{eq:VSM2}
  \end{gather} For this criterion, \(X_i\) should be below a certain
  threshold (Equation~\eqref{eq:VSMcr}): \begin{equation}
    X_i < 7.5\,\text{Wm}^{-2} \label{eq:VSMcr}
  \end{equation}
\end{enumerate}

In the final dataset, \(26\%\) of the days were identified as under
clear-sky conditions and \(48\%\) as under cloud-sky conditions. The
remaining \(26\%\) of the data correspond to mixed cases and were not
analyzed as a separate group.

\hypertarget{aggregationstatistical}{%
\subsection{Aggregation of Data and Statistical
Approach}\label{aggregationstatistical}}

In order to investigate the SDR trends that are the main focus of the
study, we implemented an aggregation scheme to the one-minute data to
derive series in coarser time scales. To preserve the representativeness
of the data, we used the following criteria: (a)~we excluded all days with
less than 50\% of the expected daytime measurements, (b)~daily means for
the clear-sky and cloudy-sky datasets were calculated only for days with
more than 60\% of the expected daytime measurements identified as clear
or cloudy, respectively, and (c) monthly means were computed from daily means.
For the all-skies dataset, monthly means were computed only when at least
20 days were available. Seasonal means were derived by averaging the
monthly mean values in each season (winter: December--February, spring:
March--May, etc.). The daily and monthly climatological means were
derived by averaging the data for each day of the year and calendar month,
respectively. The daily and monthly datasets were deseasonalized by
subtracting the corresponding climatological annual cycle (daily or
monthly) from the actual data. Finally, to estimate the SZA effect on
the SDR trends, the one-minute data were aggregated in \(1^{\circ}\) SZA
bins, separately for the morning and afternoon hours.

The linear trends were calculated using a first-order autoregressive
model with lag of 1~day using the `maximum likelihood' fitting method
\citep{Gardner1980, Jones1980} by implementing the function `arima'
from the library `stats' of the R programming language
\citep{RCT2023}. The trends were reported together with the \(2\sigma\)
errors.

\hypertarget{results}{%
\section{Results}\label{results}}

\hypertarget{long-term-sdr-trends}{%
\subsection{Long-Term SDR Trends}\label{long-term-sdr-trends}}

We calculated the linear trends of SDR from the departures of the mean
daily values from the daily climatology and for the three sky
conditions. These are presented in Table~\ref{tab:trendtable}, which also contains the \(2\sigma\) standard error, the Pearson's correlation
coefficient R, and the trend in absolute units. In
Figure~\ref{fig:trendALL}, we present only the time series under all-sky
conditions; the plots for clear-sky and cloud-sky conditions, are shown
in \mbox{\hl{Appendix} %MDPI: We revised citation. Please confirm. 
	%Natsis: confirmed
 \ref{app1}} \mbox{(Figures~\ref{fig:trendCLEAR} and~
\ref{fig:trendCLOUD}).} In the studied period, there is no significant
break or change in the variability pattern of the time series. The
linear trends in all three datasets are positive and \hl{around} %MDPI: We removed italics of units. Please confirm. The same below.  
%Natsis: confirmed
\(0.4\%/\)y for all-sky and cloudy-sky conditions, whereas for
clear-skies the trend is much smaller (\textasciitilde{}\(0.1\%/\)y).
The linear trends were calculated taking into account the
autocorrelation of the time series, and all three are statistically
significant at least at the \(95\%\) confidence level, as they are
larger than the corresponding \(2\sigma\) errors, despite the small
values of R, which is due to the large variability of the daily values.
The clear-sky trend is very small, suggesting a small effect from
aerosols and water vapor, which are the dominant factors of the SDR
variability \citep{Fountoulakis2016, Siomos2018, Yu2022}. In contrast,
the large positive trend of SDR under cloudy skies can be attributed to
reduction in cloud cover and/or cloud optical depth. Lack of continuous
observations of cloud optical thickness that could support these
findings does not allow drawing firm conclusions. However, there are
indications that the total cloud cover as inferred from the ERA5
analysis for the grid point of Thessaloniki is decreasing over the
period of study. From the difference between all-sky and clear-sky SDR
trends, expressed in W/m$^2$/y using the long-term mean of the
respective datasets, the radiative effect of clouds is estimated to
0.96 W/m$^2$/y. This estimate is similar to the cloud radiative
forcing of 1.22 W/m$^2$/y  reported for Granada, Spain
\citep{Lozano2023}.

The all-sky trend is similar to the one reported in \citet{Bais2013}
from a ten-year shorter dataset, suggesting that the tendency of SDR in
Thessaloniki is systematic. Other studies for the European region
reported a change in the SDR trend around 1980 from negative to positive
with comparable magnitude \citep{Wild2021, Yuan2021, Ohmura2009}, well
before the start of our records. However, the trends reported here for
the three datasets are in accordance with the widely accepted solar
radiation brightening over Europe. For the period of our observations,
the trend in the TSI is negligible (\(-0.00022\%/\)y), and thus we
cannot attribute any significant effect on the SDR trend to solar
variability.

\begin{table}[H]

\caption{\label{tab:trendtable}Trends in SDR daily means for different sky conditions for the period 1993--2023.}
\begin{tabu} to \linewidth {>{\centering\arraybackslash}p{8em}>{\raggedleft}X>{\raggedleft}X>{\raggedleft}X>{\raggedleft}X>{\raggedleft}X}
\toprule
\multirow{-1}{*}{\textbf{Sky Conditions}} & \textbf{Trend [\%/year]} & \textbf{Trend S.E. (\boldmath$2\sigma$)} & \textbf{Pearson Correl.} & \textbf{Trend [W/m\textsuperscript{2}/year]} & \multirow{-1}{*}{\textbf{Days}}\\
\midrule
All skies & 0.380 & 0.120 & 0.091 & 1.460 & 10251\\
Clear skies & 0.097 & 0.033 & 0.140 & 0.501 & 2684\\
Cloudy skies & 0.410 & 0.180 & 0.081 & 1.180 & 4937\\
\bottomrule
\end{tabu}
\end{table}

\vspace{-6pt}

\begin{figure}[H]

{ \includegraphics[width=.75\linewidth]{./images/LongtermTrends-1} 

}

\caption{Anomalies (\%) of the daily all-sky SDR from the climatological mean for the period 1993--2023. The black line is the long-term linear trend.}\label{fig:trendALL}
\end{figure}

Although the year-to-year variability of the anomalies (Figure
\ref{fig:trendALL} and \mbox{Figures~\ref{fig:trendCLEAR} and
\ref{fig:trendCLOUD}} in \mbox{Appendix \ref{app1})} shows a rather homogeneous behavior,
plots of the cumulative sums (CUSUM) \citep{Regier2019} of the anomalies
can reveal different structures in the records of all three sky
conditions. For time series with a uniform trend, we would expect the
CUSUMs of the anomalies to have a symmetric `V' shape centered around
the middle of the series. This would indicate that the anomalies are
evenly distributed around the climatological mean, and, for a positive
uniform trend, the first half is below and the second half above the
climatological mean. In our case, there is a more complex evolution of
the anomalies. For all skies (Figure~\ref{fig:cusummonth}a), we observe
three rather distinct periods: (a) a downward part between the start of
the datasets and about 2000, denoting that all anomalies are negative,
thus below the climatology; (b) a relatively steady part lasting for
almost 20 years, suggesting little variability in SDR anomalies; and (c) a
steep upward part to the present, indicating anomalies above the climatology.
The CUSUMs for cloudy skies (Figure~\ref{fig:cusummonth}c) show a
similar behavior with some short-term differences that do not change the
overall pattern. For clear skies (Figure~\ref{fig:cusummonth}b), a
monotonic downward tendency is evident until 2004, suggesting that the
anomalies are all negative. After 2004, the anomalies turn positive at
a fast rate for about five years and at a slower rate thereafter.

\begin{figure}[H]
    \begin{adjustwidth}{-\extralength}{0cm}
        {\centering 
       % \subfloat[\label{fig:cusummonth-1}]
            {\includegraphics[width=.32\linewidth]{./images/CumulativeMonthlyCuSum-1}}\hfill
       % \subfloat[\label{fig:cusummonth-2}]
            {\includegraphics[width=.32\linewidth]{./images/CumulativeMonthlyCuSum-5}}\hfill
       % \subfloat[\label{fig:cusummonth-3}]  %Natsis: commented this line
            {\includegraphics[width=.32\linewidth]{./images/CumulativeMonthlyCuSum-9}}\hfill
        }
\end{adjustwidth}
\caption{\hl{Cumulative} %MDPI: We moved the subfigure explanations into the figure caption. Please confirm.  %Natsis: confirmed
 sum plots of the monthly SDR anomalies in (\%) for different sky conditions: (\textbf{a})~all skies; (\textbf{b}) clear skies; (\textbf{c}) cloudy skies.}\label{fig:cusummonth}

\end{figure}

In order to unveil further the features of the variability of the three
datasets, Figure~\ref{fig:cusumnotrendmonthly} presents another set of
CUSUM plots using anomalies after the long-term linear trend is removed.
With this approach, periods when the CUSUMs diverge from zero can be
interpreted as a systematic variation of SDR from the climatological
mean. When the CUSUM is increasing, the anomalies values are above the
climatology, and vice versa. Overall, for all- and cloudy-sky conditions
(Figure~\ref{fig:cusumnotrendmonthly}a,c), we observe periods with anomalies
diverging from the climatological values, each lasting for several
years. These fluctuations are probably within the natural variability,
and no distinct changes are identified. The pattern in both datasets is
similar, suggesting prevalence in cloudy skies over Thessaloniki. For
clear skies (Figure~\ref{fig:cusumnotrendmonthly}b), the distinct change
in 2004 is now clearer. The most likely reason for this change is the
monotonic reduction of aerosols in Thessaloniki. In that year, there was a
change in the rate of decrease in aerosol optical depth, as illustrated
in Figure 7 of \citet{Siomos2020}. This abrupt change in CUSUMs lasted
until about 2010, when the anomalies become again variable.

\begin{figure}[H]
    \begin{adjustwidth}{-\extralength}{0cm}
        {\centering 
          %  \subfloat[All skies.\label{fig:cusumnotrendmonthly-1}]
                {\includegraphics[width=.32\linewidth]{./images/CumulativeMonthlyCuSumNOtrend-1} }\hfill
           % \subfloat[Clear skies.\label{fig:cusumnotrendmonthly-2}]
                {\includegraphics[width=.32\linewidth]{./images/CumulativeMonthlyCuSumNOtrend-5} }\hfill
           % \subfloat[Cloudy skies.\label{fig:cusumnotrendmonthly-3}]
                {\includegraphics[width=.32\linewidth]{./images/CumulativeMonthlyCuSumNOtrend-9} }
        }

\end{adjustwidth}
        \caption{\hl{Cumulative} %MDPI: We moved the subfigure explanations into the figure caption. Please confirm.  %Natsis: confirmed
 sum plots of monthly SDR anomalies in (\%) for different sky conditions after removing the long-term linear trend: (\textbf{a})~all skies; (\textbf{b}) clear skies; (\textbf{c}) cloudy skies.}\label{fig:cusumnotrendmonthly}
\end{figure}

\hypertarget{effects-of-the-solar-zenith-angle-on-sdr}{%
\subsection{Effects of the Solar Zenith Angle on
SDR}\label{effects-of-the-solar-zenith-angle-on-sdr}}

The solar zenith angle is a major factor affecting the SDR, as
increases in SZA leads to enhancement of the radiation path in the
atmosphere, especially in urban environments with human activities
emitting aerosols \citep{Wang2021}. In order to estimate the effect of
SZA on the SDR trends, we grouped the data in bins of \(1^\circ\) SZA
and calculated the overall trend for each bin separately for the daily
periods before noon and after noon (Figure~\ref{fig:szatrends}).
Although there are seasonal dependencies of the minimum SZA (see
Appendix \ref{app1}, Figure~\ref{fig:SZAtrendSeason}), these dependencies are not
discussed further.

For all-sky conditions, the brightening effect of SDR (positive trend)
increases with SZAs (Figure~\ref{fig:szatrends}a), ranging from about
\(0.1\%/\)y to about \(0.7\%/\)y for the statistically significant
trends. The trends in the morning and afternoon hours are more or less
consistent with small differences at small SZAs, which can be attributed
to effects on clear sky SDR from systematic diurnal patterns of aerosols
during the warm period of the year, consistent with the results
reported for China by \citet{Wang2021}. Note that SZAs less than
\(25^\circ\) can only occur during the warm period of the year around
noon when clear skies are more frequent. The increasing trend with SZA
is likely caused by the increased attenuation of SDR with SZA. The
effect is larger when aerosol and/or cloud layers are optically thicker;
therefore, decreases in aerosol and clouds through the study period will
result in larger positive trends of SDR at larger SZAs.

Under clear skies (Figure~\ref{fig:szatrends}b), the trends are
smaller and less variable, ranging between \(0.1\) and \(0.15\%/\)y up
to \(77^\circ\) SZA. At higher SZAs and in the afternoon hours, there is a
sharp increase in the trend up to \(0.3\%/\)y, which may have been
caused by the long path length of radiation through the atmosphere as
discussed above for the all-sky conditions. The small differences in the
trend between morning and afternoon between \(35^\circ\) and
\(60^\circ\) SZA is likely a result of less attenuation of SDR in the
morning hours due to lesser amounts of aerosols and a shallower boundary
layer.

For cloudy-sky conditions (Figure~\ref{fig:szatrends}c), we cannot
discern any significant dependence of the SDR trend with SZA, as the
variability of irradiance is dominated by the cloud effects leading to
insignificant trends. Statistically significant trends appear only in
the afternoon and for SZAs larger than \(60^\circ\). The sharp increase
of the trend at SZAs larger than $\sim${75}$^{\circ}$, observed also for
clear skies, is probably associated with stronger attenuation by clouds
under oblique incidence angles, which also result in smaller
variability.

\begin{figure}[H]
    \begin{adjustwidth}{-\extralength}{0cm}
        {\centering 
          %  \subfloat[All skies.\label{fig:szatrends-1}]
                {\includegraphics[width=.32\linewidth]{./images/SzaTrends-1}}\hfill
         %   \subfloat[Clear skies.\label{fig:szatrends-2}]
                {\includegraphics[width=.32\linewidth]{./images/SzaTrends-4}}\hfill
           % \subfloat[Cloudy skies.\label{fig:szatrends-3}]
                {\includegraphics[width=.32\linewidth]{./images/SzaTrends-7}}
        }

    \end{adjustwidth}
        \caption{\hl{Long}%MDPI: We removed duplicate explanation in figure. %Natsis: confirmed
 -term trends of daily SDR as a function of SZA for (\textbf{a}) all-sky, (\textbf{b}) clear-sky and (\textbf{c}) cloudy-sky conditions, separately for morning and afternoon periods. Solid shapes represent statistically significant trends ($p < 0.005$).}\label{fig:szatrends}
\end{figure}



\hypertarget{long-term-sdr-trends-by-season}{%
\subsection{Long-Term SDR Trends by
Season}\label{long-term-sdr-trends-by-season}}

Similarly to the long term trends from daily means of SDR discussed
above, we have calculated the trend for the three sky conditions and for
each season of the year using the corresponding mean monthly anomalies
(Figure~\ref{fig:seasonalALL} and Table~\ref{tab:trendseasontable}).
Table~\ref{tab:trendseasontable} also contains the \(2\sigma\) standard
error, the Pearson's correlation coefficient R, and the corresponding
p-value. The winter linear trends generally exhibit the largest R values,
ranging between \(0.54\) and \(0.60\%/\)y. For all-sky conditions, the
trend in SDR in winter is the largest (\(0.7\%/\)y), followed by the
trend in autumn (\(0.42\%/\)y, a value close to the long-term trend),
both statistically significant at the \(95\%\) confidence level. In
spring and summer, the trends are much smaller and of lesser statistical
significance. These seasonal differences indicate a possible relation of
the trends in SDR to trends in clouds during winter and autumn. For
clear skies, the trend in winter is \(0.4\%/\)y and is associated with
the decreasing trend in aerosol optical depth \citep{Siomos2020}.
Moreover, it is almost half of that for all skies, which is another
indication of a decreasing trend in cloud optical thickness. In other
seasons, the clear-sky trend is very small (below \(0.1\%/\)y).
Finally, for cloudy skies, the winter trend is the largest
(\(0.76\%/\)y) and greater than for all skies, followed by a much
smaller trend in autumn (\(0.19\%/\)y).

The trends under clear- and cloudy-sky conditions are in the same
direction, and it would be expected that their sum is similar to the
all-sky trend. This does not happen, especially for winter, likely due
to the way the monthly means for clear and cloudy skies were calculated.
Daily means were calculated only when at least \(60\%\) of the clear-
or cloudy-sky data were available (see Section \ref{aggregationstatistical}).



\begin{table}[H]

\caption{\label{tab:trendseasontable}SDR linear trends of monthly anomalies for each season of the year and related statistical parameters.}
\begin{tabu} to \linewidth {>{\centering\arraybackslash}p{8em}>{\centering}X>{\raggedleft}X>{\raggedleft}X>{\raggedleft}X>{\raggedleft}X}
\toprule
\multirow{-1}{*}{\textbf{Sky Condition}} & \multirow{-1}{*}{\textbf{Season}} & \textbf{Trend [\%/year]} & \textbf{Trend S.E. (\boldmath$2\sigma$)} & \textbf{Pearson Correl.} & \textbf{Trend \emph{p}-Value}\\
\midrule
\cellcolor{gray!6}{} & \cellcolor{gray!6}{Winter} & \cellcolor{gray!6}{0.70} & \cellcolor{gray!6}{0.43} & \cellcolor{gray!6}{0.54} & \cellcolor{gray!6}{0.003}\\

 & Spring & 0.11 & 0.24 & 0.17 & 0.371\\

\cellcolor{gray!6}{} & \cellcolor{gray!6}{Summer} & \cellcolor{gray!6}{0.11} & \cellcolor{gray!6}{0.15} & \cellcolor{gray!6}{0.25} & \cellcolor{gray!6}{0.175}\\

\multirow{-4}{*}{\centering\arraybackslash All skies} & Autumn & 0.42 & 0.30 & 0.47 & 0.009\\
\cmidrule{1-6}
\cellcolor{gray!6}{} & \cellcolor{gray!6}{Winter} & \cellcolor{gray!6}{0.40} & \cellcolor{gray!6}{0.20} & \cellcolor{gray!6}{0.60} & \cellcolor{gray!6}{0.001}\\

 & Spring & 0.06 & 0.17 & 0.13 & 0.497\\

\cellcolor{gray!6}{} & \cellcolor{gray!6}{Summer} & \cellcolor{gray!6}{$-$0.05} & \cellcolor{gray!6}{0.06} & \cellcolor{gray!6}{$-$0.30} & \cellcolor{gray!6}{0.106}\\

\multirow{-4}{*}{\centering\arraybackslash Clear skies} & Autumn & 0.05 & 0.12 & 0.17 & 0.366\\
%\cmidrule{1-6}
\bottomrule
\end{tabu}
\end{table}

\begin{table}[H]\ContinuedFloat

\caption{\label{tab:trendseasontable}\emph{Cont.}}
\begin{tabu} to \linewidth {>{\centering\arraybackslash}p{8em}>{\centering}X>{\raggedleft}X>{\raggedleft}X>{\raggedleft}X>{\raggedleft}X}
\toprule
\multirow{-1}{*}{\textbf{Sky Condition}} & \multirow{-1}{*}{\textbf{Season}} & \textbf{Trend [\%/year]} & \textbf{Trend S.E. (\boldmath$2\sigma$)} & \textbf{Pearson Correl.} & \textbf{Trend \emph{p}-Value}\\
\midrule
\cellcolor{gray!6}{} & \cellcolor{gray!6}{Winter} & \cellcolor{gray!6}{0.76} & \cellcolor{gray!6}{0.40} & \cellcolor{gray!6}{0.59} & \cellcolor{gray!6}{0.001}\\

 & Spring & 0.06 & 0.23 & 0.10 & 0.593\\

\cellcolor{gray!6}{} & \cellcolor{gray!6}{Summer} & \cellcolor{gray!6}{$-$0.08} & \cellcolor{gray!6}{0.27} & \cellcolor{gray!6}{$-$0.11} & \cellcolor{gray!6}{0.560}\\

\multirow{-4}{*}{\centering\arraybackslash Cloudy skies} & Autumn & 0.19 & 0.43 & 0.16 & 0.384\\
\bottomrule
\end{tabu}
\end{table}


\vspace{-7pt}

\begin{figure}[H]
    \begin{adjustwidth}{-\extralength}{0cm}
        {\centering 
            \includegraphics[width=1\linewidth]{./images/SeasonalMTrendsTogether3-2}   %Natsis: Image update Automn -> Autumn
        }
       
    \end{adjustwidth}
 \caption{Time series of \colorbox{red}{monthly} %MDPI: Please add the explanation for subfifgures in the figure.  %Natsis: Done that, and also rephrased the caption
 mean anomalies of SDR by season (rows of plots) for the three sky conditions (columns of plots). The black lines represent the linear trends, (\textbf{a}-\textbf{c}) winter, (\textbf{d}-\textbf{f}) spring, (\textbf{g}-\textbf{i}) summer, (\textbf{j}-\textbf{l}) autumn, and also (\textbf{a}, \textbf{d}, \textbf{g}, \textbf{j}) all-skies, (\textbf{b}, \textbf{e}, \textbf{h}, \textbf{k}) clear-skies and (\textbf{c}, \textbf{f}, \textbf{i}, \textbf{l}) cloudy-skies.}\label{fig:seasonalALL}
\end{figure}


\hypertarget{conclusions}{%
\section{\colorbox{green}{Conclusions}\label{conclusions}}} %MDPI: Please make sure no figure in paper is repeated

We have analyzed a 30-year dataset of SRD measurements in Thessaloniki,
Greece (1993--2023), aiming to identify the long-term variability of
solar irradiance under different sky conditions. Under all-sky
conditions, there is a positive trend in SDR of \(0.38\%/\)y
(brightening). A previous study \citep{Bais2013} for the period 1993--2011 also reported a positive trend of \(0.33\%/\)y. The slight
increase in this trend indicates that the brightening of SDR continues
and is likely caused by continuing decreases in aerosol optical depth
and the optical thickness of clouds over the area. A smaller trend has
been found under clear-sky conditions (\(0.097\%/\)y), which supports
the notion that part of the brightening is caused by decreasing
aerosols. \citet{Siomos2020} showed that aerosol optical depth over
Thessaloniki was decreasing constantly, at least up until 2018. The
attenuation of SDR by aerosols over Europe has been proposed as major
factor by \citet{Wild2021}. Unfortunately, for this study, aerosol data
for the entire period were not available in order to quantify their
effect on SDR. The brightening effect on SDR under cloudy-sky conditions
(\(0.41\%/\)y) suggests that cloud optical thickness is also
decreasing during this period. As long-term data of cloud optical
thickness are also not available for the region, we cannot draw
quantitative~conclusions.

The observed brightening on SDR over Thessaloniki is dependent on SZA
(larger SZAs lead to stronger brightening). The trend is also dependent
on season, with winter showing the strongest statistically significant
trend of \(0.7\) and \(0.76\%/\)y for all- and cloudy-skies,
respectively, in contrast to spring and summer. The trends for autumn
are also significant but smaller (\(0.42\) and \(0.19\%/\)y for all-
and cloudy-skies, respectively). The trend for clear skies is largest in
winter (\(0.4\%/\)y) and negligible in spring, summer, and autumn.

Using the CUSUMs of the monthly departures for all and cloudy skies, we
observed a 20-year period starting around 2000 where the CUSUMs remain
relatively stable, with a steep decline before and a steep increase
after. The rather smooth course of the CUSUMs suggests that no important
change in the SDR pattern has occurred in the entire record.

Continued observations with a collocated pyrheliometer, which started in
2016, will allow us to further investigate the variability of solar
radiation at ground level in Thessaloniki. Also, additional data of
cloudiness, aerosols, atmospheric water vapor, etc., will allow better
attribution and quantification of the effects of these factors on SRD.

%%%%%%%%%%%%%%%%%%%%%%%%%%%%%%%%%%%%%%%%%%

\vspace{6pt}

%%%%%%%%%%%%%%%%%%%%%%%%%%%%%%%%%%%%%%%%%%
%% optional

% Only for the journal Methods and Protocols:
% If you wish to submit a video article, please do so with any other supplementary material.
% \supplementary{The following supporting information can be downloaded at: \linksupplementary{s1}, Figure S1: title; Table S1: title; Video S1: title. A supporting video article is available at doi: link.}
\authorcontributions{\hl{  } %MDPI: For research articles with several authors, a short paragraph specifying their individual contributions must be provided. The following statements should be used ``Conceptualization, X.X. and Y.Y.; methodology, X.X.; software, X.X.; validation, X.X., Y.Y. and Z.Z.; formal analysis, X.X.; investigation, X.X.; resources, X.X.; data curation, X.X.; writing---original draft preparation, X.X.; writing---review and editing, X.X.; visualization, X.X.; supervision, X.X.; project administration, X.X.; funding acquisition, Y.Y. All authors have read and agreed to the published version of the manuscript.'', please turn to the  \href{http://img.mdpi.org/data/contributor-role-instruction.pdf}{CRediT taxonomy} for the term explanation. Authorship must be limited to those who have contributed substantially to the work~reported.
Conceptualization, methodology, software, data analysis and original writing---draft preparation A.N.; writing---review and editing, supervision, A.B.; data curation, C.M. All authors have read and agreed to the published version of the manuscript.} %Natsis: section added
%%%%%%%%%%%%%%%%%%%%%%%%%%%%%%%%%%%%%%%%%%

\funding{This research received no external funding.}

\institutionalreview{\hl{  } %MDPI: In this section, you should add the Institutional Review Board Statement and approval number, if relevant to your study. You might choose to exclude this statement if the study did not require ethical approval. Please note that the Editorial Office might ask you for further information. Please add “The study was conducted in accordance with the Declaration of Helsinki, and approved by the Institutional Review Board (or Ethics Committee) of NAME OF INSTITUTE (protocol code XXX and date of approval).” for studies involving humans. OR “The animal study protocol was approved by the Institutional Review Board (or Ethics Committee) of NAME OF INSTITUTE (protocol code XXX and date of approval).” for studies involving animals. OR “Ethical review and approval were waived for this study due to REASON (please provide a detailed justification).” OR “Not applicable” for studies not involving humans or animals.
Not applicable}  %Natsis: section added

\informedconsent{\hl{  } %MDPI: Any research article describing a study involving humans should contain this statement. Please add ``Informed consent was obtained from all subjects involved in the study.'' OR ``Patient consent was waived due to REASON (please provide a detailed justification).'' OR ``Not applicable'' for studies not involving humans. You might also choose to exclude this statement if the study did not involve humans.
%Written informed consent for publication must be obtained from participating patients who can be identified (including by the patients themselves). Please state ``Written informed consent has been obtained from the patient(s) to publish this paper'' if applicable.
Not applicable} %Natsis: section added

\dataavailability{Data as daily sums are available through the WRDC
database, \url{http://wrdc.mgo.rssi.ru}. One-minute data are available on
request from the corresponding author. The data are not publicly
available for protection against unmonitored commercial use.}


\conflictsofinterest{\hl{  } %MDPI: Declare conflicts of interest or state ``The authors declare no conflict of interest.'' Authors must identify and declare any personal circumstances or interest that may be perceived as inappropriately influencing the representation or interpretation of reported research results. Any role of the funders in the design of the study; in the collection, analyses or interpretation of data; in the writing of the manuscript; or in the decision to publish the results must be declared in this section. If there is no role, please state ``The funders had no role in the design of the study; in the collection, analyses, or interpretation of data; in the writing of the manuscript; or in the decision to publish the results''.
The authors declare no conflict of interest.} %Natsis: section added
%%%%%%%%%%%%%%%%%%%%%%%%%%%%%%%%%%%%%%%%%%
%% Optional

%% Only for journal Encyclopedia

\abbreviations{Abbreviations}{
The following abbreviations are used in this manuscript:\\

\noindent
\begin{tabular}{@{}ll}
DNI & Direct beam/normal irradiance \\
ERA5 & ECMWF Reanalysis v5 \\
CSid & Clear sky identification algorithm \\
CUSUM & Cumulative sum \\
SDR & Solar downward radiation \\
SZA & Solar zenith angle \\
\end{tabular}}

%%%%%%%%%%%%%%%%%%%%%%%%%%%%%%%%%%%%%%%%%%
%% Optional
%\input{"appendix.tex"}

% % \renewcommand\thefigure{\thesection.\arabic{figure}} %% my
% % \renewcommand\thefigure{\thesection.\arabic{figure}}
%
% %% From R rticle MDPI template
% %% optional
% \appendixtitles{no} %Leave argument "no" if all appendix headings stay EMPTY (then no dot is printed after "Appendix A"). If the appendix sections contain a heading then change the argument to "yes".
% \appendixsections{one} %Leave argument "multiple" if there are multiple sections. Then a counter is printed ("Appendix A"). If there is only one appendix section then change the argument to "one" and no counter is printed ("Appendix").
%
% % \appendix
% % \section{}
%
% \appendix
% \setcounter{secnumdepth}{0}
% \section{Appendix}
%
% \setcounter{figure}{0}    %% my



%% From raw MDPI template
%% Optional
\appendixtitles{no} % Leave argument "no" if all appendix headings stay EMPTY (then no dot is printed after "Appendix A"). If the appendix sections contain a heading then change the argument to "yes".
\appendixstart
\appendix
\section[\appendixname~\thesection]{}

\label{app1}

% copy output from Rmd

\begin{figure}[H]
    {
        \includegraphics[width=0.75\linewidth]{./images/LongtermTrends-2}

    }
    \caption{Anomalies (\%) of the daily clear-sky SDR, relative to climatological values for 1993--2023. The black line shows the long-term linear trend for clear-sky conditions.}\label{fig:trendCLEAR}
\end{figure}



\begin{figure}[H]
    {
        \includegraphics[width=0.75\linewidth]{./images/LongtermTrends-3}

    }
    \caption{Anomalies (\%) of the daily cloud-sky SDR, relative to climatological values for 1993--2023. The black line shows the long-term linear trend for cloud-sky conditions.}\label{fig:trendCLOUD}
\end{figure}

\vspace{-6pt}


\begin{figure}[H]
    \begin{adjustwidth}{-\extralength}{0cm}
        {\centering
            \includegraphics[width=1.0\linewidth]{./images/SzaTrendsSeasTogether-2}
        }
   \end{adjustwidth}
        \caption{\colorbox{red}{Long}%MDPI: Please add the explanation for subfifgures in the figure.
 -term trends of SDR as a function of SZA separately form morning
               and afternoon periods, by season (rows of plots) for the three sky
               conditions (columns of plots).
               Solid shapes represent statistical significant trends ($p<0.005$), (\textbf{a}-\textbf{c}) winter, (\textbf{d}-\textbf{f}) spring, (\textbf{g}-\textbf{i}) summer, (\textbf{j}-\textbf{l}) autumn, and also (\textbf{a}, \textbf{d}, \textbf{g}, \textbf{j}) all-skies, (\textbf{b}, \textbf{e}, \textbf{h}, \textbf{k}) clear-skies and (\textbf{c}, \textbf{f}, \textbf{i}, \textbf{l}) cloudy-skies.
               Cases where $p<0.005$ or with less than $85$~observations may be
               missing from view.}\label{fig:SZAtrendSeason}
 
\end{figure}




\FloatBarrier

% \subsection{}
% The appendix is an optional section that can contain details and data supplemental to the main text. For example, explanations of experimental details that would disrupt the flow of the main text, but nonetheless remain crucial to understanding and reproducing the research shown; figures of replicates for experiments of which representative data is shown in the main text can be added here if brief, or as Supplementary data. Mathematical proofs of results not central to the paper can be added as an appendix.
%
% \section{}
% All appendix sections must be cited in the main text. In the appendixes, Figures, Tables, etc. should be labeled starting with `A', e.g., Figure A1, Figure A2, etc.

%%%%%%%%%%%%%%%%%%%%%%%%%%%%%%%%%%%%%%%%%%
\begin{adjustwidth}{-\extralength}{0cm}

%\printendnotes[custom] % Un-comment to print a list of endnotes


\reftitle{References}
%\bibliography{manualreferences.bib}
\begin{thebibliography}{999}

\bibitem[Wild(2009)]{Wild2009}
Wild, M.
\newblock Global dimming and brightening: A review.
\newblock {\em J. Geophys. Res. Atmos.} {\bf 2009}, {\em
  114},~1--31.
\newblock {\url{https://doi.org/10/bcq}}.

\bibitem[Yang et~al.(2021)Yang, Zhou, Yu, and Wild]{Yang2021}
Yang, S.; Zhou, Z.; Yu, Y.; Wild, M.
\newblock Cloud {\textquotedblleft}shrinking{\textquotedblright} and
  {\textquotedblleft}optical thinning{\textquotedblright} in the
  {\textquotedblleft}dimming{\textquotedblright} period and a subsequent
  recovery in the {\textquotedblleft}brightening{\textquotedblright} period
  over China.
\newblock {\em Environ. Res. Lett.} {\bf 2021}, \emph{16}, 034013.
\newblock {\url{https://doi.org/10.1088/1748-9326/abdf89}}.

\bibitem[Wild et~al.(2021)Wild, Wacker, Yang, and Sanchez-Lorenzo]{Wild2021}
Wild, M.; Wacker, S.; Yang, S.; Sanchez-Lorenzo, A.
\newblock Evidence for Clear‐Sky Dimming and Brightening in Central Europe.
\newblock {\em Geophys. Res. Lett.} {\bf 2021}, {\em 48}, e2020GL092216.
\newblock {\url{https://doi.org/10.1029/2020GL092216}}.

\bibitem[Yamasoe et~al.(2021)Yamasoe, Ros{\'{a}}rio, Almeida, and
  Wild]{Yamasoe2021}
Yamasoe, M.A.; Ros{\'{a}}rio, N.M.{\'{E}}.; Almeida, S.N.S.M.; Wild, M.
\newblock Fifty-six years of surface solar radiation and sunshine duration over
  S{\~{a}}o Paulo, Brazil: 1961--2016.
\newblock {\em Atmos. Chem. Phys.} {\bf 2021}, {\em
  21},~6593--6603.
\newblock {\url{https://doi.org/10.5194/acp-21-6593-2021}}.

\bibitem[Yuan et~al.(2021)Yuan, Leirvik, and Wild]{Yuan2021}
Yuan, M.; Leirvik, T.; Wild, M.
\newblock Global trends in downward surface solar radiation from spatial
  interpolated ground observations during 1961--2019.
\newblock {\em J. Clim.} {\bf 2021}, \emph{34}, 9501--9521.
\newblock {\url{https://doi.org/10.1175/JCLI-D-21-0165.1}}.

\bibitem[Li et~al.(2016)Li, Lau, Ramanathan, Wu, Ding, Manoj, Liu, Qian, Li,
  Zhou, Fan, Rosenfeld, Ming, Wang, Huang, Wang, Xu, Lee, Cribb, Zhang, Yang,
  Zhao, Takemura, Wang, Xia, Yin, Zhang, Guo, Zhai, Sugimoto, Babu, and
  Brasseur]{Li2016}
Li, Z.; Lau, W.K.; Ramanathan, V.; Wu, G.; Ding, Y.; Manoj, M.G.; Liu, J.;
  Qian, Y.; Li, J.; Zhou, T.;  et~al.
\newblock Aerosol and monsoon climate interactions over Asia.
\newblock {\em Rev. Geophys.} {\bf 2016}, {\em 54},~866--929.
\newblock {\url{https://doi.org/10.1002/2015RG000500}}.

\bibitem[Samset et~al.(2018)Samset, Sand, Smith, Bauer, Forster, Fuglestvedt,
  Osprey, and Schleussner]{Samset2018}
Samset, B.H.; Sand, M.; Smith, C.J.; Bauer, S.E.; Forster, P.M.; Fuglestvedt,
  J.S.; Osprey, S.; Schleussner, C.
\newblock Climate Impacts From a Removal of Anthropogenic Aerosol Emissions.
\newblock {\em Geophys. Res. Lett.} {\bf 2018}, {\em 45},~1020--1029.
\newblock {\url{https://doi.org/10.1002/2017GL076079}}.

\bibitem[Schwarz et~al.(2020)Schwarz, Folini, Yang, Allan, and
  Wild]{Schwarz2020}
Schwarz, M.; Folini, D.; Yang, S.; Allan, R.P.; Wild, M.
\newblock Changes in atmospheric shortwave absorption as important driver of
  dimming and brightening.
\newblock {\em Nat. Geosci.} {\bf 2020}, {\em 13},~110--115.
\newblock {\url{https://doi.org/10.1038/s41561-019-0528-y}}.

\bibitem[Ohvril et~al.(2009)Ohvril, Teral, Neiman, Kannel, Uustare, Tee,
  Russak, Okulov, Jõeveer, Kallis, Ohvril, Terez, Terez, Gushchin, Abakumova,
  Gorbarenko, Tsvetkov, and Laulainen]{Ohvril2009}
Ohvril, H.; Teral, H.; Neiman, L.; Kannel, M.; Uustare, M.; Tee, M.; Russak,
  V.; Okulov, O.; Jõeveer, A.; Kallis, A.;  et~al.
\newblock Global dimming and brightening versus atmospheric column
  transparency, Europe, 1906–2007.
\newblock {\em J. Geophys. Res.} {\bf 2009}, {\em 114}.
\newblock {\url{https://doi.org/10.1029/2008JD010644}}.

\bibitem[Zerefos et~al.(2009)Zerefos, Eleftheratos, Meleti, Kazadzis, Romanou,
  Ichoku, Tselioudis, and Bais]{Zerefos2009}
Zerefos, C.S.; Eleftheratos, K.; Meleti, C.; Kazadzis, S.; Romanou, A.; Ichoku,
  C.; Tselioudis, G.; Bais, A.
\newblock Solar dimming and brightening over Thessaloniki, Greece, and Beijing,
  China.
\newblock {\em Tellus B Chem. Phys. Meteorol.} {\bf 2009}, {\em
  61},~657.
\newblock {\url{https://doi.org/10.1111/j.1600-0889.2009.00425.x}}.

\bibitem[Xia et~al.(2007)Xia, Chen, Li, Wang, and Wang]{Xia2007}
Xia, X.; Chen, H.; Li, Z.; Wang, P.; Wang, J.
\newblock Significant reduction of surface solar irradiance induced by aerosols
  in a suburban region in northeastern China.
\newblock {\em J. Geophys. Res. Atmos.} {\bf 2007}, {\em
  112},~1--9.
\newblock {\url{https://doi.org/10/cdtntw}}.

\bibitem[Fountoulakis et~al.(2016)Fountoulakis, Redondas, Bais,
  Rodriguez-Franco, Fragkos, and Cede]{Fountoulakis2016}
Fountoulakis, I.; Redondas, A.; Bais, A.F.; Rodriguez-Franco, J.J.; Fragkos,
  K.; Cede, A.
\newblock Dead time effect on the Brewer measurements: Correction and estimated
  uncertainties.
\newblock {\em Atmos. Meas. Tech.} {\bf 2016}, {\em
  9},~1799--1816.
\newblock {\url{https://doi.org/10/gcc32t}}.

\bibitem[Siomos et~al.(2018)Siomos, Balis, Voudouri, Giannakaki, Filioglou,
  Amiridis, Papayannis, and Fragkos]{Siomos2018}
Siomos, N.; Balis, D.S.; Voudouri, K.A.; Giannakaki, E.; Filioglou, M.;
  Amiridis, V.; Papayannis, A.; Fragkos, K.
\newblock Are {EARLINET} and {AERONET} climatologies consistent? The case of
  Thessaloniki, Greece.
\newblock {\em Atmos. Chem. Phys.} {\bf 2018}, {\em
  18},~11885--11903.
\newblock {\url{https://doi.org/10.5194/acp-18-11885-2018}}.

\bibitem[Gkikas et~al.(2013)Gkikas, Hatzianastassiou, Mihalopoulos, Katsoulis,
  Kazadzis, Pey, Querol, and Torres]{Gkikas2013}
Gkikas, A.; Hatzianastassiou, N.; Mihalopoulos, N.; Katsoulis, V.; Kazadzis,
  S.; Pey, J.; Querol, X.; Torres, O.
\newblock The regime of intense desert dust episodes in the Mediterranean based
  on contemporary satellite observations and ground measurements.
\newblock {\em Atmos. Chem. Phys.} {\bf 2013}, {\em
  13},~12135--12154.
\newblock {\url{https://doi.org/10.5194/acp-13-12135-2013}}.

\bibitem[Lozano et~al.(2021)Lozano, Sánchez-Hernández, Guerrero-Rascado,
  Alados, and Foyo-Moreno]{Lozano2021}
Lozano, I.L.; Sánchez-Hernández, G.; Guerrero-Rascado, J.L.; Alados, I.;
  Foyo-Moreno, I.
\newblock Aerosol radiative effects in photosynthetically active radiation and
  total irradiance at a Mediterranean site from an 11-year database.
\newblock {\em Atmos. Res.} {\bf 2021}, {\em 255},~105538.
\newblock {\url{https://doi.org/10.1016/j.atmosres.2021.105538}}.

\bibitem[Bais et~al.(2013)Bais, Drosoglou, Meleti, Tourpali, and
  Kouremeti]{Bais2013}
Bais, A.F.; Drosoglou, T.; Meleti, C.; Tourpali, K.; Kouremeti, N.
\newblock Changes in surface shortwave solar irradiance from 1993 to 2011 at
  Thessaloniki (Greece).
\newblock {\em Int. J. Climatol.} {\bf 2013}, {\em
  33},~2871--2876.
\newblock {\url{https://doi.org/10/f5dzz5}}.

\bibitem[Reno and Hansen(2016)]{Reno2016}
Reno, M.J.; Hansen, C.W.
\newblock Identification of periods of clear sky irradiance in time series of
  GHI measurements.
\newblock {\em Renew. Energy} {\bf 2016}, {\em 90},~520--531.
\newblock {\url{https://doi.org/10/gq3sbg}}.

\bibitem[Reno et~al.(2012)Reno, Hansen, and Stein]{Reno2012a}
\colorbox{green}{Reno,} %MDPI: Refs. 18 and 26 are duplicated. Please remove duplicated references and rearrange all the references to appear in numerical order. Please ensure that there are no duplicated references.
 M.J.; Hansen, C.W.; Stein, J.S.
\newblock {\emph{Global Horizontal Irradiance Clear Sky Models: Implementation and
  Analysis}};
\newblock \hl{Technical report;}  %MDPI: Please confirm if it can be removed. Please add the name of the publisher and their location.
 2012.

\bibitem[Long and Shi(2006)]{Long2006}
Long, C.N.; Shi, Y.
\newblock \emph{The QCRad Value Added Product: Surface Radiation Measurement Quality
  Control Testing, Including Climatology Configurable Limits};
\newblock \hl{Technical Report DOE/SC-ARM/TR-074;} %MDPI: Please confirm if it can be removed. Please add the name of the publisher and their location.
  2006.

\bibitem[Long and Shi(2008)]{Long2008a}
Long, C.N.; Shi, Y.
\newblock An Automated Quality Assessment and Control Algorithm for Surface
  Radiation Measurements.
\newblock {\em  Open Atmos. Sci. J.} {\bf 2008}, \emph{2}, 23--37.

\bibitem[Coddington et~al.(2005)Coddington, Lean, Lindholm, Pilewskie, Snow,
  and {NOAA CDR Program}]{Coddington2005}
Coddington, O.; Lean, J.L.; Lindholm, D.; Pilewskie, P.; Snow, M.; {NOAA CDR
  Program}.
\newblock \emph{{NOAA} Climate Data Record ({CDR}) of Total Solar Irradiance ({TSI}),
  {NRLTSI} Version 2}; \hl{{D}aily;} %MDPI: Please confirm if it can be removed. Please add the name of the publisher and their location.
  2005.
\newblock {\url{https://doi.org/10.7289/V55B00C1}}.

\bibitem[Haurwitz(1945)]{Haurwitz1945}
Haurwitz, B.
\newblock Insolation in {Relation} to {Cloudiness} and {Cloud} {Density}.
\newblock {\em J. Meteorol.} {\bf 1945}, {\em 2},~154--166.

\bibitem[Long and Ackerman(2000)]{Long2000}
Long, C.N.; Ackerman, T.P.
\newblock Identification of clear skies from broadband pyranometer measurements
  and calculation of downwelling shortwave cloud effects.
\newblock {\em J. Geophys. Res. Atmos.} {\bf 2000}, {\em
  105},~15609--15626.
\newblock {\url{https://doi.org/10.1029/2000jd900077}}.

\bibitem[Brent(1973)]{Brent1973}
Brent, R.P.
\newblock \emph{Algorithms for Minimization without Derivatives};
\newblock {\em PrenticeHall: Englewood Cliffs, NJ, USA,} {1973}.

\bibitem[{R Core Team}(2023)]{RCT2023}
{R Core Team}.
\newblock {\em R: A Language and Environment for Statistical Computing};
\newblock R Foundation for Statistical Computing: Vienna, Austria,  2023.

\bibitem[Reno et~al.(2012)Reno, Hansen, and Stein]{Reno2012}
\colorbox{red}{Reno,} 
 M.J.; Hansen, C.W.; Stein, J.S.
\newblock \emph{Global Horizontal Irradiance Clear Sky Models: Implementation and
  Analysis};
\newblock \hl{Technical Report SAND2012-2389, 1039404;} %MDPI: Please confirm if it can be removed. Please add the name of the publisher and their location.
  2012.
\newblock {\url{https://doi.org/10/gq5npv}}.

\bibitem[Gardner et~al.(1980)Gardner, Harvey, and Phillips]{Gardner1980}
Gardner, G.; Harvey, A.C.; Phillips, G.D.A.
\newblock Algorithm {AS} 154: An Algorithm for Exact Maximum Likelihood
  Estimation of Autoregressive-Moving Average Models by Means of Kalman
  Filtering.
\newblock {\em Appl. Stat.} {\bf 1980}, {\em 29},~311.
\newblock {\url{https://doi.org/10.2307/2346910}}.

\bibitem[Jones(1980)]{Jones1980}
Jones, R.H.
\newblock Maximum Likelihood Fitting of {ARMA} Models to Time Series With
  Missing Observations.
\newblock {\em Technometrics} {\bf 1980}, {\em 22},~389--395.
\newblock {\url{https://doi.org/10.1080/00401706.1980.10486171}}.

\bibitem[Yu et~al.(2022)Yu, Zhang, Wang, Qin, Jiang, and Li]{Yu2022}
Yu, L.; Zhang, M.; Wang, L.; Qin, W.; Jiang, D.; Li, J.
\newblock Variability of surface solar radiation under clear skies over
  Qinghai-Tibet Plateau: Role of aerosols and water vapor.
\newblock {\em Atmos. Environ.} {\bf 2022}, {\em 287},~119286.
\newblock {\url{https://doi.org/10.1016/j.atmosenv.2022.119286}}.

\bibitem[Lozano et~al.(2023)Lozano, Alados, and Foyo-Moreno]{Lozano2023}
Lozano, I.L.; Alados, I.; Foyo-Moreno, I.
\newblock Analysis of the solar radiation/atmosphere interaction at a
  Mediterranean site: The role of clouds.
\newblock {\em Atmos. Res.} {\bf 2023}, {\em 296},~107072.
\newblock {\url{https://doi.org/10.1016/j.atmosres.2023.107072}}.

\bibitem[Ohmura(2009)]{Ohmura2009}
Ohmura, A.
\newblock Observed decadal variations in surface solar radiation and their
  causes.
\newblock {\em J. Geophys. Res. Atmos.} {\bf 2009}, {\em 114}.
\newblock {\url{https://doi.org/10.1029/2008JD011290}}.

\bibitem[Regier et~al.(2019)Regier, Brice{\~{n}}o, and Boyer]{Regier2019}
Regier, P.; Brice{\~{n}}o, H.; Boyer, J.N.
\newblock Analyzing and comparing complex environmental time series using a
  cumulative sums approach.
\newblock {\em {MethodsX}} {\bf 2019}, {\em 6},~779--787.
\newblock {\url{https://doi.org/10.1016/j.mex.2019.03.014}}.

\bibitem[Siomos et~al.(2020)Siomos, Fountoulakis, Natsis, Drosoglou, and
  Bais]{Siomos2020}
Siomos, N.; Fountoulakis, I.; Natsis, A.; Drosoglou, T.; Bais, A.
\newblock Automated Aerosol Classification from Spectral {UV} Measurements
  Using Machine Learning Clustering.
\newblock {\em Remote Sens.} {\bf 2020}, {\em 12},~965.
\newblock {\url{https://doi.org/10.3390/rs12060965}}.

\bibitem[Wang et~al.(2021)Wang, Zhang, Sanchez‐Lorenzo, Tanaka, Trentmann,
  Yuan, and Wild]{Wang2021}
Wang, Y.; Zhang, J.; Sanchez‐Lorenzo, A.; Tanaka, K.; Trentmann, J.; Yuan,
  W.; Wild, M.
\newblock Hourly Surface Observations Suggest Stronger Solar Dimming and
  Brightening at Sunrise and Sunset Over China.
\newblock {\em Geophys. Res. Lett.} {\bf 2021}, {\em 48}, e2020GL091422.
\newblock {\url{https://doi.org/10.1029/2020GL091422}}.

\end{thebibliography}

% If authors have biography, please use the format below
%\section*{Short Biography of Authors}
%\bio
%{\raisebox{-0.35cm}{\includegraphics[width=3.5cm,height=5.3cm,clip,keepaspectratio]{Definitions/author1.pdf}}}
%{\textbf{Firstname Lastname} Biography of first author}
%
%\bio
%{\raisebox{-0.35cm}{\includegraphics[width=3.5cm,height=5.3cm,clip,keepaspectratio]{Definitions/author2.jpg}}}
%{\textbf{Firstname Lastname} Biography of second author}

%%%%%%%%%%%%%%%%%%%%%%%%%%%%%%%%%%%%%%%%%%
%% for journal Sci
%\reviewreports{\\
%Reviewer 1 comments and authors’ response\\
%Reviewer 2 comments and authors’ response\\
%Reviewer 3 comments and authors’ response
%}
%%%%%%%%%%%%%%%%%%%%%%%%%%%%%%%%%%%%%%%%%%
\PublishersNote{}
\end{adjustwidth}


\end{document}
