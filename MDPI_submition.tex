%  LaTeX support: latex@mdpi.com
%  For support, please attach all files needed for compiling as well as the log file, and specify your operating system, LaTeX version, and LaTeX editor.

%=================================================================
% pandoc conditionals added to preserve backwards compatibility with previous versions of rticles

\documentclass[applsci,article,submit,moreauthors,pdftex]{Definitions/mdpi}


%% Some pieces required from the pandoc template
\setlist[itemize]{leftmargin=*,labelsep=5.8mm}
\setlist[enumerate]{leftmargin=*,labelsep=4.9mm}


%--------------------
% Class Options:
%--------------------

%---------
% article
%---------
% The default type of manuscript is "article", but can be replaced by:
% abstract, addendum, article, book, bookreview, briefreport, casereport, comment, commentary, communication, conferenceproceedings, correction, conferencereport, entry, expressionofconcern, extendedabstract, datadescriptor, editorial, essay, erratum, hypothesis, interestingimage, obituary, opinion, projectreport, reply, retraction, review, perspective, protocol, shortnote, studyprotocol, systematicreview, supfile, technicalnote, viewpoint, guidelines, registeredreport, tutorial
% supfile = supplementary materials

%----------
% submit
%----------
% The class option "submit" will be changed to "accept" by the Editorial Office when the paper is accepted. This will only make changes to the frontpage (e.g., the logo of the journal will get visible), the headings, and the copyright information. Also, line numbering will be removed. Journal info and pagination for accepted papers will also be assigned by the Editorial Office.

%------------------
% moreauthors
%------------------
% If there is only one author the class option oneauthor should be used. Otherwise use the class option moreauthors.

%---------
% pdftex
%---------
% The option pdftex is for use with pdfLaTeX. Remove "pdftex" for (1) compiling with LaTeX & dvi2pdf (if eps figures are used) or for (2) compiling with XeLaTeX.

%=================================================================
% MDPI internal commands - do not modify
\firstpage{1}
\makeatletter
\setcounter{page}{\@firstpage}
\makeatother
\pubvolume{1}
\issuenum{1}
\articlenumber{0}
\pubyear{2023}
\copyrightyear{2023}
%\externaleditor{Academic Editor: Firstname Lastname}
\datereceived{ }
\daterevised{ } % Comment out if no revised date
\dateaccepted{ }
\datepublished{ }
%\datecorrected{} % For corrected papers: "Corrected: XXX" date in the original paper.
%\dateretracted{} % For corrected papers: "Retracted: XXX" date in the original paper.
\hreflink{https://doi.org/} % If needed use \linebreak
%\doinum{}
%\pdfoutput=1 % Uncommented for upload to arXiv.org

%=================================================================
% Add packages and commands here. The following packages are loaded in our class file: fontenc, inputenc, calc, indentfirst, fancyhdr, graphicx, epstopdf, lastpage, ifthen, float, amsmath, amssymb, lineno, setspace, enumitem, mathpazo, booktabs, titlesec, etoolbox, tabto, xcolor, colortbl, soul, multirow, microtype, tikz, totcount, changepage, attrib, upgreek, array, tabularx, pbox, ragged2e, tocloft, marginnote, marginfix, enotez, amsthm, natbib, hyperref, cleveref, scrextend, url, geometry, newfloat, caption, draftwatermark, seqsplit
% cleveref: load \crefname definitions after \begin{document}

%=================================================================
% Please use the following mathematics environments: Theorem, Lemma, Corollary, Proposition, Characterization, Property, Problem, Example, ExamplesandDefinitions, Hypothesis, Remark, Definition, Notation, Assumption
%% For proofs, please use the proof environment (the amsthm package is loaded by the MDPI class).

%=================================================================
% Full title of the paper (Capitalized)
\Title{Full title of the paper (Capitalized)}

% MDPI internal command: Title for citation in the left column
\TitleCitation{Full title of the paper (Capitalized)}

% Author Orchid ID: enter ID or remove command
%\newcommand{\orcidauthorA}{0000-0000-0000-000X} % Add \orcidA{} behind the author's name
%\newcommand{\orcidauthorB}{0000-0000-0000-000X} % Add \orcidB{} behind the author's name


% Authors, for the paper (add full first names)
\Author{Athanasios
Natsis$^{1,*}$\href{https://orcid.org/0000-0002-5199-4119}
{\orcidicon}, Alkiviadis Bais$^{2,}$}


%\longauthorlist{yes}


% MDPI internal command: Authors, for metadata in PDF
\AuthorNames{Athanasios Natsis, Alkiviadis Bais}

% MDPI internal command: Authors, for citation in the left column
%\AuthorCitation{Lastname, F.; Lastname, F.; Lastname, F.}
% If this is a Chicago style journal: Lastname, Firstname, Firstname Lastname, and Firstname Lastname.
\AuthorCitation{Natsis, A.; Bais, A.}

% Affiliations / Addresses (Add [1] after \address if there is only one affiliation.)
\address{%
$^{1}$ \quad Aristotle University of Thessaloniki - Laboratory of
Atmospheric Physics, Campus Box 149, 54124 Thessaloniki,
Greece; \href{mailto:natsisphysicist@gmail.com}{\nolinkurl{natsisphysicist@gmail.com}}\\
$^{2}$ \quad Your department Street, City, Country; \\
}

% Contact information of the corresponding author
\corres{Correspondence: \href{mailto:natsisphysicist@gmail.com}{\nolinkurl{natsisphysicist@gmail.com}};}

% Current address and/or shared authorship








% The commands \thirdnote{} till \eighthnote{} are available for further notes

% Simple summary
\simplesumm{A Simple summary goes here.}

%\conference{} % An extended version of a conference paper

% Abstract (Do not insert blank lines, i.e. \\)
\abstract{The shortwave downward solar irradiance (SDR) is an important
factor that drives climate processes, production and can affect all
living organisms. While monitoring the long term variability, there are
observations of upward and downward SDR trends on different locations
around the world for different time periods. Periods of positive tredns
are refered as brightening periods and with negative tredns as dimming
periods. We studied 29 years of CHP1 data from Thessaloniki, Greece,
under three sky conditions (clear sky, cloudy sky and all sky
conditions), applying a cloud sky identification algorithm. We found a
positive trend for all-sky and clear-sky conditions, and also,
investigated the consistency of those trends, the effect of the solar
zenith angle, and the variation of the trends for the seasons of the
year. We indentified that there are some anomalies in the long term SDR
trends, for all sky conditions.}


% Keywords
\keyword{GHI; SDR; Solar Brigthening/Dimming (list three to ten
pertinent keywords specific to the article, yet reasonably common within
the subject discipline.).}

% The fields PACS, MSC, and JEL may be left empty or commented out if not applicable
%\PACS{J0101}
%\MSC{}
%\JEL{}

%%%%%%%%%%%%%%%%%%%%%%%%%%%%%%%%%%%%%%%%%%
% Only for the journal Diversity
%\LSID{\url{http://}}

%%%%%%%%%%%%%%%%%%%%%%%%%%%%%%%%%%%%%%%%%%
% Only for the journal Applied Sciences

%%%%%%%%%%%%%%%%%%%%%%%%%%%%%%%%%%%%%%%%%%

%%%%%%%%%%%%%%%%%%%%%%%%%%%%%%%%%%%%%%%%%%
% Only for the journal Data



%%%%%%%%%%%%%%%%%%%%%%%%%%%%%%%%%%%%%%%%%%
% Only for the journal Toxins


%%%%%%%%%%%%%%%%%%%%%%%%%%%%%%%%%%%%%%%%%%
% Only for the journal Encyclopedia


%%%%%%%%%%%%%%%%%%%%%%%%%%%%%%%%%%%%%%%%%%
% Only for the journal Advances in Respiratory Medicine
%\addhighlights{yes}
%\renewcommand{\addhighlights}{%

%\noindent This is an obligatory section in “Advances in Respiratory Medicine”, whose goal is to increase the discoverability and readability of the article via search engines and other scholars. Highlights should not be a copy of the abstract, but a simple text allowing the reader to quickly and simplified find out what the article is about and what can be cited from it. Each of these parts should be devoted up to 2~bullet points.\vspace{3pt}\\
%\textbf{What are the main findings?}
% \begin{itemize}[labelsep=2.5mm,topsep=-3pt]
% \item First bullet.
% \item Second bullet.
% \end{itemize}\vspace{3pt}
%\textbf{What is the implication of the main finding?}
% \begin{itemize}[labelsep=2.5mm,topsep=-3pt]
% \item First bullet.
% \item Second bullet.
% \end{itemize}
%}


%%%%%%%%%%%%%%%%%%%%%%%%%%%%%%%%%%%%%%%%%%


% tightlist command for lists without linebreak
\providecommand{\tightlist}{%
  \setlength{\itemsep}{0pt}\setlength{\parskip}{0pt}}



\usepackage{longtable}
\usepackage{booktabs}
\usepackage{array}
\usepackage{multirow}
\usepackage{wrapfig}
\usepackage{float}
\usepackage{colortbl}
\usepackage{pdflscape}
\usepackage{tabu}
\usepackage{threeparttable}
\usepackage{threeparttablex}
\usepackage[normalem]{ulem}
\usepackage{makecell}
\usepackage{xcolor}

\begin{document}



%%%%%%%%%%%%%%%%%%%%%%%%%%%%%%%%%%%%%%%%%%

\hypertarget{introduction.}{%
\section{Introduction.}\label{introduction.}}

The shortwave downward solar irradiance (SDR) at Earth's surface play a
significant role, on its climate. Changes of the SDR can be related to
changes on Earth's energy budget, the mechanisms of climate change, and
water and carbon cycle \citep{Wild2009}. Can also affect, solar and
agricultural production, and all living organisms. Studies of SDR
variability, have identified some distinct SDR trends on different
regions of the world on different time periods. The term `brightening'
is generally used to describe periods of positive SDR trend, and
`dimming' for negative trend. There are many cases on the long term
records of irradiance, showing a systematic change of SDR's trend slope,
occurring roughly at the last decades of the 20th century. On multiple
station in China, a dimming period was reported until about 2000,
followed by a brightening period \citep{Yang2021}. A similar pattern was
identified, with the breaking point around 1980, for stations in Central
Europe \citep{Wild2021} and Brazil \citep{Yamasoe2021}. Also, on global
scale, an AI aided continental level spatial analysis, with data from
multiple station, reach similar conclusions for the above regions and
for the global trend \citep{Yuan2021}.

There is a consensus, among researchers, that the major factors of SDR
attenuation is the interaction of Sun radiation with atmospheric
aerosols and clouds. Those interactions, among other factors, have been
analysed with models \citep{Li2016, Samset2018}, showing the existence
of feedback mechanisms between the two. Similar finds, have been showed
in observational data \citep[ and references
therein]{Schwarz2020, Ohvril2009, Zerefos2009, Xia2007}.

Due to the variability of the phenomenon, and its contributing factors,
there is a constant need to investigate and monitor SDR, in different
sites, to estimate its magnitude, and its relation to the local
conditions. In this study, we examine the trends of SDR, with
ground-based measurements at Thessaloniki, Greece for the period 1993 to
2023, as derived from a CM-21 pyranometer. We reevaluated and extended
the dataset used by \citet{Bais2013}, applying a different algorithm for
the identification of clear-/cloud-sky instances
\citep{Reno2016, Reno2012a}, and we derive the radiation trends for the
period 1993 to 2023, under different sky conditions (all-sky, clear-sky
and cloud-sky).

\hypertarget{observational-data-and-methodology.}{%
\section{Observational data and
methodology.}\label{observational-data-and-methodology.}}

The SDR data were measured with a Kipp \& Zonen CM-21 pyranometer
operating continuously at the Laboratory of Atmospheric Physics of the
Aristotle University of Thessaloniki (\(40^\circ\,38'\,\)N,
\(22^\circ\,57'\,\)E, \(80\,\)m~a.s.l.) in the period from 1993-04-13 to
2023-04-13. The monitoring site is located near the city centre, and we
expect to be affected by the urban environment. During the study period,
the pyranometer has been independently calibrated three times at the
Meteorologisches Observatorium Lindenberg, DWD, when it was verified the
stability of the instrument to within better than \(0.7\%\) relative to
the initial calibration by the manufacturer. Along with SDR, the direct
beam radiation (DNI) was also measured by a collocated Kipp \& Zonen
CHP-1 pyrheliometer, for the period 2016-04-01 to 2023-04-13. Although,
we have performed a similar analysis to the DNI data the results are not
presented here, as they lack the appropriate statistical significance,
due to the sorter duration of the data. However, the DNI data were used
as auxiliary data, in the clear sky identification algorithm (CSid),
which is discussed later, for the selection of the appropriate
thresholds. It is noted that despite the capability of the CSid
algorithm to use the DNI as a characterization parameter, we haven't
utilized it here, to avoid any selection bias, due to unequal length of
the two datasets. There are four distinct steps in the creation of the
dataset analysed here: a)~the acquisition of radiation measurements from
the sensors, b)~the data quality check, c)~the identification of ``clear
sky'' conditions from the radiometric data, and d)~the aggregation of
data and trend analysis.

For the acquisition of radiometric data, the signal of the pyranometer
is sampled with a rate of \(1\,\text{Hz}\). The mean and the standard
deviation of these samples are recorded every minute. The measurements
are corrected for the zero offset (``dark signal'' in volts). The ``dark
signal'' is calculated by averaging all measurements recorded for a
period of \(3\,\text{h}\), before (morning) or after (evening) the Sun
reaches an elevation angle of \(-10^\circ\). The signal is converted to
irradiance using a ramped value of the instrument's sensitivity between
calibrations.

A manual screening was performed, to remove inconsistent and erroneous
recordings that can occur stochastically or systematically, during the
continuous operation of the instruments. The manual screening is aided
by a radiation data quality assurance procedure, adjusted for the site,
which is based on the methods of Long and
Shi~\citetext{\citeyear{Long2008a}; \citeyear{Long2006}}. Thus,
problematic recordings have been excluded from further processing.
Although it is impossible to detect all false data, the large number of
available data, and the aggregation scheme we used, ensures the good
quality of the radiometric measurements used in this study.

In order to be able to estimate the effect of the sky condition on the
long term variability of SDR, we created three datasets, by
characterizing each one-minute measurement with a corresponding sky
condition (i.e., all-sky, clear-sky and cloudy-sky). To identify the
clear-sky conditions we used a method proposed by \citet{Long2000} and
by \citet{Reno2016}, which was adapted and configured for the site, as
the authors suggest.

We have to note, that the definition of what constitutes as clear or
cloudy sky, has some subjectivity, in any method of characterization. As
a result, the details of the definition are site specific, it relies on
a combination of thresholds and comparisons with ideal actinometric
models and statistical analysis on different signal metrics. The CSid
algorithm was calibrated with the main focus, to identify the presence
of clouds on the sky dome. Although the fine-tuning of the procedure,
few marginal cases exist, that have been identified manually as false
positive or false negative but cannot affect the final results of the
study.

For completeness, we will provide below a brief overview of the clear
sky identification algorithm (CSid), along with the site specific
thresholds. To calculate the reference clear sky
\(\text{SDR}_\text{CSref}\) we used the \(\text{SDR}_\text{Haurwitz}\)
derived by the radiation model of \citet{Haurwitz1945}, adjusted for our
site with a factor \(a\) (Eq.~\ref{eq:ahau}), resulted by an iterative
optimization process, as described by \citet{Long2000} and
\citet{Reno2016}. The target of the optimization was the minimization of
a function \(f(a)\) (Eq.~\ref{eq:minf}) and was accomplished with the
algorithmic function ``optimise'', which is an implementation based on
the work of \citet{Brent1973}, from the library ``stats'' of the R
programming language \citep{RCT2023}. The optimization and the selection
of the clear sky reference model, was performed on SDR observations for
the period 2016 - 2021. During the optimization, eight simple clear sky
radiation models were tested (Daneshyar-Paltridge-Proctor,
Kasten-Czeplak, Haurwitz, Berger-Duffie, Adnot-Bourges-Campana-Gicquel,
Robledo-Soler, Kasten and Ineichen-Perez), with a wide range of factors.
These models are described in more details by \citet{Reno2012} and
evaluated by \citet{Reno2016}. We found, that Haurwitz's model, adjusted
with the factor \(a = 0.965\) yields one of the lowest root mean squared
errors (RMSE), while the procedure, manages to characterize the majority
of the data. Thus, our clear sky reference is derived by the
Eq.~\ref{eq:ahau}.

\begin{equation}
f(a) = \frac{1}{n}\sum_{i=1}^{n} ( \text{SDR}_{\text{CSid},i} - a \times \text{SDR}_{\text{testCSref},i} )^2 \label{eq:minf}
\end{equation} where: \(n\) is the total number of daylight records,
\(\text{SDR}_{\text{CSid},i}\) are the records identified as clear sky
by CSid, \(a\) is a hypothetical adjustment factor, and
\(\text{SDR}_{\text{testCSref},i}\) is any of the tested clear sky
radiation models.

\begin{equation}
\text{SDR}_\text{CSref} = a \times \text{SDR}_\text{Haurwitz} = 0.965 \times 1098 \times \cos(\theta) \times \exp \left( \frac{ - 0.057}{\cos(\theta)} \right) \label{eq:ahau}
\end{equation} where: \(\text{SDR}_\text{CSref}\) is the reference clear
sky SDR, in \(\text{w}\,\text{m}^{-2}\) and \(\theta\) is the solar
zenith angle (SZA).

The criteria that were used to identify whether a measurement was taken
under clear-sky conditions are presented below. A data point is flagged
as ``clear-sky'' if all criteria are satisfied, otherwise it is
considered to be ``cloud-sky''. Each criterion was applied for a running
window of \(11\) consecutive one-minute measurements, and the
characterization is assigned to the central value of the window. Each
parameter, was calculated both from the observations and the reference
clear sky model, for each comparison. The allowable range of variation
is defined by the model-derived value of the parameter multiplied by a
factor plus an offset. The factors and the offsets were determined
empirically, by manual inspecting each filters performance on selected
days, and adjusting them accordingly, during an iterative process.

\begin{enumerate}
\def\labelenumi{\alph{enumi})}
\tightlist
\item
  Mean of the measured \(\overline{\text{SDR}}_i\) (Eq.
  \ref{eq:MeanVIP}). \begin{equation}
  0.91 \times \overline{\text{SDR}}_{\text{CSref},i} - 20
  < \overline{\text{SDR}}_i <
  1.095 \times \overline{\text{SDR}}_{\text{CSref},i} + 30
  \label{eq:MeanVIP}
  \end{equation}
\end{enumerate}

\begin{enumerate}
\def\labelenumi{\alph{enumi})}
\setcounter{enumi}{1}
\tightlist
\item
  Maximum measured value \(M_{\text{}}\) (Eq.~\ref{eq:MaxVIP}).
  \begin{equation}
  1 \times M_{\text{CSref},i} - 75
  < M_{\text{}i} <
  1 \times M_{\text{CSref},i} + 75
  \label{eq:MaxVIP}
  \end{equation}
\end{enumerate}

\begin{enumerate}
\def\labelenumi{\alph{enumi})}
\setcounter{enumi}{2}
\tightlist
\item
  Length \(L_i\) of the sequential line segments, connecting the points
  of the \(11\) SDR values (Eq. \ref{eq:VILeq}). \begin{equation}
  L_i = \sum_{i=1}^{n-1}\sqrt{\left ( \text{SDR}_{i+1} - \text{SDR}_{i}\right )^2 + \left ( t_{i+1} - t_i \right )^2}
  \label{eq:VILeq}
  \end{equation} \begin{equation}
  1 \times L_{\text{CSref},i} - 5 < L_i < 1.3 \times L_{\text{CSref},i} + 13
  \label{eq:VILcr}
  \end{equation} where: \(t_i\) is the time each SDR measurement has
  been measured
\end{enumerate}

\begin{enumerate}
\def\labelenumi{\alph{enumi})}
\setcounter{enumi}{3}
\tightlist
\item
  Standard deviation \(\sigma_i\) of the slope (\(s_i\)) between the
  \(11\) sequential points, normalized by the mean
  \(\overline{\text{SDR}}_i\) (Eq.~\ref{eq:VCT1}). \begin{gather}
    \sigma_i = \frac {\sqrt{\frac{1}{n-1} \sum_{i=1}^{n-1} \left( s_i - \bar{s} \right)^2}} {\overline{\text{SDR}}_i} \label{eq:VCT1} \\
    s_i = \frac{\text{SDR}_{i+1} - \text{SDR}_{i}}{t_{i+1} - t_i},\;\;   \bar{s} = \frac{1}{n-1} \sum_{i=1}^{n-1} s_i,\;\;\forall i \in \left \{ 1, 2, \ldots, n-1 \right \}\;\;
  \end{gather} For this criterion, \(\sigma_i\) should be below a
  certain threshold (Eq.~\ref{eq:VCTcr}): \begin{equation}
    \sigma_i < \ensuremath{1.1\times 10^{-4}} \label{eq:VCTcr}
  \end{equation}
\end{enumerate}

\begin{enumerate}
\def\labelenumi{\alph{enumi})}
\setcounter{enumi}{4}
\tightlist
\item
  Maximum difference \(X_i\) between the change in measured irradiance
  and the change in clear sky irradiance over each measurement interval.
  \begin{gather}
    X_i = \max{\left \{ \left | x_i - x_{\text{CSref},i} \right | \right \}} \label{eq:VSM3} \\
    x_i = \text{SDR}_{i+1} - \text{SDR}_{i} \forall i \in \left \{ 1, 2, \ldots, n-1 \right \} \label{eq:VSM1} \\
    x_{\text{CSref},i} = \text{SDR}_{\text{CSref},i+1} - \text{SDR}_{\text{CSref},i} \forall i \in \left \{ 1, 2, \ldots, n-1 \right \} \label{eq:VSM2}
  \end{gather} For this criterion, \(X_i\) should be below a certain
  threshold (Eq.~\ref{eq:VSMcr}): \begin{equation}
    X_i < 7.5 \label{eq:VSMcr}
  \end{equation}
\end{enumerate}

Due to a significant measurement uncertainty near the horizon, we have
to exclude all measurements with SZA greater than \(85^\circ\).
Moreover, due to some obstructions around the site (hills and
buildings), we excluded data with Azimuth angle between \(35^\circ\) and
\(120^\circ\) with SZA greater than \(80^\circ\). On the latter
instances, Sun is systematically, not visible by the instrument's
location. To make the measurements comparable throughout the dataset, we
adjusted all one-minute radiometric values to the mean Sun - Earth
distance. Subsequently, we made all measurements relative to the Total
Solar Irradiance (TSI) at \(1\,\text{au}\), in order to compensate for
the Sun's intensity variability, using a time series of satellite TSI
observations. The TSI data we use are part of the ``NOAA Climate Data
Record of Total Solar Irradiance'' dataset \citep{Coddington2005}. Where
the initial daily values, were interpolated to match with the time step
of our measurements. The final dataset contains \(6589967\) one-minute
measurements, of which, \(84.2\%\) were identified as under clear-sky
conditions and subsequently \(15.8\%\) as under cloud-sky conditions.

In order to investigate the SDR trends, we implemented an appropriate
aggregation scheme to the 1-minute data to derive a series in coarser
timescale. To preserve the representativeness of the data we used the
following criteria: a) for the daily mean values we accept days with
more than 50\% of the daytime measurements, present and valid, b)
monthly values were computed from daily means only when at least 20 days
were available. To create the daily and monthly climatological means, we
averaged the data based on the day of year and calendar month,
respectively. For the seasonal means we averaged the mean daily values
in each season (Winter: December - February, Spring: March - May, etc.).
Finally, each data set was deseasonalized by subtracting the
corresponding climatological annual cycle (daily or monthly) from the
actual data. To estimate SZA contribution to the SDR trends, the
one-minute data were aggregated in \(1^\circ\) SZA bins, separately for
the morning and afternoon hours, and then were deseasonalized as
mentioned above.

\hypertarget{results}{%
\section{Results}\label{results}}

\hypertarget{long-term-sdr-trends}{%
\subsection{Long-term SDR trends}\label{long-term-sdr-trends}}

\hypertarget{version}{%
\section{Version}\label{version}}

This Rmd-skeleton uses the mdpi Latex template published 2023-03-25.
However, the official template gets more frequently updated than the
\textbf{rticles} package. Therefore, please make sure prior to paper
submission, that you're using the most recent .cls, .tex and .bst files
(available \href{http://www.mdpi.com/authors/latex}{here}).

\hypertarget{article-header-information}{%
\section{Article Header Information}\label{article-header-information}}

\hypertarget{journal-specific-yaml-variables}{%
\subsection{Journal Specific YAML
variables}\label{journal-specific-yaml-variables}}

\hypertarget{introduction}{%
\section{Introduction}\label{introduction}}

The introduction should briefly place the study in a broad context and
highlight why it is important. It should define the purpose of the work
and its significance. The current state of the research field should be
reviewed carefully and key publications cited. Please highlight
controversial and diverging hypotheses when necessary. Finally, briefly
mention the main aim of the work and highlight the principal
conclusions. As far as possible, please keep the introduction
comprehensible to scientists outside your particular field of research.
Citing a journal paper
\citep{bertrand-krajewski_distribution_1998, leutnant_stormwater_2016}.
And now citing a book reference \citet{gujer_systems_2008}. Some MDPI
journals use Chicago and others use APA, this template should choose the
correct citation format for you once you specify the journal in the YAML
header.

To use endnotes, change \texttt{endnotes:\ true} in the YAML header,
then use \texttt{\textbackslash{}endnote\{This\ is\ an\ endnote.\}}.

\hypertarget{materials-and-methods}{%
\section{Materials and Methods}\label{materials-and-methods}}

Materials and Methods should be described with sufficient details to
allow others to replicate and build on published results. Please note
that publication of your manuscript implicates that you must make all
materials, data, computer code, and protocols associated with the
publication available to readers. Please disclose at the submission
stage any restrictions on the availability of materials or information.
New methods and protocols should be described in detail while
well-established methods can be briefly described and appropriately
cited.

Research manuscripts reporting large datasets that are deposited in a
publicly available database should specify where the data have been
deposited and provide the relevant accession numbers. If the accession
numbers have not yet been obtained at the time of submission, please
state that they will be provided during review. They must be provided
prior to publication.

Interventionary studies involving animals or humans, and other studies
require ethical approval must list the authority that provided approval
and the corresponding ethical approval code.

\hypertarget{results-1}{%
\section{Results}\label{results-1}}

This section may be divided by subheadings. It should provide a concise
and precise description of the experimental results, their
interpretation as well as the experimental conclusions that can be
drawn.

\hypertarget{subsection-heading-here}{%
\subsection{Subsection Heading Here}\label{subsection-heading-here}}

Subsection text here.

\hypertarget{subsubsection-heading-here}{%
\subsubsection{Subsubsection Heading
Here}\label{subsubsection-heading-here}}

Bulleted lists look like this:

\begin{itemize}
\tightlist
\item
  First bullet
\item
  Second bullet
\item
  Third bullet
\end{itemize}

Numbered lists can be added as follows:

\begin{enumerate}
\def\labelenumi{\arabic{enumi}.}
\tightlist
\item
  First item
\item
  Second item
\item
  Third item
\end{enumerate}

The text continues here.

\hypertarget{figures-tables-and-schemes}{%
\subsection{Figures, Tables and
Schemes}\label{figures-tables-and-schemes}}

All figures and tables should be cited in the main text as Figure
\ref{fig:fig1}, \ref{tab:tab1}, etc. To get cross-reference to figure
generated by R chunks include the
\texttt{\textbackslash{}\textbackslash{}label\{\}} tag in the
\texttt{fig.cap} attribute of the R chunk:
\texttt{fig.cap\ =\ "Fancy\ Caption\textbackslash{}\textbackslash{}label\{fig:plot\}"}.

\begin{figure}[h!]

{\centering \includegraphics[width=0.7\linewidth]{MDPI_submition_files/figure-latex/fig1-1} 

}

\caption{A figure added with a code chunk.\label{fig:fig1}}\label{fig:fig1}
\end{figure}

When making tables using \texttt{kable}, it is suggested to use the
\texttt{format="latex"} and \texttt{tabl.envir="table"} arguments to
ensure table numbering and compatibility with the mdpi document class.

\begin{table}[H]

\caption{\label{tab:tab1}This is a table caption. Tables should be placed in the 
             main text near to the first time they are~cited.}
\begin{tabular}[t]{lccc}
\toprule
  & mpg & cyl & disp\\
\midrule
Mazda RX4 & 21.0 & 6 & 160\\
Mazda RX4 Wag & 21.0 & 6 & 160\\
Datsun 710 & 22.8 & 4 & 108\\
Hornet 4 Drive & 21.4 & 6 & 258\\
Hornet Sportabout & 18.7 & 8 & 360\\
\bottomrule
\end{tabular}
\end{table}

For a very wide table, landscape layouts are allowed.

\startlandscape

\begin{table}[H]

\caption{\label{tab:tab2}This is a very wide table}
\begin{tabular}[t]{cccc}
\toprule
Title.1 & Title.2 & Title.3 & Title.4\\
\midrule
Entry 1 & Data & Data & This cell has some longer content that runs over
                               two lines\\
Entry 2 & Data & Data & Data\\
\bottomrule
\end{tabular}
\end{table}

\finishlandscape

\hypertarget{formatting-of-mathematical-components}{%
\subsection{Formatting of Mathematical
Components}\label{formatting-of-mathematical-components}}

This is an example of an equation:

\[
a = 1.
\]

If you want numbered equations use Latex and wrap in the equation
environment:

\begin{equation}
a = 1,
\end{equation}

the text following an equation need not be a new paragraph. Please
punctuate equations as regular text.

This is the example 2 of equation:

\begin{adjustwidth}{-\extralength}{0cm}
\begin{equation}
a = b + c + d + e + f + g + h + i + j + k + l + m + n + o + p + q + r + s + t + 
u + v + w + x + y + z
\end{equation}
\end{adjustwidth}

Theorem-type environments (including propositions, lemmas, corollaries
etc.) can be formatted as follows:

Example of a theorem:

\begin{Theorem}
Example text of a theorem

\end{Theorem}

The text continues here.

\hypertarget{discussion}{%
\section{Discussion}\label{discussion}}

Authors should discuss the results and how they can be interpreted in
perspective of previous studies and of the working hypotheses. The
findings and their implications should be discussed in the broadest
context possible. Future research directions may also be highlighted.

\hypertarget{conclusion}{%
\section{Conclusion}\label{conclusion}}

This section is not mandatory, but can be added to the manuscript if the
discussion is unusually long or complex.

%%%%%%%%%%%%%%%%%%%%%%%%%%%%%%%%%%%%%%%%%%

\vspace{6pt}

%%%%%%%%%%%%%%%%%%%%%%%%%%%%%%%%%%%%%%%%%%
%% optional
\supplementary{The following supporting information can be downloaded
at:\\
\linksupplementary{s1}, Figure S1: title; Table S1: title; Video S1:
title.}

% Only for the journal Methods and Protocols:
% If you wish to submit a video article, please do so with any other supplementary material.
% \supplementary{The following supporting information can be downloaded at: \linksupplementary{s1}, Figure S1: title; Table S1: title; Video S1: title. A supporting video article is available at doi: link.}

%%%%%%%%%%%%%%%%%%%%%%%%%%%%%%%%%%%%%%%%%%
\authorcontributions{For research articles with several authors, a short
paragraph specifying their individual contributions must be provided.
The following statements should be used ``X.X. and Y.Y. conceive and
designed the experiments; X.X. performed the experiments; X.X. and Y.Y.
analyzed the data; W.W. contributed reagents/materials/analysis tools;
Y.Y. wrote the paper.'\,' Authorship must be limited to those who have
contributed substantially to the work reported.}

\funding{Please add:
\texttt{This\ research\ received\ no\ external\ funding\textquotesingle{}\textquotesingle{}\ or}This
research was funded by NAME OF FUNDER grant number XXX.'\,' and and
``The APC was funded by XXX'\,'. Check carefully that the details given
are accurate and use the standard spelling of funding agency names at
\url{https://search.crossref.org/funding}, any errors may affect your
future funding.}

\institutionalreview{In this section, you should add the Institutional
Review Board Statement and approval number, if relevant to your study.
You might choose to exclude this statement if the study did not require
ethical approval. Please note that the Editorial Office might ask you
for further information. Please add ``The study was conducted in
accordance with the Declaration of Helsinki, and approved by the
Institutional Review Board (or Ethics Committee) of NAME OF INSTITUTE
(protocol code XXX and date of approval).'' for studies involving
humans. OR ``The animal study protocol was approved by the Institutional
Review Board (or Ethics Committee) of NAME OF INSTITUTE (protocol code
XXX and date of approval).'' for studies involving animals. OR ``Ethical
review and approval were waived for this study due to REASON (please
provide a detailed justification).'' OR ``Not applicable'' for studies
not involving humans or animals.}


\dataavailability{We encourage all authors of articles published in MDPI
journals to share their research data. In this section, please provide
details regarding where data supporting reported results can be found,
including links to publicly archived datasets analyzed or generated
during the study. Where no new data were created, or where data is
unavailable due to privacy or ethical re-strictions, a statement is
still required. Suggested Data Availability Statements are available in
section ``MDPI Research Data Policies'' at
\url{https://www.mdpi.com/ethics}.}

\acknowledgments{All sources of funding of the study should be
disclosed. Please clearly indicate grants that you have received in
support of your research work. Clearly state if you received funds for
covering the costs to publish in open access.}

\conflictsofinterest{Declare conflicts of interest or state `The authors
declare no conflict of interest.' Authors must identify and declare any
personal circumstances or interest that may be perceived as
inappropriately influencing the representation or interpretation of
reported research results. Any role of the funding sponsors in the
design of the study; in the collection, analyses or interpretation of
data in the writing of the manuscript, or in the decision to publish the
results must be declared in this section. If there is no role, please
state `The founding sponsors had no role in the design of the study; in
the collection, analyses, or interpretation of data; in the writing of
the manuscript, an in the decision to publish the results'.}

%%%%%%%%%%%%%%%%%%%%%%%%%%%%%%%%%%%%%%%%%%
%% Optional

%% Only for journal Encyclopedia

\abbreviations{Abbreviations}{
The following abbreviations are used in this manuscript:\\

\noindent
\begin{tabular}{@{}ll}
MDPI & Multidisciplinary Digital Publishing Institute \\
DOAJ & Directory of open access journals \\
TLA & Three letter acronym \\
SDR & Solar downward radiation \\
\end{tabular}}

%%%%%%%%%%%%%%%%%%%%%%%%%%%%%%%%%%%%%%%%%%
%% Optional
\input{"appendix.tex"}
%%%%%%%%%%%%%%%%%%%%%%%%%%%%%%%%%%%%%%%%%%
\begin{adjustwidth}{-\extralength}{0cm}

%\printendnotes[custom] % Un-comment to print a list of endnotes


\reftitle{References}
\bibliography{manualreferences.bib}

% If authors have biography, please use the format below
%\section*{Short Biography of Authors}
%\bio
%{\raisebox{-0.35cm}{\includegraphics[width=3.5cm,height=5.3cm,clip,keepaspectratio]{Definitions/author1.pdf}}}
%{\textbf{Firstname Lastname} Biography of first author}
%
%\bio
%{\raisebox{-0.35cm}{\includegraphics[width=3.5cm,height=5.3cm,clip,keepaspectratio]{Definitions/author2.jpg}}}
%{\textbf{Firstname Lastname} Biography of second author}

%%%%%%%%%%%%%%%%%%%%%%%%%%%%%%%%%%%%%%%%%%
%% for journal Sci
%\reviewreports{\\
%Reviewer 1 comments and authors’ response\\
%Reviewer 2 comments and authors’ response\\
%Reviewer 3 comments and authors’ response
%}
%%%%%%%%%%%%%%%%%%%%%%%%%%%%%%%%%%%%%%%%%%
\PublishersNote{}
\end{adjustwidth}


\end{document}
