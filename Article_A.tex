% Options for packages loaded elsewhere
\PassOptionsToPackage{unicode}{hyperref}
\PassOptionsToPackage{hyphens}{url}
%
\documentclass[
  preprint, 3p, authoryear]{article}
\usepackage{amsmath,amssymb}
\usepackage{iftex}
\ifPDFTeX
  \usepackage[T1]{fontenc}
  \usepackage[utf8]{inputenc}
  \usepackage{textcomp} % provide euro and other symbols
\else % if luatex or xetex
  \usepackage{unicode-math} % this also loads fontspec
  \defaultfontfeatures{Scale=MatchLowercase}
  \defaultfontfeatures[\rmfamily]{Ligatures=TeX,Scale=1}
\fi
\usepackage{lmodern}
\ifPDFTeX\else
  % xetex/luatex font selection
\fi
% Use upquote if available, for straight quotes in verbatim environments
\IfFileExists{upquote.sty}{\usepackage{upquote}}{}
\IfFileExists{microtype.sty}{% use microtype if available
  \usepackage[]{microtype}
  \UseMicrotypeSet[protrusion]{basicmath} % disable protrusion for tt fonts
}{}
\makeatletter
\@ifundefined{KOMAClassName}{% if non-KOMA class
  \IfFileExists{parskip.sty}{%
    \usepackage{parskip}
  }{% else
    \setlength{\parindent}{0pt}
    \setlength{\parskip}{6pt plus 2pt minus 1pt}}
}{% if KOMA class
  \KOMAoptions{parskip=half}}
\makeatother
\usepackage{xcolor}
\usepackage[margin=1in]{geometry}
\usepackage{longtable,booktabs,array}
\usepackage{calc} % for calculating minipage widths
% Correct order of tables after \paragraph or \subparagraph
\usepackage{etoolbox}
\makeatletter
\patchcmd\longtable{\par}{\if@noskipsec\mbox{}\fi\par}{}{}
\makeatother
% Allow footnotes in longtable head/foot
\IfFileExists{footnotehyper.sty}{\usepackage{footnotehyper}}{\usepackage{footnote}}
\makesavenoteenv{longtable}
\usepackage{graphicx}
\makeatletter
\def\maxwidth{\ifdim\Gin@nat@width>\linewidth\linewidth\else\Gin@nat@width\fi}
\def\maxheight{\ifdim\Gin@nat@height>\textheight\textheight\else\Gin@nat@height\fi}
\makeatother
% Scale images if necessary, so that they will not overflow the page
% margins by default, and it is still possible to overwrite the defaults
% using explicit options in \includegraphics[width, height, ...]{}
\setkeys{Gin}{width=\maxwidth,height=\maxheight,keepaspectratio}
% Set default figure placement to htbp
\makeatletter
\def\fps@figure{htbp}
\makeatother
\setlength{\emergencystretch}{3em} % prevent overfull lines
\providecommand{\tightlist}{%
  \setlength{\itemsep}{0pt}\setlength{\parskip}{0pt}}
\setcounter{secnumdepth}{5}
\newlength{\cslhangindent}
\setlength{\cslhangindent}{1.5em}
\newlength{\csllabelwidth}
\setlength{\csllabelwidth}{3em}
\newlength{\cslentryspacingunit} % times entry-spacing
\setlength{\cslentryspacingunit}{\parskip}
\newenvironment{CSLReferences}[2] % #1 hanging-ident, #2 entry spacing
 {% don't indent paragraphs
  \setlength{\parindent}{0pt}
  % turn on hanging indent if param 1 is 1
  \ifodd #1
  \let\oldpar\par
  \def\par{\hangindent=\cslhangindent\oldpar}
  \fi
  % set entry spacing
  \setlength{\parskip}{#2\cslentryspacingunit}
 }%
 {}
\usepackage{calc}
\newcommand{\CSLBlock}[1]{#1\hfill\break}
\newcommand{\CSLLeftMargin}[1]{\parbox[t]{\csllabelwidth}{#1}}
\newcommand{\CSLRightInline}[1]{\parbox[t]{\linewidth - \csllabelwidth}{#1}\break}
\newcommand{\CSLIndent}[1]{\hspace{\cslhangindent}#1}
\usepackage{subcaption}
\usepackage{booktabs}
\usepackage{longtable}
\usepackage{array}
\usepackage{multirow}
\usepackage{wrapfig}
\usepackage{float}
\usepackage{colortbl}
\usepackage{pdflscape}
\usepackage{tabu}
\usepackage{threeparttable}
\usepackage{threeparttablex}
\usepackage[normalem]{ulem}
\usepackage{makecell}
\usepackage{xcolor}
\ifLuaTeX
  \usepackage{selnolig}  % disable illegal ligatures
\fi
\IfFileExists{bookmark.sty}{\usepackage{bookmark}}{\usepackage{hyperref}}
\IfFileExists{xurl.sty}{\usepackage{xurl}}{} % add URL line breaks if available
\urlstyle{same}
\hypersetup{
  pdftitle={Trends of SDR in Thessaloniki},
  pdfkeywords={GHI; SDR; Solar Brigthening/Dimming.},
  hidelinks,
  pdfcreator={LaTeX via pandoc}}

\title{Trends of SDR in Thessaloniki}
\author{true \and true}
\date{2023-11-02}

\begin{document}
\maketitle
\begin{abstract}
The shortwave downward solar irradiance (SDR) is an important factor that drives climate processes, production and can affect all living organisms.
While monitoring the long term variability, there are observations of upward and downward SDR trends on different locations around the world for different time periods.
Periods of positive tredns are refered as brightening periods and with negative tredns as dimming periods.
We studied 29 years of CM-21 data from Thessaloniki, Greece, under three sky conditions (clear sky, cloudy sky and all sky conditions), applying a cloud sky identification algorithm.
We found a positive trend for all-sky and clear-sky conditions, and also, investigated the consistency of those trends, the effect of the solar zenith angle, and the variation of the trends for the seasons of the year.
We indentified that there are some anomalies in the long term SDR trends, for all sky conditions.
\end{abstract}

\hypertarget{introduction.}{%
\section{Introduction.}\label{introduction.}}

The shortwave downward solar irradiance (SDR) at Earth's surface play a significant role, on its climate.
Changes of the SDR can be related to changes on Earth's energy budget, the mechanisms of climate change, and water and carbon cycle (Wild 2009).
Can also affect, solar and agricultural production, and all living organisms.
Studies of SDR variability, have identified some distinct SDR trends on different regions of the world on different time periods.
The term `brightening' is generally used to describe periods of positive SDR trend, and `dimming' for negative trend.
There are many cases on the long term records of irradiance, showing a systematic change of SDR's trend slope, occurring roughly at the last decades of the 20th century.
On multiple station in China, a dimming period was reported until about 2000, followed by a brightening period (Yang et al. 2021).
A similar pattern was identified, with the breaking point around 1980, for stations in Central Europe (Wild et al. 2021) and Brazil (Yamasoe et al. 2021).
Also, on global scale, an AI aided continental level spatial analysis, with data from multiple station, reach similar conclusions for the above regions and for the global trend (Yuan, Leirvik, and Wild 2021).

There is a consensus, among researchers, that the major factors of SDR attenuation is the interaction of Sun radiation with atmospheric aerosols and clouds.
Those interactions, among other factors, have been analysed with models (Li et al. 2016; Samset et al. 2018), showing the existence of feedback mechanisms between the two.
Similar finds, have been showed in observational data (Schwarz et al. 2020; Ohvril et al. 2009; Zerefos et al. 2009; Xia et al. 2007 and references therein).

Due to the variability of the phenomenon, and its contributing factors, there is a constant need to investigate and monitor SDR, in different sites, to estimate its magnitude, and its relation to the local conditions.
In this study, we examine the trends of SDR, with ground-based measurements at Thessaloniki, Greece for the period 1993 to 2023, as derived from a CM-21 pyranometer.
We reevaluated and extended the dataset used by Bais et al. (2013), applying a different algorithm for the identification of clear-/cloud-sky instances (Reno and Hansen 2016; Reno, Hansen, and Stein 2012a), and we derive the radiation trends for the period 1993 to 2023, under different sky conditions (all-sky, clear-sky and cloud-sky).

\hypertarget{observational-data-and-methodology.}{%
\section{Observational data and methodology.}\label{observational-data-and-methodology.}}

The SDR data were measured with a Kipp \& Zonen CM-21 pyranometer operating continuously at the Laboratory of Atmospheric Physics of the Aristotle University of Thessaloniki
(\(40^\circ\,38'\,\)N, \(22^\circ\,57'\,\)E, \(80\,\)m~a.s.l.)
in the period from
1993-04-13
to
2023-04-13.
The monitoring site is located near the city centre, and we expect to be affected by the urban environment.
During the study period, the pyranometer has been independently calibrated three times at the Meteorologisches Observatorium Lindenberg, DWD, when it was verified the stability of the instrument to within better than \(0.7\%\) relative to the initial calibration by the manufacturer.
Along with SDR, the direct beam radiation (DNI) was also measured by a collocated Kipp \& Zonen CHP-1 pyrheliometer, for the period
2016-04-01
to
2023-04-13.
Although, we have performed a similar analysis to the DNI data the results are not presented here, as they lack the appropriate statistical significance, due to the sorter duration of the data.
However, the DNI data were used as auxiliary data, in the clear sky identification algorithm (CSid), which is discussed later, for the selection of the appropriate thresholds.
It is noted that despite the capability of the CSid algorithm to use the DNI as a characterization parameter, we haven't utilized it here, to avoid any selection bias, due to unequal length of the two datasets.
There are four distinct steps in the creation of the dataset analysed here:
a)~the acquisition of radiation measurements from the sensors,
b)~the data quality check,
c)~the identification of ``clear sky'' conditions from the radiometric data, and
d)~the aggregation of data and trend analysis.

For the acquisition of radiometric data, the signal of the pyranometer is sampled with a rate of \(1\,\text{Hz}\).
The mean and the standard deviation of these samples are recorded every minute.
The measurements are corrected for the zero offset (``dark signal'' in volts).
The ``dark signal'' is calculated by averaging all measurements recorded for a period of
\(3\,\text{h}\),
before (morning) or after (evening) the Sun reaches an elevation angle of
\(-10^\circ\).
The signal is converted to irradiance using a ramped value of the instrument's sensitivity between calibrations.

A manual screening was performed, to remove inconsistent and erroneous recordings that can occur stochastically or systematically, during the continuous operation of the instruments.
The manual screening is aided by a radiation data quality assurance procedure, adjusted for the site, which is based on the methods of
Long and Shi~(2008, 2006).
Thus, problematic recordings have been excluded from further processing.
Although it is impossible to detect all false data, the large number of available data, and the aggregation scheme we used, ensures the good quality of the radiometric measurements used in this study.

In order to be able to estimate the effect of the sky condition on the long term variability of SDR, we created three datasets, by characterizing each one-minute measurement with a corresponding sky condition (i.e., all-sky, clear-sky and cloudy-sky).
To identify the clear-sky conditions we used a method proposed by
Long and Ackerman (2000)
and by
Reno and Hansen (2016),
which was adapted and configured for the site, as the authors suggest.

We have to note, that the definition of what constitutes as clear or cloudy sky, has some subjectivity, in any method of characterization.
As a result, the details of the definition are site specific, it relies on a combination of thresholds and comparisons with ideal actinometric models and statistical analysis on different signal metrics.
The CSid algorithm was calibrated with the main focus, to identify the presence of clouds on the sky dome.
Although the fine-tuning of the procedure, few marginal cases exist, that have been identified manually as false positive or false negative but cannot affect the final results of the study.

For completeness, we will provide below a brief overview of the
clear sky identification algorithm (CSid),
along with the site specific thresholds.
To calculate the reference clear sky
\(\text{SDR}_\text{CSref}\) we used the \(\text{SDR}_\text{Haurwitz}\) derived by
the radiation model of Haurwitz (1945), adjusted for our site with a
factor \(a\) (Eq.~\ref{eq:ahau}), resulted by an iterative optimization process, as described
by Long and Ackerman (2000) and Reno and Hansen (2016).
The target of
the optimization was the minimization of a function \(f(a)\) (Eq.~\ref{eq:minf}) and
was accomplished with the algorithmic function ``optimise'', which is an implementation based on the work of Brent (1973), from the library ``stats'' of the R programming language (R Core Team 2023).
The optimization and the selection of the clear sky reference model, was performed on SDR observations for the period 2016 - 2021.
During the optimization, eight simple clear sky radiation models were tested (Daneshyar-Paltridge-Proctor, Kasten-Czeplak, Haurwitz, Berger-Duffie, Adnot-Bourges-Campana-Gicquel, Robledo-Soler, Kasten and Ineichen-Perez), with a wide range of factors.
These models are described in more details by Reno, Hansen, and Stein (2012b) and evaluated by Reno and Hansen (2016).
We found, that Haurwitz's model, adjusted with the factor \(a = 0.965\) yields one of the lowest root mean squared errors (RMSE),
while the procedure, manages to characterize the majority of the data.
Thus, our clear sky reference is derived by the Eq.~\ref{eq:ahau}.
\begin{equation}
f(a) = \frac{1}{n}\sum_{i=1}^{n} ( \text{SDR}_{\text{CSid},i} - a \times \text{SDR}_{\text{testCSref},i} )^2 \label{eq:minf}
\end{equation}
where: \(n\) is the total number of daylight records, \(\text{SDR}_{\text{CSid},i}\) are the records identified as clear sky by CSid, \(a\) is a hypothetical adjustment factor, and \(\text{SDR}_{\text{testCSref},i}\) is any of the tested clear sky radiation models.
\begin{equation}
\text{SDR}_\text{CSref} = a \times \text{SDR}_\text{Haurwitz} = 0.965 \times 1098 \times \cos(\theta) \times \exp \left( \frac{ - 0.057}{\cos(\theta)} \right) \label{eq:ahau}
\end{equation}
where: \(\text{SDR}_\text{CSref}\) is the reference clear sky SDR, in \(\text{w}\,\text{m}^{-2}\) and \(\theta\) is the solar zenith angle (SZA).

The criteria that were used to identify whether a measurement was taken
under clear-sky conditions are presented below.
A data point is flagged
as ``clear-sky'' if all criteria are satisfied, otherwise it is considered to be ``cloud-sky''.
Each criterion was applied
for a running window of \(11\) consecutive one-minute measurements, and
the characterization is assigned to the central value of the window.
Each parameter, was calculated both from the observations and the
reference clear sky model, for each comparison.
The allowable range of variation is defined by the
model-derived value of the parameter multiplied by a factor plus an
offset.
The factors and the offsets were
determined empirically, by manual inspecting each filters performance on
selected days, and adjusting them accordingly, during an iterative
process.

\begin{enumerate}
\def\labelenumi{\alph{enumi})}
\tightlist
\item
  Mean of the measured \(\overline{\text{SDR}}_i\) (Eq. \ref{eq:MeanVIP}).
  \begin{equation}
  0.91 \times \overline{\text{SDR}}_{\text{CSref},i} - 20
  < \overline{\text{SDR}}_i <
  1.095 \times \overline{\text{SDR}}_{\text{CSref},i} + 30
  \label{eq:MeanVIP}
  \end{equation}
\item
  Maximum measured value \(M_{\text{}}\) (Eq.~\ref{eq:MaxVIP}).
  \begin{equation}
  1 \times M_{\text{CSref},i} - 75
  < M_{\text{}i} <
  1 \times M_{\text{CSref},i} + 75
  \label{eq:MaxVIP}
  \end{equation}
\item
  Length \(L_i\) of the sequential line segments, connecting the points of the \(11\) SDR values (Eq. \ref{eq:VILeq}).
  \begin{equation}
  L_i = \sum_{i=1}^{n-1}\sqrt{\left ( \text{SDR}_{i+1} - \text{SDR}_{i}\right )^2 + \left ( t_{i+1} - t_i \right )^2}
  \label{eq:VILeq}
  \end{equation}
  \begin{equation}
  1 \times L_{\text{CSref},i} - 5 < L_i < 1.3 \times L_{\text{CSref},i} + 13
  \label{eq:VILcr}
  \end{equation}
  where: \(t_i\) is the time each SDR measurement has been measured
\item
  Standard deviation \(\sigma_i\) of the slope (\(s_i\)) between the \(11\) sequential points, normalized by the mean \(\overline{\text{SDR}}_i\) (Eq.~\ref{eq:VCT1}).
  \begin{gather}
    \sigma_i = \frac {\sqrt{\frac{1}{n-1} \sum_{i=1}^{n-1} \left( s_i - \bar{s} \right)^2}} {\overline{\text{SDR}}_i} \label{eq:VCT1} \\
    s_i = \frac{\text{SDR}_{i+1} - \text{SDR}_{i}}{t_{i+1} - t_i},\;\;   \bar{s} = \frac{1}{n-1} \sum_{i=1}^{n-1} s_i,\;\;\forall i \in \left \{ 1, 2, \ldots, n-1 \right \}\;\;
  \end{gather}
  For this criterion, \(\sigma_i\) should be below a certain threshold (Eq.~\ref{eq:VCTcr}):
  \begin{equation}
    \sigma_i < \ensuremath{1.1\times 10^{-4}} \label{eq:VCTcr}
  \end{equation}
\item
  Maximum difference \(X_i\) between the change in measured irradiance and the change in clear sky irradiance over each measurement interval.
  \begin{gather}
    X_i = \max{\left \{ \left | x_i - x_{\text{CSref},i} \right | \right \}} \label{eq:VSM3} \\
    x_i = \text{SDR}_{i+1} - \text{SDR}_{i} \forall i \in \left \{ 1, 2, \ldots, n-1 \right \} \label{eq:VSM1} \\
    x_{\text{CSref},i} = \text{SDR}_{\text{CSref},i+1} - \text{SDR}_{\text{CSref},i} \forall i \in \left \{ 1, 2, \ldots, n-1 \right \} \label{eq:VSM2}
  \end{gather}
  For this criterion, \(X_i\) should be below a certain threshold (Eq.~\ref{eq:VSMcr}):
  \begin{equation}
    X_i < 7.5 \label{eq:VSMcr}
  \end{equation}
\end{enumerate}

Due to a significant measurement uncertainty near the horizon, we have to exclude all measurements with SZA greater than \(85^\circ\).
Moreover, due to some obstructions around the site (hills and buildings), we excluded data with Azimuth angle between
\(35^\circ\) and \(120^\circ\) with SZA greater than \(80^\circ\).
On the latter instances, Sun is systematically, not visible by the instrument's location.
To make the measurements comparable throughout the dataset, we adjusted all one-minute radiometric values to the mean Sun - Earth distance.
Subsequently, we made all measurements relative to the Total Solar Irradiance (TSI) at \(1\,\text{au}\), in order to compensate for the Sun's intensity variability, using a time series of satellite TSI observations.
The TSI data we use are part of the ``NOAA Climate Data Record of Total Solar Irradiance'' dataset (Coddington et al. 2005).
Where the initial daily values, were interpolated to match with the time step of our measurements.
The final dataset contains
\(6589967\)
one-minute measurements, of which,
\(84.2\%\)
were identified as under clear-sky conditions and subsequently
\(15.8\%\)
as under cloud-sky conditions.

In order to investigate the SDR trends, we implemented an appropriate aggregation scheme to the 1-minute data to derive a series in coarser timescale.
To preserve the representativeness of the data we used the following criteria:
a) for the daily mean values we accept days with more than 50\% of the daytime measurements, present and valid,
b) monthly values were computed from daily means
only when at least 20 days were available.
To create the daily and monthly climatological means, we averaged the data based on the day of year and calendar month, respectively.
For the seasonal means we averaged the mean daily values in each season (Winter: December - February, Spring: March - May, etc.).
Finally, each data set was deseasonalized by subtracting the corresponding climatological annual cycle (daily or monthly) from the actual data.
To estimate SZA contribution to the SDR trends, the one-minute data were aggregated in \(1^\circ\) SZA bins, separately for the morning and afternoon hours, and then were deseasonalized as mentioned above.

\hypertarget{results}{%
\section{Results}\label{results}}

\hypertarget{long-term-sdr-trends}{%
\subsection{Long-term SDR trends}\label{long-term-sdr-trends}}

We calculated the linear SDR trends, from the departures of the mean daily values from the seasonal values, and for the three sky conditions (Table~\ref{tab:trendtable}).
In Figure~\ref{fig:trendALL} we present only the time series under all-sky conditions; the plots for clear-sky and cloud-sky conditions, are very similar and are shown in the Appendix (Figures~\ref{fig:trendCLEAR} and~ \ref{fig:trendCLOUD}).
We observe a positive trend for all-sky conditions
(\(0.38\,\%/y\)), a very close but lower trend for clear-skies (\(0.35\,\%/y\)) and a negative weaker trend for cloud-skies (\(-0.28\,\%/y\)).
In the studied period, there is no significant break or change in the variability of the time series.
Other studies for the European region report a change of the SDR slope, around 1980, a few years before the start of our records (Wild et al. 2021; Yuan, Leirvik, and Wild 2021; Ohmura 2009). It is interesting to note, that for the observation period, the trend of the TSI is
\(-0.0002\,\%/y\), and thus we can not attribute any major effect on SDR trend to Solar variability.

\begin{table}[H]

\caption{(\#tab:trendtable)Trends in SDR daily means for different sky conditions for the period 1993 - 2023.}
\begin{tabu} to \linewidth {>{\centering\arraybackslash}p{8em}>{\raggedleft}X>{\raggedleft}X>{\raggedleft}X}
\toprule
Sky conditions & Trend [\%/year] & Statistical signif. [\%] & Days with data\\
\midrule
All sky & 0.376 & 100.00 & 10256\\
Clear sky & 0.349 & 100.00 & 10256\\
Cloudy sky & -0.276 & 99.97 & 5067\\
\bottomrule
\end{tabu}
\end{table}

\begin{figure}[h!]

{\centering \includegraphics[width=.70\linewidth]{./images/LongtermTrends-2} 

}

\caption{Anomaly (\%) of the daily all-sky SDR, relative to climatological values for 1993 - 2023. The black line shows the long term linear trend.}(\#fig:trendALL)
\end{figure}

Although the year-to-year variability of the anomalies (Figure \ref{fig:trendALL} and Figures \ref{fig:trendCLEAR}, \ref{fig:trendCLOUD} on Appendix), shows a rather homogeneous behavior, plots of the cumulative sums (CUSUM) of the anomalies can reveal different structures within all the three sky conditions (Regier, Briceño, and Boyer 2019).
In the cases of all-sky and clear-sky (Figures \ref{fig:cusummonth-1} and \ref{fig:cusummonth-2}), we observe three macroscopic periods.
A downward part from the start until about 2005, an relative steady part until about 2016 and finally, a steep upward part until the present.
For cloud-sky (Figure~\ref{fig:cusummonth-3}), we have a different sequence, it begins with a relative steady part until 1997, followed by a upward part until 2005, a long decline until 2020, with a small positive slope until the present.
For a uniform trend, we would expect the CUSUMs of the anomalies to have a symmetric `V' shape.
This would indicate that the anomalies are evenly distributed around the climatological mean, and for a positive uniform trend, the first half to be below and the other half above the climatological mean.
In our case, there is more complex evolution of the phenomenon.
Another distinct feature of the CUSUMs, is the different pattern of the cloudy-sky dataset which peaks around the middle of the period.
Although, it seems to exist a complementary relation to the CUSUMs of the clear- and all-sky cases, we can not assert that clouds are the main drive for this relation, due to the great difference in the number of observational data between the datasets (Table~\ref{tab:trendtable}).

\begin{figure}[h!]

{\centering \subfloat[All-skies.(\#fig:cusummonth-1)]{\includegraphics[width=.32\linewidth]{./images/CumulativeMonthlyCuSum-1} }\subfloat[Clear-skies.(\#fig:cusummonth-2)]{\includegraphics[width=.32\linewidth]{./images/CumulativeMonthlyCuSum-5} }\subfloat[Cloud-skies.(\#fig:cusummonth-3)]{\includegraphics[width=.32\linewidth]{./images/CumulativeMonthlyCuSum-9} }

}

\caption{Running cumulative sum plots of the monthly SDR.}(\#fig:cusummonth)
\end{figure}

In order to investigate more, the features within the trends, we created another set of CUSUMs plots by subtracting the corresponding long term trend from the SDR anomaly data, prior to the CUSUM calculation (Figure~\ref{fig:cusumnotrendmonthly}).
With this approach periods when the CUSUMs diverge from zero can be interpreted as systematic variation of SDR from the climatological mean. When the CUSUM is increasing, the added values are above the climatological values of SDR trend and vice versa.
Overall, for all- and clear-sky conditions (Figure~\ref{fig:cusumnotrendmonthly-1} and~\ref{fig:cusumnotrendmonthly-2}) we observe periods when the anomalies diverge from the climatological value, each lasting for several years.
The pattern for the all-sky and clear-sky conditions is very similar, suggesting prevalence in clear skies over Thessaloniki.
It is interesting that in the period 1993 - 2016 the anomalies of all-sky and clear-sky, have a high variability around zero. After 2016, the range of the variability is decreased at about one third of the prior period.
For cloud-sky conditions (Figure~\ref{fig:cusumnotrendmonthly-3}) the period 1997 - 2008 is dominated by positive CUSUMs suggesting a reduced effect of clouds on SDR.
From 1997 to mid-2000s CUSUMs are increasing, likely due to a continuous decrease in the thickness of clouds, followed by a period of rapid increase (within 3 years) in cloud optical thickness lasting up to 2008.
The following stable period spans for about 15 years up to 2021 when CUSUMs start increasing again.

\begin{figure}[h!]

{\centering \subfloat[All-skies.(\#fig:cusumnotrendmonthly-1)]{\includegraphics[width=.32\linewidth]{./images/CumulativeMonthlyCuSumNOtrend-1} }\subfloat[Clear-skies.(\#fig:cusumnotrendmonthly-2)]{\includegraphics[width=.32\linewidth]{./images/CumulativeMonthlyCuSumNOtrend-5} }\subfloat[Cloud-skies.(\#fig:cusumnotrendmonthly-3)]{\includegraphics[width=.32\linewidth]{./images/CumulativeMonthlyCuSumNOtrend-9} }

}

\caption{Running cumulative sum plots of the monthly SDR after removing the long term trend from the monthly values.}(\#fig:cusumnotrendmonthly)
\end{figure}

\hypertarget{effects-of-the-solar-zenith-angle-on-sdr.}{%
\subsection{Effects of the solar zenith angle on SDR.}\label{effects-of-the-solar-zenith-angle-on-sdr.}}

The solar zenith angle is a major driver of the attenuation of SDR reaching the ground, as SDR is affected by the enhancement of the radiation path in the atmosphere, especially in urban environment with human activities emitting aerosols (Wang et al. 2021).
In order to estimate the effect of the SZA on the SDR trends, we grouped the anomaly data in bins of \(1^\circ\) SZA, and calculated the overall trend for each bin before noon and after noon (Figure~\ref{fig:szatrends}).
Although there are seasonal dependencies of the minimum SZA (see Appendix, Figure~\ref{fig:SZAtrendSeason}), these dependencies would not be further examined here.
For all-sky and clear-sky conditions the brightening effect of SDR (positive trend) is stronger for large SZAs (Figures~\ref{fig:szatrends-1} and \ref{fig:szatrends-2}).
The trends in the morning and afternoon hours are more or less consistent with small differences, which can be attributed to systematic diurnal variations of aerosols, particularly during the warm period of the year (Wang et al. 2021).
For cloud-sky conditions (Figure~\ref{fig:szatrends-3}), we can not discern any significant dependence of the SDR trend with SZA.
For SZAs \(16^\circ\) - \(50^\circ\), the trends range within about \(\pm 0.2\,\%/y\), with a weak statistical significance.
Between \(50^\circ\) and \(75^\circ\) SZA the trends for the period before noon are stronger and negative, possibly associated with stronger attenuation by clouds under oblique incidence angles.

\begin{figure}[h!]

{\centering \subfloat[All-sky conditions.(\#fig:szatrends-1)]{\includegraphics[width=.32\linewidth]{./images/SzaTrends-1} }\subfloat[Clear-sky conditions.(\#fig:szatrends-2)]{\includegraphics[width=.32\linewidth]{./images/SzaTrends-4} }\subfloat[Cloud-sky conditions.(\#fig:szatrends-3)]{\includegraphics[width=.32\linewidth]{./images/SzaTrends-7} }

}

\caption{Distribution of the SDR's long term trends by SZA before and after noon. The solid data points represent values with an acceptable statistical significance ($p<0.005$), the trends were computed with SZA bins of $1^\circ$ for each day, before and after local noon.}(\#fig:szatrends)
\end{figure}

\hypertarget{long-term-trends-by-season-of-year}{%
\subsection{Long term trends by season of year}\label{long-term-trends-by-season-of-year}}

Similar to the long term trends discussed above, we have calculated the trend of the anomalies for the three different sky conditions, and for each season of the year, using the corresponding mean monthly values
(Figure~\ref{fig:seasonalALL} and Table~\ref{tab:trendseasontable}).
For all-sky conditions the trend in SDR in winter is the largest
(\(0.69\,\%/y\)),
followed by the trend in autumn
(\(0.43\,\%/y\),
value similar to the long term trend) both statistically significant above the \(99\,\%\) confidence level.
In spring and summer, the trends are much smaller and less statistical significant.
These seasonal differences indicate a possible relation of the trends to clouds during winter and autumn.
For clear-skies, the trend in winter is \(0.83\,\%/y\), larger than for all-skies (\(0.69\,\%/y\)), which is another indication of increasing cloud thickness.
Moreover, the trends under clear- and cloud-sky conditions are almost complementary to each other, particularly for winter and autumn, where the signal is stronger.
During spring and summer the statistical significance is very low and the actual trend too small for a meaningful comparison.

\begin{figure}[h!]

{\centering \includegraphics[width=1\linewidth]{./images/SeasonalTrendsTogether3-2} 

}

\caption{SDR trends by season of the year for the three sky conditions. Each point is a monthly mean calculated from daily means. The vertical scale is common for all graphs, each column of graphs corresponds to a sky condition and each row to a season of the year.}(\#fig:seasonalALL)
\end{figure}

\begin{table}[!h]

\caption{(\#tab:trendseasontable)SDR linear trends of monthly anomalies for each season of the year.}
\begin{tabu} to \linewidth {>{\centering\arraybackslash}p{8em}>{\centering}X>{\raggedleft}X>{\raggedleft}X}{>{\centering\arraybackslash}p{8em}crr}
\toprule
Sky condition & Season & Trend [\%/year] & Statistical signif. [\%]\\
\midrule
\cellcolor{gray!6}{} & \cellcolor{gray!6}{Winter} & \cellcolor{gray!6}{0.6860} & \cellcolor{gray!6}{99.9}\\

 & Spring & 0.1450 & 81.8\\

\cellcolor{gray!6}{} & \cellcolor{gray!6}{Summer} & \cellcolor{gray!6}{0.1200} & \cellcolor{gray!6}{89.4}\\

\multirow{-4}{8em}{\centering\arraybackslash All sky} & Autumn & 0.4310 & 99.3\\
\cmidrule{1-4}
\cellcolor{gray!6}{} & \cellcolor{gray!6}{Winter} & \cellcolor{gray!6}{0.8260} & \cellcolor{gray!6}{100.0}\\

 & Spring & 0.0613 & 38.6\\

\cellcolor{gray!6}{} & \cellcolor{gray!6}{Summer} & \cellcolor{gray!6}{-0.0307} & \cellcolor{gray!6}{25.9}\\

\multirow{-4}{8em}{\centering\arraybackslash Clear sky} & Autumn & 0.3670 & 97.2\\
\cmidrule{1-4}
\cellcolor{gray!6}{} & \cellcolor{gray!6}{Winter} & \cellcolor{gray!6}{-0.8820} & \cellcolor{gray!6}{98.9}\\

 & Spring & -0.0991 & 37.2\\

\cellcolor{gray!6}{} & \cellcolor{gray!6}{Summer} & \cellcolor{gray!6}{0.0444} & \cellcolor{gray!6}{21.5}\\

\multirow{-4}{8em}{\centering\arraybackslash Cloudy sky} & Autumn & -0.4000 & 89.5\\
\bottomrule
\end{tabu}
\end{table}

\hypertarget{conclusions}{%
\section{Conclusions}\label{conclusions}}

We demonstrate that in the period
1993 - 2023, there is a brightening trend of SDR for all-sky conditions (\(0.38\,\%/y\)).
On a previous study (Bais et al. 2013), for the period 1993 - 2011 the trend was found to be \(0.33\,\%/y\).
The increase of this trend, indicates that the phenomenon and probably the causes, are still occurring.
Accordingly, we found a similar trend for clear-sky conditions (\(0.35\,\%/y\)) that further supports that the brightening can be independent of the cloud occurrences, although, cloud formation can also attenuate SDR.
We have calculated the effect of TSI variability to be a negligible factor in SDR variability.
Unfortunately, there is no available data for this period for the aerosols, in order to directly estimate their effect on SDR, which is the most probable main cause.
The attenuation of SDR by aerosols over Europe have been proposed as major factor by Wild et al. (2021).
The dimming effect of SDR under cloud-sky condition (\(-0.28\,\%/y\)), shows that clouds can also attenuate the amount of SDR reaching the ground, and changes on their occurrences can effect the net energy balance.
Because we have no adequate data to investigate the long term changes of cloud formation in the region, we can not infer if the SDR trend we observe under cloud-sky is biased by changes in clouds and local climate.
The observed brightening of SDR over Thessaloniki; is dependent on SZA (higher SZA, have stronger brightening trends), and also, with the seasons of the year, where Winter and Autumn show a significant strong trend, in contrast to Spring and Summer.
Using CUSUMs of the trends for all- and clear-skies, we observed periods where the CUSUMs remain relative steady,
with a steep decline previous to that, and a steep incline after. This is an indication that the whole brightening phenomenon does not follow a smooth development over time.

Our findings are in agreement with other researchers for the region.
Future observations from the collocated pyrheliometer, may provide us with the means to further investigate the variability of Solar radiation at ground level in Thessaloniki.
Also, more data (cloud occurrence, aerosols, atmospheric water vapour, etc.), will help us to determine the extent of SDR attenuation by each factor.

\hypertarget{references}{%
\section*{References}\label{references}}
\addcontentsline{toc}{section}{References}

\hypertarget{refs}{}
\begin{CSLReferences}{1}{0}
\leavevmode\vadjust pre{\hypertarget{ref-Bais2013}{}}%
Bais, A. F., T. Drosoglou, C. Meleti, K. Tourpali, and N. Kouremeti. 2013. {``Changes in Surface Shortwave Solar Irradiance from 1993 to 2011 at Thessaloniki (Greece).''} \emph{International Journal of Climatology} 33 (13): 2871--76. \url{https://doi.org/f5dzz5}.

\leavevmode\vadjust pre{\hypertarget{ref-Brent1973}{}}%
Brent, Richard P. 1973. {``Algorithms for Minimization Without Derivatives.''} \emph{PrenticeHall, Englewood Cliffs, NJ}.

\leavevmode\vadjust pre{\hypertarget{ref-Coddington2005}{}}%
Coddington, Odele, Judith L. Lean, Doug Lindholm, Peter Pilewskie, Martin Snow, and NOAA CDR Program. 2005. {``{NOAA} Climate Data Record ({CDR}) of Total Solar Irradiance ({TSI}), {NRLTSI} Version 2. {D}aily.''} 2005. \url{https://doi.org/10.7289/V55B00C1}.

\leavevmode\vadjust pre{\hypertarget{ref-Haurwitz1945}{}}%
Haurwitz, Bernhard. 1945. {``Insolation in {Relation} to {Cloudiness} and {Cloud} {Density}.''} \emph{Journal of Meteorology} 2 (September): 154--66.

\leavevmode\vadjust pre{\hypertarget{ref-Li2016}{}}%
Li, Zhanqing, W. K.‐M. Lau, V. Ramanathan, G. Wu, Y. Ding, M. G. Manoj, J. Liu, et al. 2016. {``Aerosol and Monsoon Climate Interactions over Asia.''} \emph{Reviews of Geophysics} 54 (4): 866--929. \url{https://doi.org/10.1002/2015RG000500}.

\leavevmode\vadjust pre{\hypertarget{ref-Long2000}{}}%
Long, Charles N., and Thomas P. Ackerman. 2000. {``Identification of Clear Skies from Broadband Pyranometer Measurements and Calculation of Downwelling Shortwave Cloud Effects.''} \emph{Journal of Geophysical Research: Atmospheres} 105 (D12, D12): 15609--26. \url{https://doi.org/10.1029/2000jd900077}.

\leavevmode\vadjust pre{\hypertarget{ref-Long2006}{}}%
Long, Charles N., and Y. Shi. 2006. {``The QCRad Value Added Product: Surface Radiation Measurement Quality Control Testing, Including Climatology Configurable Limits.''} DOE/SC-ARM/TR-074. Office of Science, Office of Biological; Environmental Research, U.S. Department of Energy.

\leavevmode\vadjust pre{\hypertarget{ref-Long2008a}{}}%
---------. 2008. {``An Automated Quality Assessment and Control Algorithm for Surface Radiation Measurements.''} \emph{The Open Atmospheric Science Journal}, 23--37.

\leavevmode\vadjust pre{\hypertarget{ref-Ohmura2009}{}}%
Ohmura, Atsumu. 2009. {``Observed Decadal Variations in Surface Solar Radiation and Their Causes.''} \emph{Jour} 114. \url{https://doi.org/10.1029/2008JD011290}.

\leavevmode\vadjust pre{\hypertarget{ref-Ohvril2009}{}}%
Ohvril, Hanno, Hilda Teral, Lennart Neiman, Martin Kannel, Marika Uustare, Mati Tee, Viivi Russak, et al. 2009. {``Global Dimming and Brightening Versus Atmospheric Column Transparency, Europe, 1906--2007.''} \emph{Journal of Geophysical Research} 114 (May). \url{https://doi.org/10.1029/2008JD010644}.

\leavevmode\vadjust pre{\hypertarget{ref-RCT2023}{}}%
R Core Team. 2023. \emph{R: A Language and Environment for Statistical Computing}. Vienna, Austria: R Foundation for Statistical Computing. \url{https://www.R-project.org/}.

\leavevmode\vadjust pre{\hypertarget{ref-Regier2019}{}}%
Regier, Peter, Henry Briceño, and Joseph N. Boyer. 2019. {``Analyzing and Comparing Complex Environmental Time Series Using a Cumulative Sums Approach.''} \emph{{MethodsX}} 6: 779--87. \url{https://doi.org/10.1016/j.mex.2019.03.014}.

\leavevmode\vadjust pre{\hypertarget{ref-Reno2016}{}}%
Reno, Matthew J., and Clifford W. Hansen. 2016. {``Identification of Periods of Clear Sky Irradiance in Time Series of GHI Measurements.''} \emph{Renewable Energy} 90: 520--31. \url{https://doi.org/gq3sbg}.

\leavevmode\vadjust pre{\hypertarget{ref-Reno2012a}{}}%
Reno, Matthew J., Clifford W. Hansen, and Joshua S. Stein. 2012a. {``{Global Horizontal Irradiance Clear Sky Models: Implementation and Analysis}.''} \emph{SANDIA REPORT SAND2012-2389 Unlimited Release Printed March 2012}, March, 1--66.

\leavevmode\vadjust pre{\hypertarget{ref-Reno2012}{}}%
---------. 2012b. {``Global Horizontal Irradiance Clear Sky Models: Implementation and Analysis.''} SAND2012-2389, 1039404. \url{https://doi.org/gq5npv}.

\leavevmode\vadjust pre{\hypertarget{ref-Samset2018}{}}%
Samset, B. H., M. Sand, C. J. Smith, S. E. Bauer, P. M. Forster, J. S. Fuglestvedt, S. Osprey, and C.‐F. Schleussner. 2018. {``Climate Impacts from a Removal of Anthropogenic Aerosol Emissions.''} \emph{Geophysical Research Letters} 45 (2): 1020--29. \url{https://doi.org/10.1002/2017GL076079}.

\leavevmode\vadjust pre{\hypertarget{ref-Schwarz2020}{}}%
Schwarz, M., D. Folini, S. Yang, R. P. Allan, and M. Wild. 2020. {``Changes in Atmospheric Shortwave Absorption as Important Driver of Dimming and Brightening.''} \emph{Nature Geoscience} 13 (2): 110--15. \url{https://doi.org/10.1038/s41561-019-0528-y}.

\leavevmode\vadjust pre{\hypertarget{ref-Wang2021}{}}%
Wang, Yawen, Jiahua Zhang, Arturo Sanchez‐Lorenzo, Katsumasa Tanaka, Jörg Trentmann, Wenping Yuan, and Martin Wild. 2021. {``Hourly Surface Observations Suggest Stronger Solar Dimming and Brightening at Sunrise and Sunset over China.''} \emph{Geophysical Research Letters} 48 (2). \url{https://doi.org/10.1029/2020GL091422}.

\leavevmode\vadjust pre{\hypertarget{ref-Wild2009}{}}%
Wild, Martin. 2009. {``Global Dimming and Brightening: A Review.''} \emph{Journal of Geophysical Research Atmospheres} 114 (12): 1--31. \url{https://doi.org/bcq}.

\leavevmode\vadjust pre{\hypertarget{ref-Wild2021}{}}%
Wild, Martin, Stephan Wacker, Su Yang, and Arturo Sanchez‐Lorenzo. 2021. {``Evidence for Clear‐sky Dimming and Brightening in Central Europe.''} \emph{Geophysical Research Letters} 48 (6). \url{https://doi.org/10.1029/2020GL092216}.

\leavevmode\vadjust pre{\hypertarget{ref-Xia2007}{}}%
Xia, Xiangao, Hongbin Chen, Zhanqing Li, Pucai Wang, and Jiankai Wang. 2007. {``Significant Reduction of Surface Solar Irradiance Induced by Aerosols in a Suburban Region in Northeastern China.''} \emph{Journal of Geophysical Research Atmospheres} 112 (22): 1--9. \url{https://doi.org/cdtntw}.

\leavevmode\vadjust pre{\hypertarget{ref-Yamasoe2021}{}}%
Yamasoe, Marcia Akemi, Nilton Manuel Évora Rosário, Samantha Novaes Santos Martins Almeida, and Martin Wild. 2021. {``Fifty-Six Years of Surface Solar Radiation and Sunshine Duration over São Paulo, Brazil: 1961--2016.''} \emph{Atmospheric Chemistry and Physics} 21 (9): 6593--603. \url{https://doi.org/10.5194/acp-21-6593-2021}.

\leavevmode\vadjust pre{\hypertarget{ref-Yang2021}{}}%
Yang, Su, Zijiang Zhou, Yu Yu, and Martin Wild. 2021. {``Cloud {`Shrinking'} and {`Optical Thinning'} in the {`Dimming'} Period and a Subsequent Recovery in the {`Brightening'} Period over China.''} \emph{Environmental Research Letters}, January. \url{https://doi.org/10.1088/1748-9326/abdf89}.

\leavevmode\vadjust pre{\hypertarget{ref-Yuan2021}{}}%
Yuan, Menghan, Thomas Leirvik, and Martin Wild. 2021. {``Global Trends in Downward Surface Solar Radiation from Spatial Interpolated Ground Observations During 1961-2019.''} \emph{Journal of Climate}, September, 1--56. \url{https://doi.org/10.1175/JCLI-D-21-0165.1}.

\leavevmode\vadjust pre{\hypertarget{ref-Zerefos2009}{}}%
Zerefos, C. S., K. Eleftheratos, C. Meleti, S. Kazadzis, A. Romanou, C. Ichoku, G. Tselioudis, and A. Bais. 2009. {``Solar Dimming and Brightening over Thessaloniki, Greece, and Beijing, China.''} \emph{Tellus B: Chemical and Physical Meteorology} 61 (4): 657. \url{https://doi.org/10.1111/j.1600-0889.2009.00425.x}.

\end{CSLReferences}

\hypertarget{appendix}{%
\section*{Appendix}\label{appendix}}
\addcontentsline{toc}{section}{Appendix}

\begin{figure}[h!]

{\centering \includegraphics[width=0.7\linewidth]{./images/LongtermTrends-5} 

}

\caption{Anomaly (\%) of the daily clear-sky SDR, relative to climatological values for1993 - 2023. The black line shows the long term linear trend for clear-sky conditions.}(\#fig:trendCLEAR)
\end{figure}

\begin{figure}[h!]

{\centering \includegraphics[width=0.7\linewidth]{./images/LongtermTrends-8} 

}

\caption{Anomaly (\%) of the daily cloud-sky SDR, relative to climatological values for 1993 - 2023. The black line shows the long term linear trend for cloud-sky conditions.}(\#fig:trendCLOUD)
\end{figure}

\begin{figure}[h!]

{\centering \includegraphics[width=1\linewidth]{./images/SzaTrendsSeasTogether-2} 

}

\caption{Seasonal distribution of the SDR's long term trends by SZA before and after noon. The solid data points represent values with an acceptable statistical significance ($p<0.005$), the trends were computed with SZA bins of $1^\circ$ for each day, before and after local noon, all graphs have the same scale. Bins with $p<0.005$ or with less than 85 data points may be missing from the view.}(\#fig:SZAtrendSeason)
\end{figure}

\end{document}
