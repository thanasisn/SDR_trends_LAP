% Options for packages loaded elsewhere
\PassOptionsToPackage{unicode}{hyperref}
\PassOptionsToPackage{hyphens}{url}
%
\documentclass[
  preprint, 3p, authoryear]{article}
\usepackage{amsmath,amssymb}
\usepackage{lmodern}
\usepackage{iftex}
\ifPDFTeX
  \usepackage[T1]{fontenc}
  \usepackage[utf8]{inputenc}
  \usepackage{textcomp} % provide euro and other symbols
\else % if luatex or xetex
  \usepackage{unicode-math}
  \defaultfontfeatures{Scale=MatchLowercase}
  \defaultfontfeatures[\rmfamily]{Ligatures=TeX,Scale=1}
\fi
% Use upquote if available, for straight quotes in verbatim environments
\IfFileExists{upquote.sty}{\usepackage{upquote}}{}
\IfFileExists{microtype.sty}{% use microtype if available
  \usepackage[]{microtype}
  \UseMicrotypeSet[protrusion]{basicmath} % disable protrusion for tt fonts
}{}
\makeatletter
\@ifundefined{KOMAClassName}{% if non-KOMA class
  \IfFileExists{parskip.sty}{%
    \usepackage{parskip}
  }{% else
    \setlength{\parindent}{0pt}
    \setlength{\parskip}{6pt plus 2pt minus 1pt}}
}{% if KOMA class
  \KOMAoptions{parskip=half}}
\makeatother
\usepackage{xcolor}
\IfFileExists{xurl.sty}{\usepackage{xurl}}{} % add URL line breaks if available
\IfFileExists{bookmark.sty}{\usepackage{bookmark}}{\usepackage{hyperref}}
\hypersetup{
  pdftitle={Trends of SDR in Thessaloniki},
  pdfkeywords={GHI, SDR, Solar Brigthening/Dimming},
  hidelinks,
  pdfcreator={LaTeX via pandoc}}
\urlstyle{same} % disable monospaced font for URLs
\usepackage[margin=1in]{geometry}
\usepackage{longtable,booktabs,array}
\usepackage{calc} % for calculating minipage widths
% Correct order of tables after \paragraph or \subparagraph
\usepackage{etoolbox}
\makeatletter
\patchcmd\longtable{\par}{\if@noskipsec\mbox{}\fi\par}{}{}
\makeatother
% Allow footnotes in longtable head/foot
\IfFileExists{footnotehyper.sty}{\usepackage{footnotehyper}}{\usepackage{footnote}}
\makesavenoteenv{longtable}
\usepackage{graphicx}
\makeatletter
\def\maxwidth{\ifdim\Gin@nat@width>\linewidth\linewidth\else\Gin@nat@width\fi}
\def\maxheight{\ifdim\Gin@nat@height>\textheight\textheight\else\Gin@nat@height\fi}
\makeatother
% Scale images if necessary, so that they will not overflow the page
% margins by default, and it is still possible to overwrite the defaults
% using explicit options in \includegraphics[width, height, ...]{}
\setkeys{Gin}{width=\maxwidth,height=\maxheight,keepaspectratio}
% Set default figure placement to htbp
\makeatletter
\def\fps@figure{htbp}
\makeatother
\setlength{\emergencystretch}{3em} % prevent overfull lines
\providecommand{\tightlist}{%
  \setlength{\itemsep}{0pt}\setlength{\parskip}{0pt}}
\setcounter{secnumdepth}{5}
\newlength{\cslhangindent}
\setlength{\cslhangindent}{1.5em}
\newlength{\csllabelwidth}
\setlength{\csllabelwidth}{3em}
\newlength{\cslentryspacingunit} % times entry-spacing
\setlength{\cslentryspacingunit}{\parskip}
\newenvironment{CSLReferences}[2] % #1 hanging-ident, #2 entry spacing
 {% don't indent paragraphs
  \setlength{\parindent}{0pt}
  % turn on hanging indent if param 1 is 1
  \ifodd #1
  \let\oldpar\par
  \def\par{\hangindent=\cslhangindent\oldpar}
  \fi
  % set entry spacing
  \setlength{\parskip}{#2\cslentryspacingunit}
 }%
 {}
\usepackage{calc}
\newcommand{\CSLBlock}[1]{#1\hfill\break}
\newcommand{\CSLLeftMargin}[1]{\parbox[t]{\csllabelwidth}{#1}}
\newcommand{\CSLRightInline}[1]{\parbox[t]{\linewidth - \csllabelwidth}{#1}\break}
\newcommand{\CSLIndent}[1]{\hspace{\cslhangindent}#1}
\usepackage{subcaption}
\ifLuaTeX
  \usepackage{selnolig}  % disable illegal ligatures
\fi

\title{Trends of SDR in Thessaloniki}
\author{true \and true}
\date{2023-06-29}

\begin{document}
\maketitle
\begin{abstract}
Study of GHI and DNI radiation for `clear-sky' and all-sky conditions.
It consists of two paragraphs.
\end{abstract}

Cloud ``shrinking'' and ``optical thinning'' in the ``dimming'' period and a subsequent recovery in the ``brightening'' period over China

aerosols not only strengthen but also weaken the
growth of clouds with respect to their coverage and
optical thickness, depending on the levels of pollution
and the associated amounts of aerosols as postulated
in a conceptual framework (Wild 2009a, 2012, Yang
et al 2012).

\hypertarget{introduction.}{%
\section{Introduction.}\label{introduction.}}

The shortwave downward solar irradiance (SDR) at the Earth's surface plays a significant role, on Earths climate.
Changes of the SDR can be related to major changes on the Earth's energy budget, the mechanisms of climate change, and water and carbon cycle (Wild 2009).
Studies of SDR variability, have identified some distinct SDR trends on different regions of the world and for different time periods, using the term `brightening' for positive trends, and `dimming' for negative trends.
There are many cases on the long term records, showing a change of SDR trends direction, occurring roughly at the last decades of the 20th century.
Yang et al. (2021) reports a general dimming before 2000 and a brightening afterwards on multiple station in China.
Wild et al. (2021) shows a similar pattern, with the breaking point around 1980 from a 71 years record of a Central Europe station and for Brazil by Yamasoe et al. (2021).
Another approach, using AI aided spatial analysis on continent level, of numerous observational stations data, reach similar conclusions for the above regions and the global trend (Yuan, Leirvik, and Wild 2021).

There is a consensus, among researchers, that the major factors of the attenuation of SDR is the interaction with the atmospheric aerosols and the clouds.

The interaction between Aerosols and clouds, among other factors, has been analysed with modeling (Li et al. 2016; Samset et al. 2018), showing the existence of feedback mechanisms between the two.

Similar links have been showed with observational data {[}Ohvril et al. (2009); Schwarz et al. (2020); {]}

Due to the variability of those factors and the interaction between them,

\ldots\ldots{} cloud formation occurrences and constitution nuclei. cite\ldots..

it is difficult to estimate the SDR trends on each site.

Multiple studies attempt to evaluate the phenomenon and the potential causes {[}; Wild (2012); Xia et al. (2007); Zerefos et al. (2009); and references therein{]}.

Investigate multiple interaction Li et al. (2016) Samset et al. (2018) data analysis and modeling

There is a also a need to evaluate the trends on different location

In this study, we examine the trends of SDR with ground-based measurements at Thessaloniki for the period 1993 - 2023, as derived from a CM-21 pyranometer.
We reevaluate and extend the dataset used by Bais et al. (2013), applying a different algorithm for the identification of clear-/cloud-sky instances, and we derive the radiation trends for this period under different sky conditions (all-sky, clear-sky and cloud-sky).

\ldots{}

Site location description and aerosols effect

\ldots{}

Our method to characterize the sky conditions and the definition of each sky condition has some subjectivity.
The algorithm was calibrated with the main focus to identify the presence of clouds on the sky dome, although there are marginal cases that there will be false positives or false negatives identifications.

\hypertarget{data-and-methodology.}{%
\section{Data and methodology.}\label{data-and-methodology.}}

The SDR is equivalent to the whole sky Global radiation, also refered as Global Horizontal Irradiance (GHI), and was obtained with a horizontal leveled, CM-21 pyranometer.
SDR data spans the period of
1993-04-12 to 2023-05-31.
In some algorithms we used as auxiliary data, the direct beam radiation (DNI) for comparisons, obtained by a CHP-1 pyrheliometer, with data availability from
2016-04-01 to 2023-05-31.

\ldots{}

\ldots{}

There are three distinct steps to the creation of this dataset:
a) the acquisition of radiation measurements from the sensors,
b) a radiation data quality check, and
c) the identification of ``clear sky'' conditions from the radiometric data
d) data aggregation and trend analysis.

For the acquisition of radiometric data,
the signal of the broadband instruments is sampled with a rate of \(1 \text{Hz}\). The mean and standard deviation values are recorded for every minute.
The measurements are corrected for the zero offset of the instrument signal.
As reference of the ``dark signal'' is used the Sun elevation angle bellow \(-10^\circ\), for a period of \(3 \text{h}\).
The signal is converted to radiation flux, using a ramped value of the instrument sensitivity, derived from the eight laboratory calibrations of the instrument, during the study period.

A manual screening is performed, to remove inconsistencies and erroneous recordings, that can occur randomly or systematically, during the long continuous operation of the instrument.
The manual screening is aided by a radiation data quality assurance procedure, adjusted for this site, based on methods of Long and Shi~(C. Long and Shi 2008 ; C. Long and Shi 2006).
Thus, problematic recordings have been excluded from further processing.
Although it is not possible to detect all the bad data, the large number of data and the aggregation scheme we use, can provide us with accurate radiometric measurements, for the scope of this study.

\hypertarget{clear-sky-identification}{%
\subsection{Clear sky identification}\label{clear-sky-identification}}

\ldots\ldots{}

As global radiation clear sky reference we are using the Haurwitz's model, adjusted for our site with a factor of 0.965 (Eq. \ref{eq:ahau}).
The selection of a clear sky reference model, was based on SDR observation from the period 2016 -- 2021.
Where, after an iterative optimization of eight simple models (Daneshyar--Paltridge--Proctor, Kasten--Czeplak, Haurwitz, Berger--Duffie, Adnot--Bourges--Campana--Gicquel, Robledo-Soler, Kasten and Ineichen-Perez) with different factors.
We found, that Haurwitz's model, adjusted with a factor of 0.965 has the lower root mean squared error (RMSE).
The tried models are described by Reno, Hansen, and Stein (2012) and tested by Reno and Hansen (2016).
The iterative optimization method, for the selection of the reference is discussed by C. N. Long and Ackerman (2000) and Reno and Hansen (2016).

\begin{equation}
\text{SDR}_\text{Clear Sky} = 0.965 \times 1098 \times \cos( \text{SZA} ) \times \exp \left( \frac{ - 0.057}{\cos(\text{SZA})} \right) \label{eq:ahau}
\end{equation}

The following criteria and thresholds were used to identify clear-sky conditions.
Each criterion was applied for a running window of \(11\) consecutive measurements/minutes, and the characterization is applied at the center value of the window.
A data point is consider as under cloud-sky condition if it fails to pass any of the criteria, all the other data points are characterized as clear-sky.

\hypertarget{mean-value-of-irradiance-during-the-time-period.}{%
\subsubsection{Mean value of irradiance during the time period.}\label{mean-value-of-irradiance-during-the-time-period.}}

The mean of the measured value \(\overline{G}_i\) must be inside an envelope based on the reference model \(\text{SDR}_\text{Clear Sky}\) (Eq. \ref{eq:MeanVIP}).

\begin{equation}
0.91 \times \overline{\text{SDR}}_{i\text{Clear Sky}} - 20
< \overline{G}_i <
1.095 \times \overline{\text{SDR}}_{i\text{Clear Sky}} + 30
\label{eq:MeanVIP}
\end{equation}

\hypertarget{max-value-of-irradiance-during-the-time-period.}{%
\subsubsection{Max value of irradiance during the time period.}\label{max-value-of-irradiance-during-the-time-period.}}

The running max measured value \(M_{Gi} = max[\text{SDR}_{i}]\), is compared to a similar constructed value from the reference \(M_{CSi} = max[\text{SDR}_{i\text{Clear Sky}}]\) (Eq. \ref{eq:MaxVIP}).

\begin{equation}
1 \times M_{CSi} - 75
< M_{Gi} <
1 \times M_{CSi} + 75
\label{eq:MaxVIP}
\end{equation}

\hypertarget{variability-in-irradiance-by-line-length.}{%
\subsubsection{Variability in irradiance by line length.}\label{variability-in-irradiance-by-line-length.}}

The length \(L\) (Eq. \ref{eq:VILeq}) of the sequence of line segments connecting the points of the SDR time
series for the measured values \(L\) and similar for the reference \(L_{CS}\), must be within the limits of Eq. \ref{eq:VILcr}.

\begin{equation}
L = \sum_{i=1}^{n-1}\sqrt{\left ( \text{SDR}_{i+1} - \text{SDR}_{i}\right )^2 + \left ( t_{i+1} - t_i \right )^2}
\label{eq:VILeq}
\end{equation}

\begin{equation}
1 \times L_{CSi} - 5 < L_i < 1.3 \times L_{CSi} + 13
\label{eq:VILcr}
\end{equation}

\hypertarget{variance-of-changes-in-the-time-series.}{%
\subsubsection{Variance of Changes in the Time series.}\label{variance-of-changes-in-the-time-series.}}

We calculate the standard deviation \(\sigma\) of the slope
(\(s\)) between sequential points in the time series, normalized by the average SDR during the time interval.

\begin{gather}
s_i = \frac{\text{SDR}_{i+1} - \text{SDR}_{i}}{t_{i+1} - t_i}, \forall i \in \left \{ 1, 2, \ldots, n-1 \right \} \label{eq:VCT1} \\
\bar{s} = \frac{1}{n-1} \sum_{i=1}^{n-1} s_i \label{eq:VCT2} \\
\sigma_i = \frac {\sqrt{\frac{1}{n-1} \sum_{i=1}^{n-1} \left( s_i - \bar{s} \right)^2} } {\bar{G_i}} \label{eq:VCT3}
\end{gather}

\begin{equation}
\sigma_i < \ensuremath{1.1\times 10^{-4}}
\label{eq:VCTcr}
\end{equation}

\hypertarget{variability-in-the-shape-of-the-irradiance-measurements.}{%
\subsubsection{Variability in the Shape of the irradiance Measurements.}\label{variability-in-the-shape-of-the-irradiance-measurements.}}

The maximum difference \(X\) between the change in measured irradiance and the change in clear sky
irradiance over each measurement interval.

\begin{gather}
x_i = \text{SDR}_{i+1} - \text{SDR}_{i} \forall i \in \left \{ 1, 2, \ldots, n-1 \right \} \label{eq:VSM1} \\
x_{CS,i} = \text{SDR}_{CS,i+1} - \text{SDR}_{CS,i} \forall i \in \left \{ 1, 2, \ldots, n-1 \right \} \label{eq:VSM2} \\
X_i = \max{\left \{ \left | x_i - x_{CS,i} \right | \right \}} \label{eq:VSM3}
\end{gather}

\begin{equation}
X_i < 7.5
\label{eq:VSMcr}
\end{equation}

\ldots.

In order to estimate the effect of the clouds on the SDR we created three datasets, by characterizing each one-minute measurement with a corresponding sky condition.
The all-sky conditions, containing all the valid measurements.
The clear-sky conditions data, where we have inferred that the sky was almost clear of clouds, and the remainder part the data as cloudy sky conditions data (cloud-sky).
To identify the clear-sky conditions we used the method proposed by C. N. Long and Ackerman (2000) and by Reno and Hansen (2016), that was adapted and configured for the site.
We have to note, that the definition of what constitutes as clear or cloudy sky, has some subjectivity, due to the used method of characterization.
As a result, the details of the definition are site specific, it relies on a combination of threshold of comparisons with ideal actinometric models and statistics on different signal behavior.

\hypertarget{data-and-data-selection.}{%
\subsection{Data and data selection.}\label{data-and-data-selection.}}

Due to a significant measurement uncertainty near, the horizon, we have to exclude
all measurements with SZA greater than \(85^\circ\).
Moreover, due to some obstructions around the site (hills and buildings), we excluded data with Azimuth angle between
\(35^\circ\) and \(120^\circ\) with SZA greater than \(80^\circ\).
On the latter instances, Sun is systematically, not visible by the instrument's location.
To make the measurements comparable throughout the dataset, we adjusted all 1-minute radiometric values to the mean Sun - Earth distance.
Subsequently, we made all measurements relative to the Total Solar Irradiance (TSI) at \(1 \text{au}\), in order to compensate for the Sun's intensity variability, using a homogenized time series of satellite TSI observations.
The TSI data we use, is a combination of data from NOAA (Coddington et al. 2005) (for 1993-04-12 - 2023-03-31) and adjusted data from LASP (2023) (for 2023-03-31 - 2023-05-31).
As a result, we can present all radiation data as a fraction of Sun's TSI.

\hypertarget{aggregation-of-radiometric-data}{%
\subsubsection{Aggregation of radiometric data}\label{aggregation-of-radiometric-data}}

Before further analysis and deseasonalization, we implement an appropriate aggregation scheme on the 1-minute data. To preserve the representativeness we use the following criteria:
a) for daily mean values we exclude instances with less than 180 valid data points,
b) accordingly, monthly values are computed by daily aggregated data, where months with less than 20 days are rejected.
The daily seasonal values were indexed based on the day of year number, and the monthly by the corresponding calendar month.
For the seasons of the year trends, we grouped the mean daily values by season (December - February: Winter, March - May: Spring, etc.).
Finally, for each data set, we remove the natural occurring seasonal variation.
This is done, by calculating the mean values for the appropriate time step, and subtracting the annual cycle from the actual data.
In addition to the previous aggregation scheme, each dataset was aggregate in \(1^\circ\) SZA bins, separate for cases before and after local noon.
This gave us an approximation of Sun's SZA contribution to the SDR brightening.

Thus, for each data set we have obtained the relative departure of the seasonal mean.
The statistical significance and aggregation scheme and filtering will be noted along with the corresponding results.

\hypertarget{results}{%
\section{Results}\label{results}}

\hypertarget{long-term-trends}{%
\subsection{Long-term trends}\label{long-term-trends}}

Using the mean daily SDR we produce the trends of each data set, as departure from the seasonal value.
Only for All-sky (Fig. \ref{fig:trendALL}) conditions, the statistical significance of the trends are acceptable (Tab. \ref{tab:trendtable}), to draw some conclusions.

\begin{figure}[h!]

{\centering \includegraphics[width=0.7\linewidth]{./images/LongtermTrends-2} 

}

\caption{Anomaly (\%) of the daily SDR relative to climatological values for 1993 - 2023. The black line shows the long term trend for all-sky conditions.}\label{fig:trendALL}
\end{figure}

\begin{longtable}[]{@{}
  >{\centering\arraybackslash}p{(\columnwidth - 6\tabcolsep) * \real{0.2048}}
  >{\centering\arraybackslash}p{(\columnwidth - 6\tabcolsep) * \real{0.1928}}
  >{\centering\arraybackslash}p{(\columnwidth - 6\tabcolsep) * \real{0.2169}}
  >{\centering\arraybackslash}p{(\columnwidth - 6\tabcolsep) * \real{0.3855}}@{}}
\caption{\label{tab:trendtable}Trends of daily means by sky conditions.
}\tabularnewline
\toprule
\begin{minipage}[b]{\linewidth}\centering
Trend {[}\%/year{]}
\end{minipage} & \begin{minipage}[b]{\linewidth}\centering
Trend p-value
\end{minipage} & \begin{minipage}[b]{\linewidth}\centering
Sky condition
\end{minipage} & \begin{minipage}[b]{\linewidth}\centering
Trend statistical signif. {[}\%{]}
\end{minipage} \\
\midrule
\endfirsthead
\toprule
\begin{minipage}[b]{\linewidth}\centering
Trend {[}\%/year{]}
\end{minipage} & \begin{minipage}[b]{\linewidth}\centering
Trend p-value
\end{minipage} & \begin{minipage}[b]{\linewidth}\centering
Sky condition
\end{minipage} & \begin{minipage}[b]{\linewidth}\centering
Trend statistical signif. {[}\%{]}
\end{minipage} \\
\midrule
\endhead
0.381 & 3.479e-21 & All sky cond. & 100 \\
0.3605 & 1.068e-16 & Clear sky cond. & 100 \\
0.3472 & 8.968e-09 & Cloudy cond. & 100 \\
\bottomrule
\end{longtable}

\hypertarget{long-term-trends-by-solar-zenith-angle.}{%
\subsection{Long term trends by Solar zenith angle.}\label{long-term-trends-by-solar-zenith-angle.}}

Analyzing the long term trends of SDR to bins of SZA, we can see the contribution of the geometry and time in a diurnal level (Figures \ref{fig:szatrends}a and \ref{fig:szatrends}b). Although there is a seasonal distribution of SZA than in not presented here.

\begin{figure}[h!]

{\centering \subfloat[Trend distribution for all-sky conditions.\label{fig:szatrends-1}]{\includegraphics[width=.35\linewidth]{./images/SzaTrends-13} }\subfloat[Trend distribution for cloud-sky conditions.\label{fig:szatrends-2}]{\includegraphics[width=.35\linewidth]{./images/SzaTrends-21} }

}

\caption{Distribution of the SDR's long term trends by SZA.}\label{fig:szatrends}
\end{figure}

\hypertarget{long-term-trends-by-season-of-year}{%
\subsection{Long term trends by season of year}\label{long-term-trends-by-season-of-year}}

Similar we have produced the deseasonalized trend for different sky conditions for each season of the year, using the corresponding mean monthly values.
This can give us a better understanding of the annual variability of the trends.

\begin{figure}[h!]

{\centering \includegraphics[width=1\linewidth]{./images/SeasonalTrendsTogether-1} 

}

\caption{Trends by season for all condition. Displaying monthly means of daily means.}\label{fig:seasonalALL}
\end{figure}

\begin{longtable}[]{@{}
  >{\centering\arraybackslash}p{(\columnwidth - 6\tabcolsep) * \real{0.2361}}
  >{\centering\arraybackslash}p{(\columnwidth - 6\tabcolsep) * \real{0.2222}}
  >{\centering\arraybackslash}p{(\columnwidth - 6\tabcolsep) * \real{0.2500}}
  >{\centering\arraybackslash}p{(\columnwidth - 6\tabcolsep) * \real{0.1250}}@{}}
\caption{\label{tab:trendseasontable}Trends of daily means by sky conditions for the seasons of the year. (continued below)}\tabularnewline
\toprule
\begin{minipage}[b]{\linewidth}\centering
Trend {[}\%/year{]}
\end{minipage} & \begin{minipage}[b]{\linewidth}\centering
Trend p-value
\end{minipage} & \begin{minipage}[b]{\linewidth}\centering
Sky condition
\end{minipage} & \begin{minipage}[b]{\linewidth}\centering
Season
\end{minipage} \\
\midrule
\endfirsthead
\toprule
\begin{minipage}[b]{\linewidth}\centering
Trend {[}\%/year{]}
\end{minipage} & \begin{minipage}[b]{\linewidth}\centering
Trend p-value
\end{minipage} & \begin{minipage}[b]{\linewidth}\centering
Sky condition
\end{minipage} & \begin{minipage}[b]{\linewidth}\centering
Season
\end{minipage} \\
\midrule
\endhead
0.7215 & 0.001209 & All sky cond. & Winter \\
0.1679 & 0.1083 & All sky cond. & Spring \\
0.128 & 0.08705 & All sky cond. & Summer \\
0.4239 & 0.006659 & All sky cond. & Autumn \\
0.8715 & 0.000269 & Clear sky cond. & Winter \\
0.09719 & 0.3923 & Clear sky cond. & Spring \\
-0.02369 & 0.7972 & Clear sky cond. & Summer \\
0.3583 & 0.02996 & Clear sky cond. & Autumn \\
-0.02581 & 0.9335 & Cloudy cond. & Winter \\
0.3557 & 0.06914 & Cloudy cond. & Spring \\
0.442 & 0.004988 & Cloudy cond. & Summer \\
0.4776 & 0.01955 & Cloudy cond. & Autumn \\
\bottomrule
\end{longtable}

\begin{longtable}[]{@{}
  >{\centering\arraybackslash}p{(\columnwidth - 0\tabcolsep) * \real{0.4444}}@{}}
\toprule
\begin{minipage}[b]{\linewidth}\centering
Trend statistical signif. {[}\%{]}
\end{minipage} \\
\midrule
\endhead
99.88 \\
89.17 \\
91.29 \\
99.33 \\
99.97 \\
60.77 \\
20.28 \\
97 \\
6.646 \\
93.09 \\
99.5 \\
98.04 \\
\bottomrule
\end{longtable}

\hypertarget{consistency-of-the-trends}{%
\subsection{Consistency of the trends}\label{consistency-of-the-trends}}

A method to evaluate changes in the long term trend is to use the cumulative sum of the variable.

Figure \ref{fig:cumsum}

\ldots{} different in aggregation \ldots{}

\begin{figure}[h!]

{\centering \subfloat[one plot\label{fig:cumsum-1}]{\includegraphics[width=.35\linewidth]{./images/CumulativeDailyCumSum-1} }\subfloat[the other one\label{fig:cumsum-2}]{\includegraphics[width=.35\linewidth]{./images/CumulativeDailyCumSum-9} }

}

\caption{two plots}\label{fig:cumsum}
\end{figure}

\hypertarget{conclusions}{%
\section{Conclusions}\label{conclusions}}

Our result for all-sky condition (\(0.38\%/year\)), reaffirm the previous results of Bais et al. (2013) for the site.
The increase of this trend, shows that the phenomenon and probably the causes have been amplified.

\ldots..
Also, the trend of \(0.35\%/year\) for cloud-sky condition indicate the major part that clouds play on the
\ldots..

\ldots\ldots.

About:

\begin{itemize}
\item
  long terms trends
\item
  seasonal trends
\item
  effect of SZA
\item
  effect of clouds
\item
  Aerosols
\end{itemize}

similar results with Evidence for Clear‐Sky Dimming and Brightening in Central Europe\_Wild2021.pdf
inclue in discusion

\begin{longtable}[]{@{}c@{}}
\toprule
\endhead
\textbf{END} \\
\bottomrule
\end{longtable}

\hypertarget{refs}{}
\begin{CSLReferences}{1}{0}
\leavevmode\vadjust pre{\hypertarget{ref-Bais2013}{}}%
Bais, A. F., T. Drosoglou, C. Meleti, K. Tourpali, and N. Kouremeti. 2013. {``Changes in Surface Shortwave Solar Irradiance from 1993 to 2011 at Thessaloniki (Greece).''} \emph{International Journal of Climatology} 33 (13): 2871--76. \url{https://doi.org/f5dzz5}.

\leavevmode\vadjust pre{\hypertarget{ref-Coddington2005}{}}%
Coddington, Odele, Judith L. Lean, Doug Lindholm, Peter Pilewskie, Martin Snow, and NOAA CDR Program. 2005. {``{NOAA} Climate Data Record ({CDR}) of Total Solar Irradiance ({TSI}), {NRLTSI} Version 2. {D}aily.''} 2005. \url{https://doi.org/10.7289/V55B00C1}.

\leavevmode\vadjust pre{\hypertarget{ref-LASP2023}{}}%
LASP. 2023. {``TSIS-1 Total Solar Irradiance - Six Hour Average.''} Laboratory for Atmospheric; Space Physics, University of Colorado, Boulder. 2023. \url{https://lasp.colorado.edu/tsis/data/}.

\leavevmode\vadjust pre{\hypertarget{ref-Li2016}{}}%
Li, Zhanqing, W. K.‐M. Lau, V. Ramanathan, G. Wu, Y. Ding, M. G. Manoj, J. Liu, et al. 2016. {``Aerosol and Monsoon Climate Interactions over Asia.''} \emph{Reviews of Geophysics} 54 (4): 866--929. \url{https://doi.org/10.1002/2015RG000500}.

\leavevmode\vadjust pre{\hypertarget{ref-Long2000}{}}%
Long, Charles N., and Thomas P. Ackerman. 2000. {``Identification of Clear Skies from Broadband Pyranometer Measurements and Calculation of Downwelling Shortwave Cloud Effects.''} \emph{Journal of Geophysical Research: Atmospheres} 105 (D12, D12): 15609--26. \url{https://doi.org/10.1029/2000jd900077}.

\leavevmode\vadjust pre{\hypertarget{ref-Long2006}{}}%
Long, C., and Y. Shi. 2006. {``The QCRad Value Added Product: Surface Radiation Measurement Quality Control Testing, Including Climatology Configurable Limits.''} DOE/SC-ARM/TR-074. Office of Science, Office of Biological; Environmental Research, U.S. Department of Energy.

\leavevmode\vadjust pre{\hypertarget{ref-Long2008a}{}}%
---------. 2008. {``An Automated Quality Assessment and Control Algorithm for Surface Radiation Measurements.''} \emph{The Open Atmospheric Science Journal}, 23--37.

\leavevmode\vadjust pre{\hypertarget{ref-Ohvril2009}{}}%
Ohvril, Hanno, Hilda Teral, Lennart Neiman, Martin Kannel, Marika Uustare, Mati Tee, Viivi Russak, et al. 2009. {``Global Dimming and Brightening Versus Atmospheric Column Transparency, Europe, 1906--2007.''} \emph{Journal of Geophysical Research} 114 (May). \url{https://doi.org/10.1029/2008JD010644}.

\leavevmode\vadjust pre{\hypertarget{ref-Reno2016}{}}%
Reno, Matthew J., and Clifford W. Hansen. 2016. {``Identification of Periods of Clear Sky Irradiance in Time Series of GHI Measurements.''} \emph{Renewable Energy} 90: 520--31. \url{https://doi.org/gq3sbg}.

\leavevmode\vadjust pre{\hypertarget{ref-Reno2012}{}}%
Reno, Matthew J., Clifford W. Hansen, and Joshua S. Stein. 2012. {``Global Horizontal Irradiance Clear Sky Models: Implementation and Analysis.''} SAND2012-2389, 1039404. \url{https://doi.org/gq5npv}.

\leavevmode\vadjust pre{\hypertarget{ref-Samset2018}{}}%
Samset, B. H., M. Sand, C. J. Smith, S. E. Bauer, P. M. Forster, J. S. Fuglestvedt, S. Osprey, and C.‐F. Schleussner. 2018. {``Climate Impacts from a Removal of Anthropogenic Aerosol Emissions.''} \emph{Geophysical Research Letters} 45 (2): 1020--29. \url{https://doi.org/10.1002/2017GL076079}.

\leavevmode\vadjust pre{\hypertarget{ref-Schwarz2020}{}}%
Schwarz, M., D. Folini, S. Yang, R. P. Allan, and M. Wild. 2020. {``Changes in Atmospheric Shortwave Absorption as Important Driver of Dimming and Brightening.''} \emph{Nature Geoscience} 13 (2): 110--15. \url{https://doi.org/10.1038/s41561-019-0528-y}.

\leavevmode\vadjust pre{\hypertarget{ref-Wild2009}{}}%
Wild, Martin. 2009. {``Global Dimming and Brightening: A Review.''} \emph{Journal of Geophysical Research Atmospheres} 114 (12): 1--31. \url{https://doi.org/bcq}.

\leavevmode\vadjust pre{\hypertarget{ref-Wild2012}{}}%
---------. 2012. {``Enlightening Global Dimming and Brightening.''} \emph{Bulletin of the American Meteorological Society} 93 (1): 27--37. \url{https://doi.org/b3w2h4}.

\leavevmode\vadjust pre{\hypertarget{ref-Wild2021}{}}%
Wild, Martin, Stephan Wacker, Su Yang, and Arturo Sanchez‐Lorenzo. 2021. {``Evidence for Clear‐sky Dimming and Brightening in Central Europe.''} \emph{Geophysical Research Letters} 48 (6). \url{https://doi.org/10.1029/2020GL092216}.

\leavevmode\vadjust pre{\hypertarget{ref-Xia2007}{}}%
Xia, Xiangao, Hongbin Chen, Zhanqing Li, Pucai Wang, and Jiankai Wang. 2007. {``Significant Reduction of Surface Solar Irradiance Induced by Aerosols in a Suburban Region in Northeastern China.''} \emph{Journal of Geophysical Research Atmospheres} 112 (22): 1--9. \url{https://doi.org/cdtntw}.

\leavevmode\vadjust pre{\hypertarget{ref-Yamasoe2021}{}}%
Yamasoe, Marcia Akemi, Nilton Manuel Évora Rosário, Samantha Novaes Santos Martins Almeida, and Martin Wild. 2021. {``Fifty-Six Years of Surface Solar Radiation and Sunshine Duration over São Paulo, Brazil: 1961--2016.''} \emph{Atmospheric Chemistry and Physics} 21 (9): 6593--603. \url{https://doi.org/10.5194/acp-21-6593-2021}.

\leavevmode\vadjust pre{\hypertarget{ref-Yang2021}{}}%
Yang, Su, Zijiang Zhou, Yu Yu, and Martin Wild. 2021. {``Cloud {`Shrinking'} and {`Optical Thinning'} in the {`Dimming'} Period and a Subsequent Recovery in the {`Brightening'} Period over China.''} \emph{Environmental Research Letters}, January. \url{https://doi.org/10.1088/1748-9326/abdf89}.

\leavevmode\vadjust pre{\hypertarget{ref-Yuan2021}{}}%
Yuan, Menghan, Thomas Leirvik, and Martin Wild. 2021. {``Global Trends in Downward Surface Solar Radiation from Spatial Interpolated Ground Observations During 1961-2019.''} \emph{Journal of Climate}, September, 1--56. \url{https://doi.org/10.1175/JCLI-D-21-0165.1}.

\leavevmode\vadjust pre{\hypertarget{ref-Zerefos2009}{}}%
Zerefos, C. S., K. Eleftheratos, C. Meleti, S. Kazadzis, A. Romanou, C. Ichoku, G. Tselioudis, and A. Bais. 2009. {``Solar Dimming and Brightening over Thessaloniki, Greece, and Beijing, China.''} \emph{Tellus B: Chemical and Physical Meteorology} 61 (4): 657. \url{https://doi.org/10.1111/j.1600-0889.2009.00425.x}.

\end{CSLReferences}

\end{document}
