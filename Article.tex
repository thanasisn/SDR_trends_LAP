\documentclass[preprint, 3p,
authoryear]{elsarticle} %review=doublespace preprint=single 5p=2 column
%%% Begin My package additions %%%%%%%%%%%%%%%%%%%

\usepackage[hyphens]{url}

  \journal{An awesome journal} % Sets Journal name

\usepackage{graphicx}
%%%%%%%%%%%%%%%% end my additions to header

\usepackage[T1]{fontenc}
\usepackage{lmodern}
\usepackage{amssymb,amsmath}
% TODO: Currently lineno needs to be loaded after amsmath because of conflict
% https://github.com/latex-lineno/lineno/issues/5
\usepackage{lineno} % add
\usepackage{ifxetex,ifluatex}
\usepackage{fixltx2e} % provides \textsubscript
% use upquote if available, for straight quotes in verbatim environments
\IfFileExists{upquote.sty}{\usepackage{upquote}}{}
\ifnum 0\ifxetex 1\fi\ifluatex 1\fi=0 % if pdftex
  \usepackage[utf8]{inputenc}
\else % if luatex or xelatex
  \usepackage{fontspec}
  \ifxetex
    \usepackage{xltxtra,xunicode}
  \fi
  \defaultfontfeatures{Mapping=tex-text,Scale=MatchLowercase}
  \newcommand{\euro}{€}
\fi
% use microtype if available
\IfFileExists{microtype.sty}{\usepackage{microtype}}{}
\usepackage[]{natbib}
\bibliographystyle{plainnat}

\ifxetex
  \usepackage[setpagesize=false, % page size defined by xetex
              unicode=false, % unicode breaks when used with xetex
              xetex]{hyperref}
\else
  \usepackage[unicode=true]{hyperref}
\fi
\hypersetup{breaklinks=true,
            bookmarks=true,
            pdfauthor={},
            pdftitle={Trends of SDR in Thessaloniki},
            colorlinks=false,
            urlcolor=blue,
            linkcolor=magenta,
            pdfborder={0 0 0}}

\setcounter{secnumdepth}{5}
% Pandoc toggle for numbering sections (defaults to be off)


% tightlist command for lists without linebreak
\providecommand{\tightlist}{%
  \setlength{\itemsep}{0pt}\setlength{\parskip}{0pt}}

% From pandoc table feature
\usepackage{longtable,booktabs,array}
\usepackage{calc} % for calculating minipage widths
% Correct order of tables after \paragraph or \subparagraph
\usepackage{etoolbox}
\makeatletter
\patchcmd\longtable{\par}{\if@noskipsec\mbox{}\fi\par}{}{}
\makeatother
% Allow footnotes in longtable head/foot
\IfFileExists{footnotehyper.sty}{\usepackage{footnotehyper}}{\usepackage{footnote}}
\makesavenoteenv{longtable}



\usepackage{subcaption}



\begin{document}


\begin{frontmatter}

  \title{Trends of SDR in Thessaloniki}
    \author[Laboratory of Atmospheric Physics]{Natsis Athanasios%
  \corref{cor1}%
  }
   \ead{natsisphysicist@gmail.com} 
    \author[Laboratory of Atmospheric Physics]{Alkiviadis Bais%
  %
  }
   \ead{abais@auth.gr} 
      \affiliation[Laboratory of Atmospheric Physics]{
    organization={Aristotle University of
Thessaloniki},addressline={Campus Box
149},city={Thessaloniki},postcode={54124},country={Greece},}
    \cortext[cor1]{Corresponding author}
  
  \begin{abstract}
  Study of GHI and DNI radiation for `clear-sky' and all-sky conditions.
  It consists of two paragraphs.
  \end{abstract}
    \begin{keyword}
    GHI \sep SDR \sep 
    Solar Brigthening/Dimming
  \end{keyword}
  
 \end{frontmatter}

Cloud ``shrinking'' and ``optical thinning'' in the ``dimming'' period
and a subsequent recovery in the ``brightening'' period over China

aerosols not only strengthen but also weaken the growth of clouds with
respect to their coverage and optical thickness, depending on the levels
of pollution and the associated amounts of aerosols as postulated in a
conceptual framework (Wild 2009a, 2012, Yang et al 2012).

\hypertarget{introduction.}{%
\section{Introduction.}\label{introduction.}}

The shortwave downward solar irradiance (SDR) at the Earth's surface
plays a significant role, on Earths climate. Changes of the SDR can be
related to major changes on factors affecting the radiative forcing
balance and human activities. Multiple studies attempt to evaluate the
phenomenon and the potential causes
\citep{Ohvril2009, Schwarz2020, Wild2009, Wild2012, Xia2007, Zerefos2009}.
Although, there is regional variation of SDR, with both positive and
negative trends, most of the studies agree that the major factors are
the interaction of aerosols and clouds.

In this study, we examine the trends of SDR with ground-based
measurements at Thessaloniki for the period 1993 - 2023, as derived from
a CM-21 pyranometer. We reevaluate and extend the dataset used by
\citet{Bais2013}, applying a different algorithm for the identification
of clear-/cloud-sky instances, and we derive the radiation trends for
this period under different sky conditions (all-sky, clear-sky and
cloud-sky).

\ldots{}

Site location description and aerosols effect

\ldots{}

Our method to characterize the sky conditions and the definition of each
sky condition has some subjectivity. The algorithm was calibrated with
the main focus to identify the presence of clouds on the sky dome,
although there are marginal cases that there will be false positives or
false negatives identifications.

\hypertarget{data-and-methodology.}{%
\section{Data and methodology.}\label{data-and-methodology.}}

The SDR is equivalent to the whole sky Global radiation, also refered as
Global Horizontal Irradiance (GHI), and was obtained with a horizontal
leveled, CM-21 pyranometer. SDR data spans the period of 1993-04-12 to
2023-05-31. In some algorithms we used as auxiliary data, the direct
beam radiation (DNI) for comparisons, obtained by a CHP-1 pyrheliometer,
with data availability from 2016-04-01 to 2023-05-31.

\ldots{}

\ldots{}

There are three distinct steps to the creation of this dataset: a) the
acquisition of radiation measurements from the sensors, b) a radiation
data quality check, and c) the identification of ``clear sky''
conditions from the radiometric data d) data aggregation and trend
analysis.

For the acquisition of radiometric data, the signal of the broadband
instruments is sampled with a rate of \(1 \text{Hz}\). The mean and
standard deviation values are recorded for every minute. The
measurements are corrected for the zero offset of the instrument signal.
As reference of the ``dark signal'' is used the Sun elevation angle
bellow \(-10^\circ\), for a period of \(3 \text{h}\). The signal is
converted to radiation flux, using a ramped value of the instrument
sensitivity, derived from the eight laboratory calibrations of the
instrument, during the study period.

A manual screening is performed, to remove inconsistencies and erroneous
recordings, that can occur randomly or systematically, during the long
continuous operation of the instrument. The manual screening is aided by
a radiation data quality assurance procedure, adjusted for this site,
based on methods of Long and
Shi~\citetext{\citeyear{long_automated_2008}; \citeyear{Long2006}}.
Thus, problematic recordings have been excluded from further processing.
Although it is not possible to detect all the bad data, the large number
of data and the aggregation scheme we use, can provide us with accurate
radiometric measurements, for the scope of this study.

\hypertarget{clear-sky-identification}{%
\subsection{Clear sky identification}\label{clear-sky-identification}}

\ldots\ldots{}

As global radiation clear sky reference we are using the Haurwitz's
model, adjusted for our site with a factor of 0.965 (Eq. \ref{eq:ahau}).
The selection of a clear sky reference model, was based on SDR
observation from the period 2016 -- 2021. Where, after an iterative
optimization of eight simple models (Daneshyar--Paltridge--Proctor,
Kasten--Czeplak, Haurwitz, Berger--Duffie,
Adnot--Bourges--Campana--Gicquel, Robledo-Soler, Kasten and
Ineichen-Perez) with different factors. We found, that Haurwitz's model,
adjusted with a factor of 0.965 has the lower root mean squared error
(RMSE). The tried models are described by \citet{Reno2012} and tested by
\citet{Reno2016}. The iterative optimization method, for the selection
of the reference is discussed by \citet{Long2000} and \citet{Reno2016}.

\begin{equation}
\text{SDR}_\text{Clear Sky} = 0.965 \times 1098 \times \cos( \text{SZA} ) \times \exp \left( \frac{ - 0.057}{\cos(\text{SZA})} \right) \label{eq:ahau}
\end{equation}

The following criteria and thresholds were used to identify clear-sky
conditions. Each criterion was applied for a running window of \(11\)
consecutive measurements/minutes, and the characterization is applied at
the center value of the window. A data point is consider as under
cloud-sky condition if it fails to pass any of the criteria, all the
other data points are characterized as clear-sky.

\hypertarget{mean-value-of-irradiance-during-the-time-period.}{%
\subsubsection{Mean value of irradiance during the time
period.}\label{mean-value-of-irradiance-during-the-time-period.}}

The mean of the measured value \(\overline{G}_i\) must be inside an
envelope based on the reference model \(\text{SDR}_\text{Clear Sky}\)
(Eq. \ref{eq:MeanVIP}).

\begin{equation}
0.91 \times \overline{\text{SDR}}_{i\text{Clear Sky}} - 20
< \overline{G}_i <
1.095 \times \overline{\text{SDR}}_{i\text{Clear Sky}} + 30
\label{eq:MeanVIP}
\end{equation}

\hypertarget{max-value-of-irradiance-during-the-time-period.}{%
\subsubsection{Max value of irradiance during the time
period.}\label{max-value-of-irradiance-during-the-time-period.}}

The running max measured value \(M_{Gi} = max[\text{SDR}_{i}]\), is
compared to a similar constructed value from the reference
\(M_{CSi} = max[\text{SDR}_{i\text{Clear Sky}}]\) (Eq. \ref{eq:MaxVIP}).

\begin{equation}
1 \times M_{CSi} - 75
< M_{Gi} <
1 \times M_{CSi} + 75
\label{eq:MaxVIP}
\end{equation}

\hypertarget{variability-in-irradiance-by-line-length.}{%
\subsubsection{Variability in irradiance by line
length.}\label{variability-in-irradiance-by-line-length.}}

The length \(L\) (Eq. \ref{eq:VILeq}) of the sequence of line segments
connecting the points of the SDR time series for the measured values
\(L\) and similar for the reference \(L_{CS}\), must be within the
limits of Eq. \ref{eq:VILcr}.

\begin{equation}
L = \sum_{i=1}^{n-1}\sqrt{\left ( \text{SDR}_{i+1} - \text{SDR}_{i}\right )^2 + \left ( t_{i+1} - t_i \right )^2}
\label{eq:VILeq}
\end{equation}

\begin{equation}
1 \times L_{CSi} - 5 < L_i < 1.3 \times L_{CSi} + 13
\label{eq:VILcr}
\end{equation}

\hypertarget{variance-of-changes-in-the-time-series.}{%
\subsubsection{Variance of Changes in the Time
series.}\label{variance-of-changes-in-the-time-series.}}

We calculate the standard deviation \(\sigma\) of the slope (\(s\))
between sequential points in the time series, normalized by the average
SDR during the time interval.

\begin{gather}
s_i = \frac{\text{SDR}_{i+1} - \text{SDR}_{i}}{t_{i+1} - t_i}, \forall i \in \left \{ 1, 2, \ldots, n-1 \right \} \label{eq:VCT1} \\
\bar{s} = \frac{1}{n-1} \sum_{i=1}^{n-1} s_i \label{eq:VCT2} \\
\sigma_i = \frac {\sqrt{\frac{1}{n-1} \sum_{i=1}^{n-1} \left( s_i - \bar{s} \right)^2} } {\bar{G_i}} \label{eq:VCT3}
\end{gather}

\begin{equation}
\sigma_i < \ensuremath{1.1\times 10^{-4}}
\label{eq:VCTcr}
\end{equation}

\hypertarget{variability-in-the-shape-of-the-irradiance-measurements.}{%
\subsubsection{Variability in the Shape of the irradiance
Measurements.}\label{variability-in-the-shape-of-the-irradiance-measurements.}}

The maximum difference \(X\) between the change in measured irradiance
and the change in clear sky irradiance over each measurement interval.

\begin{gather}
x_i = \text{SDR}_{i+1} - \text{SDR}_{i} \forall i \in \left \{ 1, 2, \ldots, n-1 \right \} \label{eq:VSM1} \\
x_{CS,i} = \text{SDR}_{CS,i+1} - \text{SDR}_{CS,i} \forall i \in \left \{ 1, 2, \ldots, n-1 \right \} \label{eq:VSM2} \\
X_i = \max{\left \{ \left | x_i - x_{CS,i} \right | \right \}} \label{eq:VSM3}
\end{gather}

\begin{equation}
X_i < 7.5
\label{eq:VSMcr}
\end{equation}

\ldots.

In order to estimate the effect of the clouds on the SDR we created
three datasets, by characterizing each one-minute measurement with a
corresponding sky condition. The all-sky conditions, containing all the
valid measurements. The clear-sky conditions data, where we have
inferred that the sky was almost clear of clouds, and the remainder part
the data as cloudy sky conditions data (cloud-sky). To identify the
clear-sky conditions we used the method proposed by \citet{Long2000} and
by \citet{Reno2016}, that was adapted and configured for the site. We
have to note, that the definition of what constitutes as clear or cloudy
sky, has some subjectivity, due to the used method of characterization.
As a result, the details of the definition are site specific, it relies
on a combination of threshold of comparisons with ideal actinometric
models and statistics on different signal behavior.

\hypertarget{data-and-data-selection.}{%
\subsection{Data and data selection.}\label{data-and-data-selection.}}

Due to a significant measurement uncertainty near, the horizon, we have
to exclude all measurements with SZA greater than \(85^\circ\).
Moreover, due to some obstructions around the site (hills and
buildings), we excluded data with Azimuth angle between \(35^\circ\) and
\(120^\circ\) with SZA greater than \(80^\circ\). On the latter
instances, Sun is systematically, not visible by the instrument's
location. To make the measurements comparable throughout the dataset, we
adjusted all 1-minute radiometric values to the mean Sun - Earth
distance. Subsequently, we made all measurements relative to the Total
Solar Irradiance (TSI) at \(1 \text{au}\), in order to compensate for
the Sun's intensity variability, using a homogenized time series of
satellite TSI observations. The TSI data we use, is a combination of
data from NOAA \citep{Coddington2005} (for 1993-04-12 - 2023-03-31) and
adjusted data from \citet{TSIStsi} (for 2023-03-31 - 2023-05-31). As a
result, we can present all radiation data as a fraction of Sun's TSI.

\hypertarget{aggregation-of-radiometric-data}{%
\subsubsection{Aggregation of radiometric
data}\label{aggregation-of-radiometric-data}}

Before further analysis and deseasonalization, we implement an
appropriate aggregation scheme on the 1-minute data. To preserve the
representativeness we use the following criteria: a) for daily mean
values we exclude instances with less than 180 valid data points, b)
accordingly, monthly values are computed by daily aggregated data, where
months with less than 20 days are rejected. The daily seasonal values
were indexed based on the day of year number, and the monthly by the
corresponding calendar month. For the seasons of the year trends, we
grouped the mean daily values by season (December - February: Winter,
March - May: Spring, etc.). Finally, for each data set, we remove the
natural occurring seasonal variation. This is done, by calculating the
mean values for the appropriate time step, and subtracting the annual
cycle from the actual data. In addition to the previous aggregation
scheme, each dataset was aggregate in \(1^\circ\) SZA bins, separate for
cases before and after local noon. This gave us an approximation of
Sun's SZA contribution to the SDR brightening.

Thus, for each data set we have obtained the relative departure of the
seasonal mean. The statistical significance and aggregation scheme and
filtering will be noted along with the corresponding results.

\hypertarget{results}{%
\section{Results}\label{results}}

\hypertarget{long-term-trends}{%
\subsection{Long-term trends}\label{long-term-trends}}

Using the mean daily SDR we produce the trends of each data set, as
departure from the seasonal value. Only for All-sky (Fig.
\ref{fig:trendALL}) conditions, the statistical significance of the
trends are acceptable (Tab. \ref{tab:trendtable}), to draw some
conclusions.

\begin{figure}[h!]

{\centering \includegraphics[width=0.7\linewidth]{./images/LongtermTrends-2} 

}

\caption{Anomaly (\%) of the daily SDR relative to climatological values for 1993 - 2023. The black line shows the long term trend for all-sky conditions.}\label{fig:trendALL}
\end{figure}

\begin{longtable}[]{@{}
  >{\centering\arraybackslash}p{(\columnwidth - 6\tabcolsep) * \real{0.2048}}
  >{\centering\arraybackslash}p{(\columnwidth - 6\tabcolsep) * \real{0.1928}}
  >{\centering\arraybackslash}p{(\columnwidth - 6\tabcolsep) * \real{0.2169}}
  >{\centering\arraybackslash}p{(\columnwidth - 6\tabcolsep) * \real{0.3855}}@{}}
\caption{\label{tab:trendtable}Trends of daily means by sky conditions.
}\tabularnewline
\toprule\noalign{}
\begin{minipage}[b]{\linewidth}\centering
Trend {[}\%/year{]}
\end{minipage} & \begin{minipage}[b]{\linewidth}\centering
Trend p-value
\end{minipage} & \begin{minipage}[b]{\linewidth}\centering
Sky condition
\end{minipage} & \begin{minipage}[b]{\linewidth}\centering
Trend statistical signif. {[}\%{]}
\end{minipage} \\
\midrule\noalign{}
\endfirsthead
\toprule\noalign{}
\begin{minipage}[b]{\linewidth}\centering
Trend {[}\%/year{]}
\end{minipage} & \begin{minipage}[b]{\linewidth}\centering
Trend p-value
\end{minipage} & \begin{minipage}[b]{\linewidth}\centering
Sky condition
\end{minipage} & \begin{minipage}[b]{\linewidth}\centering
Trend statistical signif. {[}\%{]}
\end{minipage} \\
\midrule\noalign{}
\endhead
\bottomrule\noalign{}
\endlastfoot
0.381 & 3.479e-21 & All sky cond. & 100 \\
0.3605 & 1.068e-16 & Clear sky cond. & 100 \\
0.3472 & 8.968e-09 & Cloudy cond. & 100 \\
\end{longtable}

\hypertarget{long-term-trends-by-solar-zenith-angle.}{%
\subsection{Long term trends by Solar zenith
angle.}\label{long-term-trends-by-solar-zenith-angle.}}

Analyzing the long term trends of SDR to bins of SZA, we can see the
contribution of the geometry and time in a diurnal level (Figures
\ref{fig:szatrends}a and \ref{fig:szatrends}b). Although there is a
seasonal distribution of SZA than in not presented here.

\begin{figure}[h!]

{\centering \subfloat[Trend distribution for all-sky conditions.\label{fig:szatrends-1}]{\includegraphics[width=.35\linewidth]{./images/SzaTrends-13} }\subfloat[Trend distribution for cloud-sky conditions.\label{fig:szatrends-2}]{\includegraphics[width=.35\linewidth]{./images/SzaTrends-21} }

}

\caption{Distribution of the SDR's long term trends by SZA.}\label{fig:szatrends}
\end{figure}

\hypertarget{long-term-trends-by-season-of-year}{%
\subsection{Long term trends by season of
year}\label{long-term-trends-by-season-of-year}}

Similar we have produced the deseasonalized trend for different sky
conditions for each season of the year, using the corresponding mean
monthly values. This can give us a better understanding of the annual
variability of the trends.

\begin{figure}[h!]

{\centering \includegraphics[width=1\linewidth]{./images/SeasonalTrendsTogether-1} 

}

\caption{Trends by season for all condition. Displaying monthly means of daily means.}\label{fig:seasonalALL}
\end{figure}

\begin{longtable}[]{@{}
  >{\centering\arraybackslash}p{(\columnwidth - 6\tabcolsep) * \real{0.2361}}
  >{\centering\arraybackslash}p{(\columnwidth - 6\tabcolsep) * \real{0.2222}}
  >{\centering\arraybackslash}p{(\columnwidth - 6\tabcolsep) * \real{0.2500}}
  >{\centering\arraybackslash}p{(\columnwidth - 6\tabcolsep) * \real{0.1250}}@{}}
\caption{\label{tab:trendseasontable}Trends of daily means by sky
conditions for the seasons of the year. (continued
below)}\tabularnewline
\toprule\noalign{}
\begin{minipage}[b]{\linewidth}\centering
Trend {[}\%/year{]}
\end{minipage} & \begin{minipage}[b]{\linewidth}\centering
Trend p-value
\end{minipage} & \begin{minipage}[b]{\linewidth}\centering
Sky condition
\end{minipage} & \begin{minipage}[b]{\linewidth}\centering
Season
\end{minipage} \\
\midrule\noalign{}
\endfirsthead
\toprule\noalign{}
\begin{minipage}[b]{\linewidth}\centering
Trend {[}\%/year{]}
\end{minipage} & \begin{minipage}[b]{\linewidth}\centering
Trend p-value
\end{minipage} & \begin{minipage}[b]{\linewidth}\centering
Sky condition
\end{minipage} & \begin{minipage}[b]{\linewidth}\centering
Season
\end{minipage} \\
\midrule\noalign{}
\endhead
\bottomrule\noalign{}
\endlastfoot
0.7215 & 0.001209 & All sky cond. & Winter \\
0.1679 & 0.1083 & All sky cond. & Spring \\
0.128 & 0.08705 & All sky cond. & Summer \\
0.4239 & 0.006659 & All sky cond. & Autumn \\
0.8715 & 0.000269 & Clear sky cond. & Winter \\
0.09719 & 0.3923 & Clear sky cond. & Spring \\
-0.02369 & 0.7972 & Clear sky cond. & Summer \\
0.3583 & 0.02996 & Clear sky cond. & Autumn \\
-0.02581 & 0.9335 & Cloudy cond. & Winter \\
0.3557 & 0.06914 & Cloudy cond. & Spring \\
0.442 & 0.004988 & Cloudy cond. & Summer \\
0.4776 & 0.01955 & Cloudy cond. & Autumn \\
\end{longtable}

\begin{longtable}[]{@{}
  >{\centering\arraybackslash}p{(\columnwidth - 0\tabcolsep) * \real{0.4444}}@{}}
\toprule\noalign{}
\begin{minipage}[b]{\linewidth}\centering
Trend statistical signif. {[}\%{]}
\end{minipage} \\
\midrule\noalign{}
\endhead
\bottomrule\noalign{}
\endlastfoot
99.88 \\
89.17 \\
91.29 \\
99.33 \\
99.97 \\
60.77 \\
20.28 \\
97 \\
6.646 \\
93.09 \\
99.5 \\
98.04 \\
\end{longtable}

\hypertarget{consistency-of-the-trends}{%
\subsection{Consistency of the trends}\label{consistency-of-the-trends}}

A method to evaluate changes in the long term trend is to use the
cumulative sum of the variable.

Figure \ref{fig:cumsum}

\ldots{} different in aggregation \ldots{}

\begin{figure}[h!]

{\centering \subfloat[one plot\label{fig:cumsum-1}]{\includegraphics[width=.35\linewidth]{./images/CumulativeDailyCumSum-1} }\subfloat[the other one\label{fig:cumsum-2}]{\includegraphics[width=.35\linewidth]{./images/CumulativeDailyCumSum-9} }

}

\caption{two plots}\label{fig:cumsum}
\end{figure}

\hypertarget{conclusions}{%
\section{Conclusions}\label{conclusions}}

Our result for all-sky condition (\(0.38\%/year\)), reaffirm the
previous results of \citet{Bais2013} for the site. The increase of this
trend, shows that the phenomenon and probably the causes have been
amplified.

\ldots.. Also, the trend of \(0.35\%/year\) for cloud-sky condition
indicate the major part that clouds play on the \ldots..

\ldots\ldots.

About:

\begin{itemize}
\item
  long terms trends
\item
  seasonal trends
\item
  effect of SZA
\item
  effect of clouds
\item
  Aerosols
\end{itemize}

\begin{longtable}[]{@{}c@{}}
\toprule\noalign{}
\endhead
\bottomrule\noalign{}
\endlastfoot
\textbf{END} \\
\end{longtable}

\bibliography{/home/athan/LIBRARY/A\_Atmosphere/A\_Atmosphere.bib}


\end{document}
